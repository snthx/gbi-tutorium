% ===== handout mode =====
% Comment/uncomment this line to toggle handout mode
% \newcommand{\handout}{}

% Comment/uncoment this line to toogle Mortitz mode
% \newcommand{\Moritz}{}

% Comment/uncomment this line to toggle handout mode
% \newcommand{\handout}{}

% by Stephan

%% Moritz mode or Stephan mode
\ifdefined \Moritz

% This is a configuration file with private, tutor specific information.
% It is therefore excluded from the Git repository so changes in this file will not conflict in git commits.

% Copy this Template, rename to config.tex and add your information below.

\newcommand{\mymail}{moritz.laupichler@student.kit.edu} % Consider using your named student Mail address to keep your u-Account private.

\newcommand{\myname}{\href{mailto:\mymail}{Moritz Laupichler}}

\newcommand{\mytutnumber}{25}

\newcommand{\mytutinfos}{Dienstags, 5. Block (15:45-17:15 Uhr), SR -120}

\newcommand{\aboutMeFrame}{
	\begin{frame}{Euer Tutor}
		Name: \myname \\
		Alter: 21 Jahre \\
		Studiengang: Master Informatik, 1. Semester \\
		\vspace{1cm}
		\pause 
		\centering{Kontakt: \href{mailto:\mymail}{\mymail}}
	\end{frame}
} % Moritz mode
\else
\ifdefined \Alex

% This is a configuration file with private, tutor specific information.
% It is therefore excluded from the Git repository so changes in this file will not conflict in git commits.

% Copy this Template, rename to config.tex and add your information below.

\newcommand{\mymail}{alexander.klug@student.kit.edu} % Consider using your named student Mail address to keep your u-Account private.

\newcommand{\myname}{\href{mailto:\mymail}{Alexander Klug}}

\newcommand{\mytutnumber}{30}

\newcommand{\mytutinfos}{Mittwochs, 3. Block (11:30-13:00), SR -107}

\newcommand{\aboutMeFrame}{
	\begin{frame}{Euer Tutor}
		Name: \myname \\
		Alter: 19 Jahre \\
		Studiengang: Bachelor Informatik, 3. Semester \\
		\vspace{1cm}
		\pause 
		\centering{Kontakt: \href{mailto:\mymail}{\mymail}}
	\end{frame}
}

% Toggle Handout mode by including the following line before including style_tut
% and removing the % at the start (but do NOT remove it here, otherwise handout mode will always be on!)
% Please keep handout mode on in all commits!

% \newcommand{\handout}{} % Alex Mode
\else

% This is a configuration file with private, tutor specific information.
% It is therefore excluded from the Git repository so changes in this file will not conflict in git commits.

% Copy this Template, rename to config.tex and add your information below.

\newcommand{\mymail}{stephan.bohr@student.kit.edu} % Consider using your named student Mail address to keep your u-Account private.

\newcommand{\myname}{\href{mailto:\mymail}{Stephan Bohr}}

\newcommand{\mytutnumber}{25}

\newcommand{\mytutinfos}{Dienstags, 5. Block (15:45-17:15), SR -119}

\newcommand{\aboutMeFrame}{
	\begin{frame}{Euer Tutor}
		Name: \myname \\
		Alter: 20 Jahre \\
		Studiengang: Bachelor Informatik, 3. Semester \\
		\vspace{1cm}
		\pause 
		\centering{Kontakt: \href{mailto:\mymail}{\mymail}}
	\end{frame}
} % Stephan mode
\fi
\fi

%% Beamer-Klasse im korrekten Modus
\ifdefined \handout
\documentclass[handout]{beamer} % Handout mode
\else
\documentclass{beamer}
\fi
%\documentclass[18pt,parskip]{beamer}

%% SLIDE FORMAT

% use 'beamerthemekit' for standard 4:3 ratio
% for widescreen slides (16:9), use 'beamerthemekitwide'

\usepackage{../templates/KIT-slides/beamerthemekit}
%\usepackage{../templates/KIT-slides/beamerthemekitwide}

%% TITLE PICTURE

% if a custom picture is to be used on the title page, copy it into the 'logos'
% directory, in the line below, replace 'mypicture' with the 
% filename (without extension) and uncomment the following line
% (picture proportions: 63 : 20 for standard, 169 : 40 for wide
% *.eps format if you use latex+dvips+ps2pdf, 
% *.jpg/*.png/*.pdf if you use pdflatex)

\titleimage{../figures/titleimage/brain}

%% TITLE LOGO

% for a custom logo on the front page, copy your file into the 'logos'
% directory, insert the filename in the line below and uncomment it

%\titlelogo{mylogo}

% (*.eps format if you use latex+dvips+ps2pdf,
% *.jpg/*.png/*.pdf if you use pdflatex)

%% TikZ INTEGRATION

% use these packages for PCM symbols and UML classes
% \usepackage{templates/tikzkit}
% \usepackage{templates/tikzuml}

%\usepackage{tikz}
%\usetikzlibrary{matrix}
%\usetikzlibrary{arrows.meta}
%\usetikzlibrary{automata}
%\usetikzlibrary{tikzmark}

%%%%%%%%%%%%%%%%%%%%%%%%%
% Libertine font (Original GBI font)
\usepackage{libertine}
%\renewcommand*\familydefault{\sfdefault}  %% Only if the base font of the document is to be sans serif

%% Schönere Schriften
\usepackage[TS1,T1]{fontenc}

%% Deutsche Silbentrennung und Beschriftungen
\usepackage[ngerman]{babel}

%% UTF-8-Encoding
\usepackage[utf8]{inputenc}

%% Bibliotheken für viele mathematische Symbole
\usepackage{amsmath, amsfonts, amssymb}

%% Anzeigetiefe für Inhaltsverzeichnis: 1 Stufe
\setcounter{tocdepth}{1}

%% Hyperlinks
\usepackage{hyperref}
% I don't know why, but this works and only includes sections and NOT subsections in the pdf-bookmarks.
\hypersetup{bookmarksdepth=subsection}

%% remove navigation symbols
\setbeamertemplate{navigation symbols}{}

%% switch between "ngerman" and "english" for German/English style date and logos
\selectlanguage{ngerman}

%% for invisible pause texts instead of dimming
\setbeamercovered{invisible}

%%%%%%%%%%%% Shortcuts %%%%%%%%%%%%%
\newcommand{\nM}{\mathbb{M}}
\newcommand{\nR}{\mathbb{R}}
\newcommand{\nN}{\mathbb{N}}
\newcommand{\nZ}{\mathbb{Z}}
\newcommand{\nQ}{\mathbb{Q}}
\newcommand{\nB}{\mathbb{B}}
\newcommand{\nC}{\mathbb{C}}
\newcommand{\nK}{\mathbb{K}}
\newcommand{\nF}{\mathbb{F}}
\newcommand{\nG}{\mathbb{G}}
\newcommand{\nullel}{\mathcal{O}}
\newcommand{\einsel}{\mathds{1}}
\newcommand{\nP}{\mathbb{P}}
\newcommand{\Pot}{\mathcal{P}}
\renewcommand{\O}{\text{O}}

\newcommand{\set}[1]{\{ #1 \}}
\newcommand{\setc}[2]{\set{#1 \mid #2}}
\newcommand{\setC}[2]{\set{#1 \mid \text{ #2 }}}

\newcommand{\setsize}[1]{\; \mid #1 \mid \; }

\newcommand{\q}[1]{\textquotedblleft #1\textquotedblright}

%%%%%%%%%%%% INHALT %%%%%%%%%%%%%%%%

%% Wochennummer
%\newcounter{weeknum}

%% Titelinformationen
%\title[GBI Tutorium, Woche \theweeknum]{Grundbegriffe der Informatik \\ Tutorium \mytutnumber}
%\subtitle{Termin \theweeknum \ | \mydate \\ \myname}
\author[\myname]{\myname}
\institute{Fakultät für Informatik}
%\date{\mydate}

%% Titel einfügen
\newcommand{\titleframe}{\frame{\titlepage}\addtocounter{framenumber}{-1}}


%% Alles starten mit \starttut{X}
%\newcommand{\starttut}[1]{\setcounter{weeknum}{#1}\titleframe\frame{\frametitle{Inhalt}\tableofcontents} \AtBeginSection[]{%
%\begin{frame}
%	\tableofcontents[currentsection]
%\end{frame}\addtocounter{framenumber}{-1}}}


%\newcommand{\framePrevEpisode}{
%	\begin{frame}
%		\centering
%		\textbf{In the previous episode of GBI...}
%	\end{frame}
%}

%% Roadmap frame
%table of contents
\newcommand{\roadmap}{
	\frame{\frametitle{Roadmap}\tableofcontents}}

 \AtBeginSection[]{%
\begin{frame}
	\frametitle{Roadmap}
	\tableofcontents[currentsection]
\end{frame}%\addtocounter{framenumber}{-1}
}


%% ShowMessage frame
\newcommand{\showmessage}[1]{\frame{\frametitle{\phantom{1em}}\centering\textbf{#1}}}

%% Fragen
%% Lastframe
\newcommand{\questionframe}{\showmessage{Fragen?}}

%% Lastframe
\newcommand{\lastframe}{\showmessage{Vielen Dank für Eure Aufmerksamkeit! \\Bis nächste Woche :)}}

%% Thanks frame
\newcommand{\slideThanks}{
	\begin{frame}
		\frametitle{Credits}
		\begin{block}{}
			An der Erstellung des Foliensatzes haben mitgewirkt:\\[1em]
			\ifdefined \Moritz
			Stephan Bohr \\
			Alexander Klug \\
			\else
			\ifdefined \Alex
			Stephan Bohr \\
			Moritz Laupichler \\
			\else
			Moritz Laupichler \\
			Alexander Klug \\
			\fi
			\fi
			Katharina Wurz \\
			Thassilo Helmold \\
			Philipp Basler \\
			Nils Braun \\
			Dominik Doerner \\
			Ou Yue \\
		\end{block}
	\end{frame}
}

%% Verbatim
%\usepackage{moreverb}



\title[Relationen, Prädikatenlogik]{8. Tutorium\\ Relationen, Prädikatenlogik}
\subtitle{Grundbegriffe der Informatik, Tutorium \#\mytutnumber}
\date{\today}

\begin{document}
\titleframe
\roadmap

%%%%%%%%%% %%%%%%%%%%
\section[Kontextfreie Grammatiken]{Wdh.: Kontextfreie Grammatiken}
\subsection{Aufgabe}

\begin{frame}{Kontextfreie Grammatiken}
	\begin{block}{Definition: kontextfreie Grammatik}
		Eine \textbf{kontextfreie Grammatik} ist ein 4-Tupel $G = (N, T, S ,P)$ mit
		\begin{itemize}
			\item[N] Alphabet von Nichtterminalsymbolen
			\item[T] Alphabet von Terminalsymbolen ($N \cap T = \emptyset$)
			\item[S] Startsymbol ($S \in N$)
			\item[P] Produktionsmenge ($P \subseteq N \times (N \cup T)^\ast$)
		\end{itemize}
	\end{block}

	\begin{exampleblock}{Beispiel}
		Sei $A$ das deutsche Alphabet (mit Klein-/Großbuchstaben).\\
		Sei $G_{MI} = (\{S, M, I, N\}, A \cup \nN_+, S, P)$ mit
		\[
			P = \{S \to \text{M I N}, \qquad M \to \text{monkey}, \qquad I \to \text{island}, \qquad N \to 1 \mid 2 \mid 3 \}
		\]
	\end{exampleblock}
\end{frame}
\begin{frame}
	\frametitle{Klausuraufgabe (WS 2008)}
	\begin{exampleblock}{Klausuraufgabe}
	\begin{itemize}
		\item[(a)] Geben Sie eine kontextfreie Grammatik $$G = (N, \{a, b\}, S, P )$$ an, für die $L(G)$ die Menge aller Palindrome über dem Alphabet $\{a, b\}$ ist.
		\item[(b)] Geben Sie eine Ableitung der Wörter \emph{baaab} und \emph{abaaaba} aus dem Startsymbol Ihrer Grammatik an.
		\item[(c)] Beweisen Sie, dass Ihre Grammatik jedes Palindrom über dem Alphabet $\{a, b\}$ erzeugt.\\
		Tipp: Induktion: Wenn $n$ und $n+1$ gelten, dann gilt auch $n+2$
	\end{itemize}		
	\end{exampleblock}
\end{frame}

\begin{frame}
	\frametitle{Klausuraufgabe (WS 2008)}
	\begin{block}{Lösung}
	\begin{itemize}
		\item[(a)] Die Grammatik $$G = (\{S\}, \{a, b\}, S, P = \{S \to aSa \ | \ bSb \ | \ a \ | \ b \ | \varepsilon \})$$ erzeugt gerade die Menge der Palindrome.
		\item[(b)] Die Ableitungen der Wörter mit dieser Grammatik sind 
				$$S \Rightarrow bSb \Rightarrow baSab \Rightarrow baaab$$
				$$S \Rightarrow aSa \Rightarrow abSba \Rightarrow abaSaba \Rightarrow abaaaba$$
		\item[(c)] Siehe folgende Folien.
	\end{itemize}	
	\end{block}
\end{frame}

\begin{frame}
	\frametitle{Klausuraufgabe (WS 2008)}

	\begin{block}{Lösung (c): I.A. und I.V.}
	Sei $w$ ein Palindrom über $\{a, b\}$. Wir zeigen durch Induktion über $n = \vert w \vert$, dass alle Palindrome aus $S$ abgeleitet werden können.
	\begin{description}
		\item[I.A.] \begin{description}
						\item[$n = 0$] Das leere Wort $\varepsilon$ ist in einem Schritt aus $S$ ableitbar.
						\item[$n = 1$] Die einzigen Wörter aus $\{a, b\}^\ast$ der Länge 1 sind $a$ und $b$. Auch diese sind offensichtlich aus $S$ ableitbar. 
					\end{description}
		\item[I.V.] Für ein festes, aber beliebiges $n \in \nN_0$ gelte, dass alle Palindrome der Länge $n$ aus S abgeleitet werden können. 
		
	\end{description}
	\end{block}
\end{frame}

\begin{frame}
	\frametitle{Klausuraufgabe (WS 2008)}
	\begin{block}{Lösung (c): I.S.}
		\begin{description}
			\item[I.S.] Sei $w$ ein Palindrom der Länge $n + 2$. Das erste (und damit auch das letzte) Zeichen sei oBdA ein $a$. Dann gibt es ein $w' \in \{a, b\}^\ast$, so dass $w = aw'a$ ist. Da $w$ ein Palindrom ist, muss auch $w'$ ein Palindrom sein. Weiterhin gilt $|w'| = n$. \pause Nach IV gibt es somit eine Ableitung $S \Rightarrow^\ast w'$. Somit gibt es die Ableitung $$S \Rightarrow aSa \overset{IV}{\Rightarrow^\ast} aw'a = w$$ Es folgt $w \in L(G)$. \pause Entsprechendes gilt, wenn das erste Zeichen von w ein b ist. \\ 
		\end{description}
		Mit dem doppelten I.A. haben wir also gezeigt, dass alle Palindrome beliebiger Menge aus $S$ ableitbar sind.
	\end{block}

\end{frame}

\section{Relationen}
\subsection{Relationen}
\begin{frame}{Eigenschaften}
	\begin{Definition}
		Sei $R \subseteq A \times A$ eine (binäre) Relation auf der Menge $A$. Wir nennen $R$
		\begin{itemize}[<+->]
			\item \textbf{reflexiv} falls gilt $$\forall x \in A: (x,x) \in R$$
			\item \textbf{symmetrisch} falls gilt $$\forall x,y \in A: (x,y) \in R \implies (y,x) \in R$$
			\item \textbf{transitiv} falls gilt $$\forall x,y,z \in A: (x,y) \in R \text{ und } (y,z) \in R \implies (x,z) \in R$$
		\end{itemize}
	\end{Definition}
\end{frame}

\begin{frame}{Beispiele}
	\begin{itemize}
		\item Die Relation $=$ ist \pause reflexiv, symmetrisch und transitiv. Man nennt so etwas auch Äquivalenzrelation
		\item \pause Die Relation $<$ ist \pause nicht reflexiv und nicht symmetrisch, aber transitiv
		\item \pause Die Relation $\leq$ ist \pause reflexiv, nicht symmetrisch, aber transitiv
	\end{itemize}
\end{frame}

\begin{frame}{Produkt}
	\begin{Definition}
		Das \textbf{Produkt} von zwei Relationen $R \subseteq M \times N, S \subseteq N \times L$ definieren wir als $$S \circ R = \{(x,z) \in M \times L \mid \exists y \in N \ : \ (x,y) \in R \text{ und } (y,z) \in S \}$$
	\end{Definition}	
	\pause
	
	\begin{Definition}
		Die \textbf{Potenz} einer Relation $R \subseteq M \times M$ definieren wir als
		\begin{align*}
			R^0 &= I_M = \{(x,x) \mid x \in M \} \\
			R^{i+1} &= R^i \circ R
		\end{align*}
	\end{Definition}

	\pause
	\begin{block}{Beobachtung}
		Wenn $f$ und $g$ Funktionen sind (also linkstotale, rechtseindeutige Relationen), entspricht $f \circ g$ der Hintereinanderausführung von $f$ nach $g$.
	\end{block}
\end{frame}

\begin{frame}{Reflexiv-transitive Hülle}
	\begin{Definition}
		Die \textbf{reflexiv-transitive Hülle} einer Relation $R$ ist
		$$R^\ast = \bigcup \limits_{i=0}^\infty R^i$$
	\end{Definition}

	\pause
	\begin{block}{Satz}
		$R^*$ ist die kleinste Relation, die $R$ umfasst und reflexiv und
		transitiv ist.
	\end{block}

	\pause
	\begin{Beispiel}
		Sei $A = \{a, b, c, d, e\}$ und R = $\{(a, b), (b, c), (c, e)\} \subseteq A \times A$\\ \pause
		%TODO Align
		$R^*=\{(a,a), (b,b), (c,c), (d,d), (e,e),$ \\
		$(a,b), (b,c), (c,e),$ \\
		$(a,c), (b,e),(a,e)\}$
	\end{Beispiel}
	
\end{frame}

\section{Prädikatenlogik}
\newcommand{\plB}{\plfoo{B}}
\newcommand{\plE}{\plfoo{E}}

\subsection{Einstieg}
\begin{frame}{}
	\begin{block} {Aussagenlogik}
	\begin{itemize}
		\item \enquote{Es regnet und alle Vögel sind grau.}
		\item atomar: \enquote{Es regnet.}, \enquote{ Alle Vögel sind grau.}
		\item Diese beiden Aussagen lassen sich ihrerseits nicht in weitere Teilaussagen zerlegen!
	\end{itemize}
	\end{block}

	\pause
	\begin{block} {Prädikatenlogik}	
		\begin{itemize}
		\item In der Prädikatenlogik werden atomare Aussagen hinsichtlich ihrer inneren Struktur untersucht.
		\item \enquote{Alle Vögel sind grau}
		\item lässt sich in : \enquote{Alle Vögel}, \enquote{sind grau} zerlegen.
	\end{itemize}
	
	\end{block}
\end{frame}

\section{Prädikatenlogik: Syntax}

\begin{frame}{Aufgabe 1 (15/16, Blatt 7)}
	  Es seien $\CPL = \{ \}$, $\VPL = \{ \plx, \ply, \plz \}$, $\FPL = \{ \}$ und $\RPL = \{ \plE, \pleq \}$ mit $\ar(\plE) = 2$, und es sei $F$ die prädikatenlogische Formel
	\begin{equation*}
	\alnot \plexist \plx
	{\plka
		\plE{\plka \plx \plcomma \ply \plkz}
		\alor
		\alnot \plall \plz \plall \plx \plall \ply
		{\plka
			\plE{\plka \plx \plcomma \plz \plkz} \aland \plE{\plka \ply \plcomma \plz \plkz} \alimpl \plx \pleq \ply
			\plkz}
		\plkz}
	\end{equation*}
	
	\begin{block}{Aufgabe 1.0}
	Ist diese prädikatenlogische Formel syntaktisch korrekt?\\ \pause
	Ja
	\end{block}
\end{frame}

\begin{frame}{Syntax}
	\begin{block}{Aufbau von prädikatenlogischen Formeln}
	\begin{itemize}[<+->]
		\item \textbf{Terme}: Liefern \enquote{Werte}; Aus Konstanten, Variablen und Funktionssymbolen zusammengesetzt.
		\item \textbf{Atomare Formeln}: Liefern \enquote{Wahrheitswerte}; Aus Termen und Relationssymbolen zusammengesetzt.
		\item \textbf{Prädikatenlogische Formeln}: Verknüpfen und Quantifizieren atomare Formeln; Aus atomaren Formeln und aussagenlogischen Konnektiven sowie Quantoren zusammengesetzt. 
	\end{itemize}
	\end{block}
\end{frame}


\begin{frame}{Terme - Alphabet}
\begin{figure}
	\centering
	\includegraphics[scale=0.2]{TermeAlphabet.png} \hspace{2em} 
\end{figure} 
\end{frame}

\begin{frame}{Terme - Grammatik}
\begin{figure}[h!]
	\centering
	\includegraphics[scale=0.2]{TermeGrammatik.png} \hspace{2em} 
\end{figure} 
\end{frame}

\begin{frame}{Atomare Formeln}
\begin{figure}[h!]
	\centering
	\includegraphics[scale=0.2]{AFormeln.png} \hspace{2em} 
\end{figure} 
\end{frame}

\begin{frame}{Atomare Formeln - Beispiel}
\begin{figure}[h!]
	\centering
	\includegraphics[scale=0.2]{AFormelnBsp.png} \hspace{2em} 
\end{figure} 
\end{frame}

\begin{frame}{Prädikatenlogische Formeln}
\begin{figure}[h!]
	\centering
	\includegraphics[scale=0.2]{PFormeln.png} \hspace{2em} 
\end{figure} 
\end{frame}

\begin{frame}{Quiz}
\begin{figure}[h!]
	\centering
	\only<1|handout:1>{\includegraphics[scale=0.2]{q1.png}}
	\only<2|handout:2>{\includegraphics[scale=0.2]{q2.png}} 
	\only<3|handout:3>{\includegraphics[scale=0.2]{q3.png}}
	\only<4|handout:4>{\includegraphics[scale=0.2]{q4.png}}
	\only<5|handout:5>{\includegraphics[scale=0.2]{q5.png}}
	\only<6|handout:6>{\includegraphics[scale=0.2]{q6.png}}
	\only<7|handout:7>{\includegraphics[scale=0.2]{q7.png}}
	\hspace{2em}
\end{figure} 
\end{frame}

\begin{frame}{Freie und gebundene Variablenvorkommen}
	\begin{equation*}
	F = \alnot \plexist \plx
	{\plka
		\plE{\plka \plx \plcomma \ply \plkz}
		\alor
		\alnot \plall \plz \plall \plx \plall \ply
		{\plka
			\plE{\plka \plx \plcomma \plz \plkz} \aland \plE{\plka \ply \plcomma \plz \plkz} \alimpl \plx \pleq \ply
			\plkz}
		\plkz}
	\end{equation*}
	
	\begin{block}{Aufgabe 1.1}
		Welche Variablenvorkommen sind frei ($\fv$) und welche gebunden ($\bv$)?\\
		Ist die Formel geschlossen?
	\end{block}

	\pause
	\begin{block}{Lösung}
		Nur die Variable $\fv(F) = \{\ply\}$ kommt frei in $F$ vor.\\
		Genau die Variablen $\bv(F) = \{\plx, \ply, \plz\}$ kommen gebunden in $F$ vor.\\
		Da $\fv(F) \neq \emptyset$ ist $F$ nicht geschlossen.
	\end{block}
	
\end{frame}


\begin{frame}{Substitutionen}
	\begin{figure}[h!]
		\centering
		\includegraphics[scale=0.6]{subs.png} \hspace{2em} 
	\end{figure} 
\end{frame}

\begin{frame}{Substitutionen}
	Ersetzt werden nur \textbf{freie Variablenvorkommen}!\\
	Gebundene Vorkommen, also Variablen im Wirkungsbereich eines Quantors, werden \textbf{nicht} ersetzt.
	
	\pause
	\begin{figure}[h!]
		\centering
		\includegraphics[scale=0.5]{subs_bsp.png} \hspace{2em} 
	\end{figure} 
\end{frame}

\begin{frame}{Substitutionen: Kollisionsfreiheit}
	Bei einer \textbf{kollisionsfreien} Substitution werden keine Variablen \enquote{aus Versehen} gebunden.\\[1em]
	Ersetzen wir eine freie Variable $x$ durch einen Term, in dem die Variable $y$ frei vorkommt, so darf sich $x$ nicht im Wirkungsbereich eines Quantors über $y$ befinden.
	
	\pause
	\begin{Beispiel}
		$L = \plall \plx (\plx \aland \ply)$\\
		Kollisionsfrei: $\sigma_{\{\ply/\plz\}}$\\
		Nicht kollisionsfrei: $\sigma_{\{\ply/\plx\}}$
	\end{Beispiel}
\end{frame}

\begin{frame}{Substitutionen}
	\begin{equation*}
	F = \alnot \plexist \plx
	{\plka
		\plE{\plka \plx \plcomma \ply \plkz}
		\alor
		\alnot \plall \plz \plall \plx \plall \ply
		{\plka
			\plE{\plka \plx \plcomma \plz \plkz} \aland \plE{\plka \ply \plcomma \plz \plkz} \alimpl \plx \pleq \ply
			\plkz}
		\plkz}
	\end{equation*}
	
	\begin{block}{Aufgabe 1.2}
		Geben sie eine Substitution $\sigma$ an, die \emph{nicht} kollisionsfrei für $F$ ist.\\[1em] \pause
		
		Die Substitution $\sigma_{\{(\ply/\plx)\}}$ leistet das Gewünschte.
	\end{block}
	
\end{frame}


\section{Prädikatenlogik: Semantik}

\begin{frame}{Interpretation}
	\begin{figure}[h!]
		\centering
		\includegraphics[scale=0.6]{int.png} \hspace{2em} 
	\end{figure} 
\end{frame}

\begin{frame}{Aufgabe 1 (15/16, Blatt 7)}
	  \begin{equation*}
	\alnot \plexist \plx
	{\plka
		\plE{\plka \plx \plcomma \ply \plkz}
		\alor
		\alnot \plall \plz \plall \plx \plall \ply
		{\plka
			\plE{\plka \plx \plcomma \plz \plkz} \aland \plE{\plka \ply \plcomma \plz \plkz} \alimpl \plx \pleq \ply
			\plkz}
		\plkz}
	\end{equation*}\\[1em]
	
	\begin{block}{Aufgabe 1.3}
		Geben Sie eine Interpretation $(D_1, I_1)$ und eine Variablenbelegung $\beta_1$ so an, dass $val_{D_1, I_1, \beta_1}(F) = \textbf{W}$ gilt.\\[0.5em]
		
		\visible<2-|handout:2>{Die Interpretation $(D_1, I_1) = (\{ 0, 1 \}, {<})$ und die Variablenbelegung $\beta_1 \colon \VPL \to D$, $v \mapsto 0$, leisten das Gewünschte.}
	\end{block}

	\begin{block}{Aufgabe 1.4}
		Geben Sie eine Interpretation $(D_2, I_2)$ und eine Variablenbelegung $\beta_2$ so an, dass $val_{D_2, I_2, \beta_2}(F) = \textbf{F}$ gilt.\\[0.5em]
		
		\visible<3-|handout:2>{Die Interpretation $(D_2, I_2) = (\{ 0, 1 \}, {<})$ und die Variablenbelegung $\beta_2 \colon \VPL \to D$, $v \mapsto 1$, leisten das Gewünschte.}
	\end{block}
\end{frame}



\section{Prädikatenlogik: Aufgaben}

\begin{frame}{Prädikatenlogische Formeln aufstellen}
	% TODO Übung einbinden
	Vgl. Übung WS 15/16
\end{frame}

\begin{frame}{Aufgabe 2 (15/16, Blatt 7)}
	\begin{block}{Aufgabe}
		Formulieren Sie die folgenden Aussagen als Formeln in Prädikatenlogik:
		\begin{enumerate}
			\item Nicht alle Vögel können fliegen.
			\item Wenn es irgendjemand kann, dann kann es Donald Ervin Knuth.
			\item John liebt jeden, der sich nicht selbst liebt.
		\end{enumerate}
	\end{block}
	
	\only<2-|handout:2>{
	\begin{block}{Lösung}
		\begin{enumerate}
				\item
				\begin{equation*}
				\plexist \plx {\plka \plfoo{Vogel}{\plka \plx \plkz} \aland \alnot \plfoo{flugfaehig}{\plka \plx \plkz} \plkz}
				\end{equation*}
			\only<3-|handout:2>{
				\item
				\begin{equation*}
				\plexist \plx {\plka \plfoo{kann\_es}{\plka \plx \plkz} \plkz}
				\alimpl
				\plfoo{kann\_es}{\plka \plfoo{knuth} \plkz}
				\end{equation*}
			}
			\only<4-|handout:2>{
				\item
				\begin{equation*}
				\plall \plx {\plka \alnot \plfoo{liebt}{\plka \plx \plcomma \plx \plkz} \alimpl \plfoo{liebt}{\plka \plfoo{John} \plcomma \plx \plkz} \plkz}
				\end{equation*}
			}
		\end{enumerate}
	\end{block}
	}
\end{frame}

\begin{frame}{Weitere Aufgaben}
	Siehe Übung WS 15/16
\end{frame}
%%%%%%%%%% %%%%%%%%%%
%% Zusammenfassung
\section{}
%\subsection{Zusammenfassung}
	\begin{frame}{Was ihr jetzt kennen und können solltet\dots}
			\begin{itemize}
				\item Todo
			\end{itemize}
	
	\end{frame}
%% Ausblick
%\subsection{Ausblick}
	\begin{frame}{Ausblick}
		\begin{itemize}
			\item Todo
		\end{itemize}
	\end{frame}
%%%%%%%%%% %%%%%%%%%%
\section{}
\questionframe
\lastframe
\mode<handout>{\slideThanks}
\end{document}