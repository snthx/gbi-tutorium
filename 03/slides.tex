% ===== handout mode =====
% Comment/uncomment this line to toggle handout mode
% \newcommand{\handout}{}

% Comment/uncoment this line to toogle Mortitz mode
% \newcommand{\Moritz}{}

% Comment/uncomment this line to toggle handout mode
% \newcommand{\handout}{}

% by Stephan

%% Moritz mode or Stephan mode
\ifdefined \MoritzMode

% This is a configuration file with private, tutor specific information.
% It is therefore excluded from the Git repository so changes in this file will not conflict in git commits.

% Copy this Template, rename to config.tex and add your information below.

\newcommand{\mymail}{moritz.laupichler@student.kit.edu} % Consider using your named student Mail address to keep your u-Account private.

\newcommand{\myname}{\href{mailto:\mymail}{Moritz Laupichler}}

\newcommand{\mytutnumber}{27}

\newcommand{\mytutinfos}{Dienstags, 5. Block (15:45-17:15), SR 236}

\newcommand{\aboutMeFrame}{
	\begin{frame}{Euer Tutor}
		Name: \myname \\
		Alter: 19 Jahre \\
		Studiengang: Bachelor Informatik, 3. Semester \\
		\vspace{1cm}
		\pause 
		\centering{Kontakt: \href{mailto:\mymail}{\mymail}}
	\end{frame}
}

% Toggle Handout mode by including the following line before including style_tut
% and removing the % at the start (but do NOT remove it here, otherwise handout mode will always be on!)
% Please keep handout mode on in all commits!

% \newcommand{\handout}{} % Moritz mode
\fi
\ifdefined \AlexMode

% This is a configuration file with private, tutor specific information.
% It is therefore excluded from the Git repository so changes in this file will not conflict in git commits.

% Copy this Template, rename to config.tex and add your information below.

\newcommand{\mymail}{alexander.klug@student.kit.edu} % Consider using your named student Mail address to keep your u-Account private.

\newcommand{\myname}{\href{mailto:\mymail}{Alexander Klug}}

\newcommand{\mytutnumber}{30}

\newcommand{\mytutinfos}{Mittwochs, 3. Block (11:30-13:00), SR -107}

\newcommand{\aboutMeFrame}{
	\begin{frame}{Euer Tutor}
		Name: \myname \\
		Alter: 19 Jahre \\
		Studiengang: Bachelor Informatik, 3. Semester \\
		\vspace{1cm}
		\pause 
		\centering{Kontakt: \href{mailto:\mymail}{\mymail}}
	\end{frame}
}

% Toggle Handout mode by including the following line before including style_tut
% and removing the % at the start (but do NOT remove it here, otherwise handout mode will always be on!)
% Please keep handout mode on in all commits!

% \newcommand{\handout}{} % Alex Mode
\fi
\ifdefined \StephanMode

% This is a configuration file with private, tutor specific information.
% It is therefore excluded from the Git repository so changes in this file will not conflict in git commits.

% Copy this Template, rename to config.tex and add your information below.

\newcommand{\mymail}{stephan.bohr@student.kit.edu} % Consider using your named student Mail address to keep your u-Account private.

\newcommand{\myname}{\href{mailto:\mymail}{Stephan Bohr}}

\newcommand{\mytutnumber}{19}

\newcommand{\mytutinfos}{Dienstags, 3. Block (11:30-13:00), SR -108}

\newcommand{\aboutMeFrame}{
	\begin{frame}{Euer Tutor}
		Name: \myname \\
		Alter: 21 Jahre \\
		Studiengang: Bachelor Informatik, 5. Semester \\
		\vspace{1cm}
		\pause 
		\centering{Kontakt: \href{mailto:\mymail}{\mymail}}
	\end{frame}
} % Stephan mode
\fi

%% Beamer-Klasse im korrekten Modus
\ifdefined \handout
\documentclass[handout]{beamer} % Handout mode
\else
\documentclass{beamer}
\fi
%\documentclass[18pt,parskip]{beamer}

%% SLIDE FORMAT

% use 'beamerthemekit' for standard 4:3 ratio
% for widescreen slides (16:9), use 'beamerthemekitwide'

\usepackage{../templates/KIT-slides/beamerthemekit}
%\usepackage{../templates/KIT-slides/beamerthemekitwide}

%% TITLE PICTURE

% if a custom picture is to be used on the title page, copy it into the 'logos'
% directory, in the line below, replace 'mypicture' with the 
% filename (without extension) and uncomment the following line
% (picture proportions: 63 : 20 for standard, 169 : 40 for wide
% *.eps format if you use latex+dvips+ps2pdf, 
% *.jpg/*.png/*.pdf if you use pdflatex)

\titleimage{../figures/titleimage/brain}

%% TITLE LOGO

% for a custom logo on the front page, copy your file into the 'logos'
% directory, insert the filename in the line below and uncomment it

%\titlelogo{mylogo}

% (*.eps format if you use latex+dvips+ps2pdf,
% *.jpg/*.png/*.pdf if you use pdflatex)

%% TikZ INTEGRATION

% use these packages for PCM symbols and UML classes
% \usepackage{templates/tikzkit}
% \usepackage{templates/tikzuml}

%\usepackage{tikz}
%\usetikzlibrary{matrix}
%\usetikzlibrary{arrows.meta}
%\usetikzlibrary{automata}
%\usetikzlibrary{tikzmark}

%%%%%%%%%%%%%%%%%%%%%%%%%
% Libertine font (Original GBI font)
\usepackage[mono=false]{libertine}
%\renewcommand*\familydefault{\sfdefault}  %% Only if the base font of the document is to be sans serif

%% Schönere Schriften
\usepackage[TS1,T1]{fontenc}

%% Deutsche Silbentrennung und Beschriftungen
\usepackage[ngerman]{babel}

%% UTF-8-Encoding
\usepackage[utf8]{inputenc}

%% Bibliotheken für viele mathematische Symbole
\usepackage{amsmath, amsfonts, amssymb}

%% Anzeigetiefe für Inhaltsverzeichnis: 1 Stufe
\setcounter{tocdepth}{1}

%% Hyperlinks
\usepackage{hyperref}
% I don't know why, but this works and only includes sections and NOT subsections in the pdf-bookmarks.
\hypersetup{bookmarksdepth=subsection}

%% remove navigation symbols
\setbeamertemplate{navigation symbols}{}

%% switch between "ngerman" and "english" for German/English style date and logos
\selectlanguage{ngerman}

%% for invisible pause texts instead of dimming
\setbeamercovered{invisible}

\usepackage[german=swiss]{csquotes}

\usepackage{tabularx}
\usepackage{booktabs}

\usepackage{tikz}


% Problem: disabled itemize-icons
%\usepackage{enumitem}
% %\setlist[enumerate]{topsep=0pt,itemsep=-1ex,partopsep=1ex,parsep=1ex}
% \setlist[itemize]{noitemsep, nolistsep}
% \setlist[enumerate]{noitemsep, nolistsep}

% Mathmode no vertical space (https://tex.stackexchange.com/a/47403/146825)
\setlength{\abovedisplayskip}{0pt}
\setlength{\belowdisplayskip}{0pt}
\setlength{\abovedisplayshortskip}{0pt}
\setlength{\belowdisplayshortskip}{0pt}

%%%%%%%%%%%% Slides %%%%%%%%%%%%%%%%

\newcommand{\Moritz}[1]{
	\ifdefined \MoritzMode
	#1
	\fi
}

\newcommand{\Alex}[1]{
	\ifdefined \AlexMode
	#1
	\fi
}

\newcommand{\Stephan}[1]{
	\ifdefined \StephanMode
	#1
	\fi
}

\newcommand{\notMoritz}[1]{
	\Alex{#1} \Stephan{#1}
}

\newcommand{\notAlex}[1]{
	\Moritz{#1} \Stephan{#1}
}

\newcommand{\notStephan}[1]{
	\Alex{#1} \Moritz{#1}
}

%% Wochennummer
%\newcounter{weeknum}

%% Titelinformationen
%\title[GBI Tutorium, Woche \theweeknum]{Grundbegriffe der Informatik \\ Tutorium \mytutnumber}
%\subtitle{Termin \theweeknum \ | \mydate \\ \myname}
\author[\myname]{\myname}
\institute{Fakultät für Informatik}
%\date{\mydate}

%% Titel einfügen
\newcommand{\titleframe}{\frame{\titlepage}\addtocounter{framenumber}{-1}}


%% Alles starten mit \starttut{X}
%\newcommand{\starttut}[1]{\setcounter{weeknum}{#1}\titleframe\frame{\frametitle{Inhalt}\tableofcontents} \AtBeginSection[]{%
%\begin{frame}
%	\tableofcontents[currentsection]
%\end{frame}\addtocounter{framenumber}{-1}}}


%\newcommand{\framePrevEpisode}{
%	\begin{frame}
%		\centering
%		\textbf{In the previous episode of GBI...}
%	\end{frame}
%}

%% Roadmap frame
%table of contents
\newcommand{\roadmap}{
	\frame{\frametitle{Roadmap}\tableofcontents}}

 \AtBeginSection[]{%
\begin{frame}
	\frametitle{Roadmap}
	\tableofcontents[currentsection]
\end{frame}%\addtocounter{framenumber}{-1}
}


%% ShowMessage frame
\newcommand{\showmessage}[1]{\frame{\frametitle{\phantom{1em}}\centering\textbf{#1}}}

%% Fragen
%% Lastframe
\newcommand{\questionframe}{\showmessage{Fragen?}}

%% Lastframe
\newcommand{\lastframe}{\showmessage{Vielen Dank für Eure Aufmerksamkeit! \\Bis nächste Woche :)}}

%% Thanks frame
\newcommand{\slideThanks}{
	\begin{frame}
		\frametitle{Credits}
		\begin{block}{}
			An der Erstellung des Foliensatzes haben mitgewirkt:\\[1em]
			\Moritz{
			Stephan Bohr \\
			Alexander Klug \\
			}
			\Alex{
			Stephan Bohr \\
			Moritz Laupichler \\
			}
			\Stephan{
			Moritz Laupichler \\
			Alexander Klug \\
			}
			Katharina Wurz \\
			Thassilo Helmold \\
			Daniel Jungkind \\
			% Philipp Basler \\
			% Nils Braun \\
			% Dominik Doerner \\
			% Ou Yue \\
		\end{block}
	\end{frame}
}

%% Verbatim
%\usepackage{moreverb}

% GBI related stuff, but not beamer-stuff
\newcommand{\newpar}[1]{\paragraph{#1}\mbox{}\newline}

\newcommand{\nM}{\mathbb{M}}
\newcommand{\nR}{\mathbb{R}}
\newcommand{\nN}{\mathbb{N}}
\newcommand{\nZ}{\mathbb{Z}}
\newcommand{\nQ}{\mathbb{Q}}
\newcommand{\nB}{\mathbb{B}}
\newcommand{\nC}{\mathbb{C}}
\newcommand{\nK}{\mathbb{K}}
\newcommand{\nF}{\mathbb{F}}
\newcommand{\nG}{\mathbb{G}}
\newcommand{\nullel}{\mathcal{O}}
\newcommand{\einsel}{\mathds{1}}
\newcommand{\nP}{\mathbb{P}}
\newcommand{\Pot}{\mathcal{P}}
\renewcommand{\O}{\text{O}}

\newcommand{\bfmod}{\ensuremath{\text{\textbf{ mod }}}}
\renewcommand{\mod}{\bfmod}
\newcommand{\bfdiv}{\ensuremath{\text{\textbf{ div }}}}
\renewcommand{\div}{\bfdiv}


\newcommand{\set}[1]{\left\{ #1 \right\}}
\newcommand{\setc}[2]{\set{#1 \mid #2}}
\newcommand{\setC}[2]{\set{#1 \mid \text{ #2 }}}

\newcommand{\setsize}[1]{\; \mid #1 \mid \; }

\newcommand{\q}[1]{\textquotedblleft #1\textquotedblright}

% Zu zeigen, thx to http://www.matheboard.de/archive/155832/thread.html
\newcommand{\zz}{\ensuremath{\mathrm{z\kern-.29em\raise-0.44ex\hbox{z}}}:}

% Text above symbol
% https://tex.stackexchange.com/a/74132/146825
%
% \newcommand{\eqtext}[1]{\stackrel{\mathclap{\normalfont\mbox{#1}}}{=}}
% \newcommand{\gdwtext}[1]{\stackrel{\mathclap{\normalfont\mbox{#1}}}{\Leftrightarrow}}
% \newcommand{\imptext}[1]{\stackrel{\mathclap{\normalfont\mbox{#1}}}{\Rightarrow}}
% \newcommand{\symbtext}[2]{\stackrel{\mathclap{\normalfont\mbox{#2}}}{#1}}
\newcommand{\eqtext}[1]{\mathrel{\overset{\makebox%[0pt]
{\mbox{\normalfont\tiny #1}}}{=}}}
\newcommand{\gdwtext}[1]{\mathrel{\overset{\makebox%[0pt]
{\mbox{\normalfont\tiny #1}}}{\ensuremath{\Leftrightarrow}}}}
\newcommand{\imptext}[1]{\mathrel{\overset{\makebox%[0pt]
{\mbox{\normalfont\tiny #1}}}{\ensuremath{\Rightarrow}}}}
\newcommand{\symbtext}[2]{\mathrel{\overset{\makebox%[0pt]
{\mbox{\normalfont\tiny #2}}}{#1}}}

% qed symbol
\newcommand{\qedblack}{\hfill \ensuremath{\blacksquare}}
\newcommand{\qedwhite}{\hfill \ensuremath{\Box}}

% Aussagenlogik
% Worsch
\colorlet{alcolor}{blue}
\RequirePackage{tikz}
\usetikzlibrary{arrows.meta}
\newcommand{\alimpl}{\mathrel{\tikz[x={(0.1ex,0ex)},y={(0ex,0.1ex)},>={Classical TikZ Rightarrow[]}]{\draw[alcolor,->,line width=0.7pt,line cap=round] (0,0) -- (15,0);\path (0,-6);}}}
\newcommand{\alimp}{\alimpl}
\newcommand{\aleqv}{\mathrel{\tikz[x={(0.1ex,0ex)},y={(0ex,0.1ex)},>={Classical TikZ Rightarrow[]}]{\draw[alcolor,<->,line width=0.7pt,line cap=round] (0,0) -- (18,0);\path (0,-6);}}}
\newcommand{\aland}{\mathbin{\raisebox{-0.6pt}{\rotatebox{90}{\texttt{\color{alcolor}\char62}}}}}
\newcommand{\alor}{\mathbin{\raisebox{-0.8pt}{\rotatebox{90}{\texttt{\color{alcolor}\char60}}}}}
%\newcommand{\ali}[1]{_{\mathtt{\color{alcolor}#1}}}
\newcommand{\alv}[1]{\mathtt{\color{alcolor}#1}}
\newcommand{\alnot}{\mathop{\tikz[x={(0.1ex,0ex)},y={(0ex,0.1ex)}]{\draw[alcolor,line width=0.7pt,line cap=round,line join=round] (0,0) -- (10,0) -- (10,-4);\path (0,-8) ;}}}
\newcommand{\alP}{\alv{P}} %ali{#1}}
%\newcommand{\alka}{\negthinspace\hbox{\texttt{\color{alcolor}(}}}
\newcommand{\alka}{\negthinspace\text{\texttt{\color{alcolor}(}}}
%\newcommand{\alkz}{\texttt{\color{alcolor})}}\negthinspace}
\newcommand{\alkz}{\text{\texttt{\color{alcolor})}}\negthinspace}

% Thassilo
\newcommand{\BB}{\mathbb{B}}
\newcommand{\boder}{\alor}%{\ensuremath{\text{\;}\textcolor{blue}{\vee}}\text{\;}}
\newcommand{\bund}{\aland}%{\ensuremath{\text{\;}\textcolor{blue}{\wedge}}\text{\;}}
\newcommand{\bimp}{\alimp}%{\ensuremath{\text{\;}\textcolor{blue}{\to}}\text{\;}}
\newcommand{\bnot}{\alnot}%{\ensuremath{\text{\;}\textcolor{blue}{\neg}}\text{}}
\newcommand{\bgdw}{\aleqv}%{\ensuremath{\text{\;}\textcolor{blue}{\leftrightarrow}}\text{\;}}
\newcommand{\bone}{\ensuremath{\textcolor{blue}{1}}\text{}}
\newcommand{\bzero}{\ensuremath{\textcolor{blue}{0}}\text{}}
\newcommand{\bleftBr}{\alka}%{\ensuremath{\textcolor{blue}{(}}\text{}}
\newcommand{\brightBr}{\alkz}%{\ensuremath{\textcolor{blue}{)}}\text{}}

\newcommand{\val}{\hbox{\textit{val}}}

\newcommand{\VarAL}{\hbox{\textit{Var}}_{AL}}
\newcommand{\ForAL}{\hbox{\textit{For}}_{AL}}

% Validierungsfunktion val_i
\newcommand{\vali}[1]{\ensuremath{\val_I(#1)}}

% Boolsche Funktion b_
\newcommand{\bfnot}[1]{\ensuremath{b_{\bnot}(#1)}}
\newcommand{\bfand}[2]{\ensuremath{b_{\bund}(#1,#2)}}
\newcommand{\bfor}[2]{\ensuremath{b_{\boder}(#1,#2)}}
\newcommand{\bfimp}[2]{\ensuremath{b_{\bimp}(#1,#2)}}

% Aussagenkalkül
\newcommand{\AAL}{A_{AL}}
\newcommand{\LAL}{\hbox{\textit{For}}_{AL}}
\newcommand{\AxAL}{\hbox{\textit{Ax}}_{AL}}
\newcommand{\MP}{\hbox{\textit{MP}}}

% Prädikatenlogik
% die nachfolgenden Sachen angepasst an cmtt
\newlength{\ttquantwd}
\setlength{\ttquantwd}{1ex}
\newlength{\ttquantht}
\setlength{\ttquantht}{6.75pt}
\def\plall{%
  \tikz[line width=0.67pt,line cap=round,line join=round,baseline=(B),alcolor] {
    \draw (-0.5\ttquantwd,\ttquantht) -- node[coordinate,pos=0.4] (lll){} (-0.25pt,-0.0pt) -- (0.25pt,-0.0pt) -- node[coordinate,pos=0.6] (rrr){} (0.5\ttquantwd,\ttquantht);
    \draw (lll) -- (rrr);
    \coordinate (B) at (0,-0.35pt);
  }%
}
\def\plexist{%
  \tikz[line width=0.67pt,line cap=round,line join=round,baseline=(B),alcolor] {
    \draw (-0.9\ttquantwd,\ttquantht) -- (0,\ttquantht) -- node[coordinate,pos=0.5] (mmm){} (0,0) --  (-0.9\ttquantwd,0);
    \draw (mmm) -- ++(-0.75\ttquantwd,0);
    \coordinate (B) at (0,-0.35pt);
  }\ensuremath{\,}%
}
\let\plexists=\plexist
\newcommand{\NT}[1]{\ensuremath{\langle\mathrm{#1} \rangle}}
\newcommand{\CPL}{\text{\itshape Const}_{PL}}
\newcommand{\FPL}{\text{\itshape Fun}_{PL}}
\newcommand{\RPL}{\text{\itshape Rel}_{PL}}
\newcommand{\VPL}{\text{\itshape Var}_{PL}}
\newcommand{\plka}{\alka}
\newcommand{\plkz}{\alkz}
%\newcommand{\plka}{\plfoo{(}}
%\newcommand{\plkz}{\plfoo{)}}
\newcommand{\plcomma}{\hbox{\texttt{\color{alcolor},}}}
\newcommand{\pleq}{{\color{alcolor}\,\dot=\,}}

\newcommand{\plfoo}[1]{\mathtt{\color{alcolor}#1}}
\newcommand{\plc}{\plfoo{c}}
\newcommand{\pld}{\plfoo{d}}
\newcommand{\plf}{\plfoo{f}}
\newcommand{\plg}{\plfoo{g}}
\newcommand{\plh}{\plfoo{h}}
\newcommand{\plx}{\plfoo{x}}
\newcommand{\ply}{\plfoo{y}}
\newcommand{\plz}{\plfoo{z}}
\newcommand{\plR}{\plfoo{R}}
\newcommand{\plS}{\plfoo{S}}
\newcommand{\ar}{\mathrm{ar}}

\newcommand{\bv}{\mathrm{bv}}
\newcommand{\fv}{\mathrm{fv}}

\def\word#1{\hbox{\textcolor{blue}{\texttt{#1}}}}
%\let\literal\word
\def\mword#1{\hbox{\textcolor{blue}{$\mathtt{#1}$}}}  % math word
\def\sp{\scalebox{1}[.5]{\textvisiblespace}}
\def\wordsp{\word{\sp}}


\newcommand{\W}{\ensuremath{\hbox{\textbf{w}}}\xspace}
\newcommand{\F}{\ensuremath{\hbox{\textbf{f}}}\xspace}
\newcommand{\WF}{\ensuremath{\{\W,\F\}}\xspace}
\newcommand{\valDIb}{\val_{D,I,\beta}}

\newcommand{\impl}{\ifmmode\ensuremath{\mskip\thinmuskip\Rightarrow\mskip\thinmuskip}\else$\Rightarrow$\fi\xspace}
\newcommand{\Impl}{\ifmmode\implies\else$\Longrightarrow$\fi\xspace}

\newcommand{\derives}{\Rightarrow}

\newcommand{\gdw}{\ifmmode\mskip\thickmuskip\Leftrightarrow\mskip\thickmuskip\else$\Leftrightarrow$\fi\xspace}
\newcommand{\Gdw}{\ifmmode\iff\else$\Longleftrightarrow$\fi\xspace}

\newcommand*{\from}{\colon}
\newcommand{\functionto}{\longrightarrow}


\newcommand{\LTer}{L_{\text{\itshape Ter}}}
\newcommand{\LRel}{L_{\text{\itshape Rel}}}
\newcommand{\LFor}{L_{\text{\itshape For}}}
\newcommand{\NTer}{N_{\text{\itshape Ter}}}
\newcommand{\NRel}{N_{\text{\itshape Rel}}}
\newcommand{\NFor}{N_{\text{\itshape For}}}
\newcommand{\PTer}{P_{\text{\itshape Ter}}}
\newcommand{\PRel}{P_{\text{\itshape Rel}}}
\newcommand{\PFor}{P_{\text{\itshape For}}}

\newcommand{\sgn}{\mathop{\text{sgn}}}

\newcommand{\lang}[1]{\ensuremath{\langle#1\rangle}}

\newcommand{\literal}[1]{\hbox{\textcolor{blue!95!white}{\textup{\texttt{\scalebox{1.11}{#1}}}}}}
\let\hashtag\#
\renewcommand{\#}[1]{\literal{#1}}

\def\blank{\ensuremath{\openbox}}
\def\9{\blank}
\newcommand{\io}{\!\mid\!}


\providecommand{\fspace}{\mathord{\text{space}}}
\providecommand{\fSpace}{\mathord{\text{Space}}}
\providecommand{\ftime}{\mathord{\text{time}}}
\providecommand{\fTime}{\mathord{\text{Time}}}

\newcommand{\fnum}{\text{num}}
\newcommand{\fNum}{{\text{Num}}}

\def\Pclass{\text{\bfseries P}}
\def\PSPACE{\text{\bfseries PSPACE}}



\title[Aussagenlogik, Vollständige Induktion]{3. Tutorium\\ Aussagenlogik, Vollständige Induktion}
\subtitle{Grundbegriffe der Informatik, Tutorium \hashtag\mytutnumber}
\date{\today}
\usepackage{tikz}

\begin{document}
\titleframe
% \roadmap

%%%%%%%%%% %%%%%%%%%%



\Stephan{
	\begin{frame}{Zu ÜB 1}
	    \begin{itemize}[<+->]
	    	\item 12 von 19 Punkten im Durchschnitt
	    	\item Strukturiert eure Abgaben!
	    	\item Achtet auf die Operatoren!
	    	\item Aufg. 1: $x \in 2^A$
	    	\item Aufg. 2: $|C| = |A| + |B| - |A \cap B|$ trifft nicht den Sinn der Aufgabe
	    	\item Aufg. 3: Mengennotationen
	    \end{itemize}
	\end{frame}
}

\subsection{kekokeko}

\begin{frame}{Mengennotation in GBI}
	\begin{itemize}
		 \item Um Unklarheiten zu vermeiden:
    \[
    	\nN_0 \text{ oder } \nN_+ \text{ statt } \nN
    \]
    \item und
    \[
    	\nR_0^+ \text{ oder } \nR^+
    \]
	\end{itemize}


\end{frame}

\roadmap

\section{Aussagenlogik}

\subsection{Grundlagen: Aussage, Syntax}

\begin{frame}{Grundlagen}
	\begin{block}{Def.: Aussage}
		\textbf{Aussagen} sind Sätze, die "`objektiv"' wahr oder falsch sind. Man spricht auch von der Zweiwertigkeit der Aussagenlogik.
	\end{block}
	\pause
	\begin{block}{Def.: Aussagenvariable}
		Eine \textbf{Aussagenvariable} steht für eine (elementare) Aussage. Sie kann entweder \emph{wahr} oder \emph{falsch} sein.
	\end{block}
\end{frame}

\begin{frame}{Grundlagen}
	\begin{block}{Aussagenlogische Konnektive}
		Seien \(G\) und \(H\) Aussagevariablen. Dann kann man wie folgt größere Formeln konstruieren:
		\pause
		\begin{itemize}[<+->]
			\item \(	\bleftBr 	\bnot  G \brightBr 	\) heißt \enquote{\textcolor{blue}{nicht} \(G\)}
			\item \(	\bleftBr 	G \bund  H		\brightBr 	\)	heißt \enquote{\(G\) \textcolor{blue}{und} \(H\)}
			\item \(	\bleftBr 	G \boder H		\brightBr 	\)	heißt \enquote{\(G\) \textcolor{blue}{oder} \(H\)}
			\item \(	\bleftBr 	G  \bimp H		\brightBr 	\)	heißt \enquote{\(G\) \textcolor{blue}{impliziert} \(H\)}
			\item \(	\bleftBr 	G  \bgdw H		\brightBr := \bleftBr	G  \bimp H		\brightBr  \bund  \bleftBr 	H  \bimp G		\brightBr	\)	heißt \\ \enquote{\(G\) \textcolor{blue}{impliziert} \(H\) und \(H\) \textcolor{blue}{impliziert} \(G\)}
		\end{itemize}
	\end{block}
	\pause
	\begin{alertblock}{Beachte}
		Hinter dieser Syntax steckt natürlich viel Formalismus. Die Vorlesung kennt \(\bgdw\) nur als Abkürzung, nicht als aussagenlogisches Symbol.
	\end{alertblock}
\end{frame}

\begin{frame}{Grundlagen}
	\begin{block}{Bindungsstärken}
		\begin{itemize}
			\item $\bnot$ bindet am stärksten
  			\item $\bund$ bindet am zweitstärksten
			\item $\boder$ bindet am drittstärksten
			\item $\bimp$ bindet am viertstärksten
			\item $\bgdw$ bindet am schwächsten
		\end{itemize}
	\end{block}
	\pause
	\begin{exampleblock}{Aufgabe}
		Finde die Klammern für \\
		\(	P	\bimp	Q \bund \bnot R \bgdw Q \boder R	\)\\[1ex]
		\pause
		\(	\bleftBr \bleftBr  P	\bimp	\bleftBr  Q \bund \bleftBr  \bnot R \brightBr \brightBr \brightBr \bgdw \bleftBr Q \boder R	\brightBr \brightBr \)
	\end{exampleblock}
\end{frame}

\subsection{Semantik}

\begin{frame}{Semantik aussagenlogischer Formeln}
	\begin{block}{Def.: Boolsche Funktionen}
		Für ``Wahrheitswerte'' \( \BB = \set{\mathbf{w}, \mathbf{f}} \)
		 ist eine \textbf{boolsche Funktion} eine Funktion
		 $f: \BB^n \to \BB$
	\end{block}
	\pause
	\begin{exampleblock}{Beispiele}
		\begin{center}

		  	\begin{tabular}{cc|cccc}
		    \toprule
		    $x_1$ & $x_2$ & $\bfnot{x_1}$ & $\bfand{x_1}{x_2}$ & $\bfor{x_1}{x_2}$ & $\bfimp{x_1}{x_2}$ \\
		    \midrule
		    $f$ & $f$ & $w$ & $f$ & $f$ & $w$ \\
		    $f$ & $w$ & $w$ & $f$ & $w$ & $w$ \\
		    $w$ & $f$ & $f$ & $f$ & $w$ & $f$ \\
		    $w$ & $w$ & $f$ & $w$ & $w$ & $w$ \\
		    \bottomrule
  			\end{tabular}
		\end{center}
	\end{exampleblock}
\end{frame}

\begin{frame}{Semantik aussagenlogischer Formeln}
	\begin{block}{Def.: Interpretation}
		Für eine Menge \(V\) von Aussagevariablen ist eine \textbf{Interpretation} ist eine Abbildung \(I: V \to \nB\), wobei \( \BB = \set{w, f} \). 
	\end{block}

	\begin{exampleblock}{Beim Aufstellen einer Wahrheitstabelle}
		Anzahl Interpretationen für eine Variablenmenge mit \(k \in \nN_+ \) Aussagevariablen: \(2^k\).
	\end{exampleblock}
\end{frame}

\begin{frame}{Semantik aussagenlogischer Formeln: $\vali{F}$}
	\begin{block}{Auswertung/Validierung aussagenlogischer Formeln}
		Sei \(I: V \to \nB\) eine Interpretation.\\
		Für jede aussagelogische Formel $F$ definiere $\vali{F}$ wie folgt:\\[2ex]

		Für jedes $X \in V$, $G$, $H$ weitere aussagelogische Formeln sei:
		\begin{itemize}
			\item $\vali{X}         := I(X) $
  			\item $\vali{\bnot G}   := \bfnot{\vali{G}} $
  			\item $\vali{G \bund H} := \bfand{\vali{G}}{\vali{H}}$
  			\item $\vali{G \boder H} := \bfor{\vali{G}}{\vali{H}}$
  			\item $\vali{G \bimp H} := \bfimp{\vali{G}}{\vali{H}}$
		\end{itemize}
	\end{block}
\end{frame}

% TODO: Aufgabe zum formellen Vorgehen, mit konkreter Interpretation?

\begin{frame}{Semantik aussagenlogischer Formeln: Wahrheitstabellen}
	\begin{exampleblock}{Aufgabe}
		Stelle für eine aussagenlogische Formel eine Wahrheitstabelle auf.
	\end{exampleblock}

	\begin{exampleblock}{Vorgehen}
	\pause
		\begin{enumerate}[<+->]
			\item Wahrheitswerte für Variablen
			\item Wahrheitswerte für Teilformeln
			\item Wahrheitswerte für ganze Formeln
		\end{enumerate}
	\end{exampleblock}
\end{frame}

\begin{frame}{Semantik aussagenlogischer Formeln: Wahrheitstabellen}
	\begin{exampleblock}{Aufgabe}
	Stelle für folgende aussagenlogische Formel eine Wahrheitstabelle auf: \( \bleftBr{} \bleftBr{} \bnot{} A \bimp{} B \brightBr{} \bund \bleftBr{} \bnot{} \bleftBr{} A \bgdw{} B \brightBr{} \boder{} A \brightBr{} \brightBr{}\).
	\end{exampleblock}
	%\pause
	\begin{block}{Lösung}
\begin{center}

		  	\begin{tabular}{cc|cccccccc}
		    \toprule
		    $A$ & $B$ & $\bleftBr{} \bleftBr{} \bnot{} A$ & $\bimp{}$ & $B\brightBr{}$ & $\bund $ & $\bleftBr{} \bnot{}$ & $\bleftBr{} A \bgdw{} B \brightBr{}$ & $\boder{}$ & $A \brightBr{} \brightBr{}$\\
		    \midrule
		    \pause
		    $ f $ & $ f $ & $ w $ & $ f $ & $ f $ & $ \textcolor{kit-green100}{\mathbf{f}} $ & $ f $ & $ w $ & $ f $ & $ f $\\
		    $ f $ & $ w $ & $ w $ & $ w $ & $ w $ & $ \textcolor{kit-green100}{\mathbf{w}} $ & $ w $ & $ f $ & $ w $ & $ f $\\
		    $ w $ & $ f $ & $ f $ & $ w $ & $ f $ & $ \textcolor{kit-green100}{\mathbf{w}} $ & $ w $ & $ f $ & $ w $ & $ w $ \\
		    $ w $ & $ w $ & $ f $ & $ w $ & $ w $ & $ \textcolor{kit-green100}{\mathbf{w}} $ & $ f $ & $ w $ & $ w $ & $ w $ \\
		    \bottomrule
  			\end{tabular}
		\end{center}

		\pause
		Beobachtung: \( \bleftBr{} \bleftBr{} \bnot{} A \bimp{} B \brightBr{} \bund \bleftBr{} \bnot{} \bleftBr{} A \bgdw{} B \brightBr{} \boder{} A \brightBr{} \brightBr{}\) und \(\bleftBr{} A \boder B \brightBr{}\) sind äquivalent.
	\end{block}
\end{frame}

\begin{frame}{Semantik aussagenlogischer Formeln: Äquivalenz}
	\begin{block}{Def.: äquivalente Formeln}
		Zwei Formeln $A$ und $B$ heißen \textbf{äquivalent}, wenn für jede Interpretation $I$ gilt
		\[\val_I(A)=\val_I(B)\]
		Man schreibt auch $A\equiv B$.
	\end{block}

	\begin{exampleblock}{Beispiele}
		\begin{itemize}
			\item \( \bleftBr{} \bleftBr{} \bnot{} A \bimp{} B \brightBr{} \bund \bleftBr{} \bnot{} \bleftBr{} A \bgdw{} B \brightBr{} \boder{} A \brightBr{} \brightBr{} \equiv \bleftBr{} A \boder B \brightBr{}\)
			\item \( \bleftBr{} A \bimp{} B \brightBr{} \equiv \bleftBr{} \bnot{} A \boder{} B \brightBr{} \)
		\end{itemize}
	\end{exampleblock}
\end{frame}

\begin{frame}{Semantik aussagenlogischer Formeln: Modell}
	\begin{block}{Def.: Modell}
		Eine Interpretation $I$ heißt \textbf{Modell} einer Formel $G$, wenn $\vali{G} =\mathbf{w}$.\\[2ex]

		Eine Interpretation $I$ heißt \textbf{Modell} einer Formelmenge $\Gamma$, wenn $I$ Modell jeder Formel $G\in \Gamma$ ist.
	\end{block}

	\begin{exampleblock}{}
	Wir sagen:\\
	\textcolor{black!50!red}{$\Gamma \models G$}, wenn jedes Modell von $\Gamma$ auch ein Modell von $G$ ist.\\
	\textcolor{black!50!red}{$\models G$} (statt \textcolor{black!50!red}{$\set{}\models G$}), wenn $G$ für \emph{alle} Interpretationen wahr ist.
	\end{exampleblock}
\end{frame}

\begin{frame}{Semantik aussagenlogischer Formeln: Tautologie}
	\begin{block}{Def.: Tautologie}
		Eine Formel $G$ heißt \textbf{Tautologie}, wenn jede Interpreation $I$ Modell ist. Das heißt, für jede Belegung (Interpretation $I$) der Aussagevariablen ist die gesamte aussagenlogische Formel wahr:
			\[	\vali{G}=\mathbf{w} \quad \text{ für alle } I	\]
	\end{block}

	\begin{exampleblock}{Beispiel}
		\begin{itemize}
			\item \( \bnot{} P \boder P \)
		\end{itemize}
	\end{exampleblock}
\end{frame}


\begin{frame}{}
	\begin{exampleblock}{Aufgabe (WS 16/17)}
		% Übungsblatt 2016-2, Aufgabe 2
		Es sei $\VarAL$ eine Menge von Aussagevariablen und $\ForAL$ die Menge aller aussagenlogischen Formeln über $\VarAL$. Beweise, dass für alle \(G, H \in \ForAL\) die aussagenlogische Formel
				\[\mathcal{F} :=	\bleftBr{} \bnot{} H\bimp{} \bnot{} G \brightBr{} \bimp{} \bleftBr{} G \bimp{} H \brightBr{}\]
		eine Tautologie ist. Verwende nicht das aussagenlogische Kalkül, sondern die formellen Definitionen der Auswertung von aussagenlogischen Formeln und boolschen Funktionen.
	\end{exampleblock}
	\begin{block}{Ansatz:}
		\zz Für alle Interpretationen $I$ ist die aussagenlogische Formel wahr.\\[1ex]
		Durch \emph{Abbilden}, \emph{Umformen} und \emph{Substituieren} (\emph{Ersetzen}) auf bekannte Tautologien schließen.\\[1ex]
		\textbf{Lösung:} s. Tafel.
	\end{block}
\end{frame}

\subsection{Beweisbarkeit}

\section{Beweisbarkeit in der Aussagenlogik}

\subsection{Beweisbarkeit}
\begin{frame}{Aussagenlogik: Beweisbarkeit}
	\begin{block}{Hilbertkalkül für die Aussagenlogik}
		Der \textbf{Hilbertkalkül für die Aussagenlogik}, (vereinfacht) auch \textbf{Aussagenkalkül}, besteht aus
		\begin{itemize}
			\item dem Alphabet $\AAL$,
			\item der Menge der syntakisch korrekten Formeln $\LAL \subseteq \AAL^*$,
			\item einer Menge von \textbf{Axiomen} $\AxAL \subseteq \LAL$,
			\item der (einzigen) \textbf{Schlussregel} \textbf{Modus Ponens} $\MP \subseteq \LAL^3$
		\end{itemize}
	\end{block}
\end{frame}

\begin{frame}{Aussagenlogik: Beweisbarkeit}
	\begin{block}{Axiome}
		\begin{align*}
  			\AxAL &= \bigl\{\alka G\alimpl \alka H\alimpl  G\alkz\alkz
          			\bigm| G,H\in\LAL \bigr\} \\
        		&\mathrel{\hphantom{=}} \cup \bigl\{\alka G\alimpl \alka H\alimpl  K\alkz\alkz
          			\alimpl \alka\alka G\alimpl H\alkz\alimpl \alka G\alimpl  K\alkz\alkz \bigm| G,H,K\in\LAL \bigr\}\\
        		&\mathrel{\hphantom{=}} \cup \bigl\{
          			\alka\alnot H\alimpl \alnot G\alkz\alimpl \alka\alka\alnot H\alimpl G\alkz\alimpl  H\alkz
          			\bigm| G,H \in\LAL 
          			\bigr\}
		\end{align*}

		Kurz: $\AxAL{}_1$, $\AxAL{}_2$ und $\AxAL{}_3$ für die drei Zeilen.
	\end{block}
\end{frame}

\begin{frame}{Aussagenlogik: Beweisbarkeit}
	\begin{block}{Modus Ponens $\MP \subseteq \LAL^3$ (Schlussregel)}
		\begin{itemize}
			\item $\MP = \{ (G\alimpl H, G, H) \mid  G, H \in\LAL \}$
			\item $\MP:$ \quad \begin{tabular}{c}
                $G \alimpl H$ \qquad $G$ \\
                \midrule
                $H$
              \end{tabular}
             \item Aus einer Formel der Form $G \alimp H$ und einer Formel der Form $G$ darf man auf eine Formel der Form $H$ schließen.
		\end{itemize}
	\end{block}

	\begin{exampleblock}{Beispiel}
		\begin{description}
			\item[\emph{Prämissen}:] $G \alimpl H$: \enquote{Wenn es regnet, wird die Straße nass},\\
								 $G$: \enquote{Es regnet}
			% Sind diese beiden Prämissen gültig, so ist auch der Schluss "Conclusio" gültig.
			\item[\emph{Conclusio}:] $H$: \enquote{Die Straße wird nass}.
		\end{description}
	\end{exampleblock}

	% \begin{exampleblock}{Beispiel}
	% 	\begin{itemize}
	% 		\item[Prämissen:] $G \alimpl H$: \enquote{Wenn du Mensch bist, stirbst du},\\
	% 							 $G$: \enquote{Du bist ein Mensch}
	% 		\item[Conclusio:] $H$: \enquote{Du stirbst}.
	% 	\end{itemize}
	% \end{exampleblock}
\end{frame}	

\begin{frame}{Aussagenlogik: Beweisbarkeit}
	\begin{block}{Def.: Ableitung}
		Sei $\Gamma \subseteq \LAL$ eine Menge von \textbf{Hypothesen} oder \textbf{Prämissen} und $G$ eine Formel.\\
		Eine \textbf{Ableitung} von $G$ aus $\Gamma$ ist eine endliche Folge $(G_1, \dots, G_n)$ mit
		\begin{itemize}
			\item $G_n = G$ und 
			\item für jedes $G_i (1 \leq i \leq n)$ gilt einer der folgenden Fälle:
			\begin{itemize}
				\item $G_i\in\AxAL$ oder% $G_i$ ist ein Axiom
				\item $G_i\in\Gamma$ oder% $G_i$ ist eine Prämisse
				\item es gibt $i_1,i_2 < i$ mit $(G_{i_1},G_{i_2},G_i)\in\MP$.
			\end{itemize}
		\end{itemize}

		Geschrieben: $\Gamma\vdash G$
	\end{block}

	\begin{block}{Def.: Beweis und Theorem}
	Ist $\Gamma=\{\}$, so heißt eine entsprechende Ableitung auch ein \textbf{Beweis} von $G$ und $G$ ein \textbf{Theorem} des Kalküls, in Zeichen: $\vdash G$
	\end{block}
\end{frame}

\begin{frame}{Aussagenlogik: Beweisbarkeit}
	\begin{exampleblock}{Beispiel zum Modus Ponens}
		Ableitung/Beweis des Theorems $\alka\alP\alimpl\alP\alkz$:\\[1em]
		\begin{tabular}{rll}
			1. & $\alka \alka \alP \alimpl \alka \alka \alP \alimpl  \alP \alkz\alimpl  \alP \alkz\alkz\alimpl 
       			\alka \alka \alP \alimpl \alka \alP \alimpl  \alP \alkz\alkz\alimpl \alka \alP \alimpl  \alP \alkz\alkz\alkz$ & $\AxAL{}_2$ \\
			2. & $\alka \alP \alimpl \alka \alka \alP \alimpl  \alP \alkz\alimpl  \alP \alkz\alkz$ & $\AxAL{}_1$ \\
			3. & $\alka \alka \alP \alimpl \alka \alP \alimpl  \alP \alkz\alkz\alimpl \alka \alP \alimpl  \alP \alkz\alkz$ & $\MP(1,2)$ \\
			4. & $\alka \alP \alimpl \alka \alP \alimpl  \alP \alkz\alkz$ & $\AxAL{}_1$ \\
			5. & $\alka \alP \alimpl  \alP \alkz$ & $\MP(3,4)$
			\end{tabular}

			\qedwhite{}
	\end{exampleblock}
\end{frame}

\begin{frame}{Aussagenlogik: Beweisbarkeit}    
	\begin{alertblock}{Achtung}
		\begin{itemize}
			\item \emph{Tautologie}, \emph{Modell} und \emph{Theorem} sind unterschiedliche Wörter
			\item $\models$ und $\vdash$ sind unterschiedliche Symbole
		\end{itemize}
	\end{alertblock}
\end{frame}


\section[Vollständige Induktion]{Beweisverfahren d. vollständigen Induktion}

\subsection{Einstieg u. Definition}
\begin{frame}{Vollständige Induktion}
	\begin{block}{Vollständige Induktion}
		Vollständige Induktion beschreibt ein Beweisverfahren, das man verwendet um die Gültigkeit einer Aussage \(A(n)\) für alle \(n \in \nN_+ \text{ (oder } \nN_{0}\)) zu beweisen. Sie besteht aus: \\[.5em]
		\begin{description}
			\item [\textbf{Induktionsanfang (I.A.):}] Zeige, dass \(A(n)\) für \(n=1\) (bzw. \(n=0\)) gilt.
			\item [\textbf{Induktionsschritt (I.S.):}] Zeige, dass, wenn \(A(n)\) gilt, auch \(A(n+1)\) gilt.
				\begin{enumerate}
					\item \enquote{Sei $n \in \nN_+$ (oder $n \in \nN_0$) \textcolor{red}{beliebig}.}
					\item \textbf{Induktionsvoraussetzung (I.V.):} $A(n)$ sei wahr.
					\item Zeige mit der \textbf{I.V.}, dass dann auch $A(n+1)$ wahr ist.
				\end{enumerate}
		\end{description}
	\end{block}

	\begin{alertblock}{}
		Es gibt einige Varianten vollständiger Induktion. Immer darauf achten, was man braucht und will. Tipp: \textbf{Nachdenken!}
	\end{alertblock}
\end{frame}
\subsection{Vorgerechnetes Beispiel}
\begin{frame}{Vollständige Induktion - Ein dummes Beispiel}
	\begin{exampleblock}{Aufgabe}
		Sei $x_n$ für $x \in \nN_0$ wie folgt definiert:
		\[
			x_n := (-1)^n.
		\]
		Zeige mit vollständiger Induktion, dass 
		$x_n = 
		\begin{cases}
			-1, &\text{ falls $n$ ungerade}\\
			1, &\text{ sonst}
		\end{cases}$
	\end{exampleblock}
\end{frame}
\begin{frame}{Vollständige Induktion - Ein dummes Beispiel}
	\begin{block}{Lösung}
		\begin{itemize}
			\item[I.A.] $n=0: (-1)^0=1 \quad \checkmark$
			\item[I.S.:] $A(n) \rightarrow A(n+1)$.\\
			\begin{itemize}
				\item Sei $n \in \nN_{0}$ beliebig.
				\item \textbf{I.V.:} Es gelte: 
				 			$x_n = 
					\begin{cases}
						-1, &\text{ falls $n$ ungerade}\\
						1, &\text{ sonst}
					\end{cases}$\\[1.5em]
				\item Fall 1: $x_{n+1} \text{ ist ungerade } \Rightarrow x_n \text{ ist gerade}$\\
					$x_{n+1} = (-1)^{n+1} = (-1) \cdot (-1)^n \eqtext{\textbf{I.V.}} (-1) \cdot 1 = -1 \quad \checkmark$
				\item Fall 2: $x_{n+1} \text{ ist gerade } \Rightarrow x_n \text{ ist ungerade}$\\
					$x_{n+1} = (-1)^{n+1} = (-1) \cdot (-1)^n \eqtext{\textbf{I.V.}} (-1) \cdot (-1) = 1 \quad \checkmark$
			\end{itemize}
			Mit Fall 1 und 2 folgt die Behauptung. \qedwhite{}
		\end{itemize}
	\end{block}
\end{frame}

\subsection{Ein leichter induktiver Beweis auf einer Sprache}
\begin{frame}{Induktiv Beweisen I.: Zum Warmwerden}
	\begin{exampleblock}{Behauptung}
		Es sei w ein Wort. Es gilt:\\[.5em]
		\centering{Für alle $k \in \nN_0$ gilt $|w^k| = k \cdot |w|$.}
	\end{exampleblock}
\pause
	\begin{block}{Beweis}
		\begin{itemize}
			\item[I.A.:] Es sei $k=0$. Es gilt $|w^k|=|w^0|=|\varepsilon|=0=0 \cdot |w| = k \cdot |w|$.
			\item[I.S.:] $A(k) \rightarrow A(k+1)$:
				\begin{itemize}
					\item Sei $k \in \nN_{0}$ beliebig.
					\item \textbf{I.V.:} Es gelte $|w^k|=k \cdot |w|$. \\[1em]
					\item $|w^{k+1}|=|w^k \cdot w|=|w^k|+|w| \eqtext{I.V.} k \cdot |w| + |w| = (k+1) \cdot |w|$
				\end{itemize}						
		\end{itemize}

		Nach den Regeln der vollständigen Induktion folgt aus obigem Induktionsschritt die Behauptung.
	\end{block}
\end{frame}
\subsection{Variante vollständiger Induktion} % z.B. gilt für alle <=n oder mit größerem Anfang und n->n+2, etc
\begin{frame}{Induktiv Beweisen II.: Schwieriger}
	\begin{exampleblock}{Aufgabe}
		Eine Funktion $T:\nN_0 \rightarrow \nN_0$ sei wie folgt definiert:
		\begin{equation*}
		T(n) = 
			\begin{cases}
			 2, &\text{ wenn } n=0 \\
			 3, &\text{ wenn } n=1 \\
			 3 \cdot T(n-1) - 2 \cdot T(n-2), &\text{ wenn }n \in \nN_0 \setminus \set{0,1}
		\end{cases} \pause
		\end{equation*}
		\begin{enumerate}
			\item Gib die Funktionswerte $T(n)$ für $n \in \set{2,3,4,5,6}$ an.
			\item Gib eine geschlossene Formel $F(n)$ (einen arithmetischen Ausdruck ohne Rekursion) für $T(n)$ an.
			\item Zeige durch vollständige Induktion, dass für alle $n \in \nN_0$ gilt $F(n) = T(n)$.
		\end{enumerate}
	\end{exampleblock}
\end{frame}
\subsection{Ein schwieriger mathematischer Beweis}
\begin{frame}{Induktiv Beweisen II.: Schwieriger}
	\begin{block}{Lösung}
		\begin{enumerate}
			\item $ T(2)=5,T(3)=9, T(4)=17, T(5)=33, T(6)=65$
			\item $F(n) = 2^n+1$
			\newcounter{kevin}
			\setcounter{kevin}{\value{enumi}}
		\end{enumerate}
	\end{block}
\end{frame}
\begin{frame}{Induktiv Beweisen II.: Schwieriger}
	\begin{block}{Lösung}
		\begin{enumerate}
			\setcounter{enumi}{\value{kevin}}
			\item \begin{itemize}
		 		\item Beh.: Für alle $n \in \nN_0$ gilt $F(n) = T(n)$.
		 		\item Bew. (induktiv): 
				\begin{itemize}
					\item[I.A.:]
								Sei $n=0$. Dann gilt: \\ 
								\qquad $F(n)=F(0)=2^0+1=1+1=2=T(0)=T(n)$.\\
							 	Sei $n=1$. Dann gilt: \\
							 	\qquad $F(n)=F(1)=2^1+1=2+1=3=T(1)=T(n)$.
				\end{itemize}
		 	\end{itemize}
		\end{enumerate}		 					  
	\end{block}
\end{frame}
\begin{frame}{Induktiv Beweisen II.: Schwieriger}
	\begin{block}{Lösung}
	\begin{enumerate}
		\setcounter{enumi}{\value{kevin}}
		\item
		\begin{itemize}
				\item[I.S.:] $A(n) \wedge A(n+1) \rightarrow A(n+2)$ %\zz $T(n+2) = F(n+2)$, also $n \leadsto n+2$.
					\begin{itemize}
						\item Sei $n \in \nN_{0}$ beliebig.
						\item \textbf{I.V.:} Für ein bel., aber festes $n \in \nN_0$ gelte:
				 					\[T(n)=F(n)=2^n+1 \text{ und } T(n+1)=F(n+1)=2^{n+1}+1\]
				 		\item \zz $T(n+2) = F(n+2)$
				 		\begin{align*}
							T(n+2) &= 3 \cdot T(n+1) - 2 \cdot T(n) \\
								   &\eqtext{I.V.} 3 \cdot F(n+1) - 2 \cdot F(n) \\
								   &= 3 \cdot (2^{n+1}+1) - 2 \cdot (2^n+1) \\
								   &= 3 \cdot 2^{n+1} + 3 - 2 \cdot 2^n - 2 \\
								   &= 3 \cdot 2^{n+1} - 2 \cdot 2^n + 1 \\
								   &= 3 \cdot 2^{n+1} - 2^{n+1} +1 \\
								   &= 2 \cdot 2^{n+1} + 1 = 2^{n+2} + 1 = F(n+2)
						\end{align*}
					\end{itemize}				
			\end{itemize}
		\end{enumerate}
	\end{block}
\end{frame}
\subsection{Eine Knobelaufgabe}
\begin{frame}{Induktiv Beweisen III.: Zum Knobeln}
	\begin{exampleblock}{Aufgabe}
		Alice und Bob feiern ihren Hochzeitstag. Auf ihrer Party befinden sich $n \in \nN_+$ Paare. Dabei begrüßen sich alle Paare mit Ausnahme des eigenen Partners.\\
		\begin{enumerate}
			\item Gib die Anzahl der Begrüßungen für $i \in \set{1,2,3,4,5}$ Paare an.
			\item Stelle für die Anzahl der Begrüßungen einen geschlossenen Ausdruck $p_n$ auf.
			\item Gib für die Anzahl der Begrüßungen eine induktive Definiton $q_n$ an.
			\item Zeige per vollständiger Induktion, dass für alle $n \in \nN_+$ gilt: $p_n$ = $q_n$.
		\end{enumerate}
	\end{exampleblock}
\end{frame}

\begin{frame}{Induktiv Beweisen III.: Zum Knobeln}
	\begin{block}{Lösung}
		\begin{enumerate}
			\item \begin{itemize}
				\item 1 Paar $\rightarrow$ 0 Begrüßungen
				\item 2 Paare $\rightarrow$ 4 Begrüßungen
				\item 3 Paare $\rightarrow$ 12 Begrüßungen
				\item 4 Paare $\rightarrow$ 24 Begrüßungen
				\item 5 Paare $\rightarrow$ 40 Begrüßungen
			\end{itemize} \pause
			\item $p_n = 2 \cdot n \cdot (n-1) = 2 \cdot (n^2 -n)$ \pause
			\item Jedes Paar, das neu hinzukommt, muss die $n$ sich im Raum befindenden Paare begrüßen. Das bedeutet $2n$ Begrüßungen (ein Paar besteht aus $2$ Partnern) pro neu hinzugekommenen Partner, also insgesamt $4n$ Begrüßungen. Formal heißt das: 
			\begin{align*}
				q_1 &= 0 \\
				\text{Für alle } n \in \nN_+ \text{ gilt: } q_{n+1} &= q_n + 4n
			\end{align*}
			\setcounter{kevin}{\value{enumi}}
		\end{enumerate}
	\end{block}
\end{frame}

\begin{frame}{Induktiv Beweisen III.: Zum Knobeln}
	\begin{block}{Lösung}
	\begin{enumerate}
		\setcounter{enumi}{\value{kevin}}
		\item \begin{itemize}
			\item[I.A.] $n=1: p_n = p_1 = 2 \cdot (1^2 -1 ) = 0 = q_1 = q_n \quad \checkmark$ \\[1em]
			%\item[I.V.] Für ein beliebiges, aber festes $n \in \nN_+$ gelte $p_n=q_n$.
			\item[I.S.] $A(n) \rightarrow A(n+1)$ %\zz $p_{n+1}=q_{n+1}$, also $n \leadsto n+1$
				\begin{itemize}
					\item Sei $n \in \nN_{+}$ beliebig.
					\item \textbf{I.V.:} Es gelte $p_n = q_n$.
					\item \zz $p_{n+1}=q_{n+1}$
						\begin{align*}
							p_{n+1} &= 2 \cdot (n+1) \cdot (n+1-1) \\
									&= 2n^2 + 2n \\
									&= 2n^2 -2n + 4n \\
									&= 2 \cdot n \cdot (n-1) + 4n \\
									&= p_n + 4n \\
									&\eqtext{I.V.} q_n + 4n \\
									&\eqtext{Def.} q_{n+1} & \qedwhite{}
						\end{align*}
				\end{itemize}
		\end{itemize}
	\end{enumerate}		
	\end{block}
\end{frame}



%%%%%%%%%% %%%%%%%%%%
%% Zusammenfassung
\section{}
%\subsection{Zusammenfassung}
	\begin{frame}{Was ihr jetzt kennen und können solltet\dots}
			\begin{itemize}
				\item Durch Abbildungsformalien Ableitungen und Beweise führen.
				\item Beweise per vollständiger Induktion
			\end{itemize}
	
	\end{frame}
%% Ausblick
%\subsection{Ausblick}
\begin{frame}[fragile]{Ausblick}
 		\begin{itemize}
 			\item Formale Sprachen 2: Return of the Languages
 		\end{itemize}
\end{frame}
%%%%%%%%%% %%%%%%%%%%
%%%%%%%%%% %%%%%%%%%%
\section{}
\questionframe
\lastframe
\mode<handout>{\slideThanks}
\end{document}