\section{Reguläre Ausdrücke}
\subsection{Reguläre Ausdrücke}

\begin{frame}{Reguläre Ausdrücke}
\begin{block}{Def.: Regulärer Ausdruck}
	Es sei $A$ ein Alphabet, das keines der Symbole aus $Z := \set{\mid, (, ), *, \emptyset}$ enthält.
	Ein \emph{regulärer Ausdruck} über $A$ ist eine Zeichenfolge über $A \cup Z$, das gewissen Vorschriften genügt.\\
	Die \emph{Menge der regulären Ausdrücke} ist wie folgt festgelegt:
	\begin{itemize}
		\item $\emptyset$ ist ein regulärer Ausdruck
		\item Für jedes $x \in A$ ist $x$ ein regulärer Ausdruck
		\item Sind $R_1$ und $R_2$ reguläre Ausdrücke, dann auch $(R_1 | R_2 )$ und $(R_1R_2)$
		%\item Ist $R$ ein regulärer Ausdruck, dann auch $\(R*\)$
		\item Nichts anderes sind reguläre Ausdrücke
	\end{itemize}
	Die durch $R$ beschriebene formale Sprache ist $\lang{R}$.
\end{block}
\end{frame}

\begin{frame}{Reguläre Ausdrücke}
\begin{exampleblock}{Aufgabe}
	Gib einen regulären Ausdruck $R_i$ an, für den gilt:
	\begin{enumerate}
		\item $\lang{R_\theenumi}$ enthält genau die Wörter, in denen das Teilwort $\texttt{baa}$ vorkommt,
		\item $\lang{R_\theenumi}$ enthält genau die Wörter, in denen das Teilwort $\texttt{baa}$ nicht vorkommt,
		\item $\lang{R_\theenumi}$ enthält genau die Wörter, in denen das Teilwort $\texttt{baa}$ genau zweimal vorkommt,
		\item $\lang{R_\theenumi}$ enthält genau die Wörter, in denen mindestens drei $\texttt{b}$s vorkommen,
		\item \emph{Wichtig!} $\lang{R_\theenumi} = \set{\varepsilon}$,
	\end{enumerate}
	mit $i \in \set{1,2,3,4,5}$ und $A := \set{\texttt{a}, \texttt{b}}$.
\end{exampleblock}
\end{frame}

\begin{frame}{Reguläre Ausdrücke}
\newcommand{\any}{(\texttt{a} | \texttt{b}) *}
\begin{block}{Lösung}
	\begin{enumerate}
		\item $R_\theenumi := \any \texttt{baa} \any $
		\item Die Aussage bedeutet umformuliert, dass nach jedem $\texttt{b}$ höchstens ein $\texttt{a}$ kommt, also $R_\theenumi := \texttt{a} * (\texttt{b} | \texttt{ba} ) * $
		\item In jedem Wort muss genau zweimal $\texttt{baa}$ vorkommen. Davor, dazwischen und danach dürfen Teilworte stehen, in denen das Teilwort $\texttt{baa}$ nicht vorkommt (wie in 2.). Also $R_\theenumi := R_2 \texttt{baa} R_2 \texttt{baa} R_2$.
		\item z.B. $R_\theenumi := \any \texttt{b} \any \texttt{b}\any \texttt{b} \any $
		oder $R_\theenumi := \texttt{a} * \texttt{ba} * \texttt{ba} * \texttt{b} \any$
		\item Für $R_\theenumi := \emptyset *$ ist $\lang{R_\theenumi} = \lang{\emptyset *} = \lang{\emptyset}^* = \lang{}^* = \set{}^* = \set{\varepsilon}$
	\end{enumerate}
\end{block}
\end{frame}

\begin{frame}{Reguläre Ausdrücke}
\begin{exampleblock}{Aufgabe}
	Wenn $R$ ein regulärer Ausdruck für die Sprache $L$ ist, wie sieht dann ein regulärer Ausdruck für
	\begin{enumerate}
		\item $L^*$,
		\item $L^+$
	\end{enumerate}
	aus?
\end{exampleblock}
\pause
\begin{block}{Lösung}
	\begin{enumerate}
		\item $(R*)$
		\item $R(R*)$
	\end{enumerate}
\end{block}
\end{frame}