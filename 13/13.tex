% ===== handout mode =====
% Comment/uncomment this line to toggle handout mode
% \newcommand{\handout}{}

% Comment/uncoment this line to toogle Mortitz mode
% \newcommand{\Moritz}{}

% Comment/uncomment this line to toggle handout mode
% \newcommand{\handout}{}

% by Stephan

%% Moritz mode or Stephan mode
\ifdefined \MoritzMode

% This is a configuration file with private, tutor specific information.
% It is therefore excluded from the Git repository so changes in this file will not conflict in git commits.

% Copy this Template, rename to config.tex and add your information below.

\newcommand{\mymail}{moritz.laupichler@student.kit.edu} % Consider using your named student Mail address to keep your u-Account private.

\newcommand{\myname}{\href{mailto:\mymail}{Moritz Laupichler}}

\newcommand{\mytutnumber}{27}

\newcommand{\mytutinfos}{Dienstags, 5. Block (15:45-17:15), SR 236}

\newcommand{\aboutMeFrame}{
	\begin{frame}{Euer Tutor}
		Name: \myname \\
		Alter: 19 Jahre \\
		Studiengang: Bachelor Informatik, 3. Semester \\
		\vspace{1cm}
		\pause 
		\centering{Kontakt: \href{mailto:\mymail}{\mymail}}
	\end{frame}
}

% Toggle Handout mode by including the following line before including style_tut
% and removing the % at the start (but do NOT remove it here, otherwise handout mode will always be on!)
% Please keep handout mode on in all commits!

% \newcommand{\handout}{} % Moritz mode
\fi
\ifdefined \AlexMode

% This is a configuration file with private, tutor specific information.
% It is therefore excluded from the Git repository so changes in this file will not conflict in git commits.

% Copy this Template, rename to config.tex and add your information below.

\newcommand{\mymail}{alexander.klug@student.kit.edu} % Consider using your named student Mail address to keep your u-Account private.

\newcommand{\myname}{\href{mailto:\mymail}{Alexander Klug}}

\newcommand{\mytutnumber}{30}

\newcommand{\mytutinfos}{Mittwochs, 3. Block (11:30-13:00), SR -107}

\newcommand{\aboutMeFrame}{
	\begin{frame}{Euer Tutor}
		Name: \myname \\
		Alter: 19 Jahre \\
		Studiengang: Bachelor Informatik, 3. Semester \\
		\vspace{1cm}
		\pause 
		\centering{Kontakt: \href{mailto:\mymail}{\mymail}}
	\end{frame}
}

% Toggle Handout mode by including the following line before including style_tut
% and removing the % at the start (but do NOT remove it here, otherwise handout mode will always be on!)
% Please keep handout mode on in all commits!

% \newcommand{\handout}{} % Alex Mode
\fi
\ifdefined \StephanMode

% This is a configuration file with private, tutor specific information.
% It is therefore excluded from the Git repository so changes in this file will not conflict in git commits.

% Copy this Template, rename to config.tex and add your information below.

\newcommand{\mymail}{stephan.bohr@student.kit.edu} % Consider using your named student Mail address to keep your u-Account private.

\newcommand{\myname}{\href{mailto:\mymail}{Stephan Bohr}}

\newcommand{\mytutnumber}{19}

\newcommand{\mytutinfos}{Dienstags, 3. Block (11:30-13:00), SR -108}

\newcommand{\aboutMeFrame}{
	\begin{frame}{Euer Tutor}
		Name: \myname \\
		Alter: 21 Jahre \\
		Studiengang: Bachelor Informatik, 5. Semester \\
		\vspace{1cm}
		\pause 
		\centering{Kontakt: \href{mailto:\mymail}{\mymail}}
	\end{frame}
} % Stephan mode
\fi

%% Beamer-Klasse im korrekten Modus
\ifdefined \handout
\documentclass[handout]{beamer} % Handout mode
\else
\documentclass{beamer}
\fi
%\documentclass[18pt,parskip]{beamer}

%% SLIDE FORMAT

% use 'beamerthemekit' for standard 4:3 ratio
% for widescreen slides (16:9), use 'beamerthemekitwide'

\usepackage{../templates/KIT-slides/beamerthemekit}
%\usepackage{../templates/KIT-slides/beamerthemekitwide}

%% TITLE PICTURE

% if a custom picture is to be used on the title page, copy it into the 'logos'
% directory, in the line below, replace 'mypicture' with the 
% filename (without extension) and uncomment the following line
% (picture proportions: 63 : 20 for standard, 169 : 40 for wide
% *.eps format if you use latex+dvips+ps2pdf, 
% *.jpg/*.png/*.pdf if you use pdflatex)

\titleimage{../figures/titleimage/brain}

%% TITLE LOGO

% for a custom logo on the front page, copy your file into the 'logos'
% directory, insert the filename in the line below and uncomment it

%\titlelogo{mylogo}

% (*.eps format if you use latex+dvips+ps2pdf,
% *.jpg/*.png/*.pdf if you use pdflatex)

%% TikZ INTEGRATION

% use these packages for PCM symbols and UML classes
% \usepackage{templates/tikzkit}
% \usepackage{templates/tikzuml}

%\usepackage{tikz}
%\usetikzlibrary{matrix}
%\usetikzlibrary{arrows.meta}
%\usetikzlibrary{automata}
%\usetikzlibrary{tikzmark}

%%%%%%%%%%%%%%%%%%%%%%%%%
% Libertine font (Original GBI font)
\usepackage[mono=false]{libertine}
%\renewcommand*\familydefault{\sfdefault}  %% Only if the base font of the document is to be sans serif

%% Schönere Schriften
\usepackage[TS1,T1]{fontenc}

%% Deutsche Silbentrennung und Beschriftungen
\usepackage[ngerman]{babel}

%% UTF-8-Encoding
\usepackage[utf8]{inputenc}

%% Bibliotheken für viele mathematische Symbole
\usepackage{amsmath, amsfonts, amssymb}

%% Anzeigetiefe für Inhaltsverzeichnis: 1 Stufe
\setcounter{tocdepth}{1}

%% Hyperlinks
\usepackage{hyperref}
% I don't know why, but this works and only includes sections and NOT subsections in the pdf-bookmarks.
\hypersetup{bookmarksdepth=subsection}

%% remove navigation symbols
\setbeamertemplate{navigation symbols}{}

%% switch between "ngerman" and "english" for German/English style date and logos
\selectlanguage{ngerman}

%% for invisible pause texts instead of dimming
\setbeamercovered{invisible}

\usepackage[german=swiss]{csquotes}

\usepackage{tabularx}
\usepackage{booktabs}

\usepackage{tikz}


% Problem: disabled itemize-icons
%\usepackage{enumitem}
% %\setlist[enumerate]{topsep=0pt,itemsep=-1ex,partopsep=1ex,parsep=1ex}
% \setlist[itemize]{noitemsep, nolistsep}
% \setlist[enumerate]{noitemsep, nolistsep}

% Mathmode no vertical space (https://tex.stackexchange.com/a/47403/146825)
\setlength{\abovedisplayskip}{0pt}
\setlength{\belowdisplayskip}{0pt}
\setlength{\abovedisplayshortskip}{0pt}
\setlength{\belowdisplayshortskip}{0pt}

%%%%%%%%%%%% Slides %%%%%%%%%%%%%%%%

\newcommand{\Moritz}[1]{
	\ifdefined \MoritzMode
	#1
	\fi
}

\newcommand{\Alex}[1]{
	\ifdefined \AlexMode
	#1
	\fi
}

\newcommand{\Stephan}[1]{
	\ifdefined \StephanMode
	#1
	\fi
}

\newcommand{\notMoritz}[1]{
	\Alex{#1} \Stephan{#1}
}

\newcommand{\notAlex}[1]{
	\Moritz{#1} \Stephan{#1}
}

\newcommand{\notStephan}[1]{
	\Alex{#1} \Moritz{#1}
}

%% Wochennummer
%\newcounter{weeknum}

%% Titelinformationen
%\title[GBI Tutorium, Woche \theweeknum]{Grundbegriffe der Informatik \\ Tutorium \mytutnumber}
%\subtitle{Termin \theweeknum \ | \mydate \\ \myname}
\author[\myname]{\myname}
\institute{Fakultät für Informatik}
%\date{\mydate}

%% Titel einfügen
\newcommand{\titleframe}{\frame{\titlepage}\addtocounter{framenumber}{-1}}


%% Alles starten mit \starttut{X}
%\newcommand{\starttut}[1]{\setcounter{weeknum}{#1}\titleframe\frame{\frametitle{Inhalt}\tableofcontents} \AtBeginSection[]{%
%\begin{frame}
%	\tableofcontents[currentsection]
%\end{frame}\addtocounter{framenumber}{-1}}}


%\newcommand{\framePrevEpisode}{
%	\begin{frame}
%		\centering
%		\textbf{In the previous episode of GBI...}
%	\end{frame}
%}

%% Roadmap frame
%table of contents
\newcommand{\roadmap}{
	\frame{\frametitle{Roadmap}\tableofcontents}}

 \AtBeginSection[]{%
\begin{frame}
	\frametitle{Roadmap}
	\tableofcontents[currentsection]
\end{frame}%\addtocounter{framenumber}{-1}
}


%% ShowMessage frame
\newcommand{\showmessage}[1]{\frame{\frametitle{\phantom{1em}}\centering\textbf{#1}}}

%% Fragen
%% Lastframe
\newcommand{\questionframe}{\showmessage{Fragen?}}

%% Lastframe
\newcommand{\lastframe}{\showmessage{Vielen Dank für Eure Aufmerksamkeit! \\Bis nächste Woche :)}}

%% Thanks frame
\newcommand{\slideThanks}{
	\begin{frame}
		\frametitle{Credits}
		\begin{block}{}
			An der Erstellung des Foliensatzes haben mitgewirkt:\\[1em]
			\Moritz{
			Stephan Bohr \\
			Alexander Klug \\
			}
			\Alex{
			Stephan Bohr \\
			Moritz Laupichler \\
			}
			\Stephan{
			Moritz Laupichler \\
			Alexander Klug \\
			}
			Katharina Wurz \\
			Thassilo Helmold \\
			Daniel Jungkind \\
			% Philipp Basler \\
			% Nils Braun \\
			% Dominik Doerner \\
			% Ou Yue \\
		\end{block}
	\end{frame}
}

%% Verbatim
%\usepackage{moreverb}

% GBI related stuff, but not beamer-stuff
\newcommand{\newpar}[1]{\paragraph{#1}\mbox{}\newline}

\newcommand{\nM}{\mathbb{M}}
\newcommand{\nR}{\mathbb{R}}
\newcommand{\nN}{\mathbb{N}}
\newcommand{\nZ}{\mathbb{Z}}
\newcommand{\nQ}{\mathbb{Q}}
\newcommand{\nB}{\mathbb{B}}
\newcommand{\nC}{\mathbb{C}}
\newcommand{\nK}{\mathbb{K}}
\newcommand{\nF}{\mathbb{F}}
\newcommand{\nG}{\mathbb{G}}
\newcommand{\nullel}{\mathcal{O}}
\newcommand{\einsel}{\mathds{1}}
\newcommand{\nP}{\mathbb{P}}
\newcommand{\Pot}{\mathcal{P}}
\renewcommand{\O}{\text{O}}

\newcommand{\bfmod}{\ensuremath{\text{\textbf{ mod }}}}
\renewcommand{\mod}{\bfmod}
\newcommand{\bfdiv}{\ensuremath{\text{\textbf{ div }}}}
\renewcommand{\div}{\bfdiv}


\newcommand{\set}[1]{\left\{ #1 \right\}}
\newcommand{\setc}[2]{\set{#1 \mid #2}}
\newcommand{\setC}[2]{\set{#1 \mid \text{ #2 }}}

\newcommand{\setsize}[1]{\; \mid #1 \mid \; }

\newcommand{\q}[1]{\textquotedblleft #1\textquotedblright}

% Zu zeigen, thx to http://www.matheboard.de/archive/155832/thread.html
\newcommand{\zz}{\ensuremath{\mathrm{z\kern-.29em\raise-0.44ex\hbox{z}}}:}

% Text above symbol
% https://tex.stackexchange.com/a/74132/146825
%
% \newcommand{\eqtext}[1]{\stackrel{\mathclap{\normalfont\mbox{#1}}}{=}}
% \newcommand{\gdwtext}[1]{\stackrel{\mathclap{\normalfont\mbox{#1}}}{\Leftrightarrow}}
% \newcommand{\imptext}[1]{\stackrel{\mathclap{\normalfont\mbox{#1}}}{\Rightarrow}}
% \newcommand{\symbtext}[2]{\stackrel{\mathclap{\normalfont\mbox{#2}}}{#1}}
\newcommand{\eqtext}[1]{\mathrel{\overset{\makebox%[0pt]
{\mbox{\normalfont\tiny #1}}}{=}}}
\newcommand{\gdwtext}[1]{\mathrel{\overset{\makebox%[0pt]
{\mbox{\normalfont\tiny #1}}}{\ensuremath{\Leftrightarrow}}}}
\newcommand{\imptext}[1]{\mathrel{\overset{\makebox%[0pt]
{\mbox{\normalfont\tiny #1}}}{\ensuremath{\Rightarrow}}}}
\newcommand{\symbtext}[2]{\mathrel{\overset{\makebox%[0pt]
{\mbox{\normalfont\tiny #2}}}{#1}}}

% qed symbol
\newcommand{\qedblack}{\hfill \ensuremath{\blacksquare}}
\newcommand{\qedwhite}{\hfill \ensuremath{\Box}}

% Aussagenlogik
% Worsch
\colorlet{alcolor}{blue}
\RequirePackage{tikz}
\usetikzlibrary{arrows.meta}
\newcommand{\alimpl}{\mathrel{\tikz[x={(0.1ex,0ex)},y={(0ex,0.1ex)},>={Classical TikZ Rightarrow[]}]{\draw[alcolor,->,line width=0.7pt,line cap=round] (0,0) -- (15,0);\path (0,-6);}}}
\newcommand{\alimp}{\alimpl}
\newcommand{\aleqv}{\mathrel{\tikz[x={(0.1ex,0ex)},y={(0ex,0.1ex)},>={Classical TikZ Rightarrow[]}]{\draw[alcolor,<->,line width=0.7pt,line cap=round] (0,0) -- (18,0);\path (0,-6);}}}
\newcommand{\aland}{\mathbin{\raisebox{-0.6pt}{\rotatebox{90}{\texttt{\color{alcolor}\char62}}}}}
\newcommand{\alor}{\mathbin{\raisebox{-0.8pt}{\rotatebox{90}{\texttt{\color{alcolor}\char60}}}}}
%\newcommand{\ali}[1]{_{\mathtt{\color{alcolor}#1}}}
\newcommand{\alv}[1]{\mathtt{\color{alcolor}#1}}
\newcommand{\alnot}{\mathop{\tikz[x={(0.1ex,0ex)},y={(0ex,0.1ex)}]{\draw[alcolor,line width=0.7pt,line cap=round,line join=round] (0,0) -- (10,0) -- (10,-4);\path (0,-8) ;}}}
\newcommand{\alP}{\alv{P}} %ali{#1}}
%\newcommand{\alka}{\negthinspace\hbox{\texttt{\color{alcolor}(}}}
\newcommand{\alka}{\negthinspace\text{\texttt{\color{alcolor}(}}}
%\newcommand{\alkz}{\texttt{\color{alcolor})}}\negthinspace}
\newcommand{\alkz}{\text{\texttt{\color{alcolor})}}\negthinspace}

% Thassilo
\newcommand{\BB}{\mathbb{B}}
\newcommand{\boder}{\alor}%{\ensuremath{\text{\;}\textcolor{blue}{\vee}}\text{\;}}
\newcommand{\bund}{\aland}%{\ensuremath{\text{\;}\textcolor{blue}{\wedge}}\text{\;}}
\newcommand{\bimp}{\alimp}%{\ensuremath{\text{\;}\textcolor{blue}{\to}}\text{\;}}
\newcommand{\bnot}{\alnot}%{\ensuremath{\text{\;}\textcolor{blue}{\neg}}\text{}}
\newcommand{\bgdw}{\aleqv}%{\ensuremath{\text{\;}\textcolor{blue}{\leftrightarrow}}\text{\;}}
\newcommand{\bone}{\ensuremath{\textcolor{blue}{1}}\text{}}
\newcommand{\bzero}{\ensuremath{\textcolor{blue}{0}}\text{}}
\newcommand{\bleftBr}{\alka}%{\ensuremath{\textcolor{blue}{(}}\text{}}
\newcommand{\brightBr}{\alkz}%{\ensuremath{\textcolor{blue}{)}}\text{}}

\newcommand{\val}{\hbox{\textit{val}}}

\newcommand{\VarAL}{\hbox{\textit{Var}}_{AL}}
\newcommand{\ForAL}{\hbox{\textit{For}}_{AL}}

% Validierungsfunktion val_i
\newcommand{\vali}[1]{\ensuremath{\val_I(#1)}}

% Boolsche Funktion b_
\newcommand{\bfnot}[1]{\ensuremath{b_{\bnot}(#1)}}
\newcommand{\bfand}[2]{\ensuremath{b_{\bund}(#1,#2)}}
\newcommand{\bfor}[2]{\ensuremath{b_{\boder}(#1,#2)}}
\newcommand{\bfimp}[2]{\ensuremath{b_{\bimp}(#1,#2)}}

% Aussagenkalkül
\newcommand{\AAL}{A_{AL}}
\newcommand{\LAL}{\hbox{\textit{For}}_{AL}}
\newcommand{\AxAL}{\hbox{\textit{Ax}}_{AL}}
\newcommand{\MP}{\hbox{\textit{MP}}}

% Prädikatenlogik
% die nachfolgenden Sachen angepasst an cmtt
\newlength{\ttquantwd}
\setlength{\ttquantwd}{1ex}
\newlength{\ttquantht}
\setlength{\ttquantht}{6.75pt}
\def\plall{%
  \tikz[line width=0.67pt,line cap=round,line join=round,baseline=(B),alcolor] {
    \draw (-0.5\ttquantwd,\ttquantht) -- node[coordinate,pos=0.4] (lll){} (-0.25pt,-0.0pt) -- (0.25pt,-0.0pt) -- node[coordinate,pos=0.6] (rrr){} (0.5\ttquantwd,\ttquantht);
    \draw (lll) -- (rrr);
    \coordinate (B) at (0,-0.35pt);
  }%
}
\def\plexist{%
  \tikz[line width=0.67pt,line cap=round,line join=round,baseline=(B),alcolor] {
    \draw (-0.9\ttquantwd,\ttquantht) -- (0,\ttquantht) -- node[coordinate,pos=0.5] (mmm){} (0,0) --  (-0.9\ttquantwd,0);
    \draw (mmm) -- ++(-0.75\ttquantwd,0);
    \coordinate (B) at (0,-0.35pt);
  }\ensuremath{\,}%
}
\let\plexists=\plexist
\newcommand{\NT}[1]{\ensuremath{\langle\mathrm{#1} \rangle}}
\newcommand{\CPL}{\text{\itshape Const}_{PL}}
\newcommand{\FPL}{\text{\itshape Fun}_{PL}}
\newcommand{\RPL}{\text{\itshape Rel}_{PL}}
\newcommand{\VPL}{\text{\itshape Var}_{PL}}
\newcommand{\plka}{\alka}
\newcommand{\plkz}{\alkz}
%\newcommand{\plka}{\plfoo{(}}
%\newcommand{\plkz}{\plfoo{)}}
\newcommand{\plcomma}{\hbox{\texttt{\color{alcolor},}}}
\newcommand{\pleq}{{\color{alcolor}\,\dot=\,}}

\newcommand{\plfoo}[1]{\mathtt{\color{alcolor}#1}}
\newcommand{\plc}{\plfoo{c}}
\newcommand{\pld}{\plfoo{d}}
\newcommand{\plf}{\plfoo{f}}
\newcommand{\plg}{\plfoo{g}}
\newcommand{\plh}{\plfoo{h}}
\newcommand{\plx}{\plfoo{x}}
\newcommand{\ply}{\plfoo{y}}
\newcommand{\plz}{\plfoo{z}}
\newcommand{\plR}{\plfoo{R}}
\newcommand{\plS}{\plfoo{S}}
\newcommand{\ar}{\mathrm{ar}}

\newcommand{\bv}{\mathrm{bv}}
\newcommand{\fv}{\mathrm{fv}}

\def\word#1{\hbox{\textcolor{blue}{\texttt{#1}}}}
%\let\literal\word
\def\mword#1{\hbox{\textcolor{blue}{$\mathtt{#1}$}}}  % math word
\def\sp{\scalebox{1}[.5]{\textvisiblespace}}
\def\wordsp{\word{\sp}}


\newcommand{\W}{\ensuremath{\hbox{\textbf{w}}}\xspace}
\newcommand{\F}{\ensuremath{\hbox{\textbf{f}}}\xspace}
\newcommand{\WF}{\ensuremath{\{\W,\F\}}\xspace}
\newcommand{\valDIb}{\val_{D,I,\beta}}

\newcommand{\impl}{\ifmmode\ensuremath{\mskip\thinmuskip\Rightarrow\mskip\thinmuskip}\else$\Rightarrow$\fi\xspace}
\newcommand{\Impl}{\ifmmode\implies\else$\Longrightarrow$\fi\xspace}

\newcommand{\derives}{\Rightarrow}

\newcommand{\gdw}{\ifmmode\mskip\thickmuskip\Leftrightarrow\mskip\thickmuskip\else$\Leftrightarrow$\fi\xspace}
\newcommand{\Gdw}{\ifmmode\iff\else$\Longleftrightarrow$\fi\xspace}

\newcommand*{\from}{\colon}
\newcommand{\functionto}{\longrightarrow}


\newcommand{\LTer}{L_{\text{\itshape Ter}}}
\newcommand{\LRel}{L_{\text{\itshape Rel}}}
\newcommand{\LFor}{L_{\text{\itshape For}}}
\newcommand{\NTer}{N_{\text{\itshape Ter}}}
\newcommand{\NRel}{N_{\text{\itshape Rel}}}
\newcommand{\NFor}{N_{\text{\itshape For}}}
\newcommand{\PTer}{P_{\text{\itshape Ter}}}
\newcommand{\PRel}{P_{\text{\itshape Rel}}}
\newcommand{\PFor}{P_{\text{\itshape For}}}

\newcommand{\sgn}{\mathop{\text{sgn}}}

\newcommand{\lang}[1]{\ensuremath{\langle#1\rangle}}

\newcommand{\literal}[1]{\hbox{\textcolor{blue!95!white}{\textup{\texttt{\scalebox{1.11}{#1}}}}}}
\let\hashtag\#
\renewcommand{\#}[1]{\literal{#1}}

\def\blank{\ensuremath{\openbox}}
\def\9{\blank}
\newcommand{\io}{\!\mid\!}


\providecommand{\fspace}{\mathord{\text{space}}}
\providecommand{\fSpace}{\mathord{\text{Space}}}
\providecommand{\ftime}{\mathord{\text{time}}}
\providecommand{\fTime}{\mathord{\text{Time}}}

\newcommand{\fnum}{\text{num}}
\newcommand{\fNum}{{\text{Num}}}

\def\Pclass{\text{\bfseries P}}
\def\PSPACE{\text{\bfseries PSPACE}}



\title[Endliche Automaten, Reguläre Ausdrücke und Rechtslineare Grammatiken]{13. Tutorium\\ Endliche Automaten, Reguläre Ausdrücke, Rechtslineare Grammatiken}
\subtitle{Grundbegriffe der Informatik, Tutorium \mytutnumber}
\date{\today}




\def\mybox#1{\hbox{\vrule height 2ex width 0pt depth 0.6ex#1}}
\def\taste[#1]#2{%
  \tikz[x=8mm,y=8mm,baseline=(N.base)] \tasteinnerx[#1]{#2};%
}
\def\tasteinnerx[#1]#2{%
  \node[midway,inner sep=0mm,draw,rounded corners,anchor=base,minimum width=10mm,#1] (N) {\mybox{#2}}%
}
\def\tasteinner[#1]#2{%
  node[midway,inner sep=0mm,draw,rounded corners,minimum width=10mm,#1] (N) {\mybox{#2}}%
}
\def\tasteinnerOK{\tasteinner[fill=green!20]{OK}}
\def\tasteOK{\taste[fill=green!20]{OK}}
\def\tasteC{\taste[fill=red!20]{C}}
\def\tasteinnerC{\tasteinner[fill=red!20]{C}}
\def\tasteRein{\taste[fill=blue!10]{rein}}
\def\tasteinnerRein{\tasteinner[fill=blue!10]{rein}}
\def\tasteZitro{\taste[fill=yellow!10]{zitro}}
\def\tasteinnerZitro{\tasteinner[fill=yellow!10]{zitro}}

\newcommand{\io}[2]{#1|#2}
\newcommand{\ioeps}[1]{#1|\varepsilon}

\usetikzlibrary{matrix}
\usetikzlibrary{arrows.meta}
\usetikzlibrary{automata}
\usetikzlibrary{tikzmark}

\begin{document}
\titleframe
\roadmap

%%%%%%%%%% %%%%%%%%%%
\section{Organisatorisches}
\begin{frame}{Organisatorisches}
	\begin{itemize}
		\item Für Übungsschein angemeldet?
		\item Für Klausur angemeldet?
		\item Nach Blatt 6 kommt noch ein Bonusblatt mit ausschließlich Bonuspunkten
	\end{itemize}
\end{frame}

\Moritz{
	\begin{frame}{Übungsblatt 5}
		Achten auf:
		\begin{itemize}
			\item Ableitungspfeil: $\Rightarrow$
			\item Bei Grammatiken auf genaue Notation achten: $G=(N,T,S,P)$: N, T und P sind \textbf{Mengen}.
		\end{itemize}
	\end{frame}
}

\section{Endliche Automaten}
\subsection{Mealy-Automat}
\begin{frame}{Mealy-Automat}
	\begin{block}{Def.: Mealy-Automat}
		Ein (endlicher) \textbf{Mealy-Automat} $A=(Z, z_0, X, f, Y, g)$ ist bestimmt durch:
		\begin{itemize}
			\item endliche Zustandsmenge $Z$,
			\item Anfangszustand $z_0 \in Z$,
			\item Eingabealphabet $X$,
			\item Zustandsüberführungsfunktion $f: Z \times X \rightarrow Z$,
			\item Ausgabealphabet $Y$,
			\item Ausgabefunktion $g: Z \times X \rightarrow Y^{\ast}$
		\end{itemize}
	\end{block}

\end{frame}




\begin{frame}{Mealy-Automat}

	\begin{exampleblock}{Beispiel} \small
		Betrachte einen Getränkeautomaten:
		Man kann nur 1-Euro"=Stücke einwerfen und vier Tasten drücken: Es gibt
		zwei Auswahltasten für Mineralwasser \tasteRein{} und Zitronensprudel
		\tasteZitro{}, eine Abbruch"=Taste \tasteC{} und eine \tasteOK-Taste.

	
	\begin{itemize}
	\item Jede Flasche Sprudel kostet 1 Euro.
	\item Es kann ein Guthaben von 1 Euro gespeichert werden. Wirft man
	  weitere Euro"=Stücke ein, werden sie sofort wieder ausgegeben.
	\item Wenn man mehrfach Auswahltasten drückt, wird der letzte Wunsch
	  gespeichert.
	\item Bei Drücken der Abbruch"=Taste wird alles bereits eingeworfenen
	  Geld wieder zurückgegeben und kein Getränkewunsch mehr gespeichert.
	\item Drücken der OK-Taste wird ignoriert, solange noch kein Euro
	  eingeworfen wurde oder keine Getränkesorte ausgewählt wurde.

	  Andernfalls wird das gewünschte Getränk ausgeworfen.
	\end{itemize} 
	\end{exampleblock}
\end{frame}

\begin{frame}{Mealy-Automat}

	\begin{exampleblock}{Beispiel}
		\begin{figure}[ht]
  \centering
  \begin{tikzpicture}[x=8mm,y=8mm]
    % Rahmen
    \draw (0,0) -- (8,0) -- (8,8) .. controls  (3,9) and (5,9) .. (0,8) -- cycle;
    % Geldschlitz 
    \draw (1.4,5.5) rectangle ++(0.2,1) ++(-0.1,0) node[anchor=south] {\vbox{\hbox{Geld-}\hbox{Einwurf}}};
    % Geldrückgabe 
    \draw (1,1) rectangle ++(1,1) ++(-0.5,0) node[anchor=south] {\vbox{\hbox{Geld-}\hbox{Rückgabe}}};
    % Warenauswurf
    \draw (4,1) rectangle ++(3.5,1) ++(-1.75,0) node[anchor=south] {Ware};
    
    % Tasten
    \draw (2.75,3) ++(3,3.5) node[anchor=south] {Sprudel};
    \draw (4,5.5) rectangle ++(1.5,1)\tasteinnerRein;
    \draw (6,5.5) rectangle ++(1.5,1) \tasteinnerZitro;
    \draw (4,4) rectangle ++(1.5,1) \tasteinnerOK;
    \draw (6,4) rectangle ++(1.5,1) \tasteinnerC;
  \end{tikzpicture}
  \caption{Ein primitiver Getr"ankeautomat}
  \label{fig:getraenkeautomat}
\end{figure}
	\end{exampleblock}
\end{frame}

\begin{frame}[fragile]{Mealy-Automat}

	\begin{exampleblock}{Beispiel}
		\begin{figure}[ht]
  \centering
  \small
  \begin{tikzpicture}[->,>=stealth]
    \matrix[matrix of math nodes,column sep=25mm,row sep=20mm,nodes={circle,draw,inner sep=1pt}]   {
      |(0-)| (0,-) & |(0R)| (0,R) & |(0Z)| (0,Z) \\
      |(1-)| (1,-) & |(1R)| (1,R) & |(1Z)| (1,Z) \\
    };

    \coordinate[left of=0-] (start);

    \draw (start) -- node[auto] {} (0-);

    % Schleifen
    \draw (0-) edge[loop above]  node[pos=0.5] {$\ioeps{O}$,$\ioeps{C}$} ();
    \draw (0R) edge[loop above] node[pos=0.9,anchor=west] {$\ioeps{R}$,$\ioeps{O}$} ();
    \draw (0Z) edge[loop above] node[pos=0.5] {$\ioeps{Z}$,$\ioeps{O}$} ();
    \draw (1-) edge[loop below]  node[pos=0.5] {$\io{1}{1}$,$\ioeps{O}$} ();
    \draw (1R) edge[loop below] node[pos=0.9,anchor=east] {$\io{1}{1}$,$\ioeps{R}$} ();
    \draw (1Z) edge[loop below] node[pos=0.5] {$\io{1}{1}$,$\ioeps{Z}$} ();

    % andere Kanten
    \draw (0-) -- node[right,pos=0.2] {$\ioeps{1}$} (1-);
    \draw (0R) -- node[right,pos=0.2] {$\ioeps{1}$} (1R);
    \draw (0Z) -- node[right,pos=0.2] {$\ioeps{1}$} (1Z);

    \draw (1-) to[bend left=10] node[left,pos=0.2] {$\io{C}{1}$} (0-);

    \draw (0-) to[bend right=10] node[below] {$\ioeps{R}$} (0R);
    \draw (0R) to[bend right=10] node[above,pos=0.1] {$\ioeps{C}$} (0-);
    \draw (0R) to[bend right=10] node[below] {$\ioeps{Z}$} (0Z);
    \draw (0Z) to[bend right=10] node[above] {$\ioeps{R}$} (0R);
    \draw (0-) to[bend left=32]  node[below,pos=0.2] {$\ioeps{Z}$} (0Z);
    \draw (0Z) to[bend right=40] node[above] {$\ioeps{C}$} (0-);

    \draw (1-) to[bend right=10] node[above] {$\ioeps{R}$} (1R);
    \draw (1R) -- node[below,pos=0.4,anchor=north east] {$\io{O}{R}$,$\io{C}{1}$} (0-); %!!
    \draw (1R) to[bend right=10] node[below] {$\ioeps{Z}$} (1Z);
    \draw (1Z) to[bend right=10] node[above] {$\ioeps{R}$} (1R);
    \draw (1-) to[bend left=-32] node[below,pos=0.2] {$\ioeps{Z}$} (1Z);
    \draw (1Z) -- node[above,pos=0.3,anchor=south west] {$\io{O}{Z}$,$\io{C}{1}$} (0-); %!!
  \end{tikzpicture}
\end{figure}
	\end{exampleblock}
\end{frame}

\begin{frame}{Mealy-Automat}
	\begin{block}{Def.: Verallgemeinerte Zustandsübergangsfunktion $f_{\ast}$}
		$f_{\ast} : Z \times X^{\ast} \rightarrow Z$ gibt den Zustand aus, der durch die Eingabe eines Wortes erreicht wird, und ist definiert durch:
		\begin{align*}
			f_{\ast}(z,\varepsilon) &= z \\
			\forall w \in X^{\ast} : \forall x \in X : f_{\ast}(z,wx) &= f(f_{\ast}(z,w),x)\\
		\end{align*}
		Oder als alternative Definition:
		\begin{align*}
			\bar{f}_{\ast}(z,\varepsilon) &= z \\
			\forall w \in X^{\ast} : \forall x \in X : \bar{f}_{\ast}(z,xw) &= \bar{f}_{\ast}(f(z,x),w)
		\end{align*}
	\end{block}
\end{frame}

\begin{frame}{Mealy-Automat}
	\begin{block}{Def.: Verallgemeinerte Zustandsübergangsfunktion $f_{\ast\ast}$}
		$f_{\ast\ast} : Z \times X^{\ast} \rightarrow Z^{\ast}$ gibt \textbf{alle} Zustände aus, die bei der Eingabe eines Wortes durchlaufen werden, und ist definiert durch:
		\begin{align*}
			f_{\ast\ast}(z,\varepsilon) = z\\
			\forall w \in X^{\ast} : \forall x \in X : f_{\ast\ast}(z,wx) &= f_{\ast\ast}(z,w) \cdot f(f_{\ast}(z,w),x)\\
		\end{align*}
	\end{block}
\end{frame}

\begin{frame}[fragile]{Mealy-Automat}

	\begin{exampleblock}{Aufgabe zu $f_{\ast}$ und $f_{\ast\ast}$}
		\begin{columns}
			\begin{column}{0.6\textwidth}
				\begin{figure}[ht]
  \centering
  \tiny
  \begin{tikzpicture}[->,>=stealth]
    \matrix[matrix of math nodes,column sep=20mm,row sep=20mm,nodes={circle,draw,inner sep=1pt}]   {
      |(0-)| (0,-) & |(0R)| (0,R) & |(0Z)| (0,Z) \\
      |(1-)| (1,-) & |(1R)| (1,R) & |(1Z)| (1,Z) \\
    };

    \coordinate[left of=0-] (start);

    \draw (start) -- node[auto] {} (0-);

    % Schleifen
    \draw (0-) edge[loop above]  node[pos=0.5] {$\ioeps{O}$,$\ioeps{C}$} ();
    \draw (0R) edge[loop above] node[pos=0.9,anchor=west] {$\ioeps{R}$,$\ioeps{O}$} ();
    \draw (0Z) edge[loop above] node[pos=0.5] {$\ioeps{Z}$,$\ioeps{O}$} ();
    \draw (1-) edge[loop below]  node[pos=0.5] {$\io{1}{1}$,$\ioeps{O}$} ();
    \draw (1R) edge[loop below] node[pos=0.9,anchor=east] {$\io{1}{1}$,$\ioeps{R}$} ();
    \draw (1Z) edge[loop below] node[pos=0.5] {$\io{1}{1}$,$\ioeps{Z}$} ();

    % andere Kanten
    \draw (0-) -- node[right,pos=0.2] {$\ioeps{1}$} (1-);
    \draw (0R) -- node[right,pos=0.2] {$\ioeps{1}$} (1R);
    \draw (0Z) -- node[right,pos=0.2] {$\ioeps{1}$} (1Z);

    \draw (1-) to[bend left=10] node[left,pos=0.2] {$\io{C}{1}$} (0-);

    \draw (0-) to[bend right=10] node[below] {$\ioeps{R}$} (0R);
    \draw (0R) to[bend right=10] node[above,pos=0.1] {$\ioeps{C}$} (0-);
    \draw (0R) to[bend right=10] node[below] {$\ioeps{Z}$} (0Z);
    \draw (0Z) to[bend right=10] node[above] {$\ioeps{R}$} (0R);
    \draw (0-) to[bend left=32]  node[below,pos=0.2] {$\ioeps{Z}$} (0Z);
    \draw (0Z) to[bend right=40] node[above] {$\ioeps{C}$} (0-);

    \draw (1-) to[bend right=10] node[above] {$\ioeps{R}$} (1R);
    \draw (1R) -- node[below,pos=0.4,anchor=north east] {$\io{O}{R}$,$\io{C}{1}$} (0-); %!!
    \draw (1R) to[bend right=10] node[below] {$\ioeps{Z}$} (1Z);
    \draw (1Z) to[bend right=10] node[above] {$\ioeps{R}$} (1R);
    \draw (1-) to[bend left=-32] node[below,pos=0.2] {$\ioeps{Z}$} (1Z);
    \draw (1Z) -- node[above,pos=0.3,anchor=south west] {$\io{O}{Z}$,$\io{C}{1}$} (0-); %!!
  \end{tikzpicture}
\end{figure}
			\end{column}
			\begin{column}{0.375\textwidth}
			\small
				Gib an:
				\begin{enumerate}
					\item $f_{\ast}((0,-), R1O)$
					\item $f_{\ast\ast}((0,-), R1O)$
					\item $f_{\ast}((0,-), C1Z)$
					\item $f_{\ast\ast}((0,-), RZO)$
					\item $f_{\ast\ast}((1,Z), RZO)$
					\item $f_{\ast}((0,-), RZ1R1C1ZO)$
				\end{enumerate}
			\end{column}
		\end{columns}
	\end{exampleblock}
\end{frame}

\begin{frame}{Mealy-Automat}
	\begin{block}{Lösung zur Aufgabe zu $f_{\ast}$ und $f_{\ast\ast}$}
		\begin{enumerate}
					\item $f_{\ast}((0,-), R1O) = (0,-)$
					\item $f_{\ast\ast}((0,-), R1O) = (0,-)(0,R)(1,R)(0,-)$
					\item $f_{\ast}((0,-), C1Z) = (1,Z)$
					\item $f_{\ast\ast}((0,-), RZO) = (0,-)(0,R)(0,Z)(0,Z)$
					\item $f_{\ast\ast}((1,Z), RZO) = (1,Z)(1,R)(1,Z)(0,-)$
					\item $f_{\ast}((0,-), RZ1R1C1ZO)= (0,-)$
				\end{enumerate}
	\end{block}
\end{frame}

\begin{frame}{Mealy-Automat}
	\begin{block}{Def.: Verallgemeinerte Ausgabefunktion $g_{\ast}$}
		$g_{\ast} : Z \times X^{\ast} \rightarrow Z$ gibt die letzte Ausgabe aus, die durch die Eingabe eines Wortes produziert wird, und ist definiert durch:
		\begin{align*}
			g_{\ast}(z,\varepsilon) &= \varepsilon \\
			\forall w \in X^{\ast} : \forall x \in X : g_{\ast}(z,wx) &= g(f_{\ast}(z,w),x)\\
		\end{align*}
	\end{block}

	\begin{block}{Def.: Verallgemeinerte Ausgabefunktion $g_{\ast\ast}$}
		$g_{\ast\ast} : Z \times X^{\ast} \rightarrow Z^{\ast}$ gibt \textbf{alle} Ausgaben konkateniert aus, die bei der Eingabe eines Wortes erzeugt werden, und ist definiert durch:
		\begin{align*}
			g_{\ast\ast}(z,\varepsilon) = \varepsilon\\
			\forall w \in X^{\ast} : \forall x \in X : g_{\ast\ast}(z,wx) &= g_{\ast\ast}(z,w) \cdot g_{\ast}(z,wx)\\
		\end{align*}
	\end{block}
\end{frame}

\begin{frame}[fragile]{Mealy-Automat}

	\begin{exampleblock}{Aufgabe zu $g_{\ast}$ und $g_{\ast\ast}$}
		\begin{columns}
			\begin{column}{0.6\textwidth}
				\begin{figure}[ht]
  \centering
  \tiny
  \begin{tikzpicture}[->,>=stealth]
    \matrix[matrix of math nodes,column sep=20mm,row sep=20mm,nodes={circle,draw,inner sep=1pt}]   {
      |(0-)| (0,-) & |(0R)| (0,R) & |(0Z)| (0,Z) \\
      |(1-)| (1,-) & |(1R)| (1,R) & |(1Z)| (1,Z) \\
    };

    \coordinate[left of=0-] (start);

    \draw (start) -- node[auto] {} (0-);

    % Schleifen
    \draw (0-) edge[loop above]  node[pos=0.5] {$\ioeps{O}$,$\ioeps{C}$} ();
    \draw (0R) edge[loop above] node[pos=0.9,anchor=west] {$\ioeps{R}$,$\ioeps{O}$} ();
    \draw (0Z) edge[loop above] node[pos=0.5] {$\ioeps{Z}$,$\ioeps{O}$} ();
    \draw (1-) edge[loop below]  node[pos=0.5] {$\io{1}{1}$,$\ioeps{O}$} ();
    \draw (1R) edge[loop below] node[pos=0.9,anchor=east] {$\io{1}{1}$,$\ioeps{R}$} ();
    \draw (1Z) edge[loop below] node[pos=0.5] {$\io{1}{1}$,$\ioeps{Z}$} ();

    % andere Kanten
    \draw (0-) -- node[right,pos=0.2] {$\ioeps{1}$} (1-);
    \draw (0R) -- node[right,pos=0.2] {$\ioeps{1}$} (1R);
    \draw (0Z) -- node[right,pos=0.2] {$\ioeps{1}$} (1Z);

    \draw (1-) to[bend left=10] node[left,pos=0.2] {$\io{C}{1}$} (0-);

    \draw (0-) to[bend right=10] node[below] {$\ioeps{R}$} (0R);
    \draw (0R) to[bend right=10] node[above,pos=0.1] {$\ioeps{C}$} (0-);
    \draw (0R) to[bend right=10] node[below] {$\ioeps{Z}$} (0Z);
    \draw (0Z) to[bend right=10] node[above] {$\ioeps{R}$} (0R);
    \draw (0-) to[bend left=32]  node[below,pos=0.2] {$\ioeps{Z}$} (0Z);
    \draw (0Z) to[bend right=40] node[above] {$\ioeps{C}$} (0-);

    \draw (1-) to[bend right=10] node[above] {$\ioeps{R}$} (1R);
    \draw (1R) -- node[below,pos=0.4,anchor=north east] {$\io{O}{R}$,$\io{C}{1}$} (0-); %!!
    \draw (1R) to[bend right=10] node[below] {$\ioeps{Z}$} (1Z);
    \draw (1Z) to[bend right=10] node[above] {$\ioeps{R}$} (1R);
    \draw (1-) to[bend left=-32] node[below,pos=0.2] {$\ioeps{Z}$} (1Z);
    \draw (1Z) -- node[above,pos=0.3,anchor=south west] {$\io{O}{Z}$,$\io{C}{1}$} (0-); %!!
  \end{tikzpicture}
\end{figure}
			\end{column}
			\begin{column}{0.4\textwidth}
			\small
				Gib an:
				\begin{enumerate}
					\item $g_{\ast}((0,-), R10)$
					\item $g_{\ast\ast}((0,-), R10)$
					\item $g_{\ast\ast}((0,-), R11O)$
					\item $g_{\ast}((0,-), C1Z)$
					\item $g_{\ast\ast}((0,-), RZO)$
					\item $g_{\ast\ast}((1,Z), RZO)$
					\item $g_{\ast\ast}((0,-), RZ1R1C1ZO)$
				\end{enumerate}
			\end{column}
		\end{columns}
	\end{exampleblock}
\end{frame}

\begin{frame}{Mealy-Automat}
	\begin{block}{Lösung zur Aufgabe zu $g_{\ast}$ und $g_{\ast\ast}$}
		\begin{enumerate}
					\item $g_{\ast}((0,-), R1O) = R$
					\item $g_{\ast\ast}((0,-), R1O) = R$
					\item $g_{\ast\ast}((0,-), R11O) = 1R$
					\item $g_{\ast}((0,-), C1Z) = \varepsilon$
					\item $g_{\ast\ast}((0,-), RZO) = \varepsilon$
					\item $g_{\ast\ast}((1,Z), RZO) = Z $
					\item $g_{\ast\ast}((0,-), RZ1R1C1ZO)= 11Z$
				\end{enumerate}
	\end{block}
\end{frame}



\subsection{Moore-Automat}
\begin{frame}{Moore-Automat}
	\begin{block}{Def.: Moore-Automat}
		Ein (endlicher) \textbf{Moore-Automat} $A=(Z, z_0, X, f, Y, h)$ ist bestimmt durch:
		\begin{itemize}
			\item endliche Zustandsmenge $Z$,
			\item Anfangszustand $z_0 \in Z$,
			\item Eingabealphabet $X$,
			\item Zustandsüberführungsfunktion $f: Z \times X \rightarrow Z$,
			\item Ausgabealphabet $Y$,
			\item Ausgabefunktion $h: Z \rightarrow Y^{\ast}$
		\end{itemize}

	Der Unterschied zum Mealy Automaten ist also , dass die Ausgabe nur vom Zustand abhängt, nicht von der Eingabe.

	\end{block}
\end{frame}

\begin{frame}[fragile]{Moore-Automat}
	\begin{exampleblock}{Beispiel}
		\begin{figure}[ht]
  \centering
  \begin{tikzpicture}[shorten >=1pt,node distance=2cm,auto,initial text=,->,>=stealth]
   \node[state,initial]  (q_0)                       {$q_{\varepsilon}\!\mid\! 0$};
   \node[state]          (q_1) [above right of= q_0] {$q_{a}\!\mid\! 0$};
    \node[state]          (q_2) [below right of= q_0] {$q_{b}\!\mid\! 0$};
   \node[state](q_3) [below right of=q_1] {$q_f\!\mid\! 1$};
   \node[state](q_4) [right of=q_3] {$q_r\!\mid\! 0$};
    \path[->] (q_0) edge              node        {$a$} (q_1)
                    edge              node [swap] {$b$} (q_2)
              (q_1) edge              node        {$b$} (q_3)
                    edge [loop above] node        {$a$} ()
              (q_2) edge              node [swap] {$a$} (q_3)
                    edge [loop below] node        {$b$} ()
              (q_3) edge              node        {$a,b$} (q_4)
              (q_4) edge [loop above] node        {$a,b$} ();
  \end{tikzpicture}
\end{figure}
	\end{exampleblock}
\end{frame}

\begin{frame}{Moore-Automat}
	\begin{block}{Def.: Verallgemeinerte Zustandsübergangsfunktionen $f_{\ast}$ und $f_{\ast\ast}$}
		Wie bei Mealy-Automaten.
	\end{block}

	\begin{block}{Def.: Verallgemeinerte Ausgabenfunktion $g_{\ast} = h \circ f_{\ast}$}
		$g_{\ast} : Z \times X^{\ast} \rightarrow Y$ gibt die letzte Ausgabe aus und ist definiert durch:
		\[
			\forall(z,w) \in Z \times X^{\ast} : g_{\ast}(z,w) = h(f_{\ast}(z,w))
		\]
	\end{block}

	\begin{block}{Def.: Verallgemeinerte Ausgabenfunktion $g_{\ast\ast} = h^{\ast\ast} \circ f_{\ast\ast}$}
		$g_{\ast} : Z \times X^{\ast} \rightarrow Y$ gibt alle Ausgaben konkateniert aus und ist definiert durch:
		\[
			\forall(z,w) \in Z \times X^{\ast} : g_{\ast\ast}(z,w) = h^{\ast\ast}(f_{\ast}(z,w))
		\]
		mit $h^{\ast\ast} :$\textit{induzierter Homomorphismus von h}.
	\end{block}
\end{frame}

\begin{frame}[fragile]{Mealy-Automat}

	\begin{exampleblock}{Aufgabe zu $f_{\ast}$, $f_{\ast\ast}$, $g_{\ast}$ und $g_{\ast\ast}$}
		\begin{columns}
			\begin{column}{0.6\textwidth}
				\begin{figure}[ht]
  \centering
  \small
  \begin{tikzpicture}[shorten >=1pt,node distance=2cm,auto,initial text=,->,>=stealth]
   \node[state,initial]  (q_0)                       {$q_{\varepsilon}\!\mid\! 0$};
   \node[state]          (q_1) [above right of= q_0] {$q_{a}\!\mid\! 0$};
    \node[state]          (q_2) [below right of= q_0] {$q_{b}\!\mid\! 0$};
   \node[state](q_3) [below right of=q_1] {$q_f\!\mid\! 1$};
   \node[state](q_4) [right of=q_3] {$q_r\!\mid\! 0$};
    \path[->] (q_0) edge              node        {$a$} (q_1)
                    edge              node [swap] {$b$} (q_2)
              (q_1) edge              node        {$b$} (q_3)
                    edge [loop above] node        {$a$} ()
              (q_2) edge              node [swap] {$a$} (q_3)
                    edge [loop below] node        {$b$} ()
              (q_3) edge              node        {$a,b$} (q_4)
              (q_4) edge [loop above] node        {$a,b$} ();
  \end{tikzpicture}
\end{figure}
			\end{column}
			\begin{column}{0.4\textwidth}
			\small
				Gib an:
				\begin{enumerate}
					\item $f_{\ast}(q_{\varepsilon}, aab)$
					\item $f_{\ast\ast}(q_{\varepsilon}, aab)$

					\item $g_{\ast}(q_{\varepsilon}, aab)$
					\item $g_{\ast\ast}(q_{\varepsilon}, aab)$
					\item $g_{\ast}(q_{\varepsilon}, abab)$
					\item $g_{\ast\ast}(q_{f}, abab)$
				\end{enumerate}
			\end{column}
		\end{columns}
	\end{exampleblock}
\end{frame}

\begin{frame}{Mealy-Automat}
	\begin{block}{Lösung zur Aufgabe zu $f_{\ast}$, $f_{\ast\ast}$, $g_{\ast}$ und $g_{\ast\ast}$}
		\begin{enumerate}
					\item $f_{\ast}(q_{\varepsilon}, aab) = q_{f}$
					\item $f_{\ast\ast}(q_{\varepsilon}, aab) = q_{\varepsilon}q_{a}q_{f}$

					\item $g_{\ast}(q_{\varepsilon}, aab) = 1$
					\item $g_{\ast\ast}(q_{\varepsilon}, aab) = 001$
					\item $g_{\ast}(q_{\varepsilon}, abab) = 0$
					\item $g_{\ast\ast}(q_{f}, abba) 1000$
			\end{enumerate}
	\end{block}
\end{frame}

\subsection{Akzeptoren}
\begin{frame}{Endlicher Akzeptor}
	\begin{block}{Def.: Endlicher Akzeptor}
		Ein \textbf{endlicher Akzeptor} ist eine Spezialform des Moore-Automaten mit immer genau einem Bit Ausgabe, d.h. $Y = \set{0,1}$ (und $\forall z : h(z) \in Y$).

		Man definiert deshalb eine Menge $F = \setc{z}{h(z)=1}$ der \textbf{akzeptierenden Zustände}.

		Ein endlicher Akzeptor ist also definiert durch:

		\begin{itemize}
			\item endliche Zustandsmenge $Z$,
			\item Anfangszustand $z_0 \in Z$,
			\item Eingabealphabet $X$,
			\item Zustandsüberführungsfunktion $f: Z \times X \rightarrow Z$,
			\item eine Menge akzeptierender Zustände $F \subseteq Z$
		\end{itemize}
		Die von einem endlichen Akzeptor $A$ \textbf{akzeptierte Sprache} ist:
		\[
			L(A) = \setc{w \in X^{\ast}}{f_{\ast}(z_{0},w) \in F}
		\]
	\end{block}
\end{frame}

\begin{frame}[fragile]{Endlicher Akzeptor}
	\begin{exampleblock}{Beispiel}
	\begin{columns}
		\begin{column}{0.6\textwidth}
		\begin{figure}[ht]
  \centering
  \begin{tikzpicture}[shorten >=1pt,node distance=2cm,auto,initial text=,->,>=stealth]
    \node[state,initial]  (q_0)                       {$q_{\varepsilon}$};
    \node[state]          (q_1) [above right of= q_0] {$q_{a}$};
    \node[state]          (q_2) [below right of= q_0] {$q_{b}$};
    \node[state,accepting](q_3) [below right of=q_1] {$q_f$};
    \node[state](q_4) [right of=q_3] {$q_r$};
    \path[->] (q_0) edge              node        {$a$} (q_1)
                    edge              node [swap] {$b$} (q_2)
              (q_1) edge              node        {$b$} (q_3)
                    edge [loop above] node        {$a$} ()
              (q_2) edge              node [swap] {$a$} (q_3)
                    edge [loop below] node        {$b$} ()
              (q_3) edge              node        {$a,b$} (q_4)
              (q_4) edge [loop above] node        {$a,b$} ();
  \end{tikzpicture}
\end{figure}		
		\end{column}
		\begin{column}{0.375\textwidth}
		\centering
		(Akzeptierende Zustände mit Doppelkringel)
		\begin{figure}[ht]
			\begin{tikzpicture}[shorten >=1pt,node distance=2cm,auto,initial text=,->,>=stealth]
				\node[state,accepting] (z) {$z$};
			\end{tikzpicture}
		\end{figure}
		\end{column}
		
	\end{columns}
	\end{exampleblock}
\end{frame}

\begin{frame}{Endlicher Akzeptor}
	\begin{exampleblock}{Aufgabe zu endlichen Akzeptoren}
		\begin{enumerate}
			\small
			\item Zeichne einen Akzeptor mit $X=\set{a,b}$, der alle Wörter akzeptiert, bei denen die Anzahl der \emph{a} durch 5 teilbar ist.
			\item Zeichne einen Akzeptor mit $X=\set{a,b}$, der alle Wörter akzeptiert, in denen nirgends hintereinander zwei \emph{b} vorkommen.
			\item Zeichne einen Akzeptor mit $X=\set{a,b}$, der alle Wörter akzeptiert, in denen irgendwo das Teilwort \emph{abab} vorkommt.
			\item Zeichne einen Akzeptor mit $X=\set{a,b}$, der alle Wörter akzeptiert, in denen nirgends das Teilwort \emph{abab} vorkommt.
			\item Welche Sprache wird vom folgenden Akzeptor $A$ erkannt?
		\end{enumerate}

		\begin{figure}[ht]
  			\centering
  			\begin{tikzpicture}[shorten >=1pt,node distance=2cm,auto,initial text=,->,>=stealth]
   			 	\node[state,initial]  (s_0)                       {$s_{0}$};
  			 	\node[state,accepting](s_1) [right of= s_0] {$s_{1}$};
  			 	\node[state]          (s_2) [right of= s_1] {$s_{2}$};
    			\node[state,accepting](s_3) [right of= s_2] {$s_3$};
    			\draw[->] 	(s_0) 	edge              	node        {$a,b$} (s_1);
              	\draw[->]	(s_1) 	to [bend left=20] 	node        {$a,b$} (s_2);
              	\draw[->]	(s_2) 	to [bend left=20] 	node 		{$b$} 	(s_3);
              	\draw[->]	(s_2)	to [bend left=20] 	node 		{$a$} 	(s_1);
              	\draw[->]	(s_3)	to [bend left=20] 	node 		{$a,b$} (s_2);
  \end{tikzpicture}
\end{figure}	
	\end{exampleblock}
\end{frame}

\begin{frame}{Endlicher Akzeptor}

	\begin{exampleblock}{Lösung zur Aufgabe zu endlichen Akzeptoren}
		\begin{enumerate}
			\item siehe Tafel
		\item siehe Tafel
		\item siehe Tafel
		\item siehe Tafel
		\item $L(A)= \setc{w \in \set{a,b}^{\ast}}{\,\setsize{w} mod\, 2 = 1} = \setC{w \in \set{a,b}^{\ast}}{$w$ hat ungerade Länge}$
		\end{enumerate}		
	\end{exampleblock}
\end{frame}

\section{Reguläre Ausdrücke}
\subsection{Reguläre Ausdrücke}

\begin{frame}{Reguläre Ausdrücke}
\begin{block}{Def.: Regulärer Ausdruck}
	Es sei $A$ ein Alphabet, das keines der Symbole aus $Z := \set{\mid, (, ), *, \emptyset}$ enthält.
	Ein \textbf{regulärer Ausdruck} über $A$ ist eine Zeichenfolge über $A \cup Z$, das gewissen Vorschriften genügt.\\
	Die \textbf{Menge der regulären Ausdrücke} ist wie folgt festgelegt:
	\begin{itemize}
		\item $\emptyset$ ist ein regulärer Ausdruck
		\item Für jedes $x \in A$ ist $x$ ein regulärer Ausdruck
		\item Sind $R_1$ und $R_2$ reguläre Ausdrücke, dann auch $(R_1 | R_2 )$ und $(R_1R_2)$
		\item Ist $R$ ein regulärer Ausdruck, dann auch $(R*)$
		\item Nichts anderes sind reguläre Ausdrücke
	\end{itemize}
	Die durch $R$ beschriebene formale Sprache ist $\lang{R}$.
\end{block}
\end{frame}

\begin{frame}{Reguläre Ausdrücke}
\begin{exampleblock}{Aufgabe}
	Gib einen regulären Ausdruck $R_i$ an, für den gilt:
	\begin{enumerate}
		\item $\lang{R_\theenumi}$ enthält genau die Wörter, in denen das Teilwort $\texttt{baa}$ vorkommt,
		\item $\lang{R_\theenumi}$ enthält genau die Wörter, in denen das Teilwort $\texttt{baa}$ nicht vorkommt,
		\item $\lang{R_\theenumi}$ enthält genau die Wörter, in denen das Teilwort $\texttt{baa}$ genau zweimal vorkommt,
		\item $\lang{R_\theenumi}$ enthält genau die Wörter, in denen mindestens drei $\texttt{b}$s vorkommen,
		\item \textbf{Wichtig!} $\lang{R_\theenumi} = \set{\varepsilon}$,
	\end{enumerate}
	mit $i \in \set{1,2,3,4,5}$ und $A := \set{\texttt{a}, \texttt{b}}$.
\end{exampleblock}
\end{frame}

\begin{frame}{Reguläre Ausdrücke}
\newcommand{\any}{(\texttt{a} | \texttt{b}) *}
\begin{block}{Lösung}
	\begin{enumerate}
		\item $R_\theenumi := \any \texttt{baa} \any $
		\item Die Aussage bedeutet umformuliert, dass nach jedem $\texttt{b}$ höchstens ein $\texttt{a}$ kommt, also $R_\theenumi := \texttt{a} * (\texttt{b} | \texttt{ba} ) * $
		\item In jedem Wort muss genau zweimal $\texttt{baa}$ vorkommen. Davor, dazwischen und danach dürfen Teilworte stehen, in denen das Teilwort $\texttt{baa}$ nicht vorkommt (wie in 2.). Also $R_\theenumi := R_2 \texttt{baa} R_2 \texttt{baa} R_2$.
		\item z.B. $R_\theenumi := \any \texttt{b} \any \texttt{b}\any \texttt{b} \any $
		oder $R_\theenumi := \texttt{a} * \texttt{ba} * \texttt{ba} * \texttt{b} \any$
		\item Für $R_\theenumi := \emptyset *$ ist $\lang{R_\theenumi} = \lang{\emptyset *} = \lang{\emptyset}^* = \lang{}^* = \set{}^* = \set{\varepsilon}$
	\end{enumerate}
\end{block}
\end{frame}

\begin{frame}{Reguläre Ausdrücke}
\begin{exampleblock}{Aufgabe}
	Wenn $R$ ein regulärer Ausdruck für die Sprache $L$ ist, wie sieht dann ein regulärer Ausdruck für
	\begin{enumerate}
		\item $L^*$,
		\item $L^+$
	\end{enumerate}
	aus?
\end{exampleblock}
\pause
\begin{block}{Lösung}
	\begin{enumerate}
		\item $(R*)$
		\item $R(R*)$
	\end{enumerate}
\end{block}
\end{frame}

\Moritz{\begin{frame}{Reguläre Ausdrücke}
	Zum Ableiten der Sprache $\lang{R}$ eines regulären Ausdruckes $R$ kann man folgende Regeln anwenden:
	\begin{itemize}
		\item $\lang{\emptyset} = \set{}$
		\item Für $x \in A$ gilt $\lang{x} = \set{x}$
		\item $\lang{R_1,R_2} = \lang{R_1} \cup \lang{R_2}$
		\item $\lang{R_{1}R_{2}} = \lang{R_1} \cdot \lang{R_2}$
		\item $\lang{R *} = \lang{R}^{\ast}$
	\end{itemize}
\end{frame}}
\subsection{Rechtslineare Grammatiken}

\begin{frame}{Rechtslineare Grammatiken}
\begin{block}{Def.: Kontextfreie Grammatik}
	Eine \textbf{kontextfreie Grammatik} $G$ ist ein Tupel $G = (N,T,S,P)$ mit
	\begin{itemize}
		\item $N$ ist das Alphabet der Nonterminalsymbole, auch Nichtterminalsymbole oder Variablen
		\item $T$ ist das Alphabet der Terminalsymbole, auch Zeichen, mit $N \cap T = \emptyset$
		\item $S$ ist das Startsymbol mit $S \in N$, also die Variable, mit der man beginnt
		\item $P$ ist die Menge der Produktionen mit
			\begin{itemize}
				\item $P \subseteq N \times (N \cup T)^*$, d.h. alle Produktionen besitzen folgende Form:\\
				$V \to w \text{ mit } V \in N \text{ und } w \in (N \cup T)^*$
			\end{itemize}
	\end{itemize}
\end{block}
\end{frame}

\begin{frame}{Rechtslineare Grammatiken}
\begin{block}{Def.: Rechtslineare Grammatik}
	Eine \textbf{rechtlineare Grammatik} $G$ ist eine kontextfreie Grammatik $G = (N,T,S,P)$ mit folgenden Einschränkungen für die Produktionen $P$:\\
	\begin{itemize}
			\item $P$ ist entweder von der Form \[
		V \to \omega \text{ mit } \omega \in T^*
	\]
	\item oder \[
		V \to \omega W \text{ mit } \omega \in T^* \text{ und } V,W \in N
	\]
	\item Also: Auf der rechten Seite einer Produktion darf höchstens ein Nonterminalsymbol vorkommen, und wenn dann nur als letztes Symbol.
	\end{itemize}
\end{block}
\pause
\begin{exampleblock}{Beispiel}
	\[
		G = (\set{X,Y}, \set{a,b}, X, \set{X \to aX | bY, Y \to bY | \varepsilon})
	\]
\end{exampleblock}
\end{frame}

\subsection{Reguläre Sprache}
\begin{frame}{Reguläre Sprache}
    \begin{block}{Def.: Reguläre Sprache}
    	Für eine formale Sprache $L$ sind folgende drei Aussagen äquivalent:
    	\begin{itemize}
    		\item $L$ kann von einem endlichen Akzeptor erkannt werden
    		\item $L$ kann durch einen regulären Ausdruck beschrieben werden
    		\item $L$ kann von einer rechtslinearen Grammatik erzeugt werden
    	\end{itemize}
    	Eine solche Sprache $L$ heißt dann auch \textbf{reguläre Sprache} (oder Typ-3-Sprache) 
    \end{block}

    \begin{exampleblock}{Aufgabe}
    	Sei $G=(\set{X,Y,Z},\set{a,b},X,P)$ mit $P = \set{
    		X \to aX | bY | \varepsilon,
    		Y \to aX | bZ | \varepsilon, 
    		Z \to aZ | bZ
    	}$.\\ 
    	Zeichne den Akzeptor, der $L(G)$ akzeptiert.\\
    	Wie kann man die Grammatik vereinfachen?
    \end{exampleblock}
\end{frame}

\begin{frame}{Reguläre Sprache}
    \begin{block}{Lösung}
    	\begin{itemize}
    		\item Akzeptor, der $L(G)$ akzeptiert:\\
    		\begin{figure}[ht]
  			\centering
    			\begin{tikzpicture}[shorten >=1pt,node distance=2cm,auto,initial text=,>=stealth]
      				\node[state,initial,accepting]  (q_0)                       {$X$};
	      			\node[state,accepting]          (q_1) [right of= q_0] {$Y$};
	      			\node[state]                    (q_2) [right of= q_1] {$Z$};
	     			\path[->] (q_0) edge [loop below]      node        {$a$} ()
	     			edge [bend right] node [swap] {$b$} (q_1)
	      			(q_1) edge              node        {$b$} (q_2)
	      			edge [bend right] node [swap] {$a$} (q_0)
	      			(q_2) edge [loop below] node        {$a,b$} ()
	      			;
    			\end{tikzpicture}
    		\end{figure}
    		\item Einfacher: $G=(\set{X,Y},\set{a,b},X,P)$ mit $P=\set{X \to a X \mid b Y \mid \varepsilon,\;\; Y \to a X \mid \varepsilon}$
    		\item Oder noch einfacher:
    		$G=(\set{X},\set{a,b},X,P)$ mit $P=\set{X \to a X\mid baX \mid b \mid \varepsilon}$
    	\end{itemize}
    \end{block}
\end{frame}

\subsection{Regex-Bäume}
\begin{frame}{Regex-Bäume}
    \begin{block}{Def.: Regex-Baum}
    \small
    	Es sei $A$ ein beliebiges Alphabet. Dann heißt ein Baum \textbf{Regex-Baum}, falls:
    	\begin{itemize}
    		\item Entweder ist es ein Baum, dessen Wurzel zugleich Blatt ist und mit einem $x\in A$ oder $\emptyset$ beschriftet ist,
    		\item oder es ist ein Baum, dessen Wurzel mit $*$ beschriftet ist und genau einen Nachfolgeknoten hat, der Wurzel eines Regex-Baumes ist
    		\item oder es ist ein Baum, dessen Wurzel mit $\cdot$ oder mit $|$ beschriftet ist und genau zwei Nachfolgeknoten hat, die Wurzeln zweier Regex-Bäume sind. 
    	\end{itemize}
    \end{block}

    % \only<1|handout:1>{
    % \begin{exampleblock}{Beispiel}
    % 	Zeichne den Kantorowitschbaum zum arithmetischen Ausdruck
    % 	\[
    % 		3+(a+b)*(-c)
    % 	\]
    % \end{exampleblock}
    % }

   %  \begin{center}
   %  \begin{tikzpicture}
   %    [level 1/.style={sibling distance=30mm},
   %    level 2/.style={sibling distance=20mm},
   %    level 3/.style={sibling distance=15mm},
   %    nodes={draw,circle,inner sep=0pt, minimum size=7mm},
   %    ->,>=stealth]
      
   %    \node {$+$}
   %    child { node {$3$}  edge from parent  }
   %    child { node {$*$}  edge from parent
   %      % [level 2/.style={sibling distance=15mm}]
   %      child { node {$+$}  edge from parent
   %        % [level 2/.style={sibling distance=15mm}]
   %        child { node {$a$}  edge from parent
   %          % [level 2/.style={sibling distance=15mm}]
   %        }
   %        child { node {$b$}  edge from parent
   %          % [level 2/.style={sibling distance=15mm}]
   %        }
   %      }
   %      child {node {$-$} edge from parent
   %        child { node {$c$}  edge from parent}
   %      }
   %    }
   %    ;
   %  \end{tikzpicture}
  	% \end{center}


    \only<1|handout:1>{
    	\begin{exampleblock}{Aufgabe}
    		Zeichne den Regex-Baum zu folgendem regulären Ausdruck:
    		\[
    			R = ( ( b | \emptyset * ) a ) (b * )
    		\]
    	\end{exampleblock}
    }

    
\end{frame}

\begin{frame}{Regex-Bäume}
	\begin{block}{Lösung zu $R = ( ( b | \emptyset * ) a ) (b * )$}
		\begin{figure}[ht]
  			\centering
  			\begin{tikzpicture}
    			[level 1/.style={sibling distance=30mm},
    			level 2/.style={sibling distance=20mm},
    			level 3/.style={sibling distance=15mm},
    			nodes={draw,circle,inner sep=0pt, minimum size=7mm},
    			->,>=stealth]
			    
    			\node {$\cdot$}
    			child { node {$\cdot$}  edge from parent
      			% [level 2/.style={sibling distance=15mm}]
      			child { node {$|$}  edge from parent
        			% [level 2/.style={sibling distance=15mm}]
        			child { node {$b$}  edge from parent
          			% [level 2/.style={sibling distance=15mm}]
        			}
        			child { node {$\emptyset$}  edge from parent
          			% [level 2/.style={sibling distance=15mm}]
        			}
      			}
      			child {node {$a$} edge from parent}
    			}
    			child { node {$*$}  edge from parent
      			% [level 2/.style={sibling distance=15mm}]
      			child { node {$b$}  edge from parent
        			% [level 2/.style={sibling distance=15mm}]
      			}
    			}
    			;
			\end{tikzpicture}
  		\end{figure}
	\end{block}
\end{frame}
%%%%%%%%%% %%%%%%%%%%
%% Zusammenfassung
\section{}
%\subsection{Zusammenfassung}
	\begin{frame}{Was ihr jetzt kennen und können solltet\dots}
			\begin{itemize}
				\item Mealy- und Moore Automaten sowie Endliche Akzeptoren erkennen und zeichnen
				\item Reguläre Ausdrücke bilden
				\item Rechtslineare Grammatiken erkennen und aufstellen
				\item Den Zusammenhang zwischen Endlichen Automaten, Regulären Ausdrücken und Rechtslinearen Grammatiken kennen
			\end{itemize}
	\end{frame}
%% Ausblick
%\subsection{Ausblick}
	\begin{frame}{Ausblick}
		\begin{itemize}
			\item Turing Maschinen
			\item Noch mehr Relationen
		\end{itemize}
	\end{frame}
%%%%%%%%%% %%%%%%%%%%
\section{}
\questionframe
\lastframe
\mode<handout>{\slideThanks}
\end{document}