% ===== handout mode =====
% Comment/uncomment this line to toggle handout mode
% \newcommand{\handout}{}

% Comment/uncoment this line to toogle Mortitz mode
% \newcommand{\Moritz}{}

% Comment/uncomment this line to toggle handout mode
% \newcommand{\handout}{}

% by Stephan

%% Moritz mode or Stephan mode
\ifdefined \Moritz

% This is a configuration file with private, tutor specific information.
% It is therefore excluded from the Git repository so changes in this file will not conflict in git commits.

% Copy this Template, rename to config.tex and add your information below.

\newcommand{\mymail}{moritz.laupichler@student.kit.edu} % Consider using your named student Mail address to keep your u-Account private.

\newcommand{\myname}{\href{mailto:\mymail}{Moritz Laupichler}}

\newcommand{\mytutnumber}{25}

\newcommand{\mytutinfos}{Dienstags, 5. Block (15:45-17:15 Uhr), SR -120}

\newcommand{\aboutMeFrame}{
	\begin{frame}{Euer Tutor}
		Name: \myname \\
		Alter: 21 Jahre \\
		Studiengang: Master Informatik, 1. Semester \\
		\vspace{1cm}
		\pause 
		\centering{Kontakt: \href{mailto:\mymail}{\mymail}}
	\end{frame}
} % Moritz mode
\else
\ifdefined \Alex

% This is a configuration file with private, tutor specific information.
% It is therefore excluded from the Git repository so changes in this file will not conflict in git commits.

% Copy this Template, rename to config.tex and add your information below.

\newcommand{\mymail}{alexander.klug@student.kit.edu} % Consider using your named student Mail address to keep your u-Account private.

\newcommand{\myname}{\href{mailto:\mymail}{Alexander Klug}}

\newcommand{\mytutnumber}{30}

\newcommand{\mytutinfos}{Mittwochs, 3. Block (11:30-13:00), SR -107}

\newcommand{\aboutMeFrame}{
	\begin{frame}{Euer Tutor}
		Name: \myname \\
		Alter: 19 Jahre \\
		Studiengang: Bachelor Informatik, 3. Semester \\
		\vspace{1cm}
		\pause 
		\centering{Kontakt: \href{mailto:\mymail}{\mymail}}
	\end{frame}
}

% Toggle Handout mode by including the following line before including style_tut
% and removing the % at the start (but do NOT remove it here, otherwise handout mode will always be on!)
% Please keep handout mode on in all commits!

% \newcommand{\handout}{} % Alex Mode
\else

% This is a configuration file with private, tutor specific information.
% It is therefore excluded from the Git repository so changes in this file will not conflict in git commits.

% Copy this Template, rename to config.tex and add your information below.

\newcommand{\mymail}{stephan.bohr@student.kit.edu} % Consider using your named student Mail address to keep your u-Account private.

\newcommand{\myname}{\href{mailto:\mymail}{Stephan Bohr}}

\newcommand{\mytutnumber}{25}

\newcommand{\mytutinfos}{Dienstags, 5. Block (15:45-17:15), SR -119}

\newcommand{\aboutMeFrame}{
	\begin{frame}{Euer Tutor}
		Name: \myname \\
		Alter: 20 Jahre \\
		Studiengang: Bachelor Informatik, 3. Semester \\
		\vspace{1cm}
		\pause 
		\centering{Kontakt: \href{mailto:\mymail}{\mymail}}
	\end{frame}
} % Stephan mode
\fi
\fi

%% Beamer-Klasse im korrekten Modus
\ifdefined \handout
\documentclass[handout]{beamer} % Handout mode
\else
\documentclass{beamer}
\fi
%\documentclass[18pt,parskip]{beamer}

%% SLIDE FORMAT

% use 'beamerthemekit' for standard 4:3 ratio
% for widescreen slides (16:9), use 'beamerthemekitwide'

\usepackage{../templates/KIT-slides/beamerthemekit}
%\usepackage{../templates/KIT-slides/beamerthemekitwide}

%% TITLE PICTURE

% if a custom picture is to be used on the title page, copy it into the 'logos'
% directory, in the line below, replace 'mypicture' with the 
% filename (without extension) and uncomment the following line
% (picture proportions: 63 : 20 for standard, 169 : 40 for wide
% *.eps format if you use latex+dvips+ps2pdf, 
% *.jpg/*.png/*.pdf if you use pdflatex)

\titleimage{../figures/titleimage/brain}

%% TITLE LOGO

% for a custom logo on the front page, copy your file into the 'logos'
% directory, insert the filename in the line below and uncomment it

%\titlelogo{mylogo}

% (*.eps format if you use latex+dvips+ps2pdf,
% *.jpg/*.png/*.pdf if you use pdflatex)

%% TikZ INTEGRATION

% use these packages for PCM symbols and UML classes
% \usepackage{templates/tikzkit}
% \usepackage{templates/tikzuml}

%\usepackage{tikz}
%\usetikzlibrary{matrix}
%\usetikzlibrary{arrows.meta}
%\usetikzlibrary{automata}
%\usetikzlibrary{tikzmark}

%%%%%%%%%%%%%%%%%%%%%%%%%
% Libertine font (Original GBI font)
\usepackage{libertine}
%\renewcommand*\familydefault{\sfdefault}  %% Only if the base font of the document is to be sans serif

%% Schönere Schriften
\usepackage[TS1,T1]{fontenc}

%% Deutsche Silbentrennung und Beschriftungen
\usepackage[ngerman]{babel}

%% UTF-8-Encoding
\usepackage[utf8]{inputenc}

%% Bibliotheken für viele mathematische Symbole
\usepackage{amsmath, amsfonts, amssymb}

%% Anzeigetiefe für Inhaltsverzeichnis: 1 Stufe
\setcounter{tocdepth}{1}

%% Hyperlinks
\usepackage{hyperref}
% I don't know why, but this works and only includes sections and NOT subsections in the pdf-bookmarks.
\hypersetup{bookmarksdepth=subsection}

%% remove navigation symbols
\setbeamertemplate{navigation symbols}{}

%% switch between "ngerman" and "english" for German/English style date and logos
\selectlanguage{ngerman}

%% for invisible pause texts instead of dimming
\setbeamercovered{invisible}

%%%%%%%%%%%% Shortcuts %%%%%%%%%%%%%
\newcommand{\nM}{\mathbb{M}}
\newcommand{\nR}{\mathbb{R}}
\newcommand{\nN}{\mathbb{N}}
\newcommand{\nZ}{\mathbb{Z}}
\newcommand{\nQ}{\mathbb{Q}}
\newcommand{\nB}{\mathbb{B}}
\newcommand{\nC}{\mathbb{C}}
\newcommand{\nK}{\mathbb{K}}
\newcommand{\nF}{\mathbb{F}}
\newcommand{\nG}{\mathbb{G}}
\newcommand{\nullel}{\mathcal{O}}
\newcommand{\einsel}{\mathds{1}}
\newcommand{\nP}{\mathbb{P}}
\newcommand{\Pot}{\mathcal{P}}
\renewcommand{\O}{\text{O}}

\newcommand{\set}[1]{\{ #1 \}}
\newcommand{\setc}[2]{\set{#1 \mid #2}}
\newcommand{\setC}[2]{\set{#1 \mid \text{ #2 }}}

\newcommand{\setsize}[1]{\; \mid #1 \mid \; }

\newcommand{\q}[1]{\textquotedblleft #1\textquotedblright}

%%%%%%%%%%%% INHALT %%%%%%%%%%%%%%%%

%% Wochennummer
%\newcounter{weeknum}

%% Titelinformationen
%\title[GBI Tutorium, Woche \theweeknum]{Grundbegriffe der Informatik \\ Tutorium \mytutnumber}
%\subtitle{Termin \theweeknum \ | \mydate \\ \myname}
\author[\myname]{\myname}
\institute{Fakultät für Informatik}
%\date{\mydate}

%% Titel einfügen
\newcommand{\titleframe}{\frame{\titlepage}\addtocounter{framenumber}{-1}}


%% Alles starten mit \starttut{X}
%\newcommand{\starttut}[1]{\setcounter{weeknum}{#1}\titleframe\frame{\frametitle{Inhalt}\tableofcontents} \AtBeginSection[]{%
%\begin{frame}
%	\tableofcontents[currentsection]
%\end{frame}\addtocounter{framenumber}{-1}}}


%\newcommand{\framePrevEpisode}{
%	\begin{frame}
%		\centering
%		\textbf{In the previous episode of GBI...}
%	\end{frame}
%}

%% Roadmap frame
%table of contents
\newcommand{\roadmap}{
	\frame{\frametitle{Roadmap}\tableofcontents}}

 \AtBeginSection[]{%
\begin{frame}
	\frametitle{Roadmap}
	\tableofcontents[currentsection]
\end{frame}%\addtocounter{framenumber}{-1}
}


%% ShowMessage frame
\newcommand{\showmessage}[1]{\frame{\frametitle{\phantom{1em}}\centering\textbf{#1}}}

%% Fragen
%% Lastframe
\newcommand{\questionframe}{\showmessage{Fragen?}}

%% Lastframe
\newcommand{\lastframe}{\showmessage{Vielen Dank für Eure Aufmerksamkeit! \\Bis nächste Woche :)}}

%% Thanks frame
\newcommand{\slideThanks}{
	\begin{frame}
		\frametitle{Credits}
		\begin{block}{}
			An der Erstellung des Foliensatzes haben mitgewirkt:\\[1em]
			\ifdefined \Moritz
			Stephan Bohr \\
			Alexander Klug \\
			\else
			\ifdefined \Alex
			Stephan Bohr \\
			Moritz Laupichler \\
			\else
			Moritz Laupichler \\
			Alexander Klug \\
			\fi
			\fi
			Katharina Wurz \\
			Thassilo Helmold \\
			Philipp Basler \\
			Nils Braun \\
			Dominik Doerner \\
			Ou Yue \\
		\end{block}
	\end{frame}
}

%% Verbatim
%\usepackage{moreverb}



\title[Short Title]{\#. Tutorium\\ Short Title}
\subtitle{Grundbegriffe der Informatik, Tutorium \#\mytutnumber}
\date{\today}


\def\mybox#1{\hbox{\vrule height 2ex width 0pt depth 0.6ex#1}}
\def\taste[#1]#2{%
  \tikz[x=8mm,y=8mm,baseline=(N.base)] \tasteinnerx[#1]{#2};%
}
\def\tasteinnerx[#1]#2{%
  \node[midway,inner sep=0mm,draw,rounded corners,anchor=base,minimum width=10mm,#1] (N) {\mybox{#2}}%
}
\def\tasteinner[#1]#2{%
  node[midway,inner sep=0mm,draw,rounded corners,minimum width=10mm,#1] (N) {\mybox{#2}}%
}
\def\tasteinnerOK{\tasteinner[fill=green!20]{OK}}
\def\tasteOK{\taste[fill=green!20]{OK}}
\def\tasteC{\taste[fill=red!20]{C}}
\def\tasteinnerC{\tasteinner[fill=red!20]{C}}
\def\tasteRein{\taste[fill=blue!10]{rein}}
\def\tasteinnerRein{\tasteinner[fill=blue!10]{rein}}
\def\tasteZitro{\taste[fill=yellow!10]{zitro}}
\def\tasteinnerZitro{\tasteinner[fill=yellow!10]{zitro}}

\begin{document}
\titleframe
\roadmap

%%%%%%%%%% %%%%%%%%%%
\section{Organisatorisches}
\begin{frame}{Organisatorisches}
	\begin{itemize}
		\item Für Übungsschein angemeldet?
		\item Für Klausur angemeldet?
		\item Nach Blatt 6 kommt noch ein Bonusblatt mit ausschließlich Bonuspunkten
	\end{itemize}
\end{frame}

\Moritz{
	\begin{frame}{Übungsblatt 5}
		Achten auf:
		\begin{itemize}
			\item Ableitungspfeil: $\Rightarrow$
			\item Bei Grammatiken auf genaue Notation achten: $G=(N,T,S,P)$: N, T und P sind \textbf{Mengen}.
		\end{itemize}
	\end{frame}
}

\section{Endliche Automaten}
\subsection{Mealy-Automat}
\begin{frame}{Mealy-Automat}
	\begin{block}{Def.: Mealy-Automat}
		Ein (endlicher) \textbf{Mealy-Automat} $A=(Z, z_0, X, f, Y, g)$ ist bestimmt durch:
		\begin{itemize}
			\item endliche Zustandsmenge $Z$,
			\item Anfangszustand $z_0 \in Z$,
			\item Eingabealphabet $X$,
			\item Zustandsüberführungsfunktion $f: Z \times X \rightarrow Z$,
			\item Ausgabealphabet $Y$,
			\item Ausgabefunktion $g: Z \times X \rightarrow Y^{\ast}$
		\end{itemize}
	\end{block}

\end{frame}

\begin{frame}{Mealy-Automat}

	\begin{exampleblock}{Beispiel} \small
		Betrachte einen Getränkeautomaten:
		Man kann nur 1-Euro"=Stücke einwerfen und vier Tasten drücken: Es gibt
		zwei Auswahltasten für Mineralwasser \tasteRein{} und Zitronensprudel
		\tasteZitro{}, eine Abbruch"=Taste \tasteC{} und eine \tasteOK-Taste.

	
	\begin{itemize}
	\item Jede Flasche Sprudel kostet 1 Euro.
	\item Es kann ein Guthaben von 1 Euro gespeichert werden. Wirft man
	  weitere Euro"=Stücke ein, werden sie sofort wieder ausgegeben.
	\item Wenn man mehrfach Auswahltasten drückt, wird der letzte Wunsch
	  gespeichert.
	\item Bei Drücken der Abbruch"=Taste wird alles bereits eingeworfenen
	  Geld wieder zurückgegeben und kein Getränkewunsch mehr gespeichert.
	\item Drücken der OK-Taste wird ignoriert, solange noch kein Euro
	  eingeworfen wurde oder keine Getränkesorte ausgewählt wurde.

	  Andernfalls wird das gewünschte Getränk ausgeworfen.
	\end{itemize} 
	\end{exampleblock}
\end{frame}

\begin{frame}{Mealy-Automat}

	\begin{exampleblock}{Beispiel}
		\begin{figure}[ht]
  \centering
  \begin{tikzpicture}[x=8mm,y=8mm]
    % Rahmen
    \draw (0,0) -- (8,0) -- (8,8) .. controls  (3,9) and (5,9) .. (0,8) -- cycle;
    % Geldschlitz 
    \draw (1.4,5.5) rectangle ++(0.2,1) ++(-0.1,0) node[anchor=south] {\vbox{\hbox{Geld-}\hbox{Einwurf}}};
    % Geldrückgabe 
    \draw (1,1) rectangle ++(1,1) ++(-0.5,0) node[anchor=south] {\vbox{\hbox{Geld-}\hbox{Rückgabe}}};
    % Warenauswurf
    \draw (4,1) rectangle ++(3.5,1) ++(-1.75,0) node[anchor=south] {Ware};
    
    % Tasten
    \draw (2.75,3) ++(3,3.5) node[anchor=south] {Sprudel};
    \draw (4,5.5) rectangle ++(1.5,1)\tasteinnerRein;
    \draw (6,5.5) rectangle ++(1.5,1) \tasteinnerZitro;
    \draw (4,4) rectangle ++(1.5,1) \tasteinnerOK;
    \draw (6,4) rectangle ++(1.5,1) \tasteinnerC;
  \end{tikzpicture}
  \caption{Ein primitiver Getr"ankeautomat}
  \label{fig:getraenkeautomat}
\end{figure}
	\end{exampleblock}
\end{frame}

\begin{frame}{Mealy-Automat}

	\begin{exampleblock}{Beispiel}
		\begin{figure}[ht]
  \centering
  \begin{tikzpicture}[->,>=stealth]
    \matrix[matrix of math nodes,column sep=25mm,row sep=25mm,nodes={circle,draw,inner sep=1pt}]   {
      |(0-)| (0,\#-) & |(0R)| (0,\#R) & |(0Z)| (0,\#Z) \\
      |(1-)| (1,\#-) & |(1R)| (1,\#R) & |(1Z)| (1,\#Z) \\
    };
    \draw (0-) -- node[right,pos=0.14] {\#{1}} (1-);
    \draw (0R) -- node[right,pos=0.14] {\#{1}} (1R);
    \draw (0Z) -- node[right,pos=0.14] {\#{1}} (1Z);

    \draw (0-) -- node[above,pos=0.2] {\#{R}} (0R);
    \draw (0-) to[bend left] node[above,pos=0.2] {\#{Z}} (0Z);
    \draw (0R) to[bend left=22] node[above,pos=0.2] {\#{Z}} (0Z);
    \draw (0Z) to[bend left=22] node[above,pos=0.2] {\#{R}} (0R);
    \draw (1-) -- node[above,pos=0.2] {\#{R}} (1R);
    \draw (1-) to[bend right] node[above,pos=0.2] {\#{Z}} (1Z);
    \draw (1R) to[bend right=22] node[below,pos=0.2] {\#{Z}} (1Z);
    \draw (1Z) to[bend right=22] node[above,pos=0.2] {\#{R}} (1R);

    \draw (1-) edge[loop below] node[right,pos=0.1] {\#1} ();
    \draw (1R) edge[loop below] node[right,pos=0.1] {\#1} ();
    \draw (1Z) edge[loop below] node[right,pos=0.1] {\#1} ();

    \draw (0R) edge[loop right] node[pos=0.5] {\#R} ();
    \draw (1R) edge[loop right] node[pos=0.5] {\#R} ();
    \draw (0Z) edge[loop right] node[above,pos=0.1] {\#Z} ();
    \draw (1Z) edge[loop right] node[above,pos=0.1] {\#Z} ();
  \end{tikzpicture}
  
  \caption{Graphische Darstellung der Zustandsübergänge des
    Getränkeautomaten für die drei Eingabesymbole \protect\#1, \protect\#R und \protect\#Z.}
  \label{fig:getraenke-zustaende}
\end{figure}
	\end{exampleblock}
\end{frame}

%%%%%%%%%% %%%%%%%%%%
%% Zusammenfassung
\section{}
%\subsection{Zusammenfassung}
	\begin{frame}{Was ihr jetzt kennen und können solltet\dots}
			\begin{itemize}
				\item Todo
			\end{itemize}
	
	\end{frame}
%% Ausblick
%\subsection{Ausblick}
	\begin{frame}{Ausblick}
		\begin{itemize}
			\item Todo
		\end{itemize}
	\end{frame}
%%%%%%%%%% %%%%%%%%%%
\section{}
\questionframe
\lastframe
\mode<handout>{\slideThanks}
\end{document}