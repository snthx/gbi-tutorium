\begin{frame}{Wdh.: Adjazenzmatrix}
	\begin{block}{Def.: Adjazenzmatrix}
		\begin{columns}
			\begin{column}{0.475\textwidth}
				Sei $G=(V_G,E_G)$ gerichteter Graph.\\
				\textbf{Adjazenzmatrix} $A \in \set{0,1}^{\mid V_G \mid \times \mid V_G \mid}$\\[12pt]
				\[
					A_{ij} = 
					\begin{cases}
						1, & \text{ falls } (i,j)\in E_G\\
						0, & \text{ falls } (i,j)\notin E_G
					\end{cases}
				\]
			\end{column}
			\begin{column}{0.475\textwidth}
				Sei $U=(V_U,E_U)$ ungerichteter Graph.\\
				\textbf{Adjazenzmatrix} $A \in \set{0,1}^{\mid V_U \mid \times \mid V_U \mid}$\\[12pt]
				\[
					A_{ij} = 
					\begin{cases}
						1, & \text{ falls } \set{i,j}\in E_U\\
						0, & \text{ falls } \set{i,j}\notin E_U
					\end{cases}
				\]
			\end{column}
		\end{columns}
	\end{block}
\pause
	\begin{exampleblock}{Besondere Eigenschaften der Adjazenzmatrix}
		\begin{itemize}
			\item Schlingen lassen sich an einer $1$ auf der Diagonalen erkennen (Wert von $A_{ii}$) 
			\item Bei ungerichteten Graphen ist $A$ immer symmetrisch (also $A_{ij} = A_{ji}$).
		\end{itemize}
	\end{exampleblock}
\end{frame}

\begin{frame}{Wdh.: Wegematrix}
    \begin{block}{Def.: Wegematrix}
		\begin{columns}
			\begin{column}{0.95\textwidth}
				Sei $G=(V_G,E_G)$ gerichteter Graph.\\
				\textbf{Wegematrix} $W \in \set{0,1}^{\mid V_G \mid \times \mid V_G \mid}$\\[12pt]
				\[
					W_{ij} = 
					\begin{cases}
						1, & \parbox[t]{.6\textwidth}{ falls es in $E_G$ Pfad von $i$ nach $j$ gibt.}\\
						0, & \parbox[t]{.6\textwidth}{ sonst}
					\end{cases}
				\]
			\end{column}
		\end{columns}
	\end{block}
\end{frame}