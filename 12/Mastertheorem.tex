\begin{frame}{Mastertheorem}
    \textbf{Problemstellung:}\\[1cm]    
    Gegeben sei eine \emph{rekursiv} definierte Funktion $T$.\\
    Frage: Welche Laufzeit hat $T$?\\[1cm]
    Beispiel:
    \[
    	T(n) = 8 T \left(\frac{n}{2} \right) + 1000n^2
    \]
    \\[1cm]\centering$\Rightarrow$ \emph{Mastertheorem}
\end{frame}

\begin{frame}{Mastertheorem}
    \begin{block}{Def.: Mastertheorem}
    	Seien $a\geq 1$ und $b>1$ Konstanten, $f \from \nN \to \Rnullplus$ und $T(n)$ eine Laufzeitfunktion der Form
    	\[
			T = a T\left(\frac{n}{b}\right) + f 
		\]
		Dann gilt nach dem \textbf{Mastertheorem}:
		\begin{itemize}
			\item \textbf{Fall 1:} \\ Wenn $f \in O(n^{\log_b a -\varepsilon})$ für ein $\varepsilon>0$ ist, dann ist $T\in \Theta(n^{\log_b a})$.
			\item \textbf{Fall 2:} \\ Wenn $f \in \Theta(n^{\log_b a})$ ist, dann ist $T\in \Theta(n^{\log_b a}\log n)$.
			\item\textbf{Fall 3:} \\  Wenn $f \in \Omega(n^{\log_b a +\varepsilon})$ für ein $\varepsilon>0$ ist, und wenn es eine Konstante $d$ gibt mit $0<d<1$, so dass für alle hinreichend großen $n$ gilt $af\left(\frac{n}{b}\right)\leq d f(n)$, dann ist $T\in \Theta(f)$.
		\end{itemize}
    \end{block}
\end{frame}

\begin{frame}{Mastertheorem}
    \begin{exampleblock}{Beispiel zum 1. Fall}
    	Sei $T(n) = 8 T \left(\frac{n}{2} \right) + 1000n^2$.
    	\begin{itemize}
    		\item Aus der Formel lässt sich ablesen:\\
    			$a=8$, $b=2$, $f(n)=1000n^2$
    		\item $n^{\log_b a}$ bestimmen:\\
    			$\log_b a = \log_2 8 = 3 \Rightarrow n^{\log_b a} = n^3$
    		\item $n^{\log_b a}$ mit $f(n)$ vergleichen: $1000n^2 \in O(n^3-\varepsilon)$?\\
    			Ja, für $\varepsilon = 1$ gilt $1000n^2 \in O(n^2)$.
    		\item Mit dem Mastertheorem folgt:\\
    			$T(n) = \Theta(n^3)$
    	\end{itemize}
    \end{exampleblock}
\end{frame}


\begin{frame}{Mastertheorem}
    \begin{exampleblock}{Beispiel zum 2. Fall}
    	Sei $T(n) = 2 T \left(\frac{n}{2} \right) + 10n$.
    	\begin{itemize}
    		\item Aus der Formel lässt sich ablesen:\\
    			$a=2$, $b=2$, $f(n)=10n$
    		\item $n^{\log_b a}$ bestimmen:\\
    			$\log_b a = \log_2 2 = 1 \Rightarrow n^{\log_b a} = n^1$
    		\item $n^{\log_b a}$ mit $f(n)$ vergleichen: $10n \in \Theta(n)$?\\
    			Ja!
    		\item Mit dem Mastertheorem folgt:\\
    			$T(n) = \Theta(n \log n)$
    	\end{itemize}
    \end{exampleblock}
\end{frame}



\begin{frame}{Mastertheorem}
    \begin{exampleblock}{Beispiel zum 3. Fall}
    	Sei $T(n) = 2 T \left(\frac{n}{2} \right) + n^2$.
    	\begin{itemize}
    		\item Aus der Formel lässt sich ablesen:\\
    			$a=2$, $b=2$, $f(n)=n^2$
    		\item $n^{\log_b a}$ bestimmen:\\
    			$\log_b a = \log_2 2 = 1 \Rightarrow n^{\log_b a} = n^1$
    		\item $n^{\log_b a}$ mit $f(n)$ vergleichen: $n^2 \in \Omega(n^{1+\varepsilon})$?\\
    			Ja, für $\varepsilon = 1$ gilt $n^2 \in \Omega(n^2)$.
    		\item Zusatzbedingung überprüfen: Ist $af\left(\frac{n}{b}\right)\leq d f(n)$?\\
    			Ja, für $d = \frac{1}{2}$ gilt $\forall n \geq 1 \; : \; \frac{1}{2}n^2 \leq \frac{1}{2}n^2$
    		\item Mit dem Mastertheorem folgt:\\
    			$T(n) = \Theta(n^2)$
    	\end{itemize}
    \end{exampleblock}
\end{frame}