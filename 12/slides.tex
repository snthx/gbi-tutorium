% ===== handout mode =====
% Comment/uncomment this line to toggle handout mode
% \newcommand{\handout}{}

% Comment/uncoment this line to toogle Mortitz mode
% \newcommand{\Moritz}{}

% Comment/uncomment this line to toggle handout mode
% \newcommand{\handout}{}

% by Stephan

%% Moritz mode or Stephan mode
\ifdefined \Moritz

% This is a configuration file with private, tutor specific information.
% It is therefore excluded from the Git repository so changes in this file will not conflict in git commits.

% Copy this Template, rename to config.tex and add your information below.

\newcommand{\mymail}{moritz.laupichler@student.kit.edu} % Consider using your named student Mail address to keep your u-Account private.

\newcommand{\myname}{\href{mailto:\mymail}{Moritz Laupichler}}

\newcommand{\mytutnumber}{25}

\newcommand{\mytutinfos}{Dienstags, 5. Block (15:45-17:15 Uhr), SR -120}

\newcommand{\aboutMeFrame}{
	\begin{frame}{Euer Tutor}
		Name: \myname \\
		Alter: 21 Jahre \\
		Studiengang: Master Informatik, 1. Semester \\
		\vspace{1cm}
		\pause 
		\centering{Kontakt: \href{mailto:\mymail}{\mymail}}
	\end{frame}
} % Moritz mode
\else
\ifdefined \Alex

% This is a configuration file with private, tutor specific information.
% It is therefore excluded from the Git repository so changes in this file will not conflict in git commits.

% Copy this Template, rename to config.tex and add your information below.

\newcommand{\mymail}{alexander.klug@student.kit.edu} % Consider using your named student Mail address to keep your u-Account private.

\newcommand{\myname}{\href{mailto:\mymail}{Alexander Klug}}

\newcommand{\mytutnumber}{30}

\newcommand{\mytutinfos}{Mittwochs, 3. Block (11:30-13:00), SR -107}

\newcommand{\aboutMeFrame}{
	\begin{frame}{Euer Tutor}
		Name: \myname \\
		Alter: 19 Jahre \\
		Studiengang: Bachelor Informatik, 3. Semester \\
		\vspace{1cm}
		\pause 
		\centering{Kontakt: \href{mailto:\mymail}{\mymail}}
	\end{frame}
}

% Toggle Handout mode by including the following line before including style_tut
% and removing the % at the start (but do NOT remove it here, otherwise handout mode will always be on!)
% Please keep handout mode on in all commits!

% \newcommand{\handout}{} % Alex Mode
\else

% This is a configuration file with private, tutor specific information.
% It is therefore excluded from the Git repository so changes in this file will not conflict in git commits.

% Copy this Template, rename to config.tex and add your information below.

\newcommand{\mymail}{stephan.bohr@student.kit.edu} % Consider using your named student Mail address to keep your u-Account private.

\newcommand{\myname}{\href{mailto:\mymail}{Stephan Bohr}}

\newcommand{\mytutnumber}{25}

\newcommand{\mytutinfos}{Dienstags, 5. Block (15:45-17:15), SR -119}

\newcommand{\aboutMeFrame}{
	\begin{frame}{Euer Tutor}
		Name: \myname \\
		Alter: 20 Jahre \\
		Studiengang: Bachelor Informatik, 3. Semester \\
		\vspace{1cm}
		\pause 
		\centering{Kontakt: \href{mailto:\mymail}{\mymail}}
	\end{frame}
} % Stephan mode
\fi
\fi

%% Beamer-Klasse im korrekten Modus
\ifdefined \handout
\documentclass[handout]{beamer} % Handout mode
\else
\documentclass{beamer}
\fi
%\documentclass[18pt,parskip]{beamer}

%% SLIDE FORMAT

% use 'beamerthemekit' for standard 4:3 ratio
% for widescreen slides (16:9), use 'beamerthemekitwide'

\usepackage{../templates/KIT-slides/beamerthemekit}
%\usepackage{../templates/KIT-slides/beamerthemekitwide}

%% TITLE PICTURE

% if a custom picture is to be used on the title page, copy it into the 'logos'
% directory, in the line below, replace 'mypicture' with the 
% filename (without extension) and uncomment the following line
% (picture proportions: 63 : 20 for standard, 169 : 40 for wide
% *.eps format if you use latex+dvips+ps2pdf, 
% *.jpg/*.png/*.pdf if you use pdflatex)

\titleimage{../figures/titleimage/brain}

%% TITLE LOGO

% for a custom logo on the front page, copy your file into the 'logos'
% directory, insert the filename in the line below and uncomment it

%\titlelogo{mylogo}

% (*.eps format if you use latex+dvips+ps2pdf,
% *.jpg/*.png/*.pdf if you use pdflatex)

%% TikZ INTEGRATION

% use these packages for PCM symbols and UML classes
% \usepackage{templates/tikzkit}
% \usepackage{templates/tikzuml}

%\usepackage{tikz}
%\usetikzlibrary{matrix}
%\usetikzlibrary{arrows.meta}
%\usetikzlibrary{automata}
%\usetikzlibrary{tikzmark}

%%%%%%%%%%%%%%%%%%%%%%%%%
% Libertine font (Original GBI font)
\usepackage{libertine}
%\renewcommand*\familydefault{\sfdefault}  %% Only if the base font of the document is to be sans serif

%% Schönere Schriften
\usepackage[TS1,T1]{fontenc}

%% Deutsche Silbentrennung und Beschriftungen
\usepackage[ngerman]{babel}

%% UTF-8-Encoding
\usepackage[utf8]{inputenc}

%% Bibliotheken für viele mathematische Symbole
\usepackage{amsmath, amsfonts, amssymb}

%% Anzeigetiefe für Inhaltsverzeichnis: 1 Stufe
\setcounter{tocdepth}{1}

%% Hyperlinks
\usepackage{hyperref}
% I don't know why, but this works and only includes sections and NOT subsections in the pdf-bookmarks.
\hypersetup{bookmarksdepth=subsection}

%% remove navigation symbols
\setbeamertemplate{navigation symbols}{}

%% switch between "ngerman" and "english" for German/English style date and logos
\selectlanguage{ngerman}

%% for invisible pause texts instead of dimming
\setbeamercovered{invisible}

%%%%%%%%%%%% Shortcuts %%%%%%%%%%%%%
\newcommand{\nM}{\mathbb{M}}
\newcommand{\nR}{\mathbb{R}}
\newcommand{\nN}{\mathbb{N}}
\newcommand{\nZ}{\mathbb{Z}}
\newcommand{\nQ}{\mathbb{Q}}
\newcommand{\nB}{\mathbb{B}}
\newcommand{\nC}{\mathbb{C}}
\newcommand{\nK}{\mathbb{K}}
\newcommand{\nF}{\mathbb{F}}
\newcommand{\nG}{\mathbb{G}}
\newcommand{\nullel}{\mathcal{O}}
\newcommand{\einsel}{\mathds{1}}
\newcommand{\nP}{\mathbb{P}}
\newcommand{\Pot}{\mathcal{P}}
\renewcommand{\O}{\text{O}}

\newcommand{\set}[1]{\{ #1 \}}
\newcommand{\setc}[2]{\set{#1 \mid #2}}
\newcommand{\setC}[2]{\set{#1 \mid \text{ #2 }}}

\newcommand{\setsize}[1]{\; \mid #1 \mid \; }

\newcommand{\q}[1]{\textquotedblleft #1\textquotedblright}

%%%%%%%%%%%% INHALT %%%%%%%%%%%%%%%%

%% Wochennummer
%\newcounter{weeknum}

%% Titelinformationen
%\title[GBI Tutorium, Woche \theweeknum]{Grundbegriffe der Informatik \\ Tutorium \mytutnumber}
%\subtitle{Termin \theweeknum \ | \mydate \\ \myname}
\author[\myname]{\myname}
\institute{Fakultät für Informatik}
%\date{\mydate}

%% Titel einfügen
\newcommand{\titleframe}{\frame{\titlepage}\addtocounter{framenumber}{-1}}


%% Alles starten mit \starttut{X}
%\newcommand{\starttut}[1]{\setcounter{weeknum}{#1}\titleframe\frame{\frametitle{Inhalt}\tableofcontents} \AtBeginSection[]{%
%\begin{frame}
%	\tableofcontents[currentsection]
%\end{frame}\addtocounter{framenumber}{-1}}}


%\newcommand{\framePrevEpisode}{
%	\begin{frame}
%		\centering
%		\textbf{In the previous episode of GBI...}
%	\end{frame}
%}

%% Roadmap frame
%table of contents
\newcommand{\roadmap}{
	\frame{\frametitle{Roadmap}\tableofcontents}}

 \AtBeginSection[]{%
\begin{frame}
	\frametitle{Roadmap}
	\tableofcontents[currentsection]
\end{frame}%\addtocounter{framenumber}{-1}
}


%% ShowMessage frame
\newcommand{\showmessage}[1]{\frame{\frametitle{\phantom{1em}}\centering\textbf{#1}}}

%% Fragen
%% Lastframe
\newcommand{\questionframe}{\showmessage{Fragen?}}

%% Lastframe
\newcommand{\lastframe}{\showmessage{Vielen Dank für Eure Aufmerksamkeit! \\Bis nächste Woche :)}}

%% Thanks frame
\newcommand{\slideThanks}{
	\begin{frame}
		\frametitle{Credits}
		\begin{block}{}
			An der Erstellung des Foliensatzes haben mitgewirkt:\\[1em]
			\ifdefined \Moritz
			Stephan Bohr \\
			Alexander Klug \\
			\else
			\ifdefined \Alex
			Stephan Bohr \\
			Moritz Laupichler \\
			\else
			Moritz Laupichler \\
			Alexander Klug \\
			\fi
			\fi
			Katharina Wurz \\
			Thassilo Helmold \\
			Philipp Basler \\
			Nils Braun \\
			Dominik Doerner \\
			Ou Yue \\
		\end{block}
	\end{frame}
}

%% Verbatim
%\usepackage{moreverb}



\title[Algorithmen in Graphen, Quantitative Aspekte von Algorithmen]{12. Tutorium\\ Algorithmen in Graphen, \\Quantitative Aspekte von Algorithmen}
\subtitle{Grundbegriffe der Informatik, Tutorium \hashtag\mytutnumber}
\date{\today}


\begin{document}
\titleframe
\roadmap

%%%%%%%%%% %%%%%%%%%%
\section{Algorithmen in Graphen}
\subsection{Wiederholung Adjazenz-/Wegematrix}
\begin{frame}{Wdh.: Adjazenzmatrix}
	\begin{block}{Def.: Adjazenzmatrix}
		\begin{columns}
			\begin{column}{0.475\textwidth}
				Sei $G=(V_G,E_G)$ gerichteter Graph.\\
				\textbf{Adjazenzmatrix} $A \in \set{0,1}^{\mid V_G \mid \times \mid V_G \mid}$\\[12pt]
				\[
					A_{ij} = 
					\begin{cases}
						1, & \text{ falls } (i,j)\in E_G\\
						0, & \text{ falls } (i,j)\notin E_G
					\end{cases}
				\]
			\end{column}
			\begin{column}{0.475\textwidth}
				Sei $U=(V_U,E_U)$ ungerichteter Graph.\\
				\textbf{Adjazenzmatrix} $A \in \set{0,1}^{\mid V_U \mid \times \mid V_U \mid}$\\[12pt]
				\[
					A_{ij} = 
					\begin{cases}
						1, & \text{ falls } \set{i,j}\in E_U\\
						0, & \text{ falls } \set{i,j}\notin E_U
					\end{cases}
				\]
			\end{column}
		\end{columns}
	\end{block}
\pause
	\begin{exampleblock}{Besondere Eigenschaften der Adjazenzmatrix}
		\begin{itemize}
			\item Schlingen lassen sich an einer $1$ auf der Diagonalen erkennen (Wert von $A_{ii}$) 
			\item Bei ungerichteten Graphen ist $A$ immer symmetrisch (also $A_{ij} = A_{ji}$).
		\end{itemize}
	\end{exampleblock}
\end{frame}

\begin{frame}{Wdh.: Wegematrix}
    \begin{block}{Def.: Wegematrix}
		\begin{columns}
			\begin{column}{0.95\textwidth}
				Sei $G=(V_G,E_G)$ gerichteter Graph.\\
				\textbf{Wegematrix} $W \in \set{0,1}^{\mid V_G \mid \times \mid V_G \mid}$\\[12pt]
				\[
					W_{ij} = 
					\begin{cases}
						1, & \parbox[t]{.6\textwidth}{ falls es in $E_G$ Pfad von $i$ nach $j$ gibt.}\\
						0, & \parbox[t]{.6\textwidth}{ sonst}
					\end{cases}
				\]
			\end{column}
		\end{columns}
	\end{block}
\end{frame}
\subsection{Von der Adjazenz- zur Wegematrix, Ausblick Warshall-Algorithmus}
\subsection{Von der Adjazenz- zur Wegematrix, Ausblick Warshall-Algorithmus}
\begin{frame}{Kantenrelation}
	\begin{block}{Kantenmenge als Relation}
		Sei $G$ ein gerichteter Graph mit der Knotenmenge $V$. Die Kantenmenge ist wie folgt definiert:\\
		Sei $E \subseteq V \times V$ eine Relation auf $V$ definiert durch $(x,y) \in E \gdw \text{es gibt eine Kante von $x$ nach }y \text{ in } G$.
	\end{block}

	\begin{exampleblock}{Beobachtungen}
		\begin{itemize}
			\item Was gilt für $E^2$? \pause $\Rightarrow$ Knoten, die über einen Pfad der Länge 2 verbunden sind
			\item Analog $E^3$, $E^4$, \dots
			\item Also $(x,y) \in E^*$ genau dann, wenn es einen Pfad (beliebiger Länge) von $x$ nach $y$ gibt.
		\end{itemize}
	\end{exampleblock}
\end{frame}

\begin{frame}{Berechnung der Erreichbarkeitsrelation}
    \begin{block}{Erreichbarkeitsrelation}
        Sei $n = |V|$
    	\[
    		E^* = \bigcup_{i=0}^{\infty} E^i = \bigcup_{i=0}^{n-1} E^i
    	\]
    \end{block}

    \begin{block}{Matrizen für die Relation $E^k$}
    	Sei $G$ ein gerichteter Graph mit Adjazenzmatrix $A$. Für alle $k \in \nN_0$ gilt:\\
    	\[
    		\sgn((A^k)_{ij}):=
    		\begin{cases}
      		1 & \text{ falls in $G$ ein Pfad der Länge $k$ von $i$ nach $j$ existiert}\\
      		0 & \text{ falls in $G$ kein Pfad der Länge $k$ von $i$ nach $j$ existiert}\\
    		\end{cases}
    	\]
    \end{block}
\end{frame}

\begin{frame}{Berechnung der Wegematrix}	
    \begin{block}{Berechnung der Wegematrix}
    	Es sei $G$ ein gerichteter Graph mit Adjazenzmatrix $A$. Dann gilt für alle $k\geq n-1$: 
    	\[
    		W = \sgn\left(\sum_{i=0}^{k} A^i\right) 
    	\]
    	ist die Wegematrix des Graphen $G$.
    \end{block}
\end{frame}


\section{Quantitative Aspekte von Algorithmen}
\newcommand{\Rplus}{\ensuremath{\nR_+}}
\newcommand{\Rnullplus}{\ensuremath{\nR^+_0}}
\subsection{Ressourcenverbrauch, Abängigkeit von n, Best- Worst- und Average-Case}
\begin{frame}{Laufzeit}
\centering\includegraphics[height=4cm]{algorithm}

\begin{exampleblock}{Aufgabe}
	Wie viele \emph{Instruktionen} werden aufgerufen? \\% Worst-Case: n+1, Best-Case: 1, Average Case ?
	\pause
	Im \emph{worst case}? Im \emph{best case}? Im \emph{average case}?
\end{exampleblock}
\end{frame}
\subsection{O-Kalkül}
\begin{frame}{O-Notation}
	\begin{block}{Def. Asymptotisches Wachstum $\asymp$}
		Seien $f,g \from \nN_0 \to \Rnullplus$. Dann gilt
		\[
			f \asymp g \Leftrightarrow \exists c,c'\in \Rplus: \exists n_0\in \nN_0: \forall n\geq n_0: c f(n) \leq g(n) \leq c' f(n)
		\]
		Man sagt auch $g$ wächst genauso schnell wie $f$. $\asymp$ ist eine Äquivalenzrelation.
	\end{block}

	\begin{exampleblock}{Beispiele}
		\begin{itemize}
			\item $42n^6-33n^3+222n^2 -15 \asymp 66n^6+55555n^5$
			\item $n^{3+1}+5n^2\asymp 3n^3-n$
		\end{itemize}
	\end{exampleblock}
\end{frame}

\begin{frame}{$O$-Notation}
    \begin{block}{$\Theta$-Kalkül}
    	\begin{align*}
  			\Theta(f) &= \{ g \mid f \asymp g \} \\
  				   &= \{ g \mid \exists c,c'\in \Rplus: \exists n_0\in \nN_0: \forall  n\geq n_0: c f(n) \leq g(n) \leq c' f(n) \} 
		\end{align*}
    \end{block}

    \begin{exampleblock}{Bemerkungen}
    	\begin{itemize}
    		\item Im $\Theta$-Kalkül von $f(n)$ sind genau die Funktionen enthalten, die asymptotisch gleich schnell wachsen wie $f(n)$.
    		\item Schreibe $g(n) \in \Theta(f(n))$, wenn $g(n)$ asymptotisch gleichschnell wächst wie $f(n)$.
    		\item Ist $f$ ein Polynom, so sind insbesondere in $\Theta(f(n))$ alle Polynome enthalten, die den gleichen Grad wie $f$ haben.
    		\item Es gilt $\log_b(n) \in\Theta(\log_a(n))$. Die Basis ist also egal und man kann auch $\Theta(\log n)$ schreiben. % TODO Aufgabe und Beweis hierzu
    		% TODO Theta ist != Average Case
    	\end{itemize}
    \end{exampleblock}
\end{frame}

\begin{frame}{$O$-Notation}
    \begin{block}{Def.: $O$-Kalkül}
    	\[
    		O(f) = \{ g \mid \exists c\in \Rplus:\exists n_0\in\nN_0: \forall n\geq n_0: g(n) \leq c f(n)\}
    	\]

    	$g(n) \in O(f(n))$ (oder $g \preceq f$) genau dann, wenn $g$ asymptotisch höchstens so schnell wächst wie $f$.
    \end{block}
\pause
    \begin{block}{Def.: $\Omega$-Kalkül}
    	\[
    		\Omega(f) = \{ g \mid \exists c\in \Rplus:\exists n_0\in\nN_0: \forall n\geq n_0: g(n) \geq c f(n)\}
    	\]
    	
    	$g(n) \in \Omega(f(n))$ (oder $g \succeq f$) genau dann, wenn $g$ asymptotisch mindestens so schnell wächst wie $f$.
    \end{block}
\pause
    Beobachtung: $\Theta(f) = O(f) \cap \Omega(f)$
\end{frame}

\begin{frame}{$O$-Notation}
	\begin{exampleblock}{Aufgabe}
		\begin{enumerate}
			\item Für welches $c \in \Rplus$ gilt $5n^4 \in O(n^c)$ bzw $5n^4 \in \Omega(n^c)$?
			\item Für welches $c \in \Rplus$ gilt $5n^4 \in O(c^n)$ bzw $5n^4 \in \Omega(c^n)$?
			\item Für welches $c \in \Rplus$ gilt $2^n \in O(c^n)$ bzw $2^n \in \Omega(c^n)$?
			\item Zeige oder widerlege: $n \in \Theta(\sqrt{n})$
		\end{enumerate}
	\end{exampleblock}
\pause
	\begin{block}{Lösung}
		\begin{enumerate}
			\item Es gilt: $\forall c \geq 4$ bzw. $\forall c \leq 4$.
			\item Es gilt: $\forall c > 1$ bzw. $\forall c \leq 1$.
			\item Es gilt: $\forall c \geq 2$ bzw. $\forall c \leq 2$.
			\item \small Annahme: Die Behauptung ist richtig.\\
				Dann gilt: $n \in O(\sqrt{n}) \wedge n \in \Omega(\sqrt{n})$, insbesondere $n \in \Omega(\sqrt{n})$.\\
				$\Rightarrow \exists c\in \Rplus:\exists n_0\in\nN_0: \forall n\geq n_0: n \leq c \sqrt{n} \Leftrightarrow \frac{n}{\sqrt{n}} \leq c \Leftrightarrow \sqrt{n} \leq c$. Widerspruch.
				
		\end{enumerate}
	\end{block}
\end{frame}

\begin{frame}{$O$-Notation}
    \begin{exampleblock}{Aufgabe}
    	Zeige oder widerlege:
    	\[
    		f(n) + g(n) \in O(g(f(n)))
    	\]
    \end{exampleblock}
\pause
	\begin{block}{Lösung}
		Die Behauptung stimmt nicht. Wähle z.B. $f(n) = n^2$ und $g(n) = \sqrt{n}$ und führe dies zu einem Widerspruch.
	\end{block}
\end{frame}

\begin{frame}{$O$-Notation} % TODO: Logarithmusregeln, Typische Abschätzungskette (1 < log n < sqrt(n) < n < n log n < n c (c>1) < c^n < n!)
    \begin{itemize}
    	\item $\Theta$ entspricht \emph{nicht} dem average case.
    	\item $O(1)$ bedeutet konstante Laufzeit
    	\item Beachtet den \emph{Trick} mit den Limites\footnote{Ich weiß nicht, ob ihr den als Beweis in der Klausur oder auf dem Blättern verwenden dürft. Zur Kontrolle sollte man ihn aber kennen.}
        \item Es gilt mit Konstante $c \in \Rplus$ mit $c>1$
        \[
            1 \preceq \log n \preceq \sqrt{n} \preceq n \preceq n \log n \preceq n^c \preceq c^n \preceq n!
        \]
        \nachgucken https://martin-thoma.com/die-landau-symbole/
    \end{itemize}
\end{frame}


\subsection{Master-Theorem}
\begin{frame}{Mastertheorem}
    \textbf{Problemstellung:}\\[1cm]    
    Gegeben sei eine \emph{rekursiv} definierte Funktion $T$.\\
    Frage: Welche Laufzeit hat $T$?\\[1cm]
    Beispiel:
    \[
    	T(n) = 8 T \left(\frac{n}{2} \right) + 1000n^2
    \]
    \\[1cm]\centering$\Rightarrow$ \emph{Mastertheorem}
\end{frame}

\begin{frame}{Mastertheorem}
    \begin{block}{Def.: Mastertheorem}
    	Seien $a\geq 1$ und $b>1$ Konstanten, $f \from \nN \to \Rnullplus$ und $T(n)$ eine Laufzeitfunktion der Form
    	\[
			T = a T\left(\frac{n}{b}\right) + f 
		\]
		Dann gilt nach dem \textbf{Mastertheorem}:
		\begin{itemize}
			\item \textbf{Fall 1:} \\ Wenn $f \in O(n^{\log_b a -\varepsilon})$ für ein $\varepsilon>0$ ist, dann ist $T\in \Theta(n^{\log_b a})$.
			\item \textbf{Fall 2:} \\ Wenn $f \in \Theta(n^{\log_b a})$ ist, dann ist $T\in \Theta(n^{\log_b a}\log n)$.
			\item\textbf{Fall 3:} \\  Wenn $f \in \Omega(n^{\log_b a +\varepsilon})$ für ein $\varepsilon>0$ ist, und wenn es eine Konstante $d$ gibt mit $0<d<1$, so dass für alle hinreichend großen $n$ gilt $af\left(\frac{n}{n}\right)\leq d f$, dann ist $T\in \Theta(f)$.
		\end{itemize}
    \end{block}
\end{frame}

\begin{frame}{Mastertheorem}
    \begin{exampleblock}{Beispiel zum 1. Fall}
    	Sei $T(n) = 8 T \left(\frac{n}{2} \right) + 1000n^2$.
    	\begin{itemize}
    		\item Aus der Formel lässt sich ablesen:\\
    			$a=8$, $b=2$, $f(n)=1000n^2$
    		\item $n^{\log_b a}$ bestimmen:\\
    			$\log_b a = \log_2 8 = 3 \Rightarrow n^{\log_b a} = n^3$
    		\item $n^{\log_b a}$ mit $f(n)$ vergleichen: $1000n^2 \in O(n^3-\varepsilon)$?\\
    			Ja, für $\varepsilon = 1$ gilt $1000n^2 \in O(n^2)$.
    		\item Mit dem Mastertheorem folgt:\\
    			$T(n) = \Theta(n^3)$
    	\end{itemize}
    \end{exampleblock}
\end{frame}


\begin{frame}{Mastertheorem}
    \begin{exampleblock}{Beispiel zum 2. Fall}
    	Sei $T(n) = 2 T \left(\frac{n}{2} \right) + 10n$.
    	\begin{itemize}
    		\item Aus der Formel lässt sich ablesen:\\
    			$a=2$, $b=2$, $f(n)=10n$
    		\item $n^{\log_b a}$ bestimmen:\\
    			$\log_b a = \log_2 2 = 1 \Rightarrow n^{\log_b a} = n^1$
    		\item $n^{\log_b a}$ mit $f(n)$ vergleichen: $10n \in \Theta(n)$?\\
    			Ja!
    		\item Mit dem Mastertheorem folgt:\\
    			$T(n) = \Theta(n \log n)$
    	\end{itemize}
    \end{exampleblock}
\end{frame}



\begin{frame}{Mastertheorem}
    \begin{exampleblock}{Beispiel zum 3. Fall}
    	Sei $T(n) = 2 T \left(\frac{n}{2} \right) + n^2$.
    	\begin{itemize}
    		\item Aus der Formel lässt sich ablesen:\\
    			$a=2$, $b=2$, $f(n)=n^2$
    		\item $n^{\log_b a}$ bestimmen:\\
    			$\log_b a = \log_2 2 = 1 \Rightarrow n^{\log_b a} = n^1$
    		\item $n^{\log_b a}$ mit $f(n)$ vergleichen: $n^2 \in \Omega(n^{1+\varepsilon})$?\\
    			Ja, für $\varepsilon = 1$ gilt $n^2 \in \Omega(n^2)$.
    		\item Zusatzbedingung überprüfen: Ist $af\left(\frac{n}{n}\right)\leq d f$?\\
    			Ja, für $d = \frac{1}{2}$ gilt $\forall n \geq 1 \; : \; \frac{1}{2}n^2 \leq \frac{1}{2}n^2$
    		\item Mit dem Mastertheorem folgt:\\
    			$T(n) = \Theta(n^2)$
    	\end{itemize}
    \end{exampleblock}
\end{frame}


%%%%%%%%%% %%%%%%%%%%
%% Zusammenfassung
\section{}
%\subsection{Zusammenfassung}
	\begin{frame}{Was ihr jetzt kennen und können solltet\dots}
			\begin{itemize}
				\item Von der \emph{Adjazenzmatrix} zur \emph{Wegematrix}
				\item \emph{Laufzeiten} von Algorithmen angeben und abschätzen
				\item Mit dem \emph{$O$-Kalkül} arbeiten
				\item Die Laufzeit rekursiver Algorithmen mit dem \emph{Mastertheorem} bestimmen
			\end{itemize}
	
	\end{frame}
%% Ausblick
%\subsection{Ausblick}
	\begin{frame}{Ausblick}
		\begin{itemize}
			\item Erste Nutzung von Graphen: \emph{endliche Automaten}
			\item Neue Möglichkeiten mit formalen Sprachen: \emph{reguläre Ausdrücke} und \emph{rechtslineare Grammatriken} 
		\end{itemize}
	\end{frame}
%%%%%%%%%% %%%%%%%%%%
\section{}
\questionframe
\lastframe
\mode<handout>{\slideThanks}
\end{document}