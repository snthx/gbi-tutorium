% ===== handout mode =====
% Comment/uncomment this line to toggle handout mode
% \newcommand{\handout}{}


% Comment/uncoment this line to toogle Mortitz mode
% \newcommand{\Moritz}{}

% Comment/uncomment this line to toggle handout mode
% \newcommand{\handout}{}

% by Stephan

%% Moritz mode or Stephan mode
\ifdefined \Moritz

% This is a configuration file with private, tutor specific information.
% It is therefore excluded from the Git repository so changes in this file will not conflict in git commits.

% Copy this Template, rename to config.tex and add your information below.

\newcommand{\mymail}{moritz.laupichler@student.kit.edu} % Consider using your named student Mail address to keep your u-Account private.

\newcommand{\myname}{\href{mailto:\mymail}{Moritz Laupichler}}

\newcommand{\mytutnumber}{25}

\newcommand{\mytutinfos}{Dienstags, 5. Block (15:45-17:15 Uhr), SR -120}

\newcommand{\aboutMeFrame}{
	\begin{frame}{Euer Tutor}
		Name: \myname \\
		Alter: 21 Jahre \\
		Studiengang: Master Informatik, 1. Semester \\
		\vspace{1cm}
		\pause 
		\centering{Kontakt: \href{mailto:\mymail}{\mymail}}
	\end{frame}
} % Moritz mode
\else
\ifdefined \Alex

% This is a configuration file with private, tutor specific information.
% It is therefore excluded from the Git repository so changes in this file will not conflict in git commits.

% Copy this Template, rename to config.tex and add your information below.

\newcommand{\mymail}{alexander.klug@student.kit.edu} % Consider using your named student Mail address to keep your u-Account private.

\newcommand{\myname}{\href{mailto:\mymail}{Alexander Klug}}

\newcommand{\mytutnumber}{30}

\newcommand{\mytutinfos}{Mittwochs, 3. Block (11:30-13:00), SR -107}

\newcommand{\aboutMeFrame}{
	\begin{frame}{Euer Tutor}
		Name: \myname \\
		Alter: 19 Jahre \\
		Studiengang: Bachelor Informatik, 3. Semester \\
		\vspace{1cm}
		\pause 
		\centering{Kontakt: \href{mailto:\mymail}{\mymail}}
	\end{frame}
}

% Toggle Handout mode by including the following line before including style_tut
% and removing the % at the start (but do NOT remove it here, otherwise handout mode will always be on!)
% Please keep handout mode on in all commits!

% \newcommand{\handout}{} % Alex Mode
\else

% This is a configuration file with private, tutor specific information.
% It is therefore excluded from the Git repository so changes in this file will not conflict in git commits.

% Copy this Template, rename to config.tex and add your information below.

\newcommand{\mymail}{stephan.bohr@student.kit.edu} % Consider using your named student Mail address to keep your u-Account private.

\newcommand{\myname}{\href{mailto:\mymail}{Stephan Bohr}}

\newcommand{\mytutnumber}{25}

\newcommand{\mytutinfos}{Dienstags, 5. Block (15:45-17:15), SR -119}

\newcommand{\aboutMeFrame}{
	\begin{frame}{Euer Tutor}
		Name: \myname \\
		Alter: 20 Jahre \\
		Studiengang: Bachelor Informatik, 3. Semester \\
		\vspace{1cm}
		\pause 
		\centering{Kontakt: \href{mailto:\mymail}{\mymail}}
	\end{frame}
} % Stephan mode
\fi
\fi

%% Beamer-Klasse im korrekten Modus
\ifdefined \handout
\documentclass[handout]{beamer} % Handout mode
\else
\documentclass{beamer}
\fi
%\documentclass[18pt,parskip]{beamer}

%% SLIDE FORMAT

% use 'beamerthemekit' for standard 4:3 ratio
% for widescreen slides (16:9), use 'beamerthemekitwide'

\usepackage{../templates/KIT-slides/beamerthemekit}
%\usepackage{../templates/KIT-slides/beamerthemekitwide}

%% TITLE PICTURE

% if a custom picture is to be used on the title page, copy it into the 'logos'
% directory, in the line below, replace 'mypicture' with the 
% filename (without extension) and uncomment the following line
% (picture proportions: 63 : 20 for standard, 169 : 40 for wide
% *.eps format if you use latex+dvips+ps2pdf, 
% *.jpg/*.png/*.pdf if you use pdflatex)

\titleimage{../figures/titleimage/brain}

%% TITLE LOGO

% for a custom logo on the front page, copy your file into the 'logos'
% directory, insert the filename in the line below and uncomment it

%\titlelogo{mylogo}

% (*.eps format if you use latex+dvips+ps2pdf,
% *.jpg/*.png/*.pdf if you use pdflatex)

%% TikZ INTEGRATION

% use these packages for PCM symbols and UML classes
% \usepackage{templates/tikzkit}
% \usepackage{templates/tikzuml}

%\usepackage{tikz}
%\usetikzlibrary{matrix}
%\usetikzlibrary{arrows.meta}
%\usetikzlibrary{automata}
%\usetikzlibrary{tikzmark}

%%%%%%%%%%%%%%%%%%%%%%%%%
% Libertine font (Original GBI font)
\usepackage{libertine}
%\renewcommand*\familydefault{\sfdefault}  %% Only if the base font of the document is to be sans serif

%% Schönere Schriften
\usepackage[TS1,T1]{fontenc}

%% Deutsche Silbentrennung und Beschriftungen
\usepackage[ngerman]{babel}

%% UTF-8-Encoding
\usepackage[utf8]{inputenc}

%% Bibliotheken für viele mathematische Symbole
\usepackage{amsmath, amsfonts, amssymb}

%% Anzeigetiefe für Inhaltsverzeichnis: 1 Stufe
\setcounter{tocdepth}{1}

%% Hyperlinks
\usepackage{hyperref}
% I don't know why, but this works and only includes sections and NOT subsections in the pdf-bookmarks.
\hypersetup{bookmarksdepth=subsection}

%% remove navigation symbols
\setbeamertemplate{navigation symbols}{}

%% switch between "ngerman" and "english" for German/English style date and logos
\selectlanguage{ngerman}

%% for invisible pause texts instead of dimming
\setbeamercovered{invisible}

%%%%%%%%%%%% Shortcuts %%%%%%%%%%%%%
\newcommand{\nM}{\mathbb{M}}
\newcommand{\nR}{\mathbb{R}}
\newcommand{\nN}{\mathbb{N}}
\newcommand{\nZ}{\mathbb{Z}}
\newcommand{\nQ}{\mathbb{Q}}
\newcommand{\nB}{\mathbb{B}}
\newcommand{\nC}{\mathbb{C}}
\newcommand{\nK}{\mathbb{K}}
\newcommand{\nF}{\mathbb{F}}
\newcommand{\nG}{\mathbb{G}}
\newcommand{\nullel}{\mathcal{O}}
\newcommand{\einsel}{\mathds{1}}
\newcommand{\nP}{\mathbb{P}}
\newcommand{\Pot}{\mathcal{P}}
\renewcommand{\O}{\text{O}}

\newcommand{\set}[1]{\{ #1 \}}
\newcommand{\setc}[2]{\set{#1 \mid #2}}
\newcommand{\setC}[2]{\set{#1 \mid \text{ #2 }}}

\newcommand{\setsize}[1]{\; \mid #1 \mid \; }

\newcommand{\q}[1]{\textquotedblleft #1\textquotedblright}

%%%%%%%%%%%% INHALT %%%%%%%%%%%%%%%%

%% Wochennummer
%\newcounter{weeknum}

%% Titelinformationen
%\title[GBI Tutorium, Woche \theweeknum]{Grundbegriffe der Informatik \\ Tutorium \mytutnumber}
%\subtitle{Termin \theweeknum \ | \mydate \\ \myname}
\author[\myname]{\myname}
\institute{Fakultät für Informatik}
%\date{\mydate}

%% Titel einfügen
\newcommand{\titleframe}{\frame{\titlepage}\addtocounter{framenumber}{-1}}


%% Alles starten mit \starttut{X}
%\newcommand{\starttut}[1]{\setcounter{weeknum}{#1}\titleframe\frame{\frametitle{Inhalt}\tableofcontents} \AtBeginSection[]{%
%\begin{frame}
%	\tableofcontents[currentsection]
%\end{frame}\addtocounter{framenumber}{-1}}}


%\newcommand{\framePrevEpisode}{
%	\begin{frame}
%		\centering
%		\textbf{In the previous episode of GBI...}
%	\end{frame}
%}

%% Roadmap frame
%table of contents
\newcommand{\roadmap}{
	\frame{\frametitle{Roadmap}\tableofcontents}}

 \AtBeginSection[]{%
\begin{frame}
	\frametitle{Roadmap}
	\tableofcontents[currentsection]
\end{frame}%\addtocounter{framenumber}{-1}
}


%% ShowMessage frame
\newcommand{\showmessage}[1]{\frame{\frametitle{\phantom{1em}}\centering\textbf{#1}}}

%% Fragen
%% Lastframe
\newcommand{\questionframe}{\showmessage{Fragen?}}

%% Lastframe
\newcommand{\lastframe}{\showmessage{Vielen Dank für Eure Aufmerksamkeit! \\Bis nächste Woche :)}}

%% Thanks frame
\newcommand{\slideThanks}{
	\begin{frame}
		\frametitle{Credits}
		\begin{block}{}
			An der Erstellung des Foliensatzes haben mitgewirkt:\\[1em]
			\ifdefined \Moritz
			Stephan Bohr \\
			Alexander Klug \\
			\else
			\ifdefined \Alex
			Stephan Bohr \\
			Moritz Laupichler \\
			\else
			Moritz Laupichler \\
			Alexander Klug \\
			\fi
			\fi
			Katharina Wurz \\
			Thassilo Helmold \\
			Philipp Basler \\
			Nils Braun \\
			Dominik Doerner \\
			Ou Yue \\
		\end{block}
	\end{frame}
}

%% Verbatim
%\usepackage{moreverb}


\usepackage{tcolorbox}

%Hoare-Tripel Commands
\newcommand{\HTB}[1]{\ensuremath{\colorbox{lightgray!90}{\{ #1 \}}}}
\newcommand{\HT}[3]{\ensuremath{\HTB{#1} \; #2 \; \HTB{#3}}}

\newcolumntype{C}{>{$}c<{$}} % math-mode version of "C" column type

\title[Hoare-Kalkül]{10. Tutorium\\ Hoare-Kalkül}
\subtitle{Grundbegriffe der Informatik, Tutorium \#\mytutnumber}
\date{\today}

\begin{document}
\titleframe
\roadmap

%%%%%%%%%% %%%%%%%%%%
\section{Hoare-Kalkül}
\begin{frame}{Regeln im Hoare-Kalkül}
	Es gibt einige ``Regeln'' im Hoare-Kalkül, die für gültige Hoare-Tripel gelten.

	\begin{block}{HT-E: Verstärkung der Vorbedingung und Abschwächung der Nachbedingung}
		Es gelte $P' \vdash P$ und $Q\vdash Q'$. Dann gilt:
		 \begin{center}\begin{tabular}{c}
						$\set{P} \, S \, \set{Q}$\\
						\midrule
						$\set{P'} \, S \, \set{Q'}$
		\end{tabular}\end{center}
		(Wenn $\set{P} \, S \, \set{Q}$ gültig, dann $\set{P'} \, S \, \set{Q'}$ gültig.)
	\end{block}

\end{frame}

\begin{frame}{Regeln im Hoare-Kalkül}

	\begin{block}{HT-S: Hintereinanderausführung}
		Es gilt: \begin{center}\begin{tabular}{c}
			$\set{P} \, S_1 \, \set{Q}$ \qquad $\set{Q} \, S_2 \, \set{R}$\\
			\midrule
			$\set{P} \, S_1 ; S_2 \, \set{R}$
		\end{tabular}\end{center}
		Das heißt die Gültigkeit von Hoare-Tripeln ist transitiv. 
		
	\end{block}
\end{frame}

\begin{frame}{Regeln im Hoare-Kalkül}
	\begin{block}{HT-A: Zuweisungen}
		\begin{itemize}
			\item Zuweisung $x \leftarrow E$ ($x$ wird durch Ausdruck $E \in L_Ter$ ``ersetzt'')
			\item $Q$ sei Nachbedingung zu $x \leftarrow E$
			\item $\sigma_{\set{x/E}}$ sei auf $Q$ kollisionsfrei.
			\item Dann ist \begin{center}
				$\set{\sigma_{\set{x/E}}(Q)} \, x\leftarrow E \, \set{Q}$
			\end{center} ein gültiges Hoare-Tripel.
		\end{itemize}
	\end{block}
	\pause
	\begin{exampleblock}{Beispiel}
		\only<2>{\HTB{?}\\}
		\only<3>{\HTB{$y^{2}>0$}\\}
		$x \leftarrow y^{2}$\\
		\HTB{$x>0$}
	\end{exampleblock}
	
\end{frame}

\begin{frame}{Schwächste Vorbedingung und stärkste Nachbedingung}
	\begin{block}{Def.: Schwächste Vorbedingung}
		Zu einer Anweisungsfolge $S$ und einer Nachbedingung $Q$ heißt $P$ eine \textbf{schwächste Vorbedingung}, wenn $\set{P} \, S \, \set{Q}$ ein gültiges Hoare-Tripel ist und für jedes gültige Hoare-Tripel $\set{P'} \, S \, \set{Q}$ gilt: $P'$ impliziert $P$.
	\end{block}

	\begin{block}{Def.: Stärkste Nachbedingung}
		Zu einer Anweisungsfolge $S$ und einer Vorbedingung $P$ heißt $Q$ eine \textbf{stärkste Nachbedingung}, wenn $\set{P} \, S \, \set{Q}$ ein gültiges Hoare-Tripel ist und für jedes gültige Hoare-Tripel $\set{P} \, S \, \set{Q'}$ gilt: $Q$ impliziert $Q'$.
	\end{block}
\end{frame}

\begin{frame}{Aufgabe zum Finden einer schwächsten Vorbedingung}
	\begin{exampleblock}{Aufgabe}
		Es seien $y$ und $z$ zwei verschiedene Variablen und es seien $b$ und $c$ zwei ganze
		Zahlen. Bestimme anhand des Hoare-Kalküls eine schwächste Vorbedingung von
		\begin{align*}
			& x \leftarrow x+y\\
			& y \leftarrow x-y\\
			& x \leftarrow x-y\\
			& \HTB{$x=b \, \wedge \, y=a$}
		\end{align*}
		\\[18pt]
		indem du wiederholt die Zuweisungsregel (HT-A) anwendest.
			
	\end{exampleblock}
\end{frame}

\begin{frame}{Lösung zur Aufgabe}
	\begin{block}{Lösung}
		Schwächste Vorbedingung: \HTB{$x=a \, \wedge \, y=b$}\\
		\begin{align*}
			&\HTB{$x=a \, \wedge \, y=b$}\\
			&\HTB{$y=b \, \wedge \, (x+y)-y=x=a$}\\
			& x \leftarrow x+y\\
			&\HTB{$x-(x-y)=y=b \, \wedge \, x-y=a$}\\
			& y \leftarrow x-y\\
			&\HTB{$x-y=b \, \wedge \, y=a$}\\
			& x \leftarrow x-y\\
			&\HTB{$x=b \, \wedge \, y=a$}
		\end{align*}
	\end{block}
\end{frame}

\begin{frame}{Mehr Regeln im Hoare-Kalkül}
	\begin{block}{HT-I: Bedingte Anweisungen}
	\begin{columns}
		\begin{column}{0.3\textwidth}
		\small
		\begin{tabular}{rl}
			&\HTB{P}\\
			&\textbf{if} $B$\\
			&\textbf{then}\\
			&\qquad \HTB{$P \wedge B$}\\
			&\qquad $S_1$\\
			&\qquad \HTB{Q}\\
			&\textbf{else}\\
			&\qquad \HTB{$P \wedge \neg B$}\\
			&\qquad $S_2$\\
			&\qquad \HTB{Q}\\
			&\textbf{fi}\\
			&\HTB{Q}
		\end{tabular}			
		\end{column}

		\begin{column}{0.7\textwidth}
			\begin{center}\begin{tabular}{c}
			$\set{P \wedge B} \, S_1 \, \set{Q}$ \qquad $\set{P \wedge \neg B} \, S_2 \, \set{Q}$\\
			\midrule
			$\set{P} \, $\textbf{if} $B$ \textbf{then} $S_1$ \textbf{else} $S_2$ \textbf{fi}$\, \set{R}$
		\end{tabular}\end{center}
		\end{column}		
	\end{columns}	

	\end{block}
\end{frame}

\begin{frame}{Aufgabe zu HT-I}
	\begin{exampleblock}{Aufgabe}
		Ergänze die fehlenden Zuweisungen:
		\small 
		\begin{tabular}{rl}
			&\HTB{$x=a \wedge y=b$}\\
			&\textbf{if} $x>y$\\
			&\textbf{then}\\
			&\qquad \HTB{...}\\
			&\qquad $z \leftarrow y$\\
			&\qquad \HTB{...}\\
			&\textbf{else}\\
			&\qquad \HTB{...}\\
			&\qquad $z \leftarrow x$\\
			&\qquad \HTB{...}\\
			&\textbf{fi}\\
			&\HTB{$x=a \wedge y=b \wedge z=min(a,b)$}
		\end{tabular}
	\end{exampleblock}
\end{frame}

\begin{frame}{Lösung zur Aufgabe}
	\begin{block}{Lösung}
		\small 
		\begin{tabular}{rl}
			&\HTB{$x=a \wedge y=b$}\\
			&\textbf{if} $x>y$\\
			&\textbf{then}\\
			&\qquad \HTB{$x=a \wedge y=b \wedge x>y$}\\
			&\qquad $z \leftarrow y$\\
			&\qquad \HTB{$x=a \wedge y=b \wedge z=min(a,b)$}\\
			&\textbf{else}\\
			&\qquad \HTB{$x=a \wedge y=b \wedge x\le y$}\\
			&\qquad $z \leftarrow x$\\
			&\qquad \HTB{$x=a \wedge y=b \wedge z=min(a,b)$}\\
			&\textbf{fi}\\
			&\HTB{$x=a \wedge y=b \wedge z=min(a,b)$}
		\end{tabular}
	\end{block}
\end{frame}


\begin{frame}{Mehr Regeln im Hoare-Kalkül}
	\begin{block}{HT-W: While-Schleifen}
	\begin{columns}
		\begin{column}{0.3\textwidth}
		\small
		\begin{tabular}{rl}
			&\HTB{I}\\
			&\textbf{while} $B$\\
			&\textbf{do}\\
			&\qquad \HTB{$I \wedge B$}\\
			&\qquad $S$\\
			&\qquad \HTB{I}\\
			&\textbf{od}\\
			&\HTB{$I \wedge \neg B$}
		\end{tabular}			
		\end{column}

		\begin{column}{0.7\textwidth}
			\begin{center}\begin{tabular}{c}
			$\set{I \wedge B} \; S \; \set{I}$\\
			\midrule
			$\set{I}$ \textbf{while} $B$ \textbf{do} $S$ \textbf{od} $\set{I \wedge \neg B}$ 
		\end{tabular}\end{center}
		\centering Zusicherung $I$ heißt \textbf{Schleifeninvariante}.
		\end{column}		
	\end{columns}	

	\end{block}
\end{frame}

\begin{frame}{Beispiel zur Schleifeninvariante}
	\begin{exampleblock}{Beispiel}
		Eingabe: $a,b \in \nN_0$
		\begin{align*}
			&X_0 \leftarrow a\\
			&Y_0 \leftarrow b\\
			&\text{\textbf{for }} i \leftarrow 0 \text{\textbf{ to }} b-1 \text{\textbf{ do}}\\
			&\qquad X_{i+1} \leftarrow X_{i} + 1\\
			&\qquad Y_{i+1} \leftarrow Y_{i} - 1\\
			&\text{\textbf{od}}
		\end{align*}

		Sei $a=6$ und $b=4$. Vervollständige zunächst folgende Tabelle:
		\begin{center}\begin{tabular}{l c r}
			 &$X_i$&$Y_i$\\
			\hline
			$i=0$ & & \\
			$i=1$ & & \\
			...& & 
		\end{tabular}\end{center}
	\end{exampleblock}
\end{frame}

\begin{frame}{Beispiel zur Schleifeninvariante}

	\begin{exampleblock}{Beispiel}
		\begin{columns}
			\begin{column}{0.4\textwidth}
				\begin{align*}
				&X_0 \leftarrow a\\
				&Y_0 \leftarrow b\\
				&\text{\textbf{for }} i \leftarrow 0 \text{\textbf{ to }} b-1 \text{\textbf{ do}}\\
				&\qquad X_{i+1} \leftarrow X_{i} + 1\\
				&\qquad Y_{i+1} \leftarrow Y_{i} - 1\\
				&\text{\textbf{od}}
				\end{align*}
			\end{column}

			\begin{column}{0.6\textwidth}
		\begin{center}
		Sei $a=6$ und $b=4$.\\[18pt]
		\begin{tabular}{C C C}
			 &X_i&Y_i\\
			\hline
			i=0&6&4 \\
			i=1&7&3 \\
			i=2&8&2\\
			i=3&9&1\\
			i=4&10&0\\ 
		\end{tabular}\end{center}
			\end{column}
			
		\end{columns}
		Was berechnet der Algorithmus?\\
		Was ist eine passende Schleifeninvariante?
	\end{exampleblock}
\end{frame}

\begin{frame}{Beispiel zur Schleifeninvariante}
	\begin{block}{Lösung}
		\begin{columns}
			\begin{column}{0.4\textwidth}
				\begin{align*}
				&X_0 \leftarrow a\\
				&Y_0 \leftarrow b\\
				&\text{\textbf{for }} i \leftarrow 0 \text{\textbf{ to }} b-1 \text{\textbf{ do}}\\
				&\qquad X_{i+1} \leftarrow X_{i} + 1\\
				&\qquad Y_{i+1} \leftarrow Y_{i} - 1\\
				&\text{\textbf{od}}
				\end{align*}
			\end{column}

			\begin{column}{0.6\textwidth}
		\begin{center}
		Sei $a=6$ und $b=4$.\\[18pt]
		\begin{tabular}{C C C}
			 &X_i&Y_i\\
			\hline
			i=0&6&4 \\
			i=1&7&3 \\
			i=2&8&2\\
			i=3&9&1\\
			i=4&10&0\\ 
		\end{tabular}\end{center}
			\end{column}
			
		\end{columns}
		Der Algorithmus berechnet $a+b$.\\
		Schleifeninvariante: $\forall i \in \nN_0 \text{ mit } i<b: X_i + Y_i = a+b$
	\end{block}
\end{frame}

\begin{frame}{Beispiel zur Schleifeninvariante}

	\begin{exampleblock}{Algorithmus von oben als while-Schleife}
	\small
		\begin{tabular}{rl}
			&\HTB{$x=a \wedge y=b$}\\
			&\HTB{...}\\
			&\textbf{while} $y \neq 0$\\
			&\textbf{do}\\
			&\qquad \HTB{...}\\
			&\qquad \HTB{...}\\
			&\qquad $y \leftarrow y-1$\\
			&\qquad \HTB{...}\\
			&\qquad $x \leftarrow x+1$\\
			&\qquad \HTB{...}\\
			&\textbf{od}\\
			&\HTB{...}\\
			&\HTB{$x=a+b$}\\
		\end{tabular}
		Erinnerung: Schleifeninvariante $x+y=a+b$.
	\end{exampleblock}
\end{frame}

\begin{frame}{Beispiel zur Schleifeninvariante}
	\begin{block}{Algorithmus von oben als while-Schleife}
		\small
		\begin{tabular}{rl}
			&\HTB{$x=a \wedge y=b$}\\
			&\HTB{$x+y=a+b$}\\
			&\textbf{while} $y \neq 0$\\
			&\textbf{do}\\
			&\qquad \HTB{$x+y=a+b \wedge y \neq 0$}\\
			&\qquad \HTB{$x+1+y-1=a+b$}\\
			&\qquad $y \leftarrow y-1$\\
			&\qquad \HTB{$x+1+y=a+b$}\\
			&\qquad $x \leftarrow x+1$\\
			&\qquad \HTB{$x+y=a+b$}\\
			&\textbf{od}\\
			&\HTB{$x+y=a+b  \wedge \neg(y \neq 0)$}\\
			&\HTB{$x=a+b$}\\
		\end{tabular}
	\end{block}
\end{frame}

\begin{frame}{Aufgabe zur Schleifeninvariante}

	\begin{exampleblock}{Aufgabe}
		\begin{columns}
			\begin{column}{0.35\textwidth}
				\small
				\begin{align*}
				&X_0 \leftarrow a, Y_0 \leftarrow b, P_0 \leftarrow 1\\
				&Z_0 \leftarrow X_0 \; mod \; 2\\
				&n \leftarrow 1 + \lceil log_2a \rceil\\
				&\text{\textbf{for }} i \leftarrow 0 \text{\textbf{ to }} n-1 \text{\textbf{ do}}\\
				&\qquad P_{i+1} \leftarrow P_{i} \cdot Y_{i}^{Z_i}\\
				&\qquad X_{i+1} \leftarrow X_{i} \; div \; 2\\
				&\qquad Y_{i+1} \leftarrow Y_{i}^{2}\\
				&\qquad Z_{i+1} \leftarrow X_{i+1} \; mod \; 2\\
				&\text{\textbf{od}}
				\end{align*}
			\end{column}

			\begin{column}{0.6\textwidth}
			\begin{center}
			Eingabe: $a,b \in \nN_+$\\[18pt]
			\begin{tabular}{C C C C C C}
			 	&X_i&Y_i&P_i&Z_i&n\\
				\hline
				i=0& & & & &\\
				i=1& & & & &\\
				...& & & & & \\
			\end{tabular}\end{center}
			\end{column}
			
		\end{columns}
		a) Erstelle Tabellen für $(i)a=1,b=2$; $(ii) a=2,b=2$; $(iii) a=2,b=3$.\\
		b) Welchen Wert hat $P_n$ am Ende?\\
		c) Gib eine Schleifeninvariante an.

	\end{exampleblock}
\end{frame}

\begin{frame}{Lösung der Aufgabe zur Schleifeninvariante}
	\begin{block}{Lösung a)}
	\small
	a)
		\begin{center}\begin{tabular}{C C C C C C}
			 	a=1,b=2&X_i&Y_i&P_i&Z_i&n\\
				\hline
				i=0&1&2&1&1&1\\
				i=1&0&4&2&0&1\\
			\end{tabular}\\[10pt]
		\begin{tabular}{C C C C C C}
			 	a=2,b=2&X_i&Y_i&P_i&Z_i&n\\
				\hline
				i=0&2&2&1&0&2\\
				i=1&1&4&1&1&2\\
				i=2&0&16&4&0&2\\
		\end{tabular}\\[10pt]
		\begin{tabular}{C C C C C C}
			 	a=2,b=3&X_i&Y_i&P_i&Z_i&n\\
				\hline
				i=0&2&3&1&0&2\\
				i=1&1&9&1&1&2\\
				i=2&0&81&9&0&2\\
			\end{tabular}\end{center}
	\end{block}
\end{frame}

\begin{frame}{Lösung der Aufgabe zur Schleifeninvariante}
	\begin{block}{Lösung b) + c)}
		b) $P_n = b^{a}$\\
		c) Schleifeninvariante: $P_i \cdot Y_{i}^{X_i} = b^{a}$
	\end{block}
\end{frame}
%%%%%%%%%% %%%%%%%%%%
%% Zusammenfassung
\section{}
%\subsection{Zusammenfassung}
	\begin{frame}{Was ihr jetzt kennen und können solltet\dots}
			\begin{itemize}
				\item Todo
			\end{itemize}
	
	\end{frame}
%% Ausblick
%\subsection{Ausblick}
	\begin{frame}{Ausblick}
		\begin{itemize}
			\item Todo
		\end{itemize}
	\end{frame}
%%%%%%%%%% %%%%%%%%%%
\section{}
\questionframe
\lastframe
\mode<handout>{\slideThanks}
\end{document}