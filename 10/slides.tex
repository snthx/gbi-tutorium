% ===== handout mode =====
% Comment/uncomment this line to toggle handout mode
% \newcommand{\handout}{}

% Comment/uncoment this line to toogle Mortitz mode
% \newcommand{\Moritz}{}

% Comment/uncomment this line to toggle handout mode
% \newcommand{\handout}{}

% by Stephan

%% Moritz mode or Stephan mode
\ifdefined \MoritzMode

% This is a configuration file with private, tutor specific information.
% It is therefore excluded from the Git repository so changes in this file will not conflict in git commits.

% Copy this Template, rename to config.tex and add your information below.

\newcommand{\mymail}{moritz.laupichler@student.kit.edu} % Consider using your named student Mail address to keep your u-Account private.

\newcommand{\myname}{\href{mailto:\mymail}{Moritz Laupichler}}

\newcommand{\mytutnumber}{27}

\newcommand{\mytutinfos}{Dienstags, 5. Block (15:45-17:15), SR 236}

\newcommand{\aboutMeFrame}{
	\begin{frame}{Euer Tutor}
		Name: \myname \\
		Alter: 19 Jahre \\
		Studiengang: Bachelor Informatik, 3. Semester \\
		\vspace{1cm}
		\pause 
		\centering{Kontakt: \href{mailto:\mymail}{\mymail}}
	\end{frame}
}

% Toggle Handout mode by including the following line before including style_tut
% and removing the % at the start (but do NOT remove it here, otherwise handout mode will always be on!)
% Please keep handout mode on in all commits!

% \newcommand{\handout}{} % Moritz mode
\fi
\ifdefined \AlexMode

% This is a configuration file with private, tutor specific information.
% It is therefore excluded from the Git repository so changes in this file will not conflict in git commits.

% Copy this Template, rename to config.tex and add your information below.

\newcommand{\mymail}{alexander.klug@student.kit.edu} % Consider using your named student Mail address to keep your u-Account private.

\newcommand{\myname}{\href{mailto:\mymail}{Alexander Klug}}

\newcommand{\mytutnumber}{30}

\newcommand{\mytutinfos}{Mittwochs, 3. Block (11:30-13:00), SR -107}

\newcommand{\aboutMeFrame}{
	\begin{frame}{Euer Tutor}
		Name: \myname \\
		Alter: 19 Jahre \\
		Studiengang: Bachelor Informatik, 3. Semester \\
		\vspace{1cm}
		\pause 
		\centering{Kontakt: \href{mailto:\mymail}{\mymail}}
	\end{frame}
}

% Toggle Handout mode by including the following line before including style_tut
% and removing the % at the start (but do NOT remove it here, otherwise handout mode will always be on!)
% Please keep handout mode on in all commits!

% \newcommand{\handout}{} % Alex Mode
\fi
\ifdefined \StephanMode

% This is a configuration file with private, tutor specific information.
% It is therefore excluded from the Git repository so changes in this file will not conflict in git commits.

% Copy this Template, rename to config.tex and add your information below.

\newcommand{\mymail}{stephan.bohr@student.kit.edu} % Consider using your named student Mail address to keep your u-Account private.

\newcommand{\myname}{\href{mailto:\mymail}{Stephan Bohr}}

\newcommand{\mytutnumber}{19}

\newcommand{\mytutinfos}{Dienstags, 3. Block (11:30-13:00), SR -108}

\newcommand{\aboutMeFrame}{
	\begin{frame}{Euer Tutor}
		Name: \myname \\
		Alter: 21 Jahre \\
		Studiengang: Bachelor Informatik, 5. Semester \\
		\vspace{1cm}
		\pause 
		\centering{Kontakt: \href{mailto:\mymail}{\mymail}}
	\end{frame}
} % Stephan mode
\fi

%% Beamer-Klasse im korrekten Modus
\ifdefined \handout
\documentclass[handout]{beamer} % Handout mode
\else
\documentclass{beamer}
\fi
%\documentclass[18pt,parskip]{beamer}

%% SLIDE FORMAT

% use 'beamerthemekit' for standard 4:3 ratio
% for widescreen slides (16:9), use 'beamerthemekitwide'

\usepackage{../templates/KIT-slides/beamerthemekit}
%\usepackage{../templates/KIT-slides/beamerthemekitwide}

%% TITLE PICTURE

% if a custom picture is to be used on the title page, copy it into the 'logos'
% directory, in the line below, replace 'mypicture' with the 
% filename (without extension) and uncomment the following line
% (picture proportions: 63 : 20 for standard, 169 : 40 for wide
% *.eps format if you use latex+dvips+ps2pdf, 
% *.jpg/*.png/*.pdf if you use pdflatex)

\titleimage{../figures/titleimage/brain}

%% TITLE LOGO

% for a custom logo on the front page, copy your file into the 'logos'
% directory, insert the filename in the line below and uncomment it

%\titlelogo{mylogo}

% (*.eps format if you use latex+dvips+ps2pdf,
% *.jpg/*.png/*.pdf if you use pdflatex)

%% TikZ INTEGRATION

% use these packages for PCM symbols and UML classes
% \usepackage{templates/tikzkit}
% \usepackage{templates/tikzuml}

%\usepackage{tikz}
%\usetikzlibrary{matrix}
%\usetikzlibrary{arrows.meta}
%\usetikzlibrary{automata}
%\usetikzlibrary{tikzmark}

%%%%%%%%%%%%%%%%%%%%%%%%%
% Libertine font (Original GBI font)
\usepackage[mono=false]{libertine}
%\renewcommand*\familydefault{\sfdefault}  %% Only if the base font of the document is to be sans serif

%% Schönere Schriften
\usepackage[TS1,T1]{fontenc}

%% Deutsche Silbentrennung und Beschriftungen
\usepackage[ngerman]{babel}

%% UTF-8-Encoding
\usepackage[utf8]{inputenc}

%% Bibliotheken für viele mathematische Symbole
\usepackage{amsmath, amsfonts, amssymb}

%% Anzeigetiefe für Inhaltsverzeichnis: 1 Stufe
\setcounter{tocdepth}{1}

%% Hyperlinks
\usepackage{hyperref}
% I don't know why, but this works and only includes sections and NOT subsections in the pdf-bookmarks.
\hypersetup{bookmarksdepth=subsection}

%% remove navigation symbols
\setbeamertemplate{navigation symbols}{}

%% switch between "ngerman" and "english" for German/English style date and logos
\selectlanguage{ngerman}

%% for invisible pause texts instead of dimming
\setbeamercovered{invisible}

\usepackage[german=swiss]{csquotes}

\usepackage{tabularx}
\usepackage{booktabs}

\usepackage{tikz}


% Problem: disabled itemize-icons
%\usepackage{enumitem}
% %\setlist[enumerate]{topsep=0pt,itemsep=-1ex,partopsep=1ex,parsep=1ex}
% \setlist[itemize]{noitemsep, nolistsep}
% \setlist[enumerate]{noitemsep, nolistsep}

% Mathmode no vertical space (https://tex.stackexchange.com/a/47403/146825)
\setlength{\abovedisplayskip}{0pt}
\setlength{\belowdisplayskip}{0pt}
\setlength{\abovedisplayshortskip}{0pt}
\setlength{\belowdisplayshortskip}{0pt}

%%%%%%%%%%%% Slides %%%%%%%%%%%%%%%%

\newcommand{\Moritz}[1]{
	\ifdefined \MoritzMode
	#1
	\fi
}

\newcommand{\Alex}[1]{
	\ifdefined \AlexMode
	#1
	\fi
}

\newcommand{\Stephan}[1]{
	\ifdefined \StephanMode
	#1
	\fi
}

\newcommand{\notMoritz}[1]{
	\Alex{#1} \Stephan{#1}
}

\newcommand{\notAlex}[1]{
	\Moritz{#1} \Stephan{#1}
}

\newcommand{\notStephan}[1]{
	\Alex{#1} \Moritz{#1}
}

%% Wochennummer
%\newcounter{weeknum}

%% Titelinformationen
%\title[GBI Tutorium, Woche \theweeknum]{Grundbegriffe der Informatik \\ Tutorium \mytutnumber}
%\subtitle{Termin \theweeknum \ | \mydate \\ \myname}
\author[\myname]{\myname}
\institute{Fakultät für Informatik}
%\date{\mydate}

%% Titel einfügen
\newcommand{\titleframe}{\frame{\titlepage}\addtocounter{framenumber}{-1}}


%% Alles starten mit \starttut{X}
%\newcommand{\starttut}[1]{\setcounter{weeknum}{#1}\titleframe\frame{\frametitle{Inhalt}\tableofcontents} \AtBeginSection[]{%
%\begin{frame}
%	\tableofcontents[currentsection]
%\end{frame}\addtocounter{framenumber}{-1}}}


%\newcommand{\framePrevEpisode}{
%	\begin{frame}
%		\centering
%		\textbf{In the previous episode of GBI...}
%	\end{frame}
%}

%% Roadmap frame
%table of contents
\newcommand{\roadmap}{
	\frame{\frametitle{Roadmap}\tableofcontents}}

 \AtBeginSection[]{%
\begin{frame}
	\frametitle{Roadmap}
	\tableofcontents[currentsection]
\end{frame}%\addtocounter{framenumber}{-1}
}


%% ShowMessage frame
\newcommand{\showmessage}[1]{\frame{\frametitle{\phantom{1em}}\centering\textbf{#1}}}

%% Fragen
%% Lastframe
\newcommand{\questionframe}{\showmessage{Fragen?}}

%% Lastframe
\newcommand{\lastframe}{\showmessage{Vielen Dank für Eure Aufmerksamkeit! \\Bis nächste Woche :)}}

%% Thanks frame
\newcommand{\slideThanks}{
	\begin{frame}
		\frametitle{Credits}
		\begin{block}{}
			An der Erstellung des Foliensatzes haben mitgewirkt:\\[1em]
			\Moritz{
			Stephan Bohr \\
			Alexander Klug \\
			}
			\Alex{
			Stephan Bohr \\
			Moritz Laupichler \\
			}
			\Stephan{
			Moritz Laupichler \\
			Alexander Klug \\
			}
			Katharina Wurz \\
			Thassilo Helmold \\
			Daniel Jungkind \\
			% Philipp Basler \\
			% Nils Braun \\
			% Dominik Doerner \\
			% Ou Yue \\
		\end{block}
	\end{frame}
}

%% Verbatim
%\usepackage{moreverb}

% GBI related stuff, but not beamer-stuff
\newcommand{\newpar}[1]{\paragraph{#1}\mbox{}\newline}

\newcommand{\nM}{\mathbb{M}}
\newcommand{\nR}{\mathbb{R}}
\newcommand{\nN}{\mathbb{N}}
\newcommand{\nZ}{\mathbb{Z}}
\newcommand{\nQ}{\mathbb{Q}}
\newcommand{\nB}{\mathbb{B}}
\newcommand{\nC}{\mathbb{C}}
\newcommand{\nK}{\mathbb{K}}
\newcommand{\nF}{\mathbb{F}}
\newcommand{\nG}{\mathbb{G}}
\newcommand{\nullel}{\mathcal{O}}
\newcommand{\einsel}{\mathds{1}}
\newcommand{\nP}{\mathbb{P}}
\newcommand{\Pot}{\mathcal{P}}
\renewcommand{\O}{\text{O}}

\newcommand{\bfmod}{\ensuremath{\text{\textbf{ mod }}}}
\renewcommand{\mod}{\bfmod}
\newcommand{\bfdiv}{\ensuremath{\text{\textbf{ div }}}}
\renewcommand{\div}{\bfdiv}


\newcommand{\set}[1]{\left\{ #1 \right\}}
\newcommand{\setc}[2]{\set{#1 \mid #2}}
\newcommand{\setC}[2]{\set{#1 \mid \text{ #2 }}}

\newcommand{\setsize}[1]{\; \mid #1 \mid \; }

\newcommand{\q}[1]{\textquotedblleft #1\textquotedblright}

% Zu zeigen, thx to http://www.matheboard.de/archive/155832/thread.html
\newcommand{\zz}{\ensuremath{\mathrm{z\kern-.29em\raise-0.44ex\hbox{z}}}:}

% Text above symbol
% https://tex.stackexchange.com/a/74132/146825
%
% \newcommand{\eqtext}[1]{\stackrel{\mathclap{\normalfont\mbox{#1}}}{=}}
% \newcommand{\gdwtext}[1]{\stackrel{\mathclap{\normalfont\mbox{#1}}}{\Leftrightarrow}}
% \newcommand{\imptext}[1]{\stackrel{\mathclap{\normalfont\mbox{#1}}}{\Rightarrow}}
% \newcommand{\symbtext}[2]{\stackrel{\mathclap{\normalfont\mbox{#2}}}{#1}}
\newcommand{\eqtext}[1]{\mathrel{\overset{\makebox%[0pt]
{\mbox{\normalfont\tiny #1}}}{=}}}
\newcommand{\gdwtext}[1]{\mathrel{\overset{\makebox%[0pt]
{\mbox{\normalfont\tiny #1}}}{\ensuremath{\Leftrightarrow}}}}
\newcommand{\imptext}[1]{\mathrel{\overset{\makebox%[0pt]
{\mbox{\normalfont\tiny #1}}}{\ensuremath{\Rightarrow}}}}
\newcommand{\symbtext}[2]{\mathrel{\overset{\makebox%[0pt]
{\mbox{\normalfont\tiny #2}}}{#1}}}

% qed symbol
\newcommand{\qedblack}{\hfill \ensuremath{\blacksquare}}
\newcommand{\qedwhite}{\hfill \ensuremath{\Box}}

% Aussagenlogik
% Worsch
\colorlet{alcolor}{blue}
\RequirePackage{tikz}
\usetikzlibrary{arrows.meta}
\newcommand{\alimpl}{\mathrel{\tikz[x={(0.1ex,0ex)},y={(0ex,0.1ex)},>={Classical TikZ Rightarrow[]}]{\draw[alcolor,->,line width=0.7pt,line cap=round] (0,0) -- (15,0);\path (0,-6);}}}
\newcommand{\alimp}{\alimpl}
\newcommand{\aleqv}{\mathrel{\tikz[x={(0.1ex,0ex)},y={(0ex,0.1ex)},>={Classical TikZ Rightarrow[]}]{\draw[alcolor,<->,line width=0.7pt,line cap=round] (0,0) -- (18,0);\path (0,-6);}}}
\newcommand{\aland}{\mathbin{\raisebox{-0.6pt}{\rotatebox{90}{\texttt{\color{alcolor}\char62}}}}}
\newcommand{\alor}{\mathbin{\raisebox{-0.8pt}{\rotatebox{90}{\texttt{\color{alcolor}\char60}}}}}
%\newcommand{\ali}[1]{_{\mathtt{\color{alcolor}#1}}}
\newcommand{\alv}[1]{\mathtt{\color{alcolor}#1}}
\newcommand{\alnot}{\mathop{\tikz[x={(0.1ex,0ex)},y={(0ex,0.1ex)}]{\draw[alcolor,line width=0.7pt,line cap=round,line join=round] (0,0) -- (10,0) -- (10,-4);\path (0,-8) ;}}}
\newcommand{\alP}{\alv{P}} %ali{#1}}
%\newcommand{\alka}{\negthinspace\hbox{\texttt{\color{alcolor}(}}}
\newcommand{\alka}{\negthinspace\text{\texttt{\color{alcolor}(}}}
%\newcommand{\alkz}{\texttt{\color{alcolor})}}\negthinspace}
\newcommand{\alkz}{\text{\texttt{\color{alcolor})}}\negthinspace}

% Thassilo
\newcommand{\BB}{\mathbb{B}}
\newcommand{\boder}{\alor}%{\ensuremath{\text{\;}\textcolor{blue}{\vee}}\text{\;}}
\newcommand{\bund}{\aland}%{\ensuremath{\text{\;}\textcolor{blue}{\wedge}}\text{\;}}
\newcommand{\bimp}{\alimp}%{\ensuremath{\text{\;}\textcolor{blue}{\to}}\text{\;}}
\newcommand{\bnot}{\alnot}%{\ensuremath{\text{\;}\textcolor{blue}{\neg}}\text{}}
\newcommand{\bgdw}{\aleqv}%{\ensuremath{\text{\;}\textcolor{blue}{\leftrightarrow}}\text{\;}}
\newcommand{\bone}{\ensuremath{\textcolor{blue}{1}}\text{}}
\newcommand{\bzero}{\ensuremath{\textcolor{blue}{0}}\text{}}
\newcommand{\bleftBr}{\alka}%{\ensuremath{\textcolor{blue}{(}}\text{}}
\newcommand{\brightBr}{\alkz}%{\ensuremath{\textcolor{blue}{)}}\text{}}

\newcommand{\val}{\hbox{\textit{val}}}

\newcommand{\VarAL}{\hbox{\textit{Var}}_{AL}}
\newcommand{\ForAL}{\hbox{\textit{For}}_{AL}}

% Validierungsfunktion val_i
\newcommand{\vali}[1]{\ensuremath{\val_I(#1)}}

% Boolsche Funktion b_
\newcommand{\bfnot}[1]{\ensuremath{b_{\bnot}(#1)}}
\newcommand{\bfand}[2]{\ensuremath{b_{\bund}(#1,#2)}}
\newcommand{\bfor}[2]{\ensuremath{b_{\boder}(#1,#2)}}
\newcommand{\bfimp}[2]{\ensuremath{b_{\bimp}(#1,#2)}}

% Aussagenkalkül
\newcommand{\AAL}{A_{AL}}
\newcommand{\LAL}{\hbox{\textit{For}}_{AL}}
\newcommand{\AxAL}{\hbox{\textit{Ax}}_{AL}}
\newcommand{\MP}{\hbox{\textit{MP}}}

% Prädikatenlogik
% die nachfolgenden Sachen angepasst an cmtt
\newlength{\ttquantwd}
\setlength{\ttquantwd}{1ex}
\newlength{\ttquantht}
\setlength{\ttquantht}{6.75pt}
\def\plall{%
  \tikz[line width=0.67pt,line cap=round,line join=round,baseline=(B),alcolor] {
    \draw (-0.5\ttquantwd,\ttquantht) -- node[coordinate,pos=0.4] (lll){} (-0.25pt,-0.0pt) -- (0.25pt,-0.0pt) -- node[coordinate,pos=0.6] (rrr){} (0.5\ttquantwd,\ttquantht);
    \draw (lll) -- (rrr);
    \coordinate (B) at (0,-0.35pt);
  }%
}
\def\plexist{%
  \tikz[line width=0.67pt,line cap=round,line join=round,baseline=(B),alcolor] {
    \draw (-0.9\ttquantwd,\ttquantht) -- (0,\ttquantht) -- node[coordinate,pos=0.5] (mmm){} (0,0) --  (-0.9\ttquantwd,0);
    \draw (mmm) -- ++(-0.75\ttquantwd,0);
    \coordinate (B) at (0,-0.35pt);
  }\ensuremath{\,}%
}
\let\plexists=\plexist
\newcommand{\NT}[1]{\ensuremath{\langle\mathrm{#1} \rangle}}
\newcommand{\CPL}{\text{\itshape Const}_{PL}}
\newcommand{\FPL}{\text{\itshape Fun}_{PL}}
\newcommand{\RPL}{\text{\itshape Rel}_{PL}}
\newcommand{\VPL}{\text{\itshape Var}_{PL}}
\newcommand{\plka}{\alka}
\newcommand{\plkz}{\alkz}
%\newcommand{\plka}{\plfoo{(}}
%\newcommand{\plkz}{\plfoo{)}}
\newcommand{\plcomma}{\hbox{\texttt{\color{alcolor},}}}
\newcommand{\pleq}{{\color{alcolor}\,\dot=\,}}

\newcommand{\plfoo}[1]{\mathtt{\color{alcolor}#1}}
\newcommand{\plc}{\plfoo{c}}
\newcommand{\pld}{\plfoo{d}}
\newcommand{\plf}{\plfoo{f}}
\newcommand{\plg}{\plfoo{g}}
\newcommand{\plh}{\plfoo{h}}
\newcommand{\plx}{\plfoo{x}}
\newcommand{\ply}{\plfoo{y}}
\newcommand{\plz}{\plfoo{z}}
\newcommand{\plR}{\plfoo{R}}
\newcommand{\plS}{\plfoo{S}}
\newcommand{\ar}{\mathrm{ar}}

\newcommand{\bv}{\mathrm{bv}}
\newcommand{\fv}{\mathrm{fv}}

\def\word#1{\hbox{\textcolor{blue}{\texttt{#1}}}}
%\let\literal\word
\def\mword#1{\hbox{\textcolor{blue}{$\mathtt{#1}$}}}  % math word
\def\sp{\scalebox{1}[.5]{\textvisiblespace}}
\def\wordsp{\word{\sp}}


\newcommand{\W}{\ensuremath{\hbox{\textbf{w}}}\xspace}
\newcommand{\F}{\ensuremath{\hbox{\textbf{f}}}\xspace}
\newcommand{\WF}{\ensuremath{\{\W,\F\}}\xspace}
\newcommand{\valDIb}{\val_{D,I,\beta}}

\newcommand{\impl}{\ifmmode\ensuremath{\mskip\thinmuskip\Rightarrow\mskip\thinmuskip}\else$\Rightarrow$\fi\xspace}
\newcommand{\Impl}{\ifmmode\implies\else$\Longrightarrow$\fi\xspace}

\newcommand{\derives}{\Rightarrow}

\newcommand{\gdw}{\ifmmode\mskip\thickmuskip\Leftrightarrow\mskip\thickmuskip\else$\Leftrightarrow$\fi\xspace}
\newcommand{\Gdw}{\ifmmode\iff\else$\Longleftrightarrow$\fi\xspace}

\newcommand*{\from}{\colon}
\newcommand{\functionto}{\longrightarrow}


\newcommand{\LTer}{L_{\text{\itshape Ter}}}
\newcommand{\LRel}{L_{\text{\itshape Rel}}}
\newcommand{\LFor}{L_{\text{\itshape For}}}
\newcommand{\NTer}{N_{\text{\itshape Ter}}}
\newcommand{\NRel}{N_{\text{\itshape Rel}}}
\newcommand{\NFor}{N_{\text{\itshape For}}}
\newcommand{\PTer}{P_{\text{\itshape Ter}}}
\newcommand{\PRel}{P_{\text{\itshape Rel}}}
\newcommand{\PFor}{P_{\text{\itshape For}}}

\newcommand{\sgn}{\mathop{\text{sgn}}}

\newcommand{\lang}[1]{\ensuremath{\langle#1\rangle}}

\newcommand{\literal}[1]{\hbox{\textcolor{blue!95!white}{\textup{\texttt{\scalebox{1.11}{#1}}}}}}
\let\hashtag\#
\renewcommand{\#}[1]{\literal{#1}}

\def\blank{\ensuremath{\openbox}}
\def\9{\blank}
\newcommand{\io}{\!\mid\!}


\providecommand{\fspace}{\mathord{\text{space}}}
\providecommand{\fSpace}{\mathord{\text{Space}}}
\providecommand{\ftime}{\mathord{\text{time}}}
\providecommand{\fTime}{\mathord{\text{Time}}}

\newcommand{\fnum}{\text{num}}
\newcommand{\fNum}{{\text{Num}}}

\def\Pclass{\text{\bfseries P}}
\def\PSPACE{\text{\bfseries PSPACE}}



\title[Graphen]{10. Tutorium\\ Graphen}
\subtitle{Grundbegriffe der Informatik, Tutorium \hashtag\mytutnumber}
\date{\today}

\usetikzlibrary{matrix}
\usetikzlibrary{arrows.meta}
\usetikzlibrary{automata}
\usetikzlibrary{tikzmark}

\begin{document}
\titleframe

\begin{frame}{Organisatorisches}
\begin{itemize}
    \item Probeklausur diesen Freitag
\end{itemize}
\end{frame}

\roadmap

%%%%%%%%%% %%%%%%%%%%

\section{Graphen}
\subsection{Gerichtete Graphen}
\begin{frame}{Gerichtete Graphen}
	\begin{block}{Def.: Gerichteter Graph}
		Ein \textbf{gerichteter Graph} $G$ ist ein Paar $G=(V,E)$.
		\begin{itemize}
			\item $V$ heißt \textbf{Knotenmenge}.
				\begin{itemize}
					\item Knoten haben Bezeichner (entspricht Beschriftung), z.B. $V=\set{a,b,c,...}, V=\set{1,2,3,...}, V=\set{start, z1, z2, end}, ...$
				\end{itemize}
			\item $E \subseteq V \times V$ heißt \textbf{Kantenmenge}.
				\begin{itemize}
					\item Kanten sind also Paare aus Knoten in $V$, z.B. $E=\set{(a,b),\; (b,c)}, ...$
					\item $(x,y) \in E \Rightarrow$ Es gibt eine Kante vom Knoten $x$ zum Knoten $y$.
				\end{itemize}
		\end{itemize}
	\end{block}
\end{frame}

\begin{frame}{Gerichtete Graphen}

	\begin{exampleblock}{Beispiel}
		Graph von eben:
		\begin{figure}[H]
		\begin{tikzpicture}[->,>=stealth,baseline=-5mm]
		\matrix[matrix of math nodes,nodes={draw,circle,minimum size=10mm,inner sep=2pt},row sep=15mm,column sep=15mm,ampersand replacement=\&]
		{
			|(1)| 1 \& |(5)| 5 \& |(3)| 3 \\
			|(0)| 0 \& |(2)| 2 \& |(4)| 4 \\
		};
		\draw  (0) to [bend left] (1);
		\draw  (1) to [bend left] (0);
		\draw  (1) -- (2);
		
		\draw  (4) -- (5);
		\draw  (4) to [bend left] (3);
		\draw  (3) to [bend left] (4);
		\end{tikzpicture}
		\end{figure}

	Graph definiert durch $G = (V, E)$ mit $V = \{0,1,2,3,4,5\}$ und $E = \{(0,1), (1,0), (1,2), (3,4), (4,3) ,(4,5)\}$.
	\end{exampleblock}
\end{frame}

\begin{frame}{Aufgabe: Gerichtete Graphen zeichnen}
	\begin{exampleblock}{Aufgabe}
		Zeichnet die Graphen $G_i = (V, E_i)$ mit $V = \nZ_4$ und
		\begin{enumerate}
			\item $E_1 = \{(0,1), (0,2), (0,3), (1,2), (1,3), (2,2), (2,3), (3,2)\}$
			\item $E_2 = \{(0,1), (0,2), (0,3) \}$
			\item $E_3 = \emptyset$
			\item $E_4 = V \times V$
			\item $E_5 = \{(0,1), (1,2), (1,3)\}$
		\end{enumerate}
	\end{exampleblock}
	
\end{frame}

\begin{frame}{Gerichtete Graphen: Pfade}
	\begin{block}{Def.: Pfad}
		Sei $G=(V,E)$ ein gerichteter Graph. Ein n-Tupel $p=(v_0,v_1,...,v_n)$ mit $v_i \in V \; (i \in \set{0,...,n})$ heißt \textbf{Pfad von $v_0$ nach $v_n$ der Länge $n$}, gdw. \\
		\[
			\forall i \in \set{0, ..., n-1}: (v_i, v_{i+1}) \in E
		\]
		\\[12pt]
		Die Länge des Pfades ist also gleich der Anzahl der Kanten (!).
	\end{block}

	\begin{exampleblock}{Beispiele}
		Auf Graph oben existieren z.B. Pfade $p_1=(3,4,5)$ oder $p_2=(1,0,1,2)$.
	\end{exampleblock}
\end{frame}

\begin{frame}{Aufgabe: Pfade erkennen}
	\begin{exampleblock}{Aufgabe}
	\begin{center}
		
	\begin{columns}
		\begin{column}{0.475\textwidth}
			$G_1:$
			\begin{figure}[H]
			\begin{tikzpicture}[->,>=stealth,baseline=-5mm]
				\matrix[matrix of math nodes,nodes={draw,circle,minimum size=8mm,inner sep=2pt},row sep=10mm,column sep=10mm,ampersand replacement=\&]
				{
				    		\& |(4)| 4 \\
					|(0)| 0 \& |(1)| 1 \\
					|(2)| 2 \& |(3)| 3 \\
				};
				\draw  (0) -- (2);
				\draw  (3) -- (2);
				\draw  (1) -- (3);
				\draw  (4) -- (1);
			\end{tikzpicture}
			\end{figure}
		\end{column}
		\begin{column}{0.475\textwidth}
			$G_2:$
			\begin{figure}[H]
			\begin{tikzpicture}[->,>=stealth,baseline=-5mm]
				\matrix[matrix of math nodes,nodes={draw,circle,minimum size=8mm,inner sep=2pt},row sep=10mm,column sep=10mm,ampersand replacement=\&]
				{
					|(0)| 0 \& |(1)| 1 \\
					|(2)| 2 \& |(3)| 3 \\
				};
				\draw  (0) -- (2);
				\draw  (3) -- (2);
				\draw  (1) -- (3);
				\draw  (2) -- (1);
			\end{tikzpicture}
			\end{figure}
		\end{column}
	\end{columns}
	\end{center}
	Gib alle Pfade auf $G_1$ und $G_2$ an.

	\end{exampleblock}
\end{frame}
	
\begin{frame}
	\begin{block}{Lösung}
	\begin{itemize}
		\item Pfade auf $G_1$: \begin{itemize}
			\item $p_1=(4,1), \; p_2=(1,3), \; p_3=(3,2), \; p_4=(0,2)$,
			\item $p_5=(4,1,3), \; p_6=(1,3,2)$,
			\item $p_7=(4,1,3,2)$
			\end{itemize}
		\item Pfade auf $G_2$: ???
	\end{itemize}
	\end{block}
		
	Warum können wir die Pfade auf $G_2$ nicht angeben?\newline
	\pause

	$G_2$ enthält einen \textbf{Zyklus}...

\end{frame}

\begin{frame}{Gerichtete Graphen: Zyklen}
	\begin{block}{Def.: Zyklus}
		Ein Pfad $p=(v_0,...,v_n)$ mit $v_0=v_n$ heißt \textbf{geschlossen}.
		\begin{itemize}
			\item Ein geschlossener Pfad mit $n \ge 1$ heißt \textbf{Zyklus}.
			\item Enthält ein Graph keine Zyklen, so heißt er \textbf{azyklisch} (DAG).
		\end{itemize}		
	\end{block}
	\pause
	\begin{block}{Def.: Wiederholungsfreiheit}
		Ein Pfad $p=(v_0,...,v_n)$ heißt \textbf{wiederholungsfrei}, gdw. 
		\begin{itemize}
			\item $v_0,...,v_{n-1}$ und $v_1,...,v_n$ jeweils paarweise verschieden sind.
			\item D.h. alle Knoten müssen sich unterscheiden, nur $v_0$ und $v_n$ dürfen gleich sein.
		\end{itemize}
		\medskip
		Ein \textbf{einfacher Zyklus} ist ein wiederholungsfreier Zyklus.
	\end{block}
	
\end{frame}
		
\begin{frame}{Aufgabe: Graphen konstruieren}
	\begin{exampleblock}{Aufgabe}
		Finde einen Graphen $G=(V,E)$, der die folgenden Eigenschaften hat:
		\begin{itemize}
		 	\item G ist streng zusammenhängend\footnote{$G=(V,E)$ streng zusammenhängend $:\Leftrightarrow$ $\forall (x,y) \in V^{2}: \exists$ Pfad von $x$ nach $y$}.
		 	\item G hat mindestens zwei unterschiedliche Zyklen.
		 	\item G hat mindestens einen einfachen Zyklus.
		 \end{itemize}
		 \medskip
		 Gib die Mengen $V$ und $E$ an und zeichne den Graphen G.
	\end{exampleblock}
\end{frame}
	
\begin{frame}{Aufgabe: Graphen konstruieren}
	\begin{block}{Lösung}
		Eine mögliche Lösung ist $G=(V,E)$ mit $V=\set{1,2}$ und $E=\set{(1,2) \; (2,1) \; (1,1)}$. \\[18pt]
		$G:$

		\begin{center}
		\begin{figure}[H]
			\begin{tikzpicture}[->,>=stealth,baseline=-5mm]
				\matrix[matrix of math nodes,nodes={draw,circle,minimum size=8mm,inner sep=2pt},row sep=10mm,column sep=10mm,ampersand replacement=\&]
				{
					|(1)| 1 \& \\
					|(2)| 2 \& \\
				};
				\draw  (1) to [bend left] (2);
				\draw  (2) to [bend left] (1);
				\path  (1) edge [loop above] ();
			\end{tikzpicture}
			\end{figure}
		\end{center}
	\end{block}
\end{frame}

\begin{frame}{Gerichtete Graphen: Knotengrad}
	\begin{block}{Def.: Knotengrade}
		Sei $G=(V,E)$ ein gerichteter Graph und $v \in V$.
		\begin{itemize}
			\item \textbf{Eingangsgrad von $v$}: Anzahl eingehender Kanten, $d^{-}(v) = \setsize{\setc{x}{(x,v)\in E}}$
			\item \textbf{Ausgangsgrad von $v$}: Anzahl ausgehender Kanten, $d^{+}(v) = \setsize{\setc{x}{(v,x)\in E}}$
			\item \textbf{Grad von $v$}: $d(v) = d^{+} + d^{-}$
		\end{itemize}
	\end{block}
\end{frame}

\begin{frame}{Gerichtete Graphen: Gerichtete Bäume}
	\begin{block}{Def.: Baum}
		Ein gerichteter Graph $G=(V,E)$ heißt \textbf{gerichteter Baum}, gdw. es eine \textbf{Wurzel} $r \in V$ gibt, von der aus es genau einen Pfad zu jedem Knoten gibt.\\
		Oder Formell: $\exists r \in V: \forall x \in V: \exists_1 Pfad \; p=(r,...,x) $
	\end{block}

	\begin{exampleblock}{Beispiel}
		Ein Ableitungsbaum einer kontextfreien Grammatik ist ein gerichteter Baum mit dem Startsymbol als Wurzel.
	\end{exampleblock}
\end{frame}

\begin{frame}{Gerichtete Graphen: Gerichtete Bäume}
	\begin{center}
		\begin{figure}[H]
			\begin{tikzpicture}[->,>=stealth,baseline=-5mm]
				\matrix[matrix of math nodes,nodes={draw,circle,minimum size=8mm,inner sep=2pt},row sep=10mm,column sep=10mm,ampersand replacement=\&]
				{
					        \& |(1)| 1 \& 		  \&		\\
					|(2)| 2 \& |(3)| 3 \& |(4)| 4 \&		\\
					|(5)| 5 \& |(6)| 6 \& |(7)| 7 \& |(8)| 8\\
				};
				\draw  (1) -- (2);
				\draw  (1) -- (3);
				\draw  (1) -- (4);
				\draw  (2) -- (5);
				\draw  (2) -- (6);
				\draw  (4) -- (7);
				\draw  (4) -- (8);
			\end{tikzpicture}
		\end{figure}
	\end{center}

		Wir nennen in einem gerichteten Baum die Knoten mit Ausgangsgrad = 0 \textbf{Blätter} (hier: $3,5,6,7,8$).\\
		Wir nennen in einem gerichteten Baum die Knoten mit Ausgangsgrad > 0 \textbf{innere Knoten} (hier: $1,2,4$).

\end{frame}

\subsection{Ungerichtete Graphen}

\begin{frame}{Ungerichtete Graphen}
	\begin{block}{Def.: Ungerichteter Graph}
		Ein ungerichteter Graph ist ein Paar $U=(V,E)$.
		\begin{itemize}
			\item $V$ heißt \textbf{Knotenmenge}.
				\begin{itemize}
					\item Knoten haben Bezeichner (entspricht Beschriftung), z.B. $V=\set{a,b,c,...}, V=\set{1,2,3,...}, V=\set{start, z1, z2, end}, ...$
				\end{itemize}
			\item $E \subseteq \setc{\set{x,y}}{x \in V \wedge y \in V}$ heißt \textbf{Kantenmenge}.
				\begin{itemize}
					\item Kanten sind also \textit{Mengen} aus zwei Knoten in $V$, z.B. $E=\set{\set{a,b},\; \set{b,c}}, ...$
					\item Unterschied zu gerichteten Graphen ist also, dass die Reihenfolge bei Kanten egal ist.
					\item $\set{x,y} \in E \Rightarrow$ Es gibt eine Kante zwischen dem Knoten $x$ und dem Knoten $y$. $x$ und $y$ heißen dann \textbf{adjazent}.
				\end{itemize}
		\end{itemize}
	\end{block}
\end{frame}

\begin{frame}{Ungerichtete Graphen}
	Beim Zeichnen lässt man die Pfeile weg, da die Kanten keine Richtung haben:\\
	Bsp.: $U=(V,E)$ mit $V=\set{1,2,3}$ und $E=\set{\set{1,2}, \set{1,3}, \set{3}}$.
	\begin{center}
		\begin{figure}[H]
			\begin{tikzpicture}[>=stealth,baseline=-5mm, every loop/.style={}]
				\matrix[matrix of math nodes,nodes={draw,circle,minimum size=8mm,inner sep=2pt},row sep=10mm,column sep=10mm,ampersand replacement=\&]
				{
								\& |(1)| 1	\&  \\
					|(2)| 2 	\&			\& |(3)| 3  \\
				};
				\draw  (1) -- (2);
				\draw  (1) -- (3);
				\draw  (3) edge [loop right] ();
			\end{tikzpicture}
		\end{figure}
	\end{center}
	Hier kommt eine \textbf{Schlinge} vor (bei $3$), das heißt eine Kante von einem Knoten zu sich selbst: $\set{x}=\set{x,x} \in E$.\\ Ungerichtete Graphen ohne Schlingen heißen \textbf{schlingenfrei}.
\end{frame}

\begin{frame}{Ungerichtete Graphen: Wege}
	\begin{block}{Def.: Weg}
		Was beim gerichteten Graphen Pfad heißt, heißt beim ungerichteten Graphen \textbf{Weg}. Ein Weg ist also ein n-Tupel $p=(v_0,...,v_n)$ mit
		\begin{itemize}
		 	\item $v_i\in V \: (i\in\set{0,...,n})$ und
		 	\item $\forall i \in \set{0,...,n-1}: \set{v_i,v_{i+1}} \in E$.
		\end{itemize}
		\medskip
		\pause
		Einige Analoga zum Pfad:
		\begin{itemize}
			\item \textbf{Länge} eines Weges ist wieder die Anzahl der Kanten, nicht der Knoten.\\
			\item \textbf{Wiederholungsfreiheit} ist analog zum Pfad definiert.
			\item Ein Weg ist genau wie beim Pfad \textbf{geschlossen}, wenn $v_0 = v_n$. Ein geschlossener Weg heißt \textbf{Kreis}.
		\end{itemize}
		Ein wiederholungsfreier Kreis mit \textit{mindestens drei Knoten} heißt \textbf{einfacher Kreis}.
	\end{block}
\end{frame}


\begin{frame}{Ungerichtete Graphen: Äquivalente gerichtete Graphen}
	Ungerichtete Graphen können in äquivalente gerichtete Graphen umgewandelt werden durch:
	\begin{enumerate}
	 	\item Sei $U=(V,E)$ ein ungerichteter Graph. Definiere gerichteten Graphen $G=(V_G, E_G)$ mit:
	 	\item $V_G=V$
	 	\item $E_G = \setc{(x,y)}{\set{x,y} \in E}$
	 \end{enumerate} 

	 \begin{columns}
	 	\begin{column}{0.475\textwidth}
	 		\begin{center}
			\begin{figure}[H]
			\begin{tikzpicture}[>=stealth,baseline=-5mm, every loop/.style={}]
				\matrix[matrix of math nodes,nodes={draw,circle,minimum size=8mm,inner sep=2pt},row sep=10mm,column sep=10mm,ampersand replacement=\&]
				{
								\& |(1)| 1	\&  \\
					|(2)| 2 	\&			\& |(3)| 3  \\
				};
				\draw  (1) -- (2);
				\draw  (1) -- (3);
				\draw  (3) edge [loop right] ();
			\end{tikzpicture}
			\end{figure}
			\end{center}
	 	\end{column}
	 	\begin{column}{0.475\textwidth}
	 		\begin{center}
			\begin{figure}[H]
			\begin{tikzpicture}[->,>=stealth,baseline=-5mm]
				\matrix[matrix of math nodes,nodes={draw,circle,minimum size=8mm,inner sep=2pt},row sep=10mm,column sep=10mm,ampersand replacement=\&]
				{
								\& |(1)| 1	\&  \\
					|(2)| 2 	\&			\& |(3)| 3  \\
				};
				\draw  (1) to [bend right] (2);
				\draw  (2) to [bend right] (1);
				\draw  (1) to [bend right] (3);
				\draw  (3) to [bend right] (1);
				\draw  (3) edge [loop right] ();
			\end{tikzpicture}
			\end{figure}
			\end{center}
	 	\end{column}
	 \end{columns}
	 
\end{frame}

\begin{frame}{Ungerichtete Graphen: Grad}
	\begin{block}{Def.: Grad}
		Bei ungerichteten Graphen gibt es keinen Eingangs- oder Ausgangsgrad. Es wird nur der Grad $d$ definiert mit:
		\[
			d(x)=\setsize{\setc{y}{y \neq x \wedge \set{x,y}\in E}} + 
			\begin{cases} 
				2, & \text{ falls } \set{x,x} \in E \\
				0, & \text{ sonst}
			\end{cases}
		\]
		\medskip
		Also entspricht der Grad den Kantenenden am Knoten $x$.
	\end{block}
\end{frame}

\subsection{Isomorphie, Adjazenz- und Wegematrix}

\begin{frame}{Isomorphie}
	Zwei Graphen heißen \textbf{isomorph}, wenn sie \enquote{bis auf eine Umbenennung der Knoten identisch sind}, also die gleiche Struktur besitzen.\\

	Beispiel:

	\usetikzlibrary{shapes.geometric}

	\begin{columns}
		\begin{column}{.475\textwidth}
			\begin{center}
			\begin{figure}[H]
				\begin{tikzpicture}[mystyle/.style={draw,circle,minimum size=8mm,inner sep=2pt}]
				\node[regular polygon,regular polygon sides=5,minimum size=3cm] (p) {};
				\foreach\x in {1,...,5}{\node[mystyle] (\x) at (p.corner \x){\x};}
				\draw  (1) -- (2);
				\draw  (1) -- (3);
				\draw  (1) -- (4);
				\draw  (1) -- (5);
				\draw  (2) -- (4);
				\draw  (3) -- (5);
				\end{tikzpicture}

			
			\end{figure}
			\end{center}
		\end{column}

		\begin{column}{.475\textwidth}
			\begin{center}
			\begin{figure}[H]
				\begin{tikzpicture}[mystyle/.style={draw,circle,minimum size=8mm,inner sep=2pt}]
				\node[regular polygon,regular polygon sides=5,minimum size=3cm] (p) {};
				\foreach\x in {1,...,5}{\node[mystyle] (\x) at (p.corner \x){\x};}
				\draw  (1) -- (2);
				\draw  (1) -- (5);
				\draw  (2) -- (5);
				\draw  (2) -- (3);
				\draw  (2) -- (4);
				\draw  (3) -- (4);
				\end{tikzpicture}
			\end{figure}
			\end{center}
		\end{column}	
	\end{columns}
\end{frame}

\begin{frame}{Adjazenzmatrix}
	\begin{block}{Def.: Adjazenzmatrix}
		\begin{columns}
			\begin{column}{0.475\textwidth}
				Sei $G=(V_G,E_G)$ gerichteter Graph.\\
				Adjazenzmatrix $A \in \set{0,1}^{\mid V_G \mid \times \mid V_G \mid}$\\[12pt]
				\[
					A_{ij} = 
					\begin{cases}
						1, & \text{ falls } (i,j)\in E_G\\
						0, & \text{ falls } (i,j)\notin E_G
					\end{cases}
				\]
			\end{column}
			\begin{column}{0.475\textwidth}
				Sei $U=(V_U,E_U)$ ungerichteter Graph.\\
				Adjazenzmatrix $A \in \set{0,1}^{\mid V_U \mid \times \mid V_U \mid}$\\[12pt]
				\[
					A_{ij} = 
					\begin{cases}
						1, & \text{ falls } \set{i,j}\in E_U\\
						0, & \text{ falls } \set{i,j}\notin E_U
					\end{cases}
				\]
			\end{column}
		\end{columns}
	\end{block}
\end{frame}

\begin{frame}{Adjazenzmatrix}
	\begin{exampleblock}{Beispiele}
		\begin{columns}
			\begin{column}{0.475\textwidth}
				\begin{center}
					\begin{figure}[H]
				\begin{tikzpicture}[->,>=stealth,baseline=-5mm]
				\matrix[matrix of math nodes,nodes={draw,circle,minimum size=8mm,inner sep=2pt},row sep=10mm,column sep=10mm,ampersand replacement=\&]
				{
					|(0)| 0 \& |(1)| 1 \\
					|(2)| 2 \& |(3)| 3 \\
				};
				\draw  (0) -- (2);
				\draw  (0) edge [loop above] ();
				\draw  (1) -- (3);
				\draw  (2) to [bend right] (1);
				\draw  (1) to [bend right] (2);
				\end{tikzpicture}
				\end{figure}
				$\bordermatrix{
					  & 0 & 1 & 2 & 3 \cr
					0 & 1 & 0 & 1 & 0 \cr
					1 & 0 & 0 & 1 & 1 \cr
					2 & 0 & 1 & 0 & 0 \cr
					3 & 0 & 0 & 0 & 0 \cr
					}$
				\end{center}
			\end{column}
			\begin{column}{0.475\textwidth}
				\begin{center}
				\begin{figure}[H]
				\begin{tikzpicture}[>=stealth,baseline=-5mm, every loop/.style={}]
				\matrix[matrix of math nodes,nodes={draw,circle,minimum size=8mm,inner sep=2pt},row sep=10mm,column sep=10mm,ampersand replacement=\&]
				{
								\& |(1)| 1	\&  \\
					|(2)| 2 	\&			\& |(3)| 3  \\
				};
				\draw  (1) -- (2);
				\draw  (1) -- (3);
				\draw  (3) edge [loop right] ();
			\end{tikzpicture}
			\end{figure}
			$\bordermatrix{
					  & 0 & 1 & 2 & 3 \cr
					0 & 0 & 0 & 0 & 0 \cr
					1 & 0 & 0 & 1 & 1 \cr
					2 & 0 & 1 & 0 & 0 \cr
					3 & 0 & 1 & 0 & 1 \cr
					}$
			\end{center}
			\end{column}
		\end{columns}
	\end{exampleblock}
\end{frame}

\begin{frame}{Wegematrix}
	\begin{block}{Def.: Wegematrix}
		\begin{columns}
			\begin{column}{0.95\textwidth}
				Sei $G=(V_G,E_G)$ gerichteter Graph.\\
				Wegematrix $W \in \set{0,1}^{\mid V_G \mid \times \mid V_G \mid}$\\[12pt]
				\[
					W_{ij} = 
					\begin{cases}
						1, & \parbox[t]{.6\textwidth}{ falls es in $E_G$ Pfad von $i$ nach $j$ gibt.}\\
						0, & \parbox[t]{.6\textwidth}{ sonst}
					\end{cases}
				\]
			\end{column}
			% \begin{column}{0.475\textwidth}
			% 	Sei $U=(V_U,E_U)$ gerichteter Graph.\\
			% 	Wegematrix $W \in \set{0,1}^{\mid V_U \mid \times \mid V_U \mid}$\\[12pt]
			% 	\[
			% 		W_{ij} = 
			% 		\begin{cases}
			% 			1, & \parbox[t]{.6\textwidth}{ falls es in $E_U$ Pfad von $i$ nach $j$ gibt.}\\
			% 			0, & \parbox[t]{.6\textwidth}{ sonst}
			% 		\end{cases}
			% 	\]
			% \end{column}
		\end{columns}
	\end{block}
\end{frame}

\begin{frame}{Wegematrix}
	\begin{exampleblock}{Beispiele}
		\begin{columns}
			\begin{column}{0.95\textwidth}
				\begin{center}
					\begin{figure}[H]
				\begin{tikzpicture}[->,>=stealth,baseline=-5mm]
				\matrix[matrix of math nodes,nodes={draw,circle,minimum size=8mm,inner sep=2pt},row sep=10mm,column sep=10mm,ampersand replacement=\&]
				{
					|(0)| 0 \& |(1)| 1 \\
					|(2)| 2 \& |(3)| 3 \\
				};
				\draw  (0) -- (2);
				\draw  (0) edge [loop above] ();
				\draw  (1) -- (3);
				\draw  (2) to [bend right] (1);
				\draw  (1) to [bend right] (2);
				\end{tikzpicture}
				\end{figure}
				$\bordermatrix{
					  & 0 & 1 & 2 & 3 \cr
					0 & 1 & 1 & 1 & 1 \cr
					1 & 0 & 1 & 1 & 1 \cr
					2 & 0 & 1 & 1 & 1 \cr
					3 & 0 & 0 & 0 & 1 \cr
					}$
				\end{center}
			\end{column}
			% \begin{column}{0.475\textwidth}
			% 	\begin{center}
			% 	\begin{figure}[H]
			% 	\begin{tikzpicture}[>=stealth,baseline=-5mm, every loop/.style={}]
			% 	\matrix[matrix of math nodes,nodes={draw,circle,minimum size=8mm,inner sep=2pt},row sep=10mm,column sep=10mm,ampersand replacement=\&]
			% 	{
			% 					\& |(1)| 1	\&  \\
			% 		|(2)| 2 	\&			\& |(3)| 3  \\
			% 	};
			% 	\draw  (1) -- (2);
			% 	\draw  (1) -- (3);
			% 	\draw  (3) edge [loop right] ();
			% \end{tikzpicture}
			% \end{figure}
			% $\bordermatrix{
			% 		  & 0 & 1 & 2 & 3 \cr
			% 		0 & 1 & 1 & 1 & 0 \cr
			% 		1 & 1 & 1 & 1 & 0 \cr
			% 		2 & 1 & 1 & 1 & 0 \cr
			% 		3 & 1 & 1 & 1 & 0 \cr
			% 		}$
			% \end{center}
			% \end{column}
		\end{columns}
	\end{exampleblock}
\end{frame}

\section{Algorithmen in Graphen}
\subsection{Von der Adjazenz- zur Wegematrix, Ausblick Warshall-Algorithmus}
\begin{frame}{Kantenrelation}
	\begin{block}{Kantenmenge als Relation}
		Sei $G$ ein gerichteter Graph mit der Knotenmenge $V$. Die Kantenmenge ist wie folgt definiert:\\
		Sei $E \subseteq V \times V$ eine Relation auf $V$ definiert durch $(x,y) \in E \gdw \text{es gibt eine Kante von $x$ nach }y \text{ in } G$.
	\end{block}

	\begin{exampleblock}{Beobachtungen}
		\begin{itemize}
			\item Was gilt für $E^2$? \pause $\Rightarrow$ Knoten, die über einen Pfad der Länge 2 verbunden sind
			\item Analog $E^3$, $E^4$, \dots
			\item Also $(x,y) \in E^*$ genau dann, wenn es einen Pfad (beliebiger Länge) von $x$ nach $y$ gibt.
		\end{itemize}
	\end{exampleblock}
\end{frame}

\begin{frame}{Berechnung der Erreichbarkeitsrelation}
    \begin{block}{Erreichbarkeitsrelation}
        Sei $n = |V|$
    	\[
    		E^* = \bigcup_{i=0}^{\infty} E^i = \bigcup_{i=0}^{n-1} E^i
    	\]
    \end{block}

    \begin{block}{Matrizen für die Relation $E^k$}
    	Sei $G$ ein gerichteter Graph mit Adjazenzmatrix $A$. Für alle $k \in \nN_0$ gilt:\\
    	\[
    		\sgn((A^k)_{ij}):=
    		\begin{cases}
      		1 & \text{ falls in $G$ ein Pfad der Länge $k$ von $i$ nach $j$ existiert}\\
      		0 & \text{ falls in $G$ kein Pfad der Länge $k$ von $i$ nach $j$ existiert}\\
    		\end{cases}
    	\]
    \end{block}
\end{frame}

\begin{frame}{Berechnung der Wegematrix}	
    \begin{block}{Berechnung der Wegematrix}
    	Es sei $G$ ein gerichteter Graph mit Adjazenzmatrix $A$. Dann gilt für alle $k\geq n-1$: 
    	\[
    		W = \sgn\left(\sum_{i=0}^{k} A^i\right) 
    	\]
    	ist die Wegematrix des Graphen $G$.
    \end{block}
\end{frame}

%%%%%%%%%% %%%%%%%%%%
%% Zusammenfassung
\section{}
%\subsection{Zusammenfassung}
	\begin{frame}{Was ihr jetzt kennen und können solltet\dots}
			\begin{itemize}
				\item Graphen konstruieren und zeichnen
				\item Eigenschaften von Graphen erkennen
				\item Wegematrixalgorithmus
			\end{itemize}
	
	\end{frame}
%% Ausblick
%\subsection{Ausblick}
	\begin{frame}{Ausblick}
		\begin{itemize}
			\item Wie \emph{schnell} ist ein Algorithmus $\Rightarrow$ O-Kalkül 
		\end{itemize}
	\end{frame}
%%%%%%%%%% %%%%%%%%%%
\section{}
\questionframe
\lastframe
\mode<handout>{\slideThanks}
\end{document}