% ===== handout mode =====
% Comment/uncomment this line to toggle handout mode
% \newcommand{\handout}{}

% Comment/uncoment this line to toogle Mortitz mode
% \newcommand{\Moritz}{}

% Comment/uncomment this line to toggle handout mode
% \newcommand{\handout}{}

% by Stephan

%% Moritz mode or Stephan mode
\ifdefined \Moritz

% This is a configuration file with private, tutor specific information.
% It is therefore excluded from the Git repository so changes in this file will not conflict in git commits.

% Copy this Template, rename to config.tex and add your information below.

\newcommand{\mymail}{moritz.laupichler@student.kit.edu} % Consider using your named student Mail address to keep your u-Account private.

\newcommand{\myname}{\href{mailto:\mymail}{Moritz Laupichler}}

\newcommand{\mytutnumber}{25}

\newcommand{\mytutinfos}{Dienstags, 5. Block (15:45-17:15 Uhr), SR -120}

\newcommand{\aboutMeFrame}{
	\begin{frame}{Euer Tutor}
		Name: \myname \\
		Alter: 21 Jahre \\
		Studiengang: Master Informatik, 1. Semester \\
		\vspace{1cm}
		\pause 
		\centering{Kontakt: \href{mailto:\mymail}{\mymail}}
	\end{frame}
} % Moritz mode
\else
\ifdefined \Alex

% This is a configuration file with private, tutor specific information.
% It is therefore excluded from the Git repository so changes in this file will not conflict in git commits.

% Copy this Template, rename to config.tex and add your information below.

\newcommand{\mymail}{alexander.klug@student.kit.edu} % Consider using your named student Mail address to keep your u-Account private.

\newcommand{\myname}{\href{mailto:\mymail}{Alexander Klug}}

\newcommand{\mytutnumber}{30}

\newcommand{\mytutinfos}{Mittwochs, 3. Block (11:30-13:00), SR -107}

\newcommand{\aboutMeFrame}{
	\begin{frame}{Euer Tutor}
		Name: \myname \\
		Alter: 19 Jahre \\
		Studiengang: Bachelor Informatik, 3. Semester \\
		\vspace{1cm}
		\pause 
		\centering{Kontakt: \href{mailto:\mymail}{\mymail}}
	\end{frame}
}

% Toggle Handout mode by including the following line before including style_tut
% and removing the % at the start (but do NOT remove it here, otherwise handout mode will always be on!)
% Please keep handout mode on in all commits!

% \newcommand{\handout}{} % Alex Mode
\else

% This is a configuration file with private, tutor specific information.
% It is therefore excluded from the Git repository so changes in this file will not conflict in git commits.

% Copy this Template, rename to config.tex and add your information below.

\newcommand{\mymail}{stephan.bohr@student.kit.edu} % Consider using your named student Mail address to keep your u-Account private.

\newcommand{\myname}{\href{mailto:\mymail}{Stephan Bohr}}

\newcommand{\mytutnumber}{25}

\newcommand{\mytutinfos}{Dienstags, 5. Block (15:45-17:15), SR -119}

\newcommand{\aboutMeFrame}{
	\begin{frame}{Euer Tutor}
		Name: \myname \\
		Alter: 20 Jahre \\
		Studiengang: Bachelor Informatik, 3. Semester \\
		\vspace{1cm}
		\pause 
		\centering{Kontakt: \href{mailto:\mymail}{\mymail}}
	\end{frame}
} % Stephan mode
\fi
\fi

%% Beamer-Klasse im korrekten Modus
\ifdefined \handout
\documentclass[handout]{beamer} % Handout mode
\else
\documentclass{beamer}
\fi
%\documentclass[18pt,parskip]{beamer}

%% SLIDE FORMAT

% use 'beamerthemekit' for standard 4:3 ratio
% for widescreen slides (16:9), use 'beamerthemekitwide'

\usepackage{../templates/KIT-slides/beamerthemekit}
%\usepackage{../templates/KIT-slides/beamerthemekitwide}

%% TITLE PICTURE

% if a custom picture is to be used on the title page, copy it into the 'logos'
% directory, in the line below, replace 'mypicture' with the 
% filename (without extension) and uncomment the following line
% (picture proportions: 63 : 20 for standard, 169 : 40 for wide
% *.eps format if you use latex+dvips+ps2pdf, 
% *.jpg/*.png/*.pdf if you use pdflatex)

\titleimage{../figures/titleimage/brain}

%% TITLE LOGO

% for a custom logo on the front page, copy your file into the 'logos'
% directory, insert the filename in the line below and uncomment it

%\titlelogo{mylogo}

% (*.eps format if you use latex+dvips+ps2pdf,
% *.jpg/*.png/*.pdf if you use pdflatex)

%% TikZ INTEGRATION

% use these packages for PCM symbols and UML classes
% \usepackage{templates/tikzkit}
% \usepackage{templates/tikzuml}

%\usepackage{tikz}
%\usetikzlibrary{matrix}
%\usetikzlibrary{arrows.meta}
%\usetikzlibrary{automata}
%\usetikzlibrary{tikzmark}

%%%%%%%%%%%%%%%%%%%%%%%%%
% Libertine font (Original GBI font)
\usepackage{libertine}
%\renewcommand*\familydefault{\sfdefault}  %% Only if the base font of the document is to be sans serif

%% Schönere Schriften
\usepackage[TS1,T1]{fontenc}

%% Deutsche Silbentrennung und Beschriftungen
\usepackage[ngerman]{babel}

%% UTF-8-Encoding
\usepackage[utf8]{inputenc}

%% Bibliotheken für viele mathematische Symbole
\usepackage{amsmath, amsfonts, amssymb}

%% Anzeigetiefe für Inhaltsverzeichnis: 1 Stufe
\setcounter{tocdepth}{1}

%% Hyperlinks
\usepackage{hyperref}
% I don't know why, but this works and only includes sections and NOT subsections in the pdf-bookmarks.
\hypersetup{bookmarksdepth=subsection}

%% remove navigation symbols
\setbeamertemplate{navigation symbols}{}

%% switch between "ngerman" and "english" for German/English style date and logos
\selectlanguage{ngerman}

%% for invisible pause texts instead of dimming
\setbeamercovered{invisible}

%%%%%%%%%%%% Shortcuts %%%%%%%%%%%%%
\newcommand{\nM}{\mathbb{M}}
\newcommand{\nR}{\mathbb{R}}
\newcommand{\nN}{\mathbb{N}}
\newcommand{\nZ}{\mathbb{Z}}
\newcommand{\nQ}{\mathbb{Q}}
\newcommand{\nB}{\mathbb{B}}
\newcommand{\nC}{\mathbb{C}}
\newcommand{\nK}{\mathbb{K}}
\newcommand{\nF}{\mathbb{F}}
\newcommand{\nG}{\mathbb{G}}
\newcommand{\nullel}{\mathcal{O}}
\newcommand{\einsel}{\mathds{1}}
\newcommand{\nP}{\mathbb{P}}
\newcommand{\Pot}{\mathcal{P}}
\renewcommand{\O}{\text{O}}

\newcommand{\set}[1]{\{ #1 \}}
\newcommand{\setc}[2]{\set{#1 \mid #2}}
\newcommand{\setC}[2]{\set{#1 \mid \text{ #2 }}}

\newcommand{\setsize}[1]{\; \mid #1 \mid \; }

\newcommand{\q}[1]{\textquotedblleft #1\textquotedblright}

%%%%%%%%%%%% INHALT %%%%%%%%%%%%%%%%

%% Wochennummer
%\newcounter{weeknum}

%% Titelinformationen
%\title[GBI Tutorium, Woche \theweeknum]{Grundbegriffe der Informatik \\ Tutorium \mytutnumber}
%\subtitle{Termin \theweeknum \ | \mydate \\ \myname}
\author[\myname]{\myname}
\institute{Fakultät für Informatik}
%\date{\mydate}

%% Titel einfügen
\newcommand{\titleframe}{\frame{\titlepage}\addtocounter{framenumber}{-1}}


%% Alles starten mit \starttut{X}
%\newcommand{\starttut}[1]{\setcounter{weeknum}{#1}\titleframe\frame{\frametitle{Inhalt}\tableofcontents} \AtBeginSection[]{%
%\begin{frame}
%	\tableofcontents[currentsection]
%\end{frame}\addtocounter{framenumber}{-1}}}


%\newcommand{\framePrevEpisode}{
%	\begin{frame}
%		\centering
%		\textbf{In the previous episode of GBI...}
%	\end{frame}
%}

%% Roadmap frame
%table of contents
\newcommand{\roadmap}{
	\frame{\frametitle{Roadmap}\tableofcontents}}

 \AtBeginSection[]{%
\begin{frame}
	\frametitle{Roadmap}
	\tableofcontents[currentsection]
\end{frame}%\addtocounter{framenumber}{-1}
}


%% ShowMessage frame
\newcommand{\showmessage}[1]{\frame{\frametitle{\phantom{1em}}\centering\textbf{#1}}}

%% Fragen
%% Lastframe
\newcommand{\questionframe}{\showmessage{Fragen?}}

%% Lastframe
\newcommand{\lastframe}{\showmessage{Vielen Dank für Eure Aufmerksamkeit! \\Bis nächste Woche :)}}

%% Thanks frame
\newcommand{\slideThanks}{
	\begin{frame}
		\frametitle{Credits}
		\begin{block}{}
			An der Erstellung des Foliensatzes haben mitgewirkt:\\[1em]
			\ifdefined \Moritz
			Stephan Bohr \\
			Alexander Klug \\
			\else
			\ifdefined \Alex
			Stephan Bohr \\
			Moritz Laupichler \\
			\else
			Moritz Laupichler \\
			Alexander Klug \\
			\fi
			\fi
			Katharina Wurz \\
			Thassilo Helmold \\
			Philipp Basler \\
			Nils Braun \\
			Dominik Doerner \\
			Ou Yue \\
		\end{block}
	\end{frame}
}

%% Verbatim
%\usepackage{moreverb}



\usetikzlibrary{matrix}
\usetikzlibrary{arrows.meta}
\usetikzlibrary{automata}
\usetikzlibrary{tikzmark}

\title[Quantitative Aspekte von Algorithmen, Endliche Automaten]{11. Tutorium\\  Quantitative Aspekte von Algorithmen, Endliche Automaten}
\subtitle{Grundbegriffe der Informatik, Tutorium \hashtag\mytutnumber}
\date{\today}


\begin{document}
\titleframe

\roadmap

%%%%%%%%%% %%%%%%%%%%


\section{Quantitative Aspekte von Algorithmen}
\newcommand{\Rplus}{\ensuremath{\nR_+}}
\newcommand{\Rnullplus}{\ensuremath{\nR^+_0}}

\subsection{Ressourcenverbrauch, Abängigkeit von n, Best- Worst- und Average-Case}
\begin{frame}{Laufzeit}
\centering\includegraphics[height=4cm]{algorithm}

\begin{exampleblock}{Aufgabe}
	Wie viele \emph{Instruktionen} werden aufgerufen? \\% Worst-Case: n+1, Best-Case: 1, Average Case ?
	\pause
	Im \emph{worst case}? Im \emph{best case}? Im \emph{average case}?
\end{exampleblock}
\end{frame}

\subsection{O-Kalkül}
\begin{frame}{O-Notation}
	\begin{block}{Def. Asymptotisches Wachstum $\asymp$}
		Seien $f,g \from \nN_0 \to \Rnullplus$. Dann gilt
		\[
			f \asymp g \Leftrightarrow \exists c,c'\in \Rplus: \exists n_0\in \nN_0: \forall n\geq n_0: c f(n) \leq g(n) \leq c' f(n)
		\]
		Man sagt auch $g$ wächst genauso schnell wie $f$. $\asymp$ ist eine Äquivalenzrelation.
	\end{block}

	\begin{exampleblock}{Beispiele}
		\begin{itemize}
			\item $42n^6-33n^3+222n^2 -15 \asymp 66n^6+55555n^5$
			\item $n^{3}+5n^2 + 1\asymp 3n^3-n$
		\end{itemize}
	\end{exampleblock}
\end{frame}

\begin{frame}{$O$-Notation}
    \begin{block}{$\Theta$-Kalkül}
    	\begin{align*}
  			\Theta(f) &= \{ g \mid f \asymp g \} \\
  				   &= \{ g \mid \exists c,c'\in \Rplus: \exists n_0\in \nN_0: \forall  n\geq n_0: c f(n) \leq g(n) \leq c' f(n) \} 
		\end{align*}
    \end{block}

    \begin{exampleblock}{Bemerkungen}
    	\begin{itemize}
    		\item Im $\Theta$-Kalkül von $f(n)$ sind genau die Funktionen enthalten, die asymptotisch gleich schnell wachsen wie $f(n)$.
    		\item Schreibe $g(n) \in \Theta(f(n))$, wenn $g(n)$ asymptotisch gleichschnell wächst wie $f(n)$.
    		\item Ist $f$ ein Polynom, so sind insbesondere in $\Theta(f(n))$ alle Polynome enthalten, die den gleichen Grad wie $f$ haben.
    		\item Es gilt $\log_b(n) \in\Theta(\log_a(n))$. Die Basis ist also egal und man kann auch $\Theta(\log n)$ schreiben. % TODO Aufgabe und Beweis hierzu
    		% TODO Theta ist != Average Case
    	\end{itemize}
    \end{exampleblock}
\end{frame}

\begin{frame}{$O$-Notation}
    \begin{block}{Def.: $O$-Kalkül}
    	\[
    		O(f) = \{ g \mid \exists c\in \Rplus:\exists n_0\in\nN_0: \forall n\geq n_0: g(n) \leq c f(n)\}
    	\]

    	$g(n) \in O(f(n))$ (oder $g \preceq f$) genau dann, wenn $g$ asymptotisch höchstens so schnell wächst wie $f$.
    \end{block}
\pause
    \begin{block}{Def.: $\Omega$-Kalkül}
    	\[
    		\Omega(f) = \{ g \mid \exists c\in \Rplus:\exists n_0\in\nN_0: \forall n\geq n_0: g(n) \geq c f(n)\}
    	\]
    	
    	$g(n) \in \Omega(f(n))$ (oder $g \succeq f$) genau dann, wenn $g$ asymptotisch mindestens so schnell wächst wie $f$.
    \end{block}
\pause
    Beobachtung: $\Theta(f) = O(f) \cap \Omega(f)$
\end{frame}

\begin{frame}{$O$-Notation}
	\begin{exampleblock}{Aufgabe}
		\begin{enumerate}
			\item Für welches $c \in \Rplus$ gilt $5n^4 \in O(n^c)$ bzw $5n^4 \in \Omega(n^c)$?
			\item Für welches $c \in \Rplus$ gilt $5n^4 \in O(c^n)$ bzw $5n^4 \in \Omega(c^n)$?
			\item Für welches $c \in \Rplus$ gilt $2^n \in O(c^n)$ bzw $2^n \in \Omega(c^n)$?
			\item Zeige oder widerlege: $n \in \Theta(\sqrt{n})$
		\end{enumerate}
	\end{exampleblock}
\pause
	\begin{block}{Lösung}
		\begin{enumerate}
			\item Es gilt: $\forall c \geq 4$ bzw. $\forall c \leq 4$.
			\item Es gilt: $\forall c > 1$ bzw. $\forall c \leq 1$.
			\item Es gilt: $\forall c \geq 2$ bzw. $\forall c \leq 2$.
			\item \small Annahme: Die Behauptung ist richtig.\\
				Dann gilt: $n \in O(\sqrt{n}) \wedge n \in \Omega(\sqrt{n})$, insbesondere $n \in \Omega(\sqrt{n})$.\\
				$\Rightarrow \exists c\in \Rplus:\exists n_0\in\nN_0: \forall n\geq n_0: n \leq c \sqrt{n} \Leftrightarrow \frac{n}{\sqrt{n}} \leq c \Leftrightarrow \sqrt{n} \leq c$. Widerspruch.
				
		\end{enumerate}
	\end{block}
\end{frame}

\begin{frame}{$O$-Notation}
    \begin{exampleblock}{Aufgabe}
    	Zeige oder widerlege:
    	\[
    		f(n) + g(n) \in O(g(f(n)))
    	\]
    \end{exampleblock}
\pause
	\begin{block}{Lösung}
		Die Behauptung stimmt nicht. Wähle z.B. $f(n) = n^2$ und $g(n) = \sqrt{n}$ und führe dies zu einem Widerspruch.
	\end{block}
\end{frame}

\begin{frame}{$O$-Notation} % TODO: Logarithmusregeln, Typische Abschätzungskette (1 < log n < sqrt(n) < n < n log n < n c (c>1) < c^n < n!)
    \begin{itemize}
    	\item $\Theta$ entspricht \emph{nicht} dem average case.
    	\item $O(1)$ bedeutet konstante Laufzeit
    	\item Beachtet den \emph{Trick} mit den Limites\footnote{Ich weiß nicht, ob ihr den als Beweis in der Klausur oder auf dem Blättern verwenden dürft. Zur Kontrolle sollte man ihn aber kennen.}
        \item Es gilt mit Konstante $c \in \Rplus$ mit $c>1$
        \[
            1 \preceq \log n \preceq \sqrt{n} \preceq n \preceq n \log n \preceq n^c \preceq c^n \preceq n!
        \]
        %\nachgucken https://martin-thoma.com/die-landau-symbole/
    \end{itemize}
\end{frame}

\subsection{Master-Theorem}
\begin{frame}{Mastertheorem}
    \textbf{Problemstellung:}\\[1cm]    
    Gegeben sei eine \emph{rekursiv} definierte Funktion $T$.\\
    Frage: Welche Laufzeit hat $T$?\\[1cm]
    Beispiel:
    \[
    	T(n) = 8 T \left(\frac{n}{2} \right) + 1000n^2
    \]
    \\[1cm]\centering$\Rightarrow$ \emph{Mastertheorem}
\end{frame}

\begin{frame}{Mastertheorem}
    \begin{block}{Def.: Mastertheorem}
    	Seien $a\geq 1$ und $b>1$ Konstanten, $f \from \nN \to \Rnullplus$ und $T(n)$ eine Laufzeitfunktion der Form
    	\[
			T = a T\left(\frac{n}{b}\right) + f 
		\]
		Dann gilt nach dem \textbf{Mastertheorem}:
		\begin{itemize}
			\item \textbf{Fall 1:} \\ Wenn $f \in O(n^{\log_b a -\varepsilon})$ für ein $\varepsilon>0$ ist, dann ist $T\in \Theta(n^{\log_b a})$.
			\item \textbf{Fall 2:} \\ Wenn $f \in \Theta(n^{\log_b a})$ ist, dann ist $T\in \Theta(n^{\log_b a}\log n)$.
			\item\textbf{Fall 3:} \\  Wenn $f \in \Omega(n^{\log_b a +\varepsilon})$ für ein $\varepsilon>0$ ist, und wenn es eine Konstante $d$ gibt mit $0<d<1$, so dass für alle hinreichend großen $n$ gilt $af\left(\frac{n}{b}\right)\leq d f(n)$, dann ist $T\in \Theta(f)$.
		\end{itemize}
    \end{block}
\end{frame}

\begin{frame}{Mastertheorem}
    \begin{exampleblock}{Beispiel zum 1. Fall}
    	Sei $T(n) = 8 T \left(\frac{n}{2} \right) + 1000n^2$.
    	\begin{itemize}
    		\item Aus der Formel lässt sich ablesen:\\
    			$a=8$, $b=2$, $f(n)=1000n^2$
    		\item $n^{\log_b a}$ bestimmen:\\
    			$\log_b a = \log_2 8 = 3 \Rightarrow n^{\log_b a} = n^3$
    		\item $n^{\log_b a}$ mit $f(n)$ vergleichen: $1000n^2 \in O(n^{3-\varepsilon})$?\\
    			Ja, für $\varepsilon = 1$ gilt $1000n^2 \in O(n^2)$.
    		\item Mit dem Mastertheorem folgt:\\
    			$T(n) \in \Theta(n^3)$
    	\end{itemize}
    \end{exampleblock}
\end{frame}


\begin{frame}{Mastertheorem}
    \begin{exampleblock}{Beispiel zum 2. Fall}
    	Sei $T(n) = 2 T \left(\frac{n}{2} \right) + 10n$.
    	\begin{itemize}
    		\item Aus der Formel lässt sich ablesen:\\
    			$a=2$, $b=2$, $f(n)=10n$
    		\item $n^{\log_b a}$ bestimmen:\\
    			$\log_b a = \log_2 2 = 1 \Rightarrow n^{\log_b a} = n^1$
    		\item $n^{\log_b a}$ mit $f(n)$ vergleichen: $10n \in \Theta(n)$?\\
    			Ja!
    		\item Mit dem Mastertheorem folgt:\\
    			$T(n) \in \Theta(n \log n)$
    	\end{itemize}
    \end{exampleblock}
\end{frame}



\begin{frame}{Mastertheorem}
    \begin{exampleblock}{Beispiel zum 3. Fall}
    	Sei $T(n) = 2 T \left(\frac{n}{2} \right) + n^2$.
    	\begin{itemize}
    		\item Aus der Formel lässt sich ablesen:\\
    			$a=2$, $b=2$, $f(n)=n^2$
    		\item $n^{\log_b a}$ bestimmen:\\
    			$\log_b a = \log_2 2 = 1 \Rightarrow n^{\log_b a} = n^1$
    		\item $n^{\log_b a}$ mit $f(n)$ vergleichen: $n^2 \in \Omega(n^{1+\varepsilon})$?\\
    			Ja, für $\varepsilon = 1$ gilt $n^2 \in \Omega(n^2)$.
    		\item Zusatzbedingung überprüfen: Ist $af\left(\frac{n}{b}\right)\leq d f(n)$?\\
    			Ja, für $d = \frac{1}{2}$ gilt $\forall n \geq 1 \; : \; \frac{1}{2}n^2 \leq \frac{1}{2}n^2$
    		\item Mit dem Mastertheorem folgt:\\
    			$T(n) \in \Theta(n^2)$
    	\end{itemize}
    \end{exampleblock}
\end{frame}

\section{Endliche Automaten}
\def\mybox#1{\hbox{\vrule height 2ex width 0pt depth 0.6ex#1}}
\def\taste[#1]#2{%
  \tikz[x=8mm,y=8mm,baseline=(N.base)] \tasteinnerx[#1]{#2};%
}
\def\tasteinnerx[#1]#2{%
  \node[midway,inner sep=0mm,draw,rounded corners,anchor=base,minimum width=10mm,#1] (N) {\mybox{#2}}%
}
\def\tasteinner[#1]#2{%
  node[midway,inner sep=0mm,draw,rounded corners,minimum width=10mm,#1] (N) {\mybox{#2}}%
}
\def\tasteinnerOK{\tasteinner[fill=green!20]{OK}}
\def\tasteOK{\taste[fill=green!20]{OK}}
\def\tasteC{\taste[fill=red!20]{C}}
\def\tasteinnerC{\tasteinner[fill=red!20]{C}}
\def\tasteRein{\taste[fill=blue!10]{rein}}
\def\tasteinnerRein{\tasteinner[fill=blue!10]{rein}}
\def\tasteZitro{\taste[fill=yellow!10]{zitro}}
\def\tasteinnerZitro{\tasteinner[fill=yellow!10]{zitro}}

\newcommand{\ioeps}[1]{#1|\varepsilon}

\subsection{Mealy-Automat}
\begin{frame}{Der Getränkeautomat}

	\begin{exampleblock}{Beispiel} \small
		Betrachte einen Getränkeautomaten:
		Man kann nur 1-Euro"=Stücke einwerfen und vier Tasten drücken: Es gibt
		zwei Auswahltasten für Mineralwasser \tasteRein{} und Zitronensprudel
		\tasteZitro{}, eine Abbruch"=Taste \tasteC{} und eine \tasteOK-Taste.

	
	\begin{itemize}
	\item Jede Flasche Sprudel kostet 1 Euro.
	\item Es kann ein Guthaben von 1 Euro gespeichert werden. Wirft man
	  weitere Euro"=Stücke ein, werden sie sofort wieder ausgegeben.
	\item Wenn man mehrfach Auswahltasten drückt, wird der letzte Wunsch
	  gespeichert.
	\item Bei Drücken der Abbruch"=Taste wird alles bereits eingeworfenen
	  Geld wieder zurückgegeben und kein Getränkewunsch mehr gespeichert.
	\item Drücken der OK-Taste wird ignoriert, solange noch kein Euro
	  eingeworfen wurde oder keine Getränkesorte ausgewählt wurde.

	  Andernfalls wird das gewünschte Getränk ausgeworfen.
	\end{itemize} 
	\end{exampleblock}
\end{frame}

\begin{frame}{Der Getränkeautomat}

	\begin{exampleblock}{Beispiel}
		\begin{figure}[ht]
  \centering
  \begin{tikzpicture}[x=8mm,y=8mm]
    % Rahmen
    \draw (0,0) -- (8,0) -- (8,8) .. controls  (3,9) and (5,9) .. (0,8) -- cycle;
    % Geldschlitz 
    \draw (1.4,5.5) rectangle ++(0.2,1) ++(-0.1,0) node[anchor=south] {\vbox{\hbox{Geld-}\hbox{Einwurf}}};
    % Geldrückgabe 
    \draw (1,1) rectangle ++(1,1) ++(-0.5,0) node[anchor=south] {\vbox{\hbox{Geld-}\hbox{Rückgabe}}};
    % Warenauswurf
    \draw (4,1) rectangle ++(3.5,1) ++(-1.75,0) node[anchor=south] {Ware};
    
    % Tasten
    \draw (2.75,3) ++(3,3.5) node[anchor=south] {Sprudel};
    \draw (4,5.5) rectangle ++(1.5,1)\tasteinnerRein;
    \draw (6,5.5) rectangle ++(1.5,1) \tasteinnerZitro;
    \draw (4,4) rectangle ++(1.5,1) \tasteinnerOK;
    \draw (6,4) rectangle ++(1.5,1) \tasteinnerC;
  \end{tikzpicture}
  \caption{Ein primitiver Getr"ankeautomat}
  \label{fig:getraenkeautomat}
\end{figure}
	\end{exampleblock}
\end{frame}

\begin{frame}[fragile]{Der Getränkeautomat}

	\begin{exampleblock}{Beispiel}
		\begin{figure}[ht]
  \centering
  \small
  \begin{tikzpicture}[->,>=stealth]
    \matrix[matrix of math nodes,column sep=25mm,row sep=20mm,nodes={circle,draw,inner sep=1pt}]   {
      |(0-)| (0,-) & |(0R)| (0,R) & |(0Z)| (0,Z) \\
      |(1-)| (1,-) & |(1R)| (1,R) & |(1Z)| (1,Z) \\
    };

    \coordinate[left of=0-] (start);

    \draw (start) -- node[auto] {} (0-);

    % Schleifen
    \draw (0-) edge[loop above]  node[pos=0.5] {$\ioeps{O}$,$\ioeps{C}$} ();
    \draw (0R) edge[loop above] node[pos=0.9,anchor=west] {$\ioeps{R}$,$\ioeps{O}$} ();
    \draw (0Z) edge[loop above] node[pos=0.5] {$\ioeps{Z}$,$\ioeps{O}$} ();
    \draw (1-) edge[loop below]  node[pos=0.5] {$\io{1}{1}$,$\ioeps{O}$} ();
    \draw (1R) edge[loop below] node[pos=0.9,anchor=east] {$\io{1}{1}$,$\ioeps{R}$} ();
    \draw (1Z) edge[loop below] node[pos=0.5] {$\io{1}{1}$,$\ioeps{Z}$} ();

    % andere Kanten
    \draw (0-) -- node[right,pos=0.2] {$\ioeps{1}$} (1-);
    \draw (0R) -- node[right,pos=0.2] {$\ioeps{1}$} (1R);
    \draw (0Z) -- node[right,pos=0.2] {$\ioeps{1}$} (1Z);

    \draw (1-) to[bend left=10] node[left,pos=0.2] {$\io{C}{1}$} (0-);

    \draw (0-) to[bend right=10] node[below] {$\ioeps{R}$} (0R);
    \draw (0R) to[bend right=10] node[above,pos=0.1] {$\ioeps{C}$} (0-);
    \draw (0R) to[bend right=10] node[below] {$\ioeps{Z}$} (0Z);
    \draw (0Z) to[bend right=10] node[above] {$\ioeps{R}$} (0R);
    \draw (0-) to[bend left=32]  node[below,pos=0.2] {$\ioeps{Z}$} (0Z);
    \draw (0Z) to[bend right=40] node[above] {$\ioeps{C}$} (0-);

    \draw (1-) to[bend right=10] node[above] {$\ioeps{R}$} (1R);
    \draw (1R) -- node[below,pos=0.4,anchor=north east] {$\io{O}{R}$,$\io{C}{1}$} (0-); %!!
    \draw (1R) to[bend right=10] node[below] {$\ioeps{Z}$} (1Z);
    \draw (1Z) to[bend right=10] node[above] {$\ioeps{R}$} (1R);
    \draw (1-) to[bend left=-32] node[below,pos=0.2] {$\ioeps{Z}$} (1Z);
    \draw (1Z) -- node[above,pos=0.3,anchor=south west] {$\io{O}{Z}$,$\io{C}{1}$} (0-); %!!
  \end{tikzpicture}
\end{figure}
	\end{exampleblock}
\end{frame}

\begin{frame}{Mealy-Automat}
	\begin{block}{Def.: Mealy-Automat}
		Ein (endlicher) \textbf{Mealy-Automat} $A=(Z, z_0, X, f, Y, g)$ ist bestimmt durch:
		\begin{itemize}
			\item endliche Zustandsmenge $Z$,
			\item Anfangszustand $z_0 \in Z$,
			\item Eingabealphabet $X$,
			\item Zustandsüberführungsfunktion $f: Z \times X \rightarrow Z$,
			\item Ausgabealphabet $Y$,
			\item Ausgabefunktion $g: Z \times X \rightarrow Y^{\ast}$
		\end{itemize}
	\end{block}
\pause
	\begin{alertblock}{Achtung!}
	$f$ ist eine \textbf{Funktion}, also muss man von jedem Zustand mit allen Eingaben ``irgendwohin'' gelangen! Stichwort: Müllzustand.
	\end{alertblock}
\end{frame}



\begin{frame}{Mealy-Automat}
	\begin{block}{Def.: Verallgemeinerte Zustandsübergangsfunktion $f_{\ast}$}
		$f_{\ast} : Z \times X^{\ast} \rightarrow Z$ gibt den Zustand aus, der durch die Eingabe eines Wortes erreicht wird, und ist definiert durch:
		\begin{align*}
			f_{\ast}(z,\varepsilon) &= z \\
			\forall w \in X^{\ast} : \forall x \in X : f_{\ast}(z,wx) &= f(f_{\ast}(z,w),x)\\
		\end{align*}
		Oder als alternative Definition:
		\begin{align*}
			\bar{f}_{\ast}(z,\varepsilon) &= z \\
			\forall w \in X^{\ast} : \forall x \in X : \bar{f}_{\ast}(z,xw) &= \bar{f}_{\ast}(f(z,x),w)
		\end{align*}
	\end{block}
\end{frame}

\begin{frame}{Mealy-Automat}
	\begin{block}{Def.: Verallgemeinerte Zustandsübergangsfunktion $f_{\ast\ast}$}
		$f_{\ast\ast} : Z \times X^{\ast} \rightarrow Z^{\ast}$ gibt \textbf{alle} Zustände aus, die bei der Eingabe eines Wortes durchlaufen werden, und ist definiert durch:
		\begin{align*}
			f_{\ast\ast}(z,\varepsilon) = z\\
			\forall w \in X^{\ast} : \forall x \in X : f_{\ast\ast}(z,wx) &= f_{\ast\ast}(z,w) \cdot f(f_{\ast}(z,w),x)\\
		\end{align*}
	\end{block}
\end{frame}

\begin{frame}[fragile]{Mealy-Automat}

	\begin{exampleblock}{Aufgabe zu $f_{\ast}$ und $f_{\ast\ast}$}
		\begin{columns}
			\begin{column}{0.6\textwidth}
				\begin{figure}[ht]
  \centering
  \tiny
  \begin{tikzpicture}[->,>=stealth]
    \matrix[matrix of math nodes,column sep=20mm,row sep=20mm,nodes={circle,draw,inner sep=1pt}]   {
      |(0-)| (0,-) & |(0R)| (0,R) & |(0Z)| (0,Z) \\
      |(1-)| (1,-) & |(1R)| (1,R) & |(1Z)| (1,Z) \\
    };

    \coordinate[left of=0-] (start);

    \draw (start) -- node[auto] {} (0-);

    % Schleifen
    \draw (0-) edge[loop above]  node[pos=0.5] {$\ioeps{O}$,$\ioeps{C}$} ();
    \draw (0R) edge[loop above] node[pos=0.9,anchor=west] {$\ioeps{R}$,$\ioeps{O}$} ();
    \draw (0Z) edge[loop above] node[pos=0.5] {$\ioeps{Z}$,$\ioeps{O}$} ();
    \draw (1-) edge[loop below]  node[pos=0.5] {$\io{1}{1}$,$\ioeps{O}$} ();
    \draw (1R) edge[loop below] node[pos=0.9,anchor=east] {$\io{1}{1}$,$\ioeps{R}$} ();
    \draw (1Z) edge[loop below] node[pos=0.5] {$\io{1}{1}$,$\ioeps{Z}$} ();

    % andere Kanten
    \draw (0-) -- node[right,pos=0.2] {$\ioeps{1}$} (1-);
    \draw (0R) -- node[right,pos=0.2] {$\ioeps{1}$} (1R);
    \draw (0Z) -- node[right,pos=0.2] {$\ioeps{1}$} (1Z);

    \draw (1-) to[bend left=10] node[left,pos=0.2] {$\io{C}{1}$} (0-);

    \draw (0-) to[bend right=10] node[below] {$\ioeps{R}$} (0R);
    \draw (0R) to[bend right=10] node[above,pos=0.1] {$\ioeps{C}$} (0-);
    \draw (0R) to[bend right=10] node[below] {$\ioeps{Z}$} (0Z);
    \draw (0Z) to[bend right=10] node[above] {$\ioeps{R}$} (0R);
    \draw (0-) to[bend left=32]  node[below,pos=0.2] {$\ioeps{Z}$} (0Z);
    \draw (0Z) to[bend right=40] node[above] {$\ioeps{C}$} (0-);

    \draw (1-) to[bend right=10] node[above] {$\ioeps{R}$} (1R);
    \draw (1R) -- node[below,pos=0.4,anchor=north east] {$\io{O}{R}$,$\io{C}{1}$} (0-); %!!
    \draw (1R) to[bend right=10] node[below] {$\ioeps{Z}$} (1Z);
    \draw (1Z) to[bend right=10] node[above] {$\ioeps{R}$} (1R);
    \draw (1-) to[bend left=-32] node[below,pos=0.2] {$\ioeps{Z}$} (1Z);
    \draw (1Z) -- node[above,pos=0.3,anchor=south west] {$\io{O}{Z}$,$\io{C}{1}$} (0-); %!!
  \end{tikzpicture}
\end{figure}
			\end{column}
			\begin{column}{0.375\textwidth}
			\small
				Gib an:
				\begin{enumerate}
					\item $f_{\ast}((0,-), R1O)$
					\item $f_{\ast\ast}((0,-), R1O)$
					\item $f_{\ast}((0,-), C1Z)$
					\item $f_{\ast\ast}((0,-), RZO)$
					\item $f_{\ast\ast}((1,Z), RZO)$
					\item $f_{\ast}((0,-), RZ1R1C1ZO)$
				\end{enumerate}
			\end{column}
		\end{columns}
	\end{exampleblock}
\end{frame}

\begin{frame}{Mealy-Automat}
	\begin{block}{Lösung zur Aufgabe zu $f_{\ast}$ und $f_{\ast\ast}$}
		\begin{enumerate}
					\item $f_{\ast}((0,-), R1O) = (0,-)$
					\item $f_{\ast\ast}((0,-), R1O) = (0,-)(0,R)(1,R)(0,-)$
					\item $f_{\ast}((0,-), C1Z) = (1,Z)$
					\item $f_{\ast\ast}((0,-), RZO) = (0,-)(0,R)(0,Z)(0,Z)$
					\item $f_{\ast\ast}((1,Z), RZO) = (1,Z)(1,R)(1,Z)(0,-)$
					\item $f_{\ast}((0,-), RZ1R1C1ZO)= (0,-)$
				\end{enumerate}
	\end{block}
\end{frame}

\begin{frame}{Mealy-Automat}
	\begin{block}{Def.: Verallgemeinerte Ausgabefunktion $g_{\ast}$}
		$g_{\ast} : Z \times X^{\ast} \rightarrow Z$ gibt die letzte Ausgabe aus, die durch die Eingabe eines Wortes produziert wird, und ist definiert durch:
		\begin{align*}
			g_{\ast}(z,\varepsilon) &= \varepsilon \\
			\forall w \in X^{\ast} : \forall x \in X : g_{\ast}(z,wx) &= g(f_{\ast}(z,w),x)\\
		\end{align*}
	\end{block}

	\begin{block}{Def.: Verallgemeinerte Ausgabefunktion $g_{\ast\ast}$}
		$g_{\ast\ast} : Z \times X^{\ast} \rightarrow Z^{\ast}$ gibt \textbf{alle} Ausgaben konkateniert aus, die bei der Eingabe eines Wortes erzeugt werden, und ist definiert durch:
		\begin{align*}
			g_{\ast\ast}(z,\varepsilon) = \varepsilon\\
			\forall w \in X^{\ast} : \forall x \in X : g_{\ast\ast}(z,wx) &= g_{\ast\ast}(z,w) \cdot g_{\ast}(z,wx)\\
		\end{align*}
	\end{block}
\end{frame}

\begin{frame}[fragile]{Mealy-Automat}

	\begin{exampleblock}{Aufgabe zu $g_{\ast}$ und $g_{\ast\ast}$}
		\begin{columns}
			\begin{column}{0.6\textwidth}
				\begin{figure}[ht]
  \centering
  \tiny
  \begin{tikzpicture}[->,>=stealth]
    \matrix[matrix of math nodes,column sep=20mm,row sep=20mm,nodes={circle,draw,inner sep=1pt}]   {
      |(0-)| (0,-) & |(0R)| (0,R) & |(0Z)| (0,Z) \\
      |(1-)| (1,-) & |(1R)| (1,R) & |(1Z)| (1,Z) \\
    };

    \coordinate[left of=0-] (start);

    \draw (start) -- node[auto] {} (0-);

    % Schleifen
    \draw (0-) edge[loop above]  node[pos=0.5] {$\ioeps{O}$,$\ioeps{C}$} ();
    \draw (0R) edge[loop above] node[pos=0.9,anchor=west] {$\ioeps{R}$,$\ioeps{O}$} ();
    \draw (0Z) edge[loop above] node[pos=0.5] {$\ioeps{Z}$,$\ioeps{O}$} ();
    \draw (1-) edge[loop below]  node[pos=0.5] {$\io{1}{1}$,$\ioeps{O}$} ();
    \draw (1R) edge[loop below] node[pos=0.9,anchor=east] {$\io{1}{1}$,$\ioeps{R}$} ();
    \draw (1Z) edge[loop below] node[pos=0.5] {$\io{1}{1}$,$\ioeps{Z}$} ();

    % andere Kanten
    \draw (0-) -- node[right,pos=0.2] {$\ioeps{1}$} (1-);
    \draw (0R) -- node[right,pos=0.2] {$\ioeps{1}$} (1R);
    \draw (0Z) -- node[right,pos=0.2] {$\ioeps{1}$} (1Z);

    \draw (1-) to[bend left=10] node[left,pos=0.2] {$\io{C}{1}$} (0-);

    \draw (0-) to[bend right=10] node[below] {$\ioeps{R}$} (0R);
    \draw (0R) to[bend right=10] node[above,pos=0.1] {$\ioeps{C}$} (0-);
    \draw (0R) to[bend right=10] node[below] {$\ioeps{Z}$} (0Z);
    \draw (0Z) to[bend right=10] node[above] {$\ioeps{R}$} (0R);
    \draw (0-) to[bend left=32]  node[below,pos=0.2] {$\ioeps{Z}$} (0Z);
    \draw (0Z) to[bend right=40] node[above] {$\ioeps{C}$} (0-);

    \draw (1-) to[bend right=10] node[above] {$\ioeps{R}$} (1R);
    \draw (1R) -- node[below,pos=0.4,anchor=north east] {$\io{O}{R}$,$\io{C}{1}$} (0-); %!!
    \draw (1R) to[bend right=10] node[below] {$\ioeps{Z}$} (1Z);
    \draw (1Z) to[bend right=10] node[above] {$\ioeps{R}$} (1R);
    \draw (1-) to[bend left=-32] node[below,pos=0.2] {$\ioeps{Z}$} (1Z);
    \draw (1Z) -- node[above,pos=0.3,anchor=south west] {$\io{O}{Z}$,$\io{C}{1}$} (0-); %!!
  \end{tikzpicture}
\end{figure}
			\end{column}
			\begin{column}{0.4\textwidth}
			\small
				Gib an:
				\begin{enumerate}
					\item $g_{\ast}((0,-), R1O)$
					\item $g_{\ast\ast}((0,-), R1O)$
					\item $g_{\ast\ast}((0,-), R11O)$
					\item $g_{\ast}((0,-), C1Z)$
					\item $g_{\ast\ast}((0,-), RZO)$
					\item $g_{\ast\ast}((1,Z), RZO)$
					\item $g_{\ast\ast}((0,-), RZ1R1C1ZO)$
				\end{enumerate}
			\end{column}
		\end{columns}
	\end{exampleblock}
\end{frame}

\begin{frame}{Mealy-Automat}
	\begin{block}{Lösung zur Aufgabe zu $g_{\ast}$ und $g_{\ast\ast}$}
		\begin{enumerate}
					\item $g_{\ast}((0,-), R1O) = R$
					\item $g_{\ast\ast}((0,-), R1O) = R$
					\item $g_{\ast\ast}((0,-), R11O) = 1R$
					\item $g_{\ast}((0,-), C1Z) = \varepsilon$
					\item $g_{\ast\ast}((0,-), RZO) = \varepsilon$
					\item $g_{\ast\ast}((1,Z), RZO) = Z $
					\item $g_{\ast\ast}((0,-), RZ1R1C1ZO)= 11Z$
				\end{enumerate}
	\end{block}
\end{frame}



\subsection{Moore-Automat}
\begin{frame}{Moore-Automat}
	\begin{block}{Def.: Moore-Automat}
		Ein (endlicher) \textbf{Moore-Automat} $A=(Z, z_0, X, f, Y, h)$ ist bestimmt durch:
		\begin{itemize}
			\item endliche Zustandsmenge $Z$,
			\item Anfangszustand $z_0 \in Z$,
			\item Eingabealphabet $X$,
			\item Zustandsüberführungsfunktion $f: Z \times X \rightarrow Z$,
			\item Ausgabealphabet $Y$,
			\item Ausgabefunktion $h: Z \rightarrow Y^{\ast}$
		\end{itemize}

	Der Unterschied zum Mealy Automaten ist also , dass die Ausgabe nur vom Zustand abhängt, nicht von der Eingabe.

	\end{block}
\end{frame}

\begin{frame}[fragile]{Moore-Automat}
	\begin{exampleblock}{Beispiel}
		\begin{figure}[ht]
  \centering
  \begin{tikzpicture}[shorten >=1pt,node distance=2cm,auto,initial text=,->,>=stealth]
   \node[state,initial]  (q_0)                       {$q_{\varepsilon}\!\mid\! 0$};
   \node[state]          (q_1) [above right of= q_0] {$q_{a}\!\mid\! 0$};
    \node[state]          (q_2) [below right of= q_0] {$q_{b}\!\mid\! 0$};
   \node[state](q_3) [below right of=q_1] {$q_f\!\mid\! 1$};
   \node[state](q_4) [right of=q_3] {$q_r\!\mid\! 0$};
    \path[->] (q_0) edge              node        {$a$} (q_1)
                    edge              node [swap] {$b$} (q_2)
              (q_1) edge              node        {$b$} (q_3)
                    edge [loop above] node        {$a$} ()
              (q_2) edge              node [swap] {$a$} (q_3)
                    edge [loop below] node        {$b$} ()
              (q_3) edge              node        {$a,b$} (q_4)
              (q_4) edge [loop above] node        {$a,b$} ();
  \end{tikzpicture}
\end{figure}
	\end{exampleblock}
\end{frame}

\begin{frame}{Moore-Automat}
	\begin{block}{Def.: Verallgemeinerte Zustandsübergangsfunktionen $f_{\ast}$ und $f_{\ast\ast}$}
		Wie bei Mealy-Automaten.
	\end{block}

	\begin{block}{Def.: Verallgemeinerte Ausgabenfunktion $g_{\ast} = h \circ f_{\ast}$}
		$g_{\ast} : Z \times X^{\ast} \rightarrow Y$ gibt die letzte Ausgabe aus und ist definiert durch:
		\[
			\forall(z,w) \in Z \times X^{\ast} : g_{\ast}(z,w) = h(f_{\ast}(z,w))
		\]
	\end{block}

	\begin{block}{Def.: Verallgemeinerte Ausgabenfunktion $g_{\ast\ast} = h^{\ast\ast} \circ f_{\ast\ast}$}
		$g_{\ast\ast} : Z \times X^{\ast} \rightarrow Y$ gibt alle Ausgaben konkateniert aus und ist definiert durch:
		\[
			\forall(z,w) \in Z \times X^{\ast} : g_{\ast\ast}(z,w) = h^{\ast\ast}(f_{\ast\ast}(z,w))
		\]
		mit $h^{\ast\ast} :$\textit{induzierter Homomorphismus von h}.
	\end{block}
\end{frame}

\begin{frame}[fragile]{Moore-Automat}

	\begin{exampleblock}{Aufgabe zu $f_{\ast}$, $f_{\ast\ast}$, $g_{\ast}$ und $g_{\ast\ast}$}
		\begin{columns}
			\begin{column}{0.6\textwidth}
				\begin{figure}[ht]
  \centering
  \small
  \begin{tikzpicture}[shorten >=1pt,node distance=2cm,auto,initial text=,->,>=stealth]
   \node[state,initial]  (q_0)                       {$q_{\varepsilon}\!\mid\! 0$};
   \node[state]          (q_1) [above right of= q_0] {$q_{a}\!\mid\! 0$};
    \node[state]          (q_2) [below right of= q_0] {$q_{b}\!\mid\! 0$};
   \node[state](q_3) [below right of=q_1] {$q_f\!\mid\! 1$};
   \node[state](q_4) [right of=q_3] {$q_r\!\mid\! 0$};
    \path[->] (q_0) edge              node        {$a$} (q_1)
                    edge              node [swap] {$b$} (q_2)
              (q_1) edge              node        {$b$} (q_3)
                    edge [loop above] node        {$a$} ()
              (q_2) edge              node [swap] {$a$} (q_3)
                    edge [loop below] node        {$b$} ()
              (q_3) edge              node        {$a,b$} (q_4)
              (q_4) edge [loop above] node        {$a,b$} ();
  \end{tikzpicture}
\end{figure}
			\end{column}
			\begin{column}{0.4\textwidth}
			\small
				Gib an:
				\begin{enumerate}
					\item $f_{\ast}(q_{\varepsilon}, aab)$
					\item $f_{\ast\ast}(q_{\varepsilon}, aab)$

					\item $g_{\ast}(q_{\varepsilon}, aab)$
					\item $g_{\ast\ast}(q_{\varepsilon}, aab)$
					\item $g_{\ast}(q_{\varepsilon}, abab)$
					\item $g_{\ast\ast}(q_{f}, abab)$
				\end{enumerate}
			\end{column}
		\end{columns}
	\end{exampleblock}
\end{frame}

\begin{frame}{Moore-Automat}
	\begin{block}{Lösung zur Aufgabe zu $f_{\ast}$, $f_{\ast\ast}$, $g_{\ast}$ und $g_{\ast\ast}$}
		\begin{enumerate}
					\item $f_{\ast}(q_{\varepsilon}, aab) = q_{f}$
					\item $f_{\ast\ast}(q_{\varepsilon}, aab) = q_{\varepsilon}q_{a}q_{a}q_{f}$

					\item $g_{\ast}(q_{\varepsilon}, aab) = 1$
					\item $g_{\ast\ast}(q_{\varepsilon}, aab) = 0001$
					\item $g_{\ast}(q_{\varepsilon}, abab) = 0$
					\item $g_{\ast\ast}(q_{f}, abba) = 10000$
			\end{enumerate}
	\end{block}
\end{frame}

\subsection{Akzeptoren}
\begin{frame}{Endlicher Akzeptor}
	\begin{block}{Def.: Endlicher Akzeptor}
		Ein \textbf{endlicher Akzeptor} ist eine Spezialform des Moore-Automaten mit immer genau einem Bit Ausgabe, d.h. $Y = \set{0,1}$ (und $\forall z : h(z) \in Y$).

		Man definiert deshalb eine Menge $F = \setc{z}{h(z)=1}$ der \textbf{akzeptierenden Zustände}.

		Ein endlicher Akzeptor ist also definiert durch:

		\begin{itemize}
			\item endliche Zustandsmenge $Z$,
			\item Anfangszustand $z_0 \in Z$,
			\item Eingabealphabet $X$,
			\item Zustandsüberführungsfunktion $f: Z \times X \rightarrow Z$,
			\item eine Menge akzeptierender Zustände $F \subseteq Z$
		\end{itemize}
		Die von einem endlichen Akzeptor $A$ \textbf{akzeptierte Sprache} ist:
		\[
			L(A) = \setc{w \in X^{\ast}}{f_{\ast}(z_{0},w) \in F}
		\]
	\end{block}
\end{frame}

\begin{frame}[fragile]{Endlicher Akzeptor}
	\begin{exampleblock}{Beispiel}
	\begin{columns}
		\begin{column}{0.6\textwidth}
		\begin{figure}[ht]
  \centering
  \begin{tikzpicture}[shorten >=1pt,node distance=2cm,auto,initial text=,->,>=stealth]
    \node[state,initial]  (q_0)                       {$q_{\varepsilon}$};
    \node[state]          (q_1) [above right of= q_0] {$q_{a}$};
    \node[state]          (q_2) [below right of= q_0] {$q_{b}$};
    \node[state,accepting](q_3) [below right of=q_1] {$q_f$};
    \node[state](q_4) [right of=q_3] {$q_r$};
    \path[->] (q_0) edge              node        {$a$} (q_1)
                    edge              node [swap] {$b$} (q_2)
              (q_1) edge              node        {$b$} (q_3)
                    edge [loop above] node        {$a$} ()
              (q_2) edge              node [swap] {$a$} (q_3)
                    edge [loop below] node        {$b$} ()
              (q_3) edge              node        {$a,b$} (q_4)
              (q_4) edge [loop above] node        {$a,b$} ();
  \end{tikzpicture}
\end{figure}		
		\end{column}
		\begin{column}{0.375\textwidth}
		\centering
		(Akzeptierende Zustände mit Doppelkringel)
		\begin{figure}[ht]
			\begin{tikzpicture}[shorten >=1pt,node distance=2cm,auto,initial text=,->,>=stealth]
				\node[state,accepting] (z) {$z$};
			\end{tikzpicture}
		\end{figure}
		\end{column}
		
	\end{columns}
	\end{exampleblock}
\end{frame}

\begin{frame}{Endlicher Akzeptor}
	\begin{exampleblock}{Aufgabe zu endlichen Akzeptoren}
		\begin{enumerate}
			\small
			\item Zeichne einen Akzeptor mit $X=\set{a,b}$, der alle Wörter akzeptiert, bei denen die Anzahl der \emph{a} durch 5 teilbar ist.
			\item Zeichne einen Akzeptor mit $X=\set{a,b}$, der alle Wörter akzeptiert, in denen nirgends hintereinander zwei \emph{b} vorkommen.
			\item Zeichne einen Akzeptor mit $X=\set{a,b}$, der alle Wörter akzeptiert, in denen irgendwo das Teilwort \emph{abab} vorkommt.
			\item Zeichne einen Akzeptor mit $X=\set{a,b}$, der alle Wörter akzeptiert, in denen nirgends das Teilwort \emph{abab} vorkommt.
			\item Welche Sprache wird vom folgenden Akzeptor $A$ erkannt?
		\end{enumerate}

		\begin{figure}[ht]
  			\centering
  			\begin{tikzpicture}[shorten >=1pt,node distance=2cm,auto,initial text=,->,>=stealth]
   			 	\node[state,initial]  (s_0)                       {$s_{0}$};
  			 	\node[state,accepting](s_1) [right of= s_0] {$s_{1}$};
  			 	\node[state]          (s_2) [right of= s_1] {$s_{2}$};
    			\node[state,accepting](s_3) [right of= s_2] {$s_3$};
    			\draw[->] 	(s_0) 	edge              	node        {$a,b$} (s_1);
              	\draw[->]	(s_1) 	to [bend left=20] 	node        {$a,b$} (s_2);
              	\draw[->]	(s_2) 	to [bend left=20] 	node 		{$b$} 	(s_3);
              	\draw[->]	(s_2)	to [bend left=20] 	node 		{$a$} 	(s_1);
              	\draw[->]	(s_3)	to [bend left=20] 	node 		{$a,b$} (s_2);
  \end{tikzpicture}
\end{figure}	
	\end{exampleblock}
\end{frame}

\begin{frame}{Endlicher Akzeptor}

	\begin{exampleblock}{Lösung zur Aufgabe zu endlichen Akzeptoren}
		\begin{enumerate}
			\item siehe Tafel
		\item siehe Tafel
		\item siehe Tafel
		\item siehe Tafel
		\item $L(A)= \setc{w \in \set{a,b}^{\ast}}{\,\setsize{w} mod\, 2 = 1} = \setC{w \in \set{a,b}^{\ast}}{$w$ hat ungerade Länge}$
		\end{enumerate}		
	\end{exampleblock}
\end{frame}

%%%%%%%%%% %%%%%%%%%%
%% Zusammenfassung
\section{}
%\subsection{Zusammenfassung}
	\begin{frame}{Was ihr jetzt kennen und können solltet\dots}
			\begin{itemize}
				\item \emph{Laufzeiten} von Algorithmen angeben und abschätzen
				\item Mit dem \emph{$O$-Kalkül} arbeiten
				\item Die Laufzeit rekursiver Algorithmen mit dem \emph{Mastertheorem} bestimmen
				\item Mealy- und Moore Automaten sowie Endliche Akzeptoren erkennen und zeichnen
			\end{itemize}
	
	\end{frame}
\section{}
\questionframe
\lastframe
\mode<handout>{\slideThanks}
\end{document}
