\subsection{Definition}
% \begin{frame}
% 	Kann man mit einer Codierung die benötigte Anzahl der Zeichen für ein Wort reduzieren und trotzdem den Sinn erhalten?\\[0.5em]
% 	\pause
% 	Natürlich geht das (manchmal), dieses Verfahren ist überall im Einsatz:\\
% 	Komprimierung!
% \end{frame}

\begin{frame}{Huffman-Codierung}
	\textbf{Ziel}: möglichst kurze, präfixfreie Codierungen für beliebige Wörter.
	\pause
	\begin{exampleblock}{}
		Eine Huffman-Codierung ist ein präfixfreier (und demnach einfach zu decodierenden) Homomorphismus, bei der die Codierung eines Zeichens umso länger wird, je seltener das Zeichen vorkommt.
	
		Die Huffman-Codierung für ein Wort ist dabei nicht eindeutig, wie wir gleich im Konstruktionsverfahren sehen werden.
	\end{exampleblock}
	% \begin{block}{Lemma}
	% 	Unter allen präfixfreien Codes führen Huffman-Codes zu kürzesten Codierungen
	% 	\textbf{des Wortes, für das die Huffman-Codierung konstruiert wurde.}
	% \end{block}
\end{frame}
\begin{frame}
	\begin{block}{Nützliche (informelle) Definitionen zu Bäumen}
		\begin{itemize}
			\item \textbf{Baum}: besteht aus Knoten, die über Kanten miteinander verbunden sind
			\item \textbf{Wurzel}: Knoten, von dem aus alle anderen Knoten erreichbar sind
			\item \textbf{Blatt}: Ein Knoten ohne ausgehende Kanten
			\item \textbf{Innerer Knoten}: Alle Knoten, die keine Blätter sind
		\end{itemize}
	\end{block}
\end{frame}

\begin{frame}{Huffman-Codierung}
	\begin{block}{Baum konstruieren (informell)}
		\begin{enumerate}
			\item \textbf{Alle Zeichen} mit ihrer \textbf{Häufigkeit} als Blätter in die unterste Ebene zeichnen.
			\item Jeweils zwei Knoten/Bäume (nicht unbeding Blätter!) mit den \textbf{geringsten Häufigkeiten} verbinden, indem man ihnen eine gemeinsame Wurzel gibst, die die Summe der Häufigkeiten erhält.
			\item Fortfahren, bis nur noch \textbf{ein Baum} übrig ist.
			\item Die linken Äste jeweils mit \bzero \ beschriften, die rechten Äste mit \bone.
			\item Die Codierung einzelner Zeichen kann man entlang der Pfade von der Wurzel zum jeweiligen Zeichen ablesen.
		\end{enumerate}
	\end{block}
	\begin{exampleblock}{Am Beispiel}
		Codiere das Wort $w= cabadcdaac $
	\end{exampleblock}
\end{frame}

\begin{frame}{Huffman-Codierung}
    \begin{block}{Lösung}
    \begin{columns}[T] % align columns
    	\begin{column}{.58\textwidth}
			\begin{figure}[b]
				%\centering
				\begin{tikzpicture}
				[level 1/.style={sibling distance=40mm},
				level 2/.style={sibling distance=20mm},
				level 3/.style={sibling distance=15mm}]
				\node {$10$}
				child {
					node{$6$}
					child{
						node{$3$}
						child{
							node{$1,b$}
							edge from parent node[left] {\bzero}
						}
						child {
							node{$2,d$}
							edge from parent node[right] {\bone}
						}
						edge from parent node[left] {\bzero};
					}
					child{
						node{$3,c$}
						edge from parent node[right] {\bone}
					}
					edge from parent node[left] {\bzero};
				}
				child{
					node{$4,a$}
					edge from parent node[right] {\bone}
				};
				\end{tikzpicture}
			\end{figure}  
		\end{column}
			\hfill
		\begin{column}{.44\textwidth}
			\hfill
			\hfill 
			\vspace*{0.1\linewidth}
			\begin{table}[H]
				\begin{tabular}{c|cccc}
					\hline
					x & a & b & c & d  \\ \hline
					$N_x(w)$  & 4 & 1 & 3 & 2 \\ \hline
					h(x) & 1 & 000 & 01 & 001 \\ \hline 
				\end{tabular}
			\end{table}
		\end{column}
	\end{columns}
    \end{block}
\end{frame}

\begin{frame}{Huffman-Codierung}
	\begin{block}{``Operator'' Huffman-Codierung}
		\begin{enumerate}
			\item Häufigkeiten zählen
			\item Baum konstruieren
			\item Abbildung definieren
			\item Codierung des Wortes
		\end{enumerate}
	\end{block}

	\begin{exampleblock}{Aufgabe}
		Betrachte das Alphabet $A:=\set{a,b,c,d,e}.$
		\begin{enumerate}
			\item Bestimme die Huffman-Codierung $h_1(w)$ des Wortes $w:=bbeacdbbea$. Wie ist $|h_1(w)|?$
			\pause \item Bestimme die Huffman-Codierung $h_2(u)$ mit $u:=a^2 b^4 c^8 d^{16} e^{32}$.
			\pause \item Zerlege $w$ in Zweierblöcke und fasse diese als Zeichen auf. Wie ist nun die Blockcodierung $h_3(w)$? Wie verhält sich $|h_1(w)|$ zu $|h_3(w)|$?
		\end{enumerate}
	\end{exampleblock}
\end{frame}