\subsection{Vorbereitung Übungsblatt 4}
\begin{frame}{ÜB-Tipps: Konkatenationsabschluss}
    \[
    	L^\ast = \bigcup_{i = 0}^\infty L^i
    \]

    Also insbesondere:
    \begin{itemize}
    	\item $\set{\varepsilon} = L^0 \subseteq L^\ast$ für beliebige Sprachen $L$!
    	\item $\emptyset^\ast = \set{}^\ast = \emptyset^0 \cup \bigcup_{i = 1}^\infty \emptyset^i = \set{\varepsilon} \cup \bigcup_{i = 1}^\infty \emptyset^{i-1} \cdot \emptyset = \set{\varepsilon}$
    \end{itemize}
\end{frame}

\begin{frame}{ÜB-Tipps: Induktive Abbildungen auf Wörtern}
	
	\begin{itemize}
		\item Induktiv definierte Abbildungen auf Wörtern verwenden oft Konkatenation in Abbildungsvorschrift
		\item z.B. Funktion $inv : A^\ast \to A^\ast$ auf bel. Alphabet $A$, die Wörter \enquote{umdreht}:
	\end{itemize}
	\begin{align*}
		inv(\varepsilon) &= \varepsilon \\
		\forall w \in A^\ast \forall x \in A: inv(wx) &= x \cdot inv(w)
	\end{align*}
	\begin{itemize}
		\item Achten auf: Genaue Definition der Operanden in der Konkatenation \begin{itemize}
			\item Hier ein Wort $w \in A^\ast$ und ein letztes Zeichen $x \in A$
			\item Auf Übungsblatt binäre Operation auf immer zwei solcher Wörter
		\end{itemize}
	\end{itemize}

\end{frame}