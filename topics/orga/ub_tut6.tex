\subsection{Orga}
\begin{frame}{Organisatorisches}
    \begin{itemize}
    	\item Anmeldung f. Klausur und Übungsschein sind geöffnet
    \end{itemize}


\end{frame}

\Moritz{

\begin{frame}{Zu Übungsblatt 4}

	\textbf{Aufgabe 1):}
	\begin{itemize}
		\item Auf Präzedenz der Operationen auf Sprachen achten!
		\item Produkt ``$\cdot$'' bindet stärker als Vereinigung ``$\cup$''
		\item \begin{align*}
			L &= \set{ab}^\ast \cup \set{a} \cdot ({a} \cdot \set{a,ab})^\ast \cdot \set{bb}\\
			&= 	(\set{ab}^\ast) \cup (\set{a} \cdot ({a} \cdot \set{a,ab})^\ast \cdot \set{bb})
		\end{align*}
	\end{itemize}

\end{frame}
	
\begin{frame}{Zu Übungsblatt 4}
	
	\textbf{Aufgabe 3b):}
	\begin{itemize}
		\item $A(L):\; (\{a\}\cdot L)^\ast \cap A^+ = L$
		\item ``Zeigen Sie: Wenn $A(L)$ gilt, dann ist $\varepsilon \notin L$.''
		\item Gute Idee: Kontraposition (= Umkehrschluss) \begin{itemize}
			\item Also zeigen: $\varepsilon \in L \, \rightarrow $ ($A(L)$ gilt nicht)
			\item Kontraposition $\ne$ Beweis durch Widerspruch!
		\end{itemize}
		\pause
		\item Nicht so gute Idee: Einfach $\varepsilon$ für $L$ einsetzen \begin{itemize}
			\item \textcolor{red}{$(\{a\}\cdot \set{\varepsilon})^\ast \cap A^+ = \set{\varepsilon}$}
			\item Das ist so, als würde man sagen ``Weil $\varepsilon \in L$, gilt $L=\set{\varepsilon}$''
		\end{itemize}
	\end{itemize}

\end{frame}

\begin{frame}{Zu Übungsblatt 4}
	\textbf{Aufgabe 3b):}
	\begin{itemize}
		\item Bei Mengen über einzelne, beliebige Elemente argumentieren meist am Einfachsten. \begin{itemize}
			\item Sei $w \in (\{a\}\cdot L)^\ast \cap A^+$ beliebig.
			\item Dann gilt $w \in A^+$, womit $|w| > 0$. Wegen $|\varepsilon| = 0$ also $\varepsilon \notin (\{a\}\cdot L)^\ast \cap A^+$.
			\item Aber $\varepsilon \in L$, also $L \ne (\{a\}\cdot L)^\ast \cap A^+$.
			\item $A(L)$ gilt nicht.
		\end{itemize}
	\end{itemize}
\end{frame}

}