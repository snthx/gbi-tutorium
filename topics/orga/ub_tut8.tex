\Stephan{
\subsection{Übungsblatt 6}
\begin{frame}{Zum 6. Übungsblatt}
	\begin{itemize}
	\item Lief sehr gut!
	\item Aufgabe 1
	\begin{itemize}
		\item Die MIMA ist eine Maschine!
		\item Die MIMA verhält sich anders, als die mathematischen Algorithmen, die sie versucht zu berechnen.
		\item Insbesondere gibt es \textbf{Overflows und Underflows}, so wie beim echten Computer auch.
		\item Die MIMA hat Werte mit 24 Bit Länge.
		\item Steht in einem Register das Wort
		\[
			\text{Num}_{2}(8388607) = \#{01}^{23} = \text{Zkpl}_{24}(8388607)
		\]
		und wird 1 addiert, so steht dort nun
		\[
			\text{Num}_{2}(8388608) = \#{10}^{23} = \text{Zkpl}_{24}(-1)	
		\]
	\end{itemize}
	\item Aufgabe 3\\
	Schnellste Lösung: Kevin Tobias Haag (154 Anweisungen)\\
	Kürzeste Lösung: 9 Befehle
	\end{itemize}
\end{frame}
}