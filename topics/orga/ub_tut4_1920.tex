\Moritz{
	
\begin{frame}{2. Übungsblatt}
	
	\emph{Aufgabe 1:}
	\begin{itemize}
		\item Diese Aufgabe verlangt ein bisschen Kreativität
		\item Bitte überlegt wirklich gründlich selbst, bevor ihr anfangt zu googlen o.Ä.
		\item In der Klausur wird es auch keine Recherche-Möglichkeiten geben
		\item \url{https://www.stacklounge.de/5060/binare-operation-kommutativ-aber-nicht-assoziativ}
	\end{itemize}

\end{frame}
	
\begin{frame}{2. Übungsblatt}

	\emph{Aufgabe 3:}
	\begin{itemize}
		\item Bei Beweisen ist es meistens schön Gleichungsketten (oder Ungleichungsketten) zu verwenden
		\item Also z.B. beim Beweis von Assoziativität:
			\begin{itemize}
				\item Nicht schön mit Behauptung anzufangen: 
					\begin{align*}
						((t_1,b_1) \diamond (t_2,b_2)) \diamond (t_3,b_3) &= (t_1,b_1) \diamond ((t_2,b_2) \diamond (t_3,b_3)) \\
						\textcolor{orange}{\Leftrightarrow} (t_1 \cdot t_2, b_1 \cdot b_2)  \diamond (t_3,b_3) &= (t_1,b_1) \diamond (t_2 \cdot t_3, b_2 \cdot b_3) \\
						\textcolor{orange}{\Leftrightarrow} (t_1 \cdot t_2 \cdot t_3, b_1 \cdot b_2 \cdot b_3) &= (t_1 \cdot t_2 \cdot t_3, b_1 \cdot b_2 \cdot b_3)
					\end{align*}
				\item Schöner Behauptung aus Gleichungskette zu schließen:
					\begin{align*}
						&((t_1,b_1) \diamond (t_2,b_2)) \diamond (t_3,b_3) = (t_1 \cdot t_2, b_1 \cdot b_2)  \diamond (t_3,b_3) \\
						&= (t_1 \cdot t_2 \cdot t_3, b_1 \cdot b_2 \cdot b_3) = (t_1,b_1) \diamond (t_2 \cdot t_3, b_2 \cdot b_3) \\
						&= (t_1,b_1) \diamond ((t_2,b_2) \diamond (t_3,b_3)) \\
					\end{align*}
			\end{itemize}
			
	\end{itemize}

\end{frame}
	
}

\Stephan{
	\begin{frame}{2. Übungsblatt}
	    \begin{itemize}
	    	\item Von 64 \% auf 72 \% gesteigert!
	    	\item Aufg. 1: \emph{Wohldefiniertheit}
	    \end{itemize}
	\emph{Aufgabe 3:}
	\begin{itemize}
		\item Bei Beweisen ist es meistens schön Gleichungsketten (oder Ungleichungsketten) zu verwenden
		\item Also z.B. beim Beweis von Assoziativität:
			\begin{itemize}
				\item Nicht schön mit Behauptung anzufangen: 
					\begin{align*}
						((t_1,b_1) \diamond (t_2,b_2)) \diamond (t_3,b_3) &= (t_1,b_1) \diamond ((t_2,b_2) \diamond (t_3,b_3)) \\
						\textcolor{orange}{\Leftrightarrow} (t_1 \cdot t_2, b_1 \cdot b_2)  \diamond (t_3,b_3) &= (t_1,b_1) \diamond (t_2 \cdot t_3, b_2 \cdot b_3) \\
						\textcolor{orange}{\Leftrightarrow} (t_1 \cdot t_2 \cdot t_3, b_1 \cdot b_2 \cdot b_3) &= (t_1 \cdot t_2 \cdot t_3, b_1 \cdot b_2 \cdot b_3)
					\end{align*}
				\item Schöner Behauptung aus Gleichungskette zu schließen:
					\begin{align*}
						&((t_1,b_1) \diamond (t_2,b_2)) \diamond (t_3,b_3) = (t_1 \cdot t_2, b_1 \cdot b_2)  \diamond (t_3,b_3) \\
						&= (t_1 \cdot t_2 \cdot t_3, b_1 \cdot b_2 \cdot b_3) = (t_1,b_1) \diamond (t_2 \cdot t_3, b_2 \cdot b_3) \\
						&= (t_1,b_1) \diamond ((t_2,b_2) \diamond (t_3,b_3)) \\
					\end{align*}
			\end{itemize}
			
	\end{itemize}

\end{frame}







}