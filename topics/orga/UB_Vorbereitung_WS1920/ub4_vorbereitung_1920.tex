\subsection{Vorbereitng Übungsblatt 4}
\begin{frame}{ÜB-Tipps: Formale Sprachen}
    \[
    	L^+ = L \cdot L^*
    \]

    \bigskip

    \centering $\Rightarrow$ Also im Allgemeinen $L^+ \ne L^*$
\end{frame}

\begin{frame}{ÜB-Tipps: \enquote{Starke} vollständige Induktion}
	\begin{columns}

		\begin{column}[t]{.5\textwidth}
			\textbf{\enquote{Normale} VI}
			\begin{itemize}
				\item IV: Annahme, dass Aussage $A(n)$ wahr für ein $n \in  \nN$
				\item IS: $A(n)$ wahr $\rightarrow$ $A(n+1)$ wahr
			\end{itemize}			
		\end{column}

		\begin{column}[t]{.5\textwidth}
			\textbf{\enquote{Starke} VI}
			\begin{itemize}
				\item IV: Annahme, dass Aussage $A(n')$ wahr für \textbf{alle} $n' \in \nN$ mit $n' \le n$
				\item IS: $A(n')$ wahr für alle $n' \le n$ $\rightarrow$ $A(n+1)$ wahr
			\end{itemize}
		\end{column}
	
	\end{columns}

	\begin{alertblock}{Achtung}
		Immer, wenn möglich, normale vollständige Induktion verwenden! Starke vollständige Induktion nur verwenden, wenn angegeben oder unausweichbar!
	\end{alertblock}

\end{frame}