\subsection{Vorbereitung ÜB 5}
\begin{frame}{Auftauchende Zeichen und Begriffe}
	\begin{block}{Def.: $Z_k$}
		Das Alphabet $Z_k$ enthält Zeichen, die Ziffern mit den Werten von $0$ bis $k-1$ (inklusive) entsprechen. $Z_k$ enthält somit keine Zahlen, sondern Zeichen, die der Repräsentation von Zahlen als Wort dienen.
	\end{block}

	\begin{exampleblock}{Beispiele}
		\begin{align*}
			Z_{2} &= \set{\texttt{\textcolor{blue}{0,1}}}\\
			Z_{10} &= \set{\texttt{\textcolor{blue}{0,1,2,3,4,5,6,7,8,9}}}\\
			Z_{16} &= \set{\texttt{\textcolor{blue}{0,1,2,3,4,5,6,7,8,9,A,B,C,D,E,F}}}
		\end{align*}
	\end{exampleblock}
\end{frame}

\begin{frame}{Auftauchende Zeichen und Begriffe}
	
	\begin{block}{Linksinverse}
		Sei $f:A \to B$ eine Abbildung. Falls \[ g:B\to A \text{ mit } g \circ f = id_A\]
		existiert, so heißt $g$ eine \textbf{Linksinverse} zu $f$.
	\end{block}

	\begin{exampleblock}{Beispiel}
		Sei \[f: \nN_0 \to \nN_0, \; x \mapsto 2x\]
		Dann ist \[ g: \nN_0 \to \nN_0, \; y \mapsto \lfloor y/2 \rfloor \]
		eine Linksinverse von $f$.
	\end{exampleblock}

\end{frame}

\begin{frame}{Auftauchende Zeichen und Begriffe}

	\begin{block}{Rechtsinverse}
		Sei $f:A \to B$ eine Abbildung. Falls \[ g:B\to A \text{ mit } f \circ g = id_B\]
		existiert, so heißt $g$ eine \textbf{Rechtsinverse} zu $f$.
	\end{block}

\end{frame}