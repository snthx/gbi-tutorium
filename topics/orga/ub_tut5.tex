\Kilian{}

\Moritz{}

\Stephan{
\begin{frame}{Drittes Übungsblatt}
    \begin{itemize}
    	\item Knapp 70\%
    	\item Haltet euch bei Beweisen mit vollständiger Induktion an gelernte Schemen und Definitionen
    	\item Schreibt Gleichungsketten, keine Äquivalenzen
    	\item nicht abschreiben
    \end{itemize}


\end{frame}

\begin{frame}{Drittes Übungsblatt}

\emph{Aufgabe 2}
\begin{itemize}
	\item $u_n$ ist eine \textit{Zahl}, keine Menge von Formeln
	\item (c) Häufige Antwort:\\
		  Die Korrektheit folgt aus der rekursiven Definition.\\ Es gilt $u_0=1$ (IA) sowie $u_{n+1}$, wenn $u_n$ gilt (IS).
\end{itemize}
\end{frame}




\newcommand{\lul}[2]{
	\temporal<2>{#1}{\textcolor{red}{#1}}{\textcolor{kit-green100}{#2}}
}

\newcommand{\lulz}[1]{\lul{#1}{#1}}


\begin{frame}{Drittes Übungsblatt}
\emph{Aufgabe 3a}
\begin{itemize}
	\item IA: $B(0,0) = 1 \geq 2^0$ \lul{}{und $B(1,1)=B(1,0)+B(0,1)=2 \geq 2^1$}
	\item IS: Es sei $n \in \nN_{\lul{0}{+}}$\\
	IV: $B(n,n) \geq 2^n$ gelte. Dann
	\begin{align*}
		B(n+1,n+1)	&= B(n+1,n) + B(n,n+1)\\
					&= \underbrace{B(n+1,\lulz{n-1})}_{\lul{>}{\geq}0} + B(n,n) + B(n,n) + \underbrace{B(\lulz{n-1},n+1)}_{\lul{>}{\geq}0}\\
					&\geq 2 B(n,n) \overset{IV}{\geq} 2 \cdot 2^n = 2^n+1 
	\end{align*}
	\item[]
	\item \only<-2>{Was ist falsch?} \only<3>{Beachtet die \emph{schöne Gleichungskette}}
\end{itemize}
\end{frame}
}