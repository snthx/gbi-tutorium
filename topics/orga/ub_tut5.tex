\newcommand{\lul}[2]{
	\temporal<2>{#1}{\textcolor{red}{#1}}{\textcolor{kit-green100}{#2}}
}

\newcommand{\lulz}[1]{\lul{#1}{#1}}




<<<<<<< HEAD
\begin{frame}{Allgemein}
    %\begin{itemize}
    %	\item \emph{Probeklausur} im Januar
    %\end{itemize}
    %\pause
    \begin{block}{Def.: Potenz von Abbildungen}
    	Ist $f$ eine Abbildung auf $A$ und $x \in A$, dann ist
    	\begin{align*}
    		P^1(x) &:= P(x) \\
    		\text{für alle } n\in \nN_+ \quad P^{n+1}(x) &:= P(P^n(x))
    	\end{align*}
    \end{block}

    \begin{itemize}
    	\item \alert{Achtung}: Prüft, ob es sich um eine Abbildung, ein Wort, eine Menge, etc. handelt, Potenzen heißen jeweils unterschiedliches
    \end{itemize}

\end{frame}

\begin{frame}{Allgemein}
\begin{itemize}
	\item Wie ausführlich muss man bei den Abbildungen \textbf{Repr}, \textbf{Num} und \textbf{Zkpl} und weiteren kleinschrittigen Abbildungen vorgehen?
	\item[]
	\item Achtet auf den Operator
	\begin{itemize}
		\item Gib an: Nur das Ergebnis
		\item Gebt dabei die einzelnen Abbildungsschritte an (o.ä.): Die einzelnen Abbildungsschritte angeben
		\item Berechnet: wie gib an, sollte in Zukunft nicht mehr als Operator verwendet werden
	\end{itemize}
	\item Im Zweifelsfall alles angeben, mehr schadet nie
\end{itemize}
\end{frame}




=======
>>>>>>> cae5215e3bcd8f26c0fe43490450c03cfede38f9
\Kilian{}

\Moritz{

\begin{frame}{Probeklausur}
	\begin{itemize}
		\item Wahrscheinlich Probeklausur im Januar
		\item Ein Termin statt Vorlesung
		\item Vorteile:\begin{itemize}
			\item Selbstkontrolle
			\item Simulation Prüfungsumgebung
			\item freiwillig
		\end{itemize}
		\item Von mir korrigiert und im Tut zurückgegeben \begin{itemize}
			\item Punkte wirken sich \textbf{nicht} auf Übungsschein aus!
		\end{itemize}
	\end{itemize}
\end{frame}

\begin{frame}{Übungsschein}
	\begin{itemize}
		\item Zitat von Übungsblatt 5:
	\end{itemize}
		\begin{quote}
			Auf den ersten 6 Aufgabenblättern wird man insgesamt genau 120 Punkte erreichen können. Wer den Übungsschein erwerben will, kann dies also nur dann sicher schaffen, wenn auf den ersten 6 Aufgabenblättern mindestens 60 Punkte erreicht werden.
		\end{quote}
	\begin{itemize}
		\item Das heißt \textbf{nicht}, dass man mit weniger als 60 Punkten garantiert keinen Übungsschein bekommt
		\item Dr. Worsch hat darauf hingewiesen, dass auch mal beide Augen zugedrückt werden
		\item Falls also bisher knapp hinterher: Dranbleiben!
	\end{itemize}
\end{frame}

}

% \Stephan{
% \begin{frame}{Drittes Übungsblatt}
%     \begin{itemize}
%     	\item Knapp 70\%
%     	\item Haltet euch bei Beweisen mit vollständiger Induktion an gelernte Schemen und Definitionen
%     	\item Schreibt Gleichungsketten, keine Äquivalenzen
%     	\item nicht abschreiben
%     \end{itemize}


% \end{frame}

% \begin{frame}{Drittes Übungsblatt}

% \emph{Aufgabe 2}
% \begin{itemize}
% 	\item $u_n$ ist eine \textit{Zahl}, keine Menge von Formeln
% 	\item (c) Häufige Antwort:\\
% 		  Die Korrektheit folgt aus der rekursiven Definition.\\ Es gilt $u_0=1$ (IA) sowie $u_{n+1}$, wenn $u_n$ gilt (IS).
% \end{itemize}
% \end{frame}







% \begin{frame}{Drittes Übungsblatt}
% \emph{Aufgabe 3a}
% \begin{itemize}
% 	\item IA: $B(0,0) = 1 \geq 2^0$ \lul{}{und $B(1,1)=B(1,0)+B(0,1)=2 \geq 2^1$}
% 	\item IS: Es sei $n \in \nN_{\lul{0}{+}}$\\
% 	IV: $B(n,n) \geq 2^n$ gelte. Dann
% 	\begin{align*}
% 		B(n+1,n+1)	&= B(n+1,n) + B(n,n+1)\\
% 					&= \underbrace{B(n+1,\lulz{n-1})}_{\lul{>}{\geq}0} + B(n,n) + B(n,n) + \underbrace{B(\lulz{n-1},n+1)}_{\lul{>}{\geq}0}\\
% 					&\geq 2 B(n,n) \overset{IV}{\geq} 2 \cdot 2^n = 2^{n+1} 
% 	\end{align*}
% 	\item[]
% 	\item \only<-2>{Was ist falsch?} \only<3>{Beachtet die \emph{schöne Gleichungskette}}
% \end{itemize}
% \end{frame}

% \begin{frame}{Viertes Übungsblatt}
% 	\begin{itemize}
% 		\item Schnitt: 83\% (Das zweitbeste ÜB bisher!)
% 		\item Viele schöne Lösungen
% 		\item Das selbe wie im dritten ÜB: Definitionen und Formalismen einhalten, nicht abschreiben
% 		\item[]
% 		\item Unnötig: $M := \{x \in A^*\}$ statt $M := A*$
% 	\end{itemize}
% \end{frame}

% \begin{frame}{Viertes Übungsblatt}
% 	\emph{Aufgabe 4.1 + 4.2}
% 	\begin{itemize}
% 		\item Operationen auf Sprachen ergeben wieder Sprachen: \\ Konkatenation ($\cdot$), Potenzieren, Mengenoperationen ($\cup,\cap,\setminus$) und Konkatenationsabschluss
% 		\item $\Rightarrow$ Keine weiteren Mengenklammern um Sprachen!
% 		\only<2>{
% 			\item \textcolor{red}{\textbf{Falsch:} $L_{1} = \set{ \set{a,b}^{\ast} \cdot \set{aa,bb,ab,ba} }$
% 		}}
% 		\only<3-|handout:0>{
% 			\item \textcolor{red}{\sout{$L_{1} = \set{ \set{a,b}^{\ast} \cdot \set{aa,bb,ab,ba} }$}} 
% 			\item \textcolor{kit-green100}{\textbf{Korrekt: } $L_{1} = \set{a,b}^{\ast} \cdot \set{aa,bb,ab,ba}$}
% 		}
% 	\end{itemize}
% \end{frame}

% \begin{frame}{zu ÜB 4 Aufg. 1 + 2}
% 	\begin{itemize}
% 		\item Entweder Konkatenation von Sprachen oder Konkatenation von Wörtern, nicht Sprachen mit Wörtern konkatenieren!
% 		\item Beispiel: Sei $A=\set{a,b}$ ein Alphabet.
% 		\only<1>{
% 			\item \textcolor{red}{\textbf{Falsch:} $aa \cdot \set{b}^{k}$, $\varepsilon \cdot A^{+}$}
% 		}
% 		\only<2-|handout:0>{
% 			\item \textcolor{red}{\sout{\textbf{Falsch:} $aa \cdot \set{b}^{k}$, $\varepsilon \cdot A^{+}$}}
% 			\item \textcolor{kit-green100}{\textbf{Korrekt:}} $aa \cdot b^{k}$, $\set{\varepsilon} \cdot A^{+}$
% 		}
% 	\end{itemize}
% \end{frame}
% }

\Stephan{
	\begin{frame}{Probeklausur}
	\begin{itemize}
		\item Wahrscheinlich Probeklausur im Januar
		\item Ein Termin statt Vorlesung
		\item Vorteile:\begin{itemize}
			\item Selbstkontrolle
			\item Simulation Prüfungsumgebung
			\item freiwillig
		\end{itemize}
		\item Von mir korrigiert und im Tut zurückgegeben \begin{itemize}
			\item Punkte wirken sich \textbf{nicht} auf Übungsschein aus!
		\end{itemize}
	\end{itemize}
\end{frame}

\begin{frame}{Übungsschein}
	\begin{itemize}
		\item Zitat von Übungsblatt 5:
	\end{itemize}
		\begin{quote}
			Auf den ersten 6 Aufgabenblättern wird man insgesamt genau 120 Punkte erreichen können. Wer den Übungsschein erwerben will, kann dies also nur dann sicher schaffen, wenn auf den ersten 6 Aufgabenblättern mindestens 60 Punkte erreicht werden.
		\end{quote}
	\begin{itemize}
		\item Das heißt \textbf{nicht}, dass man mit weniger als 60 Punkten garantiert keinen Übungsschein bekommt
		\item Dr. Worsch hat darauf hingewiesen, dass auch mal beide Augen zugedrückt werden
		\item Falls also bisher knapp hinterher: Dranbleiben!
	\end{itemize}
\end{frame}

}