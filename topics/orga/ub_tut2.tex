\Moritz{
	
\begin{frame}{Zum ersten Übungsblatt}
	
	\textbf{Aufg. 1:}
	\begin{itemize}
		\item Denkweise bei Beweis von Implikation $A \Rightarrow B$ \\(\enquote{Es gilt A, \textbf{wenn} B gilt}): \begin{itemize}
			\item Nehme an, dass $A$ gilt
			\item Dann folgere $B$
		\end{itemize}
		\item Denkweise bei Beweis von Äquivalenz $A \Leftrightarrow B$ \\(\enquote{Es gilt A \textbf{genau dann, wenn} B gilt}): \begin{itemize}
			\item Zeige $A \Rightarrow B$ und
			\item Zeige $B \Rightarrow A$
		\end{itemize}
		\item Immer nur die Informationen verwenden, die bekannt sind aus \begin{itemize}
			\item Aufgabenstellung
			\item Annahmen (z.B. bei Implikation)
		\end{itemize}
		\item Speziell durften hier die Mengen $A,B,C$ nicht festgelegt werden, da die Aufgabenstellung das nicht hergibt
		\item Relation auf Mengen gegeben, z.B. $B \subseteq C$ in Annahme $\Rightarrow$ Informationen über \textit{Elemente der Mengen} bekannt!
	\end{itemize}

\end{frame}

}