% \Stephan{\subsection{Übungsblatt 5}
% 	\subsection{Mima Simulatoren} % (fold)
\begin{frame}{MIMA-Simulationen}
	    \begin{itemize}
	    	\item \url{github.com/WArBp-3R/kit-mima-sim}
	    	\item \url{github.com/C1bergh0st/MiMaSimlator}
	    	\item \url{github.com/cbdevnet/mima}
	    	\item \url{github.com/phiresky/mima}
	    \end{itemize}
	\end{frame}

	    
% \begin{frame}{Zum 5. Übungsblatt}
% 	\emph{Homomorphismen}
% 	\begin{itemize}
% 		\item Homomorphismen sind bei Wörtern immer als $h:A^{\ast} \to B^{\ast}$ definiert (mit $A,B$ Alphabete).
% 		\item Man kann einen Homorphismus definieren, indem man nur eine Funktion $g:A\to B^{\ast}$ angibt, die einzelne Zeichen abbildet. 
% 		\item \textbf{ABER:} Dann muss man den induzierten Homomorphismus $g^{\ast\ast}$ von $g$ als gesuchten Homomorphismus angeben!
% 		\item Beispiel (Aufg 5.3):

% 		\begin{align*}
% 			g_1&: \set{0,1} \to \set{0,1}^\ast, \; g_1(0)=1, \; g_1(0)=0 \\
% 			h_1& = g_{1}^{\ast\ast}
% 		\end{align*}
% 	\end{itemize}
% \end{frame}
% }

\Moritz{
	
\subsection{Übungsblatt 5}
\begin{frame}{Zum 5. Übungsblatt}
	\emph{Verwendung von Zeichen}
	\begin{itemize}
		\item Mathematik: \begin{itemize}
			\item Produkt: \enquote{$\cdot$}
			\item großes Produkt: \enquote{$\prod$}
		\end{itemize}
		\item GBI: \begin{itemize}
			\item Konkatenation: \enquote{$\cdot$}
			\item große Konkatenation: \textbf{NICHT definiert!} Definieren, wenn sie verwendet wird (z.B. ``\enquote{$\bigcirc$} sei die große Konkatenation'')
		\end{itemize}
	\end{itemize}
	
\end{frame}

}