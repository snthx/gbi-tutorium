\Moritz{
	
\subsection{Übungsblatt 5}
\begin{frame}{Zum 5. Übungsblatt}
	\emph{Homomorphismen}
	\begin{itemize}
		\item Homomorphismen sind bei Wörtern immer als $h:A^{\ast} \to B^{\ast}$ definiert (mit $A,B$ Alphabete).
		\item Man kann einen Homorphismus definieren, indem man nur eine Funktion $g:A\to B^{\ast}$ angibt, die einzelne Zeichen abbildet. 
		\item \textbf{ABER:} Dann muss man den induzierten Homomorphismus $g^{\ast\ast}$ von $g$ als gesuchten Homomorphismus angeben!
		\item Beispiel (Aufg 5.3):
		\[ g_1: \set{0,1}^\ast \to \set{0,1}^\ast, \; g_1(0)=1, \; g_1(0)=0\]
		\[ h_1 = g^{\ast\ast}\]
	\end{itemize}
\end{frame}

\subsection{Übungsblatt 6}
\begin{frame}{Zum 6. Übungsblatt}
	\emph{MIMA ist eine Maschine!}
	\begin{itemize}
		\item Die MIMA verhält sich anders, als die mathematischen Algorithmen, die sie versucht zu berechnen.
		\item Insbesondere gibt es \textbf{Overflows und Underflows}, so wie beim echten Computer auch.
		\item Die MIMA hat Werte mit 24 Bit Länge: \begin{equation}
			\text{\textit{Nimm an in Adresse x steht }} Zkpl_{24}(-2^{23}) \\
			LDC 0 \\
			NOT \\
			ADD x \\
		\end{equation}
		liefert $Akku = Zkpl_{24}(2^{23}-1)$
	\end{itemize}
\end{frame}

}