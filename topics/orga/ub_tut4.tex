\Kilian{}

\Moritz{}

\Stephan{
	\begin{frame}{Zweites Übungsblatt}
		\emph{Aufgabe 1}
		\begin{itemize}
			\item $M_y = \setC{x \in \nR_0^+}{f(x)=y} \neq \set{0,1} $
			\item Oft solche Angaben: $M_0 = \setc{x \in \nN_0}{f(x)=0}$.\\
	    	      Die Einschränkung ist unnötig, $M_0 = \nN_0$ reicht.
		\end{itemize}
		\pause
		\emph{Aufgabe 1c}
		\begin{itemize}
			\item Häufige Lösung: $h^n(n) = \lfloor \frac{n}{2^n} \rfloor$
			\item Das Ergebnis ist korrekt, da $h^n(n) = 0 =\lfloor \frac{n}{2^n} \rfloor$
			\item Die Erkenntnis ($\forall a,b : h^a (b) = \lfloor \frac{b}{2^a} \rfloor$) falsch, da in \textit{jedem} Schritt abgerundet wird
		\end{itemize}
		\pause
		\emph{Aufgabe 3d}
		\begin{itemize}
	    	\item Häufige Lösung: $((k \diamond n) \diamond (k-n-1)) + ((n \diamond k) \diamond (n-k-1))$
	    	\item Problem: $k-n-1$ und $n-k-1$ können negativ werden, dafür die Operation $\diamond$ nicht definiert
	    \end{itemize}	
	\end{frame}
}