\subsection{Tutorium}

	\aboutMeFrame

	\Moritz{ 
		\showmessage{Und ihr?}
	}

	% IDEEN: Was ist GBI
	% Was wird in der Übung gemacht

	\begin{frame}{Infos zum Tutorium}
		\begin{itemize}
			\item Name: \myname
			\item Tutoriumsnummer: \mytutnumber
			\item \mytutinfos
		\end{itemize}	
	\end{frame}

	\begin{frame}{Infos zum Tutorium}
		Tutorium ist...\\
		\begin{itemize}
			\item Wiederholung der Vorlesung
			\item Gemeinsames Üben des aktuellen Stoffes
			\item Erste Anlaufstelle für Fragen
			\item Ausgabestelle der korrigierten Übungsblätter
		\end{itemize}
		\pause
		Tutorium ist nicht...\\
		\begin{itemize}
			\item Ersatz für die Vorlesung
			\item Lösen des kommenden Übungsblattes
		\end{itemize}
	\end{frame}
	
	\showmessage{Mitarbeit ist erwünscht!}
	
\subsection{Übungsblätter}
	\begin{frame}{Übungsblätter}
		\textbf{Ausgabe:} Mittwochs im ILIAS-Forum \\
		\pause
		\textbf{Abgabe:}
			\begin{itemize}
				\item Dienstag d. zweiten Folgewoche, 12:30 Uhr
				\item richtiger Briefkasten im Infobau-UG (sortiert nach Tut.-Nummer)
				\pause
				\item \textbf{einzeln} und \textbf{handschriftlich} bearbeitet, Abschreiben führt zu Nichtbestehen des Scheines. Für die Übungsblätter 1-6 ist eine Abgabe zu zweit erlaubt.
				\item Blätter getackert
			\end{itemize}
		\pause	
		\textbf{Rückgabe:} Im Tutorium
	\end{frame}
\subsection{Schein/Klausur}
 \begin{frame}{Modul GBI}
		\textbf{Übungsschein}
			\begin{itemize}
				\item Erhält, wer jeweils auf den Übungsblättern 1-6 und 7-12 mindestens 50\% der Punkte erziehlt
				\item Ist keine Voraussetzung für die Teilnahme an der Klausur
				\item Kann nur im Wintersemester gemacht werden, muss also dieses Semester versucht werden
				\item \textbf{Übungsschein wird zum Bestehen des Moduls benötigt} 
			\end{itemize}
		\pause	
		\textbf{Klausur}
		\begin{itemize}
				\item Datum: 18.03.20, 14-16 Uhr
				\item Nachklausur nach dem SS
				\item Klausurnote = Modulnote
				\item \textbf{Klausur wird zum Bestehen des Moduls benötigt} 
			\end{itemize}
	\end{frame}
	
	\showmessage{Orientierungsprüfung!}