\subsection{Tutorium}

	\aboutMeFrame

	\Moritz{ 
		\begin{frame}{Wer seid ihr?}
			\begin{itemize}
				\item Wie heißt Du?
				% \item Wie alt bist Du?
				\item Was studierst Du?
				\item Wie kommt es, dass Du Info studierst bzw. GBI hörst?
			\end{itemize}
		\end{frame}
	}

	% IDEEN: Was ist GBI
	% Was wird in der Übung gemacht

	\begin{frame}{Infos zum Tutorium}
		\begin{itemize}
			\item Name: \myname
			\item Tutoriumsnummer: \mytutnumber
			\item \mytutinfos
		\end{itemize}	
	\end{frame}

	\Stephan{\begin{frame}{Infos zum Tutorium}
	Tutorium ist...\\
		\begin{itemize}
			\item Wiederholung der Vorlesung
			\item Gemeinsames Üben des aktuellen Stoffes
			\item Erste Anlaufstelle für Fragen
			\item Ausgabestelle der korrigierten Übungsblätter
		\end{itemize}
	\pause
	Tutorium ist nicht...\\
		\begin{itemize}
			\item Ersatz für die Vorlesung
			\item Lösen des kommenden Übungsblattes
		\end{itemize}
	\pause
	Ich wünsche mir...
		\begin{itemize}
			\item Interesse, Mitarbeit, \pause und eine Reinschrift
		\end{itemize}
	\pause
	Aber das Wichtigste: Ich beantworte immer gern alle Fragen!\\[.5em]
			\begin{quote}
				Es gibt keine dummen Fragen, aber es kann sein, dass ich nur eine dumme Antwort weiß!
			\end{quote}
		\end{frame}

		\begin{frame}{Was macht ihr?}
			Im Tutorium:
			\begin{itemize}
				\item Vorlesungsstoff wiederholen durch Zuhören und Mitarbeiten bei Aufgaben
				\item Eure Fragen stellen bezüglich \begin{itemize}
					\item Stoff
					\item Aufgaben im Tut
					\item Aufgaben auf Übungsblättern
					\item Organisatorisches zu VL und Übung
					\item Organisatorisches zum Studium generell
				\end{itemize}
			\end{itemize}
			\pause
			\medskip
			Ansonsten:
			\begin{itemize}
				\item Fleißig Übungsblätter bearbeiten und rechtzeitig abgeben (bitte als Reinschrift)
				\item Jegliche Fragen auch per Mail gerne an mich
			\end{itemize}

		\end{frame}




	}

	\Kilian{\begin{frame}{Infos zum Tutorium}
			Tutorium ist...\\
			\begin{itemize}
				\item Wiederholung der Vorlesung
				\item Gemeinsames Üben des aktuellen Stoffes
				\item Erste Anlaufstelle für Fragen
				\item Ausgabestelle der korrigierten Übungsblätter
			\end{itemize}
			\pause
			Tutorium ist nicht...\\
			\begin{itemize}
				\item Ersatz für die Vorlesung
				\item Lösen des kommenden Übungsblattes
			\end{itemize}
		\end{frame}

		\showmessage{Mitarbeit ist erwünscht!}
	}

	\Jan{\begin{frame}{Infos zum Tutorium}
			Tutorium ist...\\
			\begin{itemize}
				\item Wiederholung der Vorlesung
				\item Gemeinsames Üben des aktuellen Stoffes
				\item Erste Anlaufstelle für Fragen
				\item Ausgabestelle der korrigierten Übungsblätter
			\end{itemize}
			\pause
			Tutorium ist nicht...\\
			\begin{itemize}
				\item Ersatz für die Vorlesung
				\item Lösen des kommenden Übungsblattes
			\end{itemize}
		\end{frame}

		\showmessage{Mitarbeit ist erwünscht!}
	}

	\Leonard{\begin{frame}{Infos zum Tutorium}
			Tutorium ist...\\
			\begin{itemize}
				\item Wiederholung der Vorlesung
				\item Gemeinsames Üben des aktuellen Stoffes
				\item Erste Anlaufstelle für Fragen
				\item Ausgabestelle der korrigierten Übungsblätter
			\end{itemize}
			\pause
			Tutorium ist nicht...\\
			\begin{itemize}
				\item Ersatz für die Vorlesung
				\item Lösen des kommenden Übungsblattes
			\end{itemize}
		\end{frame}

		\showmessage{Mitarbeit ist erwünscht!}
	}

	\Moritz{
		\begin{frame}{Tutorium ergänzt Vorlesung}
			Ohne Tutorium ist Vorlesung doof:
			\begin{itemize}
				\item Alles trocken
				\item Keine Fragen
				\item Keine Praxis
			\end{itemize}

			\bigskip

			Tutorium ist cool, weil...
			\begin{itemize}
				\item Verdaulich
				\item Fragen willkommen!
				\item Übungsaufgaben
				\item Theorie -> Praxis
			\end{itemize}
		\end{frame}

		\begin{frame}{Was mache ich?}
			\begin{itemize}
				\item Ich bin ein Student, der irgendwann mal so wie ihr GBI gehört hat
				\item Ich will euch helfen... \begin{itemize}
					\item Zu sehen, dass GBI kein Hexenwerk ist
					\item In der Lage zu sein, Übungsblätter zu bearbeiten
					\item Formalitäten in GBI und dem ganzen Info-Studium zu verstehen
				\end{itemize}
			\end{itemize}
			\pause
			\medskip

			Aber das Wichtigste: Ich beantworte immer gern alle Fragen!\\[.5em]
			\begin{quote}
				Es gibt keine dummen Fragen, aber es kann sein, dass ich nur eine dumme Antwort weiß.
			\end{quote}

		\end{frame}

		\begin{frame}{Was macht ihr?}
			Im Tutorium:
			\begin{itemize}
				\item Vorlesungsstoff wiederholen durch Zuhören und Mitarbeiten bei Aufgaben
				\item Eure Fragen stellen bezüglich \begin{itemize}
					\item Stoff
					\item Aufgaben im Tut
					\item Aufgaben auf Übungsblättern
					\item Organisatorisches zu VL und Übung
					\item Organisatorisches zum Studium generell
				\end{itemize}
			\end{itemize}
			\pause
			\medskip
			Ansonsten:
			\begin{itemize}
				\item Fleißig Übungsblätter bearbeiten und rechtzeitig abgeben (bitte als Reinschrift)
				\item Jegliche Fragen auch per Mail gerne an mich
			\end{itemize}

		\end{frame}
	}
	
\subsection{Übungsblätter}
	\begin{frame}{Übungsblätter}
		\textbf{Ausgabe:} Im ILIAS \\
		\pause
		\textbf{Abgabe:}
			\begin{itemize}
				\item Erstes ÜB: Bis 17.11., 12:00 Uhr im ILIAS abgeben
				\pause
				\item \textbf{handschriftlich} bearbeitet und eingescannt
				\item Blätter getackert
				\item Übungsblätter 1-6: Abgabe zu zweit ausdrücklich erwünscht!
				\item Übungsblätter 7-12: Abgabe nur einzeln!
				\item Abschreiben führt zum Nichtbestehen des Übungsscheines!
			\end{itemize}
		\pause	
		\textbf{Rückgabe:} Im Tutorium
	\end{frame}

\Moritz{
	
	\begin{frame}{Partnerbörse!}
		
		\begin{itemize}
			\item Ich gebe euch kurz Zeit, Paare zu bilden
			\item Wenn ihr dann schon eine*n Arbeitspartner*in habt, meldet euch
			\item Für alle anderen losen wir eine*n Partner*in aus! \pause \begin{itemize}
				\item Ich gehe durch die Liste aller, die noch keine*n Arbeitspartner*in haben
				\item Wenn ihr dran seid, ziehe ich eine Karte, die euch eine Zahl zuweist
				\item Jede Zahl gibt es zwei Mal, also wer die gleiche Zahl hat, bildet ein Paar!
			\end{itemize}
			\pause
			\item Euch zwingt natürlich niemand, euch an die gelosten Paare zu halten, aber versucht es doch mal :)
			\item Ihr könnt euch z.B. in MS Teams im Chat austauschen (oder da Telefonnummern tauschen oder so)
		\end{itemize}

	\end{frame}

}

\subsection{Schein/Klausur}
 \begin{frame}{Modul GBI}
		\textbf{Übungsschein}
			\begin{itemize}
				\item Erhält, wer jeweils auf den Übungsblättern 1-6 und 7-12 mindestens 50\% der Punkte erziehlt
				\item Ist keine Voraussetzung für die Teilnahme an der Klausur
				\item Kann nur im Wintersemester gemacht werden, muss also dieses Semester versucht werden
				\item \textbf{Übungsschein wird zum Bestehen des Moduls benötigt} 
			\end{itemize}
		\pause	
		\textbf{Klausur}
		\begin{itemize}
				\item Datum: ??
				\item Nachklausur nach dem SS
				\item Klausurnote = Modulnote
				\item \textbf{Klausur wird zum Bestehen des Moduls benötigt} 
			\end{itemize}
	\end{frame}
	
	\showmessage{Orientierungsprüfung!}