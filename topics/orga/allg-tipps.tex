\subsection{Tipps}
\begin{frame}{Allgemeine Tipps}
    \begin{itemize}[<+->]
    	\item Jedes abgegebene Blatt: \emph{Kopfzeile} mit \\{\centering``GBI ÜB 1, \quad Tut. \hashtag\mytutnumber\ (Tutorname), \quad Max Mustermann, 1234567''}
    	% \item Aufgabenblatt muss \emph{nicht ausgedruckt} werden
		% \item Kein Bleistift,
        \item Kein roter oder grüner Stift
		\item Achtet auf die \emph{Aufgabenstellung}: Unterschied  ``Gib an'', ``Beweise'' oder ``Begründe''
		\item Bei Beweis \alert{NIEMALS} Beispiel, Widerlegen oft einfaches Gegenbeispiel
		\item Gebt nur \emph{eine Version} an, sonst bewerte ich die schlechtere
    	\end{itemize}
\end{frame}

\begin{frame}{Allgemeine Tipps}
    \begin{itemize}[<+->]
    	\item \emph{Strukturiert vorgehen!} Schreibt ``Beh.:'', ``\zz'', ``Ann.:'',\\
    	setzt ``q.e.d oder $\qedsymbol$'' ans Ende eines Beweises
    	\item \emph{Führt den Leser} bei Beweisen, macht klar, was ihr meint, was ihr in einem Schritt tut
    	\item Formalisiert so viel wie nötig, aber nicht mehr
    	\item Neue Definitionen mit \emph{$:=$}, z.B. $A_n := c^2$
    	\item \emph{Funktionen vollständig} angeben: $f \from \nN_0 \to \nR, x \mapsto x^2$ o.ä.
    \end{itemize}
\end{frame}