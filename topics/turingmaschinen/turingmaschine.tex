\subsection{Turingmaschinen}

\begin{frame}[fragile]{Turingmaschine}
   \begin{figure}[ht]
  		\centering
  		\begin{tikzpicture}[node distance=0mm,->,>={Latex[open]},thin]
    		\matrix[matrix of math nodes,column sep=0mm,minimum size=8mm,row sep=10mm,nodes={every rectangle node/.style={rectangle,draw,anchor=center}}]   {
      		& & & && \node[circle,draw] (z) {z}; & \\
      		\node[circle] {\cdots}; & \blank & \blank & |(x)| \#a & \#b & \#b & \#a & \#a & \blank & \node[circle] {\cdots};\\
    		};
    		\path[<->] (z) edge [out=270,in=90] (x);
  		\end{tikzpicture}
	\end{figure}
\end{frame}

\begin{frame}{Turingmaschine}
	\begin{block}{Def.: Turingmaschine}
		Eine \textbf{Turingmaschine} $T=(Z,z_0,X,f,$ $g,m)$ ist festgelegt durch
			\begin{itemize}
			\item eine \textbf{Zustandsmenge} $Z$
			\item einen \textbf{Anfangszustand} $z_0\in Z$
			\item ein \textbf{Bandalphabet} $X$
			\item eine partielle \textbf{Zustandsüberführungsfunktion} $f:Z\times X \dashrightarrow Z$
			\item eine partielle \textbf{Ausgabefunktion} $g:Z\times X \dashrightarrow X$ und
			\item eine partielle \textbf{Bewegungsfunktion} $m:Z\times X \dashrightarrow \{ -1, 0, 1\}$
			\begin{itemize}
				\item $f$, $g$ und $m$ seien für die gleichen Paare $(z,x)\in Z\times X$ definiert bzw. nicht definiert
			\end{itemize}
			\item[]
			\item Darstellung einer konkreten Turingmaschine:\\
					Tabelle oder graphisch (ähnlich wie bei Mealy-Automaten)
			\end{itemize} 
	\end{block}
\end{frame}

\begin{frame}{Turingmaschine}
	\dots Und was das ganze soll:\\

	\begin{itemize}
		\item \textbf{Bandalphabet} $X$
		\begin{itemize}
			\item meist mit Blanksymbol \blank
		\end{itemize}
		\item partielle \textbf{Zustandsüberführungsfunktion} $f:Z\times X \dashrightarrow Z$
		\begin{itemize}
			\item in neuen Zustand übergehen
		\end{itemize}
		\item \textbf{Ausgabefunktion} $g:Z\times X \dashrightarrow X$
		\begin{itemize}
			\item Band mit neuem Symbol beschriften
		\end{itemize}
		\item \textbf{Bewegungsfunktion} $m:Z\times X \dashrightarrow \{ -1, 0, 1\} \text{ bzw. } \{ L,0,R\}$
		\begin{itemize}
			\item Kopf bewegen
		\end{itemize}
		\item[]
		\item TM liest von Band, schreibt auf Band und kann ihren Kopf bewegen
	\end{itemize}
\end{frame}

\begin{frame}[fragile]{Turingmaschine}
    \begin{columns}\footnotesize
    	\begin{column}{0.4\textwidth}
    		\begin{tikzpicture}[shorten >=1pt,initial text=,node distance=1.7cm,auto,->,>=stealth,baseline=(B.base)]
          		% \node[state,initial]  (S)                       {$S$};
          		\node[state,initial]  (A)          {$A$};
          		% \node (nix) [right of=A] {};
          		\node[state]          (B) [above right of=A] {$B$};
          		\node[state]          (C) [right of=B] {$C$};
          		\node[state]          (E) [below right of=A] {$E$};
          		\node[state]          (D) [right of=E] {$D$};
          		\path[->]
          		% (S) edge              node  {$\9\io\9R$} (A)
          		(A) edge              node  {$\io{\#1}{\#XR}$} (B)
          		(B) edge [loop above] node  {$\io{\#1}{\#1R}$} ()
          		edge              node  {$\io{\9}{\9R}$} (C)
          		(C) edge [loop above] node  {$\io{\#1}{\#1R}$} ()
          		edge              node  {$\io{\9}{\#1L}$} (D)
          		(D) edge [loop below] node  {$\io{\#1}{\#1L}$} ()
          		edge              node  {$\io{\9}{\9L}$} (E)
          		(E) edge [loop below] node  {$\io{\#1}{\#1L}$} ()
          		edge              node  {$\io{\#X}{\#1R}$} (A)
          		% (B) edge              node        {$\9\io\9R$} (B)
          		% edge [loop right] node        {$\#1\io\#1R$} ()
          		% (B) edge [loop right] node {$\9\io\#1L$} ()
          		% edge  node [pos=0.3]       {$\#1\io\#1L$} (A)
          		;
        \end{tikzpicture}
		\end{column}
		\begin{column}{0.6\textwidth}
			\begin{tabular}[t]{>{$}c<{$}@{\quad}*{6}{>{$}c<{$}}}
          		\toprule
          		& A       & B      & C       & D       & E \\
          		\midrule
          		\9       
          		&         & \9,R,C & \#1,L,D & \9,L,E  &  \\
          		\#1         & \#X,R,B & \#1,R,B& \#1,R,C & \#1,L,D & \#1,L,E \\
          		\#X         &         &        &         &         & \#1,R,A \\
          		\bottomrule
        	\end{tabular}
        	\normalsize
        	\begin{exampleblock}{Aufgabe}
        		\begin{enumerate}
        			\item Was steht bei der Eingabe von \#{111} auf dem Band?
        			\item Was macht die TM allgemein, wenn man auf der ersten \#1 startet?
        		\end{enumerate}
        	\end{exampleblock}
		\end{column}
	\end{columns}
\end{frame}

\begin{frame}{Turingmaschinen}
    \begin{block}{Lösung}
    	\begin{enumerate}
    		\item $\dots \blank \#{111} \blank \#{111} \blank \dots$
    		\item diese TM kopiert ein Wort $\#1^k$ auf einem leeren Band, so dass hinterher $\cdots \blank \#1^k \blank \#1^k \blank \cdots$ da steht
    	\end{enumerate}
    \end{block}

    \textbf{Verallgemeinerung} fürs Kopieren von Wörtern über $\{\#0,\#1\}$:\\
    \begin{itemize}
    	\item auf dem Weg nach rechts merken, was mit \#X überschrieben wurde
    	\item auf dem Weg nach rechts und
    zurück sowohl \#1 als auch \#0 überlaufen
    \end{itemize}
\end{frame}

\begin{frame}{Turingmaschinen}
	\begin{block}{Def.: Konfiguration}
		Die \textbf{Konfiguration} $c = (z,b,p)$ beschreibt den „Gesamtzustand“ in dem sich die TM zu einem bestimmten Zeitpunkt befindet. Sie wird vollständig beschrieben durch:
			\begin{itemize}
			\item den aktuellen Zustand $z\in Z$ der Steuereinheit
			\item die aktuelle Beschriftung des gesamten Bandes, die man als
  			Abbildung $b:\nZ \to X$ formalisieren kann
			\item die aktuelle Position $p\in \nZ$ des Kopfes.
			\end{itemize}
	\end{block}

	\begin{block}{Def.: Endkonfiguration}
		Falls für eine Konfiguration $c$ die Nachfolgekonfiguration nicht definiert ist, heißt $c$ auch eine \textbf{Endkonfiguration} und man sagt, die Turingmaschine habe \textbf{gehalten}.
	\end{block}
\end{frame}

\begin{frame}{Turingmaschine}
    \begin{block}{Def.: $\Delta_t$ und $\Delta_*$}
    	Für $t\in \nN_0$ liefert $\Delta_t(c)$ die ausgehend von Konfiguration $c$ nach $t$ Schritten erreichte Konfiguration.

    		\[
			\Delta_*(c) =
			\begin{cases}
  			\Delta_t(c) & \text{, falls $\Delta_t(c)$ definiert und Endkonfiguration ist} \\
  			\text{undefiniert} & \text{, falls $\Delta_t(c)$ für alle $t\in \nN_0$ definiert ist} \\
			\end{cases}
			\]
    \end{block}
\end{frame}

\begin{frame}{Turingmaschinen}
    \begin{alertblock}{Achtung!}
    	\begin{itemize}
    		\item Turingmaschinen halten \textbf{nicht immer}!
    		\item es gibt auch unendliche Berechnungen, genauso wie z.B. in Java
    	\end{itemize}
    \end{alertblock}

    \begin{block}{Weitere Definitionen}
    	\begin{itemize}
    		\item \textbf{Endliche Berechnung}: endliche Folge von Konfigurationen $(c_0, c_1, c_2,$ $\dots, c_t)$ mit $c_i = \Delta_1(c_{i-1})$ für alle $0<i\leq t$ 
    		\item \textbf{Haltende Berechnung}: endliche Berechung, deren letzte Konfiguration eine Endkonfiguration ist
    		\item \textbf{Unendliche Berechnung/Nicht haltende Berechnung}: für jede Konfiguration ist immer die Nachfolgekonfiguration definiert
    	\end{itemize}
    \end{block}
\end{frame}

\begin{frame}{Turingmaschinen}
	\begin{exampleblock}{Aufgabe}
		 \begin{tabular}[t]{>{$}c<{$}@{\qquad}*{4}{>{$}c<{$}}}
      		\toprule
      		& r & c_0 & c_1 & h \\
      		\midrule
      		\#0 & \#0,R,r   & \#0,L,c_0 & \#1,L,c_0 \\
      		\#1 & \#1,R,r   & \#1,L,c_0 & \#0,L,c_1 \\
      		\9  & \9, L,c_1 & \9 ,R,h   & \#1,L,c_0 & \hphantom{\#1,L,C} \\
      		\bottomrule
    		\end{tabular}\\[2em]

    		$r$ ist der Anfangszustand.
    		\begin{enumerate}
    			\item Stelle diese Turingmaschine graphisch dar
    			\item Gib die Endkonfiguration für die Eingaben \#{100} und \#{111} an
    			\item Was macht die Turingmaschine?
    		\end{enumerate}
	\end{exampleblock}
\end{frame}

\begin{frame}{Turingmaschine}
  
  \begin{block}{Lösung}
    \centering \begin{tikzpicture}[shorten >=1pt,initial text=,node distance=2.25cm,auto,->,>=stealth,baseline=(B.base)]
              \node[state,initial]  (r)          {$r$};
              \node[state]          (c_1) [right of=r] {$c_1$};
              \node[state]          (c_0) [right of=c_1] {$c_0$};
              \node[state]          (h) [right of=c_0] {$h$};
              
              \draw[->] (r) to [loop above] node[above,align=center] {$\io{\#0}{\#0R}$\\$\io{\#1}{\#1R}$} (r);
              \draw[->] (c_1) to [loop above] node[above,align=center] {$\io{\#1}{\#0L}$} (c_1);
              \draw[->] (c_0) to [loop above] node[above,align=center] {$\io{\#0}{\#0L}$\\$\io{\#1}{\#1L}$} (c_0);

              \draw[->] (r) to node[above,align=center] {$\io{\9}{\9L}$} (c_1);
              \draw[->] (c_1) to node[above,align=center] {$\io{\#0}{\#1L}$\\$\io{\9}{\#1L}$} (c_0);
              \draw[->] (c_0) to node[above,align=center] {$\io{\9}{\9R}$} (h);

        \end{tikzpicture}
  \end{block}

\end{frame}

\begin{frame}{Turingmaschinen}
	\begin{block}{Lösung}
		\begin{enumerate}
			\item s. o.
			\item \begin{tabular}{ccccc}
        &h&&&\\
        \hline
        $\9$&$\#{1}$&$\#{0}$&$\#{1}$&$\9$\\
        \hline
      \end{tabular} \quad und \quad 
      \begin{tabular}{cccccc}
        &h&&&&\\
        \hline
        $\9$&$\#{1}$&$\#{0}$&$\#{0}$&$\#{0}$&$\9$\\
        \hline
      \end{tabular}
  %$\9 h \#{101} \9 $, $\9 h \#{1000} \9 $
			\item Die TM erhöht eine binär dargestellte Zahl um 1
		\end{enumerate}
	\end{block}
\end{frame}