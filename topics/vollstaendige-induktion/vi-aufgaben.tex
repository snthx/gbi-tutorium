\subsection{Ein leichter induktiver Beweis auf einer Sprache}
\begin{frame}{Induktiv Beweisen I.: Zum Warmwerden}
	\begin{exampleblock}{Behauptung}
		Es sei w ein Wort. Es gilt:\\[.5em]
		\centering{Für alle $k \in \nN_0$ gilt $|w^k| = k \cdot |w|$.}
	\end{exampleblock}
\pause
	\begin{block}{Beweis}
		\begin{itemize}
			\item[I.A.:] Es sei $k=0$. Es gilt $|w^k|=|w^0|=|\varepsilon|=0=0 \cdot |w| = k \cdot |w|$.
			\item[I.S.:] $A(k) \rightarrow A(k+1)$:
				\begin{itemize}
					\item Sei $k \in \nN_{0}$ beliebig.
					\item \textbf{I.V.:} Es gelte $|w^k|=k \cdot |w|$. \\[1em]
					\item $|w^{k+1}|=|w^k \cdot w|=|w^k|+|w| \eqtext{I.V.} k \cdot |w| + |w| = (k+1) \cdot |w|$
				\end{itemize}						
		\end{itemize}

		Nach den Regeln der vollständigen Induktion folgt aus obigem Induktionsschritt die Behauptung.
	\end{block}
\end{frame}
\subsection{Variante vollständiger Induktion} % z.B. gilt für alle <=n oder mit größerem Anfang und n->n+2, etc
\begin{frame}{Induktiv Beweisen II.: Schwieriger}
	\begin{exampleblock}{Aufgabe}
		Eine Funktion $T:\nN_0 \rightarrow \nN_0$ sei wie folgt definiert:
		\begin{equation*}
		T(n) = 
			\begin{cases}
			 2, &\text{ wenn } n=0 \\
			 3, &\text{ wenn } n=1 \\
			 3 \cdot T(n-1) - 2 \cdot T(n-2) &\text{ sonst}
		\end{cases} \pause
		\end{equation*}
		\begin{enumerate}
			\item Gib die Funktionswerte $T(n)$ für $n \in \set{2,3,4,5,6}$ an.
			\item Gib eine geschlossene Formel $F(n)$ (einen arithmetischen Ausdruck ohne Rekursion) für $T(n)$ an.
			\item Zeige durch vollständige Induktion, dass für alle $n \in \nN_0$ gilt $F(n) = T(n)$.
		\end{enumerate}
	\end{exampleblock}
\end{frame}
\subsection{Ein schwieriger mathematischer Beweis}
\begin{frame}{Induktiv Beweisen II.: Schwieriger}
	\begin{block}{Lösung}
		\begin{enumerate}
			\item $ T(2)=5,T(3)=9, T(4)=17, T(5)=33, T(6)=65$
			\item $F(n) = 2^n+1$
			\newcounter{kevin}
			\setcounter{kevin}{\value{enumi}}
		\end{enumerate}
	\end{block}
\end{frame}
\begin{frame}{Induktiv Beweisen II.: Schwieriger}
	\begin{block}{Lösung}
		\begin{enumerate}
			\setcounter{enumi}{\value{kevin}}
			\item \begin{itemize}
		 		\item Beh.: Für alle $n \in \nN_0$ gilt $F(n) = T(n)$.
		 		\item Bew. (induktiv): 
				\begin{itemize}
					\item[I.A.:]
								Sei $n=0$. Dann gilt: \\ 
								\qquad $F(n)=F(0)=2^0+1=1+1=2=T(0)=T(n)$.\\
							 	Sei $n=1$. Dann gilt: \\
							 	\qquad $F(n)=F(1)=2^1+1=2+1=3=T(1)=T(n)$.
				\end{itemize}
		 	\end{itemize}
		\end{enumerate}		 					  
	\end{block}
\end{frame}
\begin{frame}{Induktiv Beweisen II.: Schwieriger}
	\begin{block}{Lösung}
	\begin{enumerate}
		\setcounter{enumi}{\value{kevin}}
		\item
		\begin{itemize}
				\item[I.S.:] $A(n) \wedge A(n+1) \rightarrow A(n+2)$ %\zz $T(n+2) = F(n+2)$, also $n \leadsto n+2$.
					\begin{itemize}
						\item Sei $n \in \nN_{0}$ beliebig.
						\item \textbf{I.V.:} Es gelte
				 					\[T(n)=F(n)=2^n+1 \text{ und } T(n+1)=F(n+1)=2^{n+1}+1\]
				 		\item \zz $T(n+2) = F(n+2)$
				 		\begin{align*}
							T(n+2) &= 3 \cdot T(n+1) - 2 \cdot T(n) \\
								   &\eqtext{I.V.} 3 \cdot F(n+1) - 2 \cdot F(n) \\
								   &= 3 \cdot (2^{n+1}+1) - 2 \cdot (2^n+1) \\
								   &= 3 \cdot 2^{n+1} + 3 - 2 \cdot 2^n - 2 \\
								   &= 3 \cdot 2^{n+1} - 2 \cdot 2^n + 1 \\
								   &= 3 \cdot 2^{n+1} - 2^{n+1} +1 \\
								   &= 2 \cdot 2^{n+1} + 1 = 2^{n+2} + 1 = F(n+2)
						\end{align*}
					\end{itemize}				
			\end{itemize}
		\end{enumerate}
	\end{block}
\end{frame}
\subsection{Eine Knobelaufgabe}
\begin{frame}{Induktiv Beweisen III.: Zum Knobeln}
	\begin{exampleblock}{Aufgabe}
		Alice und Bob feiern ihren Hochzeitstag. Auf ihrer Party befinden sich $n \in \nN_+$ Paare. Dabei begrüßen\footnote{Eine Begrüßung ist das gegenseitige Händeschütteln. Mit einer Begrüßung haben sich beide Personen begrüßt.} sich alle Paare mit Ausnahme des eigenen Partners.\\
		\begin{enumerate}
			\item Gib die Anzahl der Begrüßungen für $i \in \set{1,2,3,4,5}$ Paare an.
			\item Stelle für die Anzahl der Begrüßungen einen geschlossenen Ausdruck $p_n$ auf.
			\item Gib für die Anzahl der Begrüßungen eine induktive Definiton $q_n$ an.
			\item Zeige per vollständiger Induktion, dass für alle $n \in \nN_+$ gilt: $p_n$ = $q_n$.
		\end{enumerate}
	\end{exampleblock}
\end{frame}

\begin{frame}{Induktiv Beweisen III.: Zum Knobeln}
	\begin{block}{Lösung}
		\begin{enumerate}
			\item \begin{itemize}
				\item 1 Paar $\rightarrow$ 0 Begrüßungen
				\item 2 Paare $\rightarrow$ 4 Begrüßungen
				\item 3 Paare $\rightarrow$ 12 Begrüßungen
				\item 4 Paare $\rightarrow$ 24 Begrüßungen
				\item 5 Paare $\rightarrow$ 40 Begrüßungen
			\end{itemize} \pause
			\item $p_n = 4 \cdot \frac{1}{2} \cdot n \cdot (n-1) = 2 \cdot (n^2 -n)$ \pause
			\item Jedes Paar, das neu hinzukommt, muss die $n$ sich im Raum befindenden Paare begrüßen. Das bedeutet $2n$ Begrüßungen (ein Paar besteht aus $2$ Partnern) pro neu hinzugekommenen Partner, also insgesamt $4n$ Begrüßungen. Formal heißt das: 
			\begin{align*}
				q_1 &= 0 \\
				\text{Für alle } n \in \nN_+ \text{ gilt: } q_{n+1} &= q_n + 4n
			\end{align*}
			\setcounter{kevin}{\value{enumi}}
		\end{enumerate}
	\end{block}
\end{frame}

\begin{frame}{Induktiv Beweisen III.: Zum Knobeln}
	\begin{block}{Lösung}
	\begin{enumerate}
		\setcounter{enumi}{\value{kevin}}
		\item \begin{itemize}
			\item[I.A.] $n=1: p_n = p_1 = 2 \cdot (1^2 -1 ) = 0 = q_1 = q_n \quad \checkmark$ \\[1em]
			%\item[I.V.] Für ein beliebiges, aber festes $n \in \nN_+$ gelte $p_n=q_n$.
			\item[I.S.] $A(n) \rightarrow A(n+1)$ %\zz $p_{n+1}=q_{n+1}$, also $n \leadsto n+1$
				\begin{itemize}
					\item Sei $n \in \nN_{+}$ beliebig.
					\item \textbf{I.V.:} Es gelte $p_n = q_n$.
					\item \zz $p_{n+1}=q_{n+1}$
						\begin{align*}
							p_{n+1} &= 2 \cdot (n+1) \cdot (n+1-1) \\
									&= 2n^2 + 2n \\
									&= 2n^2 -2n + 4n \\
									&= 2 \cdot n \cdot (n-1) + 4n \\
									&= p_n + 4n \\
									&\eqtext{I.V.} q_n + 4n \\
									&\eqtext{Def.} q_{n+1} & \qedwhite{}
						\end{align*}
				\end{itemize}
		\end{itemize}
	\end{enumerate}		
	\end{block}
\end{frame}