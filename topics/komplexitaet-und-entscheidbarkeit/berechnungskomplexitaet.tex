\section{Berechnungskomplexität}
\subsection{kekekek}
\begin{frame}{Komplexitätsmaße}
	\begin{block}{Def.: Zeitkomplexität $\ftime_T$ und $\fTime_t$}
		Für die Beurteilung des \textbf{Zeitbedarfs} definiert man zwei Funktionen \\
		$\ftime_T:A^+ \to \nN_+$ und $\fTime_T:\nN_+ \to \nN_+$ wie folgt:
		\begin{align*}
  			\ftime_T(w) &= \text{das $t$, für das $\Delta_t(c_0(w))$ Endkonfiguration ist} \\
  			\fTime_T(n)   &= \max \{\ftime_T(w) \mid w\in A^n\}
		\end{align*}
	\end{block}

	Also: $\fTime_T(n)$ maximale Anzahl Schritte, die bei einer Eingabe der Länge $n$ gemacht werden
\end{frame}

\begin{frame}{Komplexitätsmaße}
    \begin{block}{Def.: Raumkomplexität}
    	Für die Beurteilung des Speicherplatzbedarfs definiert man zwei
		Funktionen $\fspace_T(w):A^+ \to \nN_+$ und $\fSpace_T(n):\nN_+ \to \nN_+$ wie folgt:
			\begin{align*}
  			\fspace_T(w) &= \text{die Anzahl der Felder, die während der }\\
               			&\quad  \text{Berechnung für Eingabe $w$ benötigt werden}\\
  			\fSpace_T(n)   &= \max \{\fspace_T(w) \mid w\in A^n\} 
			\end{align*}
    \end{block}

    Also: $\fSpace_T(n)$ maximale Anzahl Felder, die bei einer Eingabe der Länge $n$ besucht werden\\[1em]
    Bemerkung:\\
    Ein Feld gilt als „benötigt“, wenn es zu Beginn ein Eingabesymbol enthält oder irgendwann vom Kopf der Turingmaschine besucht wird. 
\end{frame}

\begin{frame}{Komplexitätsmaße}
    \textbf{Zusammenhang zwischen Zeit- und Raumkomplexität:}
    \begin{itemize}
    	\item Polynomielle Zeitkomplexität:\\
    	es existiert ein Polynom $p(n)$, sodass $\fTime(n) \in O(p(n))$
    	\item Polynomielle Raumkomplexität:\\
    	es existiert ein Polynom $p(n)$, sodass $\fSpace(n) \in O(p(n))$
    	\pause
    	\item[]
    	\item Welcher Zusammenhang gilt?
    	\begin{itemize}
    		\item Polynomielle Laufzeit $\Rightarrow$ Polynomieller Platzbedarf?
    		\item Polynomieller Platzbedarf $\Rightarrow$ Polynomielle Laufzeit ?
    	\end{itemize}
    	\pause
    	\item[]
    	\item Es gilt:
    	\begin{itemize}
    		\item TM mit polynomieller Laufzeit hat auch nur polynomiellen Platzbedarf
    		\item Umkehrung gilt nicht!
    	\end{itemize}
    \end{itemize}
\end{frame}

\begin{frame}{Turingmaschine}
	\begin{exampleblock}{Aufgabe (WS 2015)}
		Konstruiere eine Turingmaschine $T$ mit Zuständen $A,B,C,D$ und Bandalphabet $X := \set{\#0, \#1, \blank}$, die für jede Eingabe $w \in X^+$ hält und am Ende das Wort $w'\in X^+$ auf dem Band steht, für das gilt:
		\begin{itemize}
			\item $|w|=|w'|$
			\item $\fNum_2(w')=\fNum_2(w)-1$, falls $\fNum_2(w)>0$
			\item $w' = \#0 \cdot ... \cdot \#0$ sonst
		\end{itemize}
		Wo bei der Endkonfiguration der Kopf steht, ist nicht wichtig.\\
		Zeige an Eingabe $w = 010$, dass die TM funktioniert, indem du alle Konfigurationen angibst, die deine TM durchläuft.\\
		Gib außerdem zwei Funktionen $f,g$ an, für die gilt: $\fTime_T(n) \notin O(f(n))$ und $\fTime_T(n)\notin \Omega(g(n))$
	\end{exampleblock}
\end{frame}

\begin{frame}{Komplexitätsklassen}
    \begin{block}{Def.: Komplexitätsklasse}
        Eine \textbf{Komplexitätsklasse} ist eine \textbf{Menge von Problemen}.\\
        Wir beschränken uns auf Entscheidungsprobleme, also auf formale Sprachen.
    \end{block}

    \begin{block}{Def.: $\Pclass$ und $\PSPACE$}
        \Pclass: Menge aller Entscheidungsprobleme, die von TMs entschieden werden können, deren Zeitkomplexität polynomiell ist\\[1ex]
        \PSPACE: Menge aller Entscheidungsprobleme, die von TMs entschieden werden können, deren Raumkomplexität polynomiell ist 
    \end{block}

    Es gilt: $\Pclass \subseteq \PSPACE$
\end{frame}