\subsection{Wiederholung Turing-Maschinen}

\begin{frame}[fragile]{Turingmaschinen (Wdh.)}
   \begin{figure}[ht]
  		\centering
  		\begin{tikzpicture}[node distance=0mm,->,>={Latex[open]},thin]
    		\matrix[matrix of math nodes,column sep=0mm,minimum size=8mm,row sep=10mm,nodes={every rectangle node/.style={rectangle,draw,anchor=center}}]   {
      		& & & && \node[circle,draw] (z) {z}; & \\
      		\node[circle] {\cdots}; & \blank & \blank & |(x)| \#a & \#b & \#b & \#a & \#a & \blank & \node[circle] {\cdots};\\
    		};
    		\path[<->] (z) edge [out=270,in=90] (x);
  		\end{tikzpicture}
	\end{figure}
\end{frame}

\begin{frame}{Turingmaschinen (Wdh.)}
	\begin{block}{Def.: Turingmaschine}
		Eine \textbf{Turingmaschine} $T=(Z,z_0,X,f,$ $g,m)$ ist festgelegt durch
			\begin{itemize}
			\item eine \textbf{Zustandsmenge} $Z$
			\item einen \textbf{Anfangszustand} $z_0\in Z$
			\item ein \textbf{Bandalphabet} $X$
			\item eine partielle \textbf{Zustandsüberführungsfunktion} $f:Z\times X \dashrightarrow Z$
			\item eine partielle \textbf{Ausgabefunktion} $g:Z\times X \dashrightarrow X$ und
			\item eine partielle \textbf{Bewegungsfunktion} $m:Z\times X \dashrightarrow \{L, 0, R\}$
			\begin{itemize}
				\item $f$, $g$ und $m$ seien für die gleichen Paare $(z,x)\in Z\times X$ definiert bzw. nicht definiert
			\end{itemize}
			\item[]
			\item TM liest von Band, schreibt auf Band und kann ihren Kopf bewegen
			\end{itemize} 
	\end{block}
\end{frame}

\begin{frame}[fragile]{Turingmaschinen (Wdh.)}
    \begin{columns}\footnotesize
    	\begin{column}{0.4\textwidth}
    		\begin{tikzpicture}[shorten >=1pt,initial text=,node distance=1.7cm,auto,->,>=stealth,baseline=(B.base)]
          		% \node[state,initial]  (S)                       {$S$};
          		\node[state,initial]  (A)          {$A$};
          		% \node (nix) [right of=A] {};
          		\node[state]          (B) [above right of=A] {$B$};
          		\node[state]          (C) [right of=B] {$C$};
          		\node[state]          (E) [below right of=A] {$E$};
          		\node[state]          (D) [right of=E] {$D$};
          		\path[->]
          		% (S) edge              node  {$\9\io\9R$} (A)
          		(A) edge              node  {$\#1\io\#XR$} (B)
          		(B) edge [loop above] node  {$\#1\io\#1R$} ()
          		edge              node  {$\9\io\9R$} (C)
          		(C) edge [loop above] node  {$\#1\io\#1R$} ()
          		edge              node  {$\9\io\#1L$} (D)
          		(D) edge [loop below] node  {$\#1\io\#1L$} ()
          		edge              node  {$\9\io\9L$} (E)
          		(E) edge [loop below] node  {$\#1\io\#1L$} ()
          		edge              node  {$\#X\io\#1R$} (A)
          		% (B) edge              node        {$\9\io\9R$} (B)
          		% edge [loop right] node        {$\#1\io\#1R$} ()
          		% (B) edge [loop right] node {$\9\io\#1L$} ()
          		% edge  node [pos=0.3]       {$\#1\io\#1L$} (A)
          		;
        \end{tikzpicture}
		\end{column}
		\begin{column}{0.6\textwidth}
			\begin{tabular}[t]{>{$}c<{$}@{\quad}*{6}{>{$}c<{$}}}
          		\toprule
          		& A       & B      & C       & D       & E \\
          		\midrule
          		\9       
          		&         & \9,R,C & \#1,L,D & \9,L,E  &  \\
          		\#1         & \#X,R,B & \#1,R,B& \#1,R,C & \#1,L,D & \#1,L,E \\
          		\#X         &         &        &         &         & \#1,R,A \\
          		\bottomrule
        	\end{tabular}
        	\normalsize
        	\begin{exampleblock}{Aufgabe}
        		\begin{enumerate}
        			\item Was steht bei der Eingabe von \#{111} auf dem Band?
        			\item Was macht die TM allgemein, wenn man auf der ersten \#1 startet?
        		\end{enumerate}
        	\end{exampleblock}
		\end{column}
	\end{columns}
\end{frame}

\Stephan{
\begin{frame}{Turingmaschinen}
  \begin{exampleblock}{Aufgabe}
     \begin{tabular}[t]{>{$}c<{$}@{\qquad}*{4}{>{$}c<{$}}}
          \toprule
          & r & c_0 & c_1 & h \\
          \midrule
          \#0 & \#0,R,r   & \#0,L,c_0 & \#1,L,c_0 \\
          \#1 & \#1,R,r   & \#1,L,c_0 & \#0,L,c_1 \\
          \9  & \9, L,c_1 & \9 ,R,h   & \#1,L,c_0 & \hphantom{\#1,L,C} \\
          \bottomrule
        \end{tabular}\\[2em]

        $r$ ist der Anfangszustand.
        \begin{enumerate}
          \item Stelle diese Turingmaschine graphisch dar
          \item Gib die Endkonfiguration für die Eingaben \#{100} und \#{111} an
          \item Was macht die Turingmaschine?
        \end{enumerate}
  \end{exampleblock}
\end{frame}

\begin{frame}{Turingmaschinen}
  \begin{block}{Lösung}
    \begin{enumerate}
      \item s. Tafel
      \item $\9 h \#{101} \9 $, $\9 h \#{1000} \9 $
      \item Die TM erhöht eine binär dargestellte Zahl um 1
    \end{enumerate}
  \end{block}
\end{frame}
}