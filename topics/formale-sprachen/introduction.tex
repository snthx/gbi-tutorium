\subsection{Alphabete, Wörter, Sprachen}
\begin{frame}{Alphabete, Wörter, Sprachen}
	\begin{block}{Def.: Alphabet}
		Ein \textbf{Alphabet} ist eine endliche, nichtleere Menge von Zeichen.
	\end{block}
	\pause
	\begin{block}{Def.: Wort}
		Ein \textbf{Wort} über einem Alphabet A ist eine Folge von Zeichen aus A.
	\end{block}
	\pause
	\begin{block}{Def.: leeres Wort}
		Das \textbf{leere Wort} \(\varepsilon\) ist das Wort, das aus null Zeichen besteht, d.h. es hat die Länge 0.
	\end{block}
\end{frame}

\begin{frame}{Alphabete, Wörter, Sprachen}
	\begin{block}{Def.: Wort (formell)}
		Ein \textbf{Wort} $w$ über einem Alphabet $A$ ist eine surjektive Abbildung $w \from \nZ_n \to B \subseteq A$ mit $\nZ_n := \setc{i \in \nN_0}{0 \le i < n}$.
	\end{block}

	\begin{block}{Def.: leeres Wort (formell)}
		Das \textbf{leere Wort} $\varepsilon$ ist die Abbildung $\varepsilon \from \nZ_0 = \set{} \to \set{}$.	
	\end{block}

	\begin{alertblock}{Beachte}
		$\set{\varepsilon} \neq \emptyset$	
	\end{alertblock}

\end{frame}

\begin{frame}{Alphabete, Wörter, Sprachen}
	\begin{block}{Def.: \(A^{n}\)}
		\textbf{\(A^{n}\)} ist die Menge aller Wörter der Länge n über dem Alphabet A.
	\end{block}
	\pause
	\begin{block}{Def.: \(A^{*}\)}
		\textbf{\(A^{*}\)} ist die Menge aller Wörter beliebiger Länge über dem Alphabet A.		
		\[
			A^{*}= \bigcup_{i=0}^{\infty} A^{i}
		\]
	\end{block}
	\pause
	\begin{exampleblock}{Beispiele}
		$anna \in \set{a,n}^{4}$, \quad $ana \notin \set{a,n}^{4}$
		\(A^{0}=\set{\varepsilon}, \quad \set{a,b}^{*} = \set{\varepsilon, a, b, aa, ab, ba, bb, aaa, ...}\)
	\end{exampleblock}
\end{frame}

\begin{frame}{Alphabete, Wörter, Sprachen}
	\begin{block}{Def.: Konkatenation}
		Die \textbf{Konkatenation} ist die Operation der Verknüpfung von Wörtern. Jedes Wort kann als Konkatenation seiner Zeichen dargestellt werden, z.B. \(GBI = G \cdot B \cdot I\).
	\end{block}
	\pause
	\begin{block}{Induktive Def.: Potenz von Wörtern}
		Die n-te \textbf{Potenz} eines Wortes w ist induktiv definiert durch:
		\begin{align*}
				w^{0} &= \varepsilon \hphantom{000000000000}\\
				\text{Für jedes } n \in \nN_{0}: \; w^{n+1} & = w^{n} \cdot w
		\end{align*}
	\end{block}
	\pause
	\begin{exampleblock}{Aufgabe}
	\begin{itemize}
		\item Ist die Konkatenation assoziativ?\pause
		\item Ist die Konkatenation kommutativ?\pause
		\item \(a\varepsilon\varepsilon\varepsilon\varepsilon\varepsilon\varepsilon b\varepsilon\varepsilon\varepsilon\varepsilon = \, ?\)
	\end{itemize}		
	\end{exampleblock}
\end{frame}

\begin{frame}{Alphabete, Wörter, Sprachen}
	\begin{exampleblock}{Aufgabe}
		Es sei $A$ ein Alphabet.
		\begin{enumerate}
			\item Geben Sie eine Abbildung $f \from A^{\ast} \to A^{\ast}$, die injektiv, aber nicht surjektiv ist.
			\item Geben Sie eine Abbildung $g \from A^{\ast} \to A^{\ast}$, die surjektiv, aber nicht injektiv ist.
			\item Geben Sie eine Abbildung $h \from A^{\ast} \to A^{\ast}$, die bijektiv ist , aber nicht die Identität ist.
		\end{enumerate}
	\end{exampleblock}
\end{frame}
	
\begin{frame}
	\begin{block}{Lösung}
		Mögliche Abbildungen sind:
		\begin{enumerate}
			\item \[f \from A^{\ast} \to A^{\ast}, w \mapsto w^2\]
			\item \begin{align*}
				g \from A^{\ast} &\to A^{\ast} \\
				\varepsilon &\mapsto \varepsilon \\
				x \cdot w &\mapsto w \text{ mit } x \in A, w\in A^{\ast}
			\end{align*}
			\item \begin{align*}
				h \from A^{\ast} &\to A^{\ast} \\
				\varepsilon &\mapsto \varepsilon \\
				w \cdot x &\mapsto x \cdot h(w) \text{ mit } x \in A, w\in A^{\ast}
			\end{align*}
		\end{enumerate}
	\end{block}
\end{frame}

\begin{frame}{Alphabete, Wörter, Sprachen}
	\begin{block}{Def.: Formale Sprache}
		Eine \textbf{formale Sprache} L über einem Alphabet A ist eine Teilmenge der Wörter über A, also \(L \subseteq A^{*}\).
	\end{block}

	\begin{exampleblock}{Aufgabe}
		Gib die Sprache L über dem Alphabet \(\set{a,b}\) an, die alle Wörter enthält, in denen die Zeichenfolge \q{ab} nicht vorkommt.\\
		\pause
		\textbf{Lösung:} \(L = \setc{w_{1}w_{2}}{w_{1} \in \set{b}^{*} \text{ und } w_{2} \in \set{a}^{*}}\)
	\end{exampleblock}
\end{frame}
