\subsection{Sprachen} % Produkt v. Sprachen, Potenzen, L^*, L^+ und dabei induktives Definieren klarmachen
\begin{frame}{Formale Sprachen}
	\begin{block}{Def.: Formale Sprache}
		Eine \textbf{formale Sprache} L über einem Alphabet A ist eine Teilmenge der Wörter über A, also \(L \subseteq A^{*}\).
	\end{block}

	\begin{block}{Def.: Produkt formaler Sprachen}
		Es sei $A$ ein Alphabet, sowie $L,S \subseteq A^*$ formale Sprachen.\\
		Das \textbf{Produkt} der Sprachen ist definiert durch
		\[
			L \cdot S := \setc{u \cdot v}{u \in L \text{ und } v \in S}
		\]
	\end{block}

	\begin{exampleblock}{Beispiel}
		Es sei $L:=\set{\texttt{aa}, \texttt{bb}}$. Dann ist $L\cdot L = \set{\texttt{aaaa}, \texttt{aabb}, \texttt{bbbb}, \texttt{bbaa}}$
	\end{exampleblock}
\end{frame}

\begin{frame}{Formale Sprachen}
	\begin{block}{Def.: Potenzen von Sprachen}
		Die \textbf{Potenzen} einer formalen Sprache $L$ sind induktiv definiert:
		\begin{align*}
			L^0 &= \set{\varepsilon} \\
			\text{Für alle $n \in \nN_0$ gilt: } L^{n+1} &= L^n \cdot L
		\end{align*}
	\end{block}

	\begin{block}{Def.: Konkatenationsabschluss}
		Sei $L$ eine formale Sprache. Dann ist \\
		$L^* = \bigcup_{i=0}^{\infty} L^{i}$ der \textbf{Konkatenationsabschluss} $L^*$ von $L$ und \\
		$L^+ = \bigcup_{i=1}^{\infty} L^{i}$ der \textbf{$\varepsilon$-freie Konkatenationsabschluss} $L^+$ von $L$.
	\end{block}

	\begin{alertblock}{Achtung}
		Auch beim $\varepsilon$-freien Konkatenationsabschluss kann $\varepsilon$ enthalten sein, nämlich gdw. $\varepsilon \in L$
	\end{alertblock}
\end{frame}

% \begin{frame}{Formale Sprachen}
%     Tipp für's ÜB:
%     \[
%     	L^+ = L \cdot L^*
%     \]
% \end{frame}

\begin{frame}{Formale Sprachen}
	\begin{exampleblock}{Aufgabe}
		Es sei $A = {a,b}$. Drücke folgende Sprachen $L_1, L_2, L_3 \subseteq A^*$ formell aus:
		\begin{enumerate}
			\item $L_1$ enthält nur Wörter mit mindestens drei Vorkommen von $a$
			\item $L_2$ enthält nur Wörter, in denen die Zeichenfolge $bb$ nicht vorkommt
			\item Für alle Wörter $w$ in $L_3$ gilt: Wenn $w$ mindestens drei Vorkommen von $a$ hat, so hat $w$ auch mindestens zwei Vorkommen von $b$ 
		\end{enumerate}
	\end{exampleblock}
	\pause
	\begin{block}{Lösung}
		\begin{enumerate}
			\item $L_1 = \set{a,b}^\ast \cdot \set{a} \cdot \set{a,b}^\ast \cdot \set{a} \cdot \set{a,b}^\ast \cdot \set{a} \cdot \set{a,b}^\ast$
			\item $L_2 = \set{a}^\ast \cdot \set{b, \varepsilon} \cdot (\set{a}^+ \cdot \set{b})^\ast \cdot \{ a \}^*$ 
			\item \begin{align*}L_3 = & \set{\varepsilon,b} \cup \set{a, ab, ba} \cup \set{aa, aab, aba, baa} \\ & \cup (\set{a,b}^\ast \cdot \set{b} \cdot \set{a,b}^\ast \cdot \set{b} \cdot \set{a,b}^\ast) \end{align*}
		\end{enumerate}
	\end{block}
\end{frame}