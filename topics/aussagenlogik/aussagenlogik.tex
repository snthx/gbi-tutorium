\subsection{Grundlagen: Aussage, Syntax}

\begin{frame}{Grundlagen}
	\begin{block}{Def.: Aussage}
		\textbf{Aussagen} sind Sätze, die "`objektiv"' wahr oder falsch sind. Man spricht auch von der Zweiwertigkeit der Aussagenlogik.
	\end{block}
	\pause
	\begin{block}{Def.: Aussagenvariable}
		Eine \textbf{Aussagenvariable} steht für eine (elementare) Aussage. Sie kann entweder \emph{wahr} oder \emph{falsch} sein.
	\end{block}
\end{frame}

\begin{frame}{Grundlagen}
	\begin{block}{Aussagenlogische Konnektive}
		Seien \(G\) und \(H\) Aussagevariablen. Dann kann man wie folgt größere Formeln konstruieren:
		\pause
		\begin{itemize}[<+->]
			\item \(	\bleftBr 	\bnot  G \brightBr 	\) heißt \enquote{\textcolor{blue}{nicht} \(G\)}
			\item \(	\bleftBr 	G \bund  H		\brightBr 	\)	heißt \enquote{\(G\) \textcolor{blue}{und} \(H\)}
			\item \(	\bleftBr 	G \boder H		\brightBr 	\)	heißt \enquote{\(G\) \textcolor{blue}{oder} \(H\)}
			\item \(	\bleftBr 	G  \bimp H		\brightBr 	\)	heißt \enquote{\(G\) \textcolor{blue}{impliziert} \(H\)}
			\item \(	\bleftBr 	G  \bgdw H		\brightBr := \bleftBr	G  \bimp H		\brightBr  \bund  \bleftBr 	H  \bimp G		\brightBr	\)	heißt \\ \enquote{\(G\) \textcolor{blue}{impliziert} \(H\) und \(H\) \textcolor{blue}{impliziert} \(G\)}
		\end{itemize}
	\end{block}
	\pause
	\begin{alertblock}{Beachte}
		Hinter dieser Syntax steckt natürlich viel Formalismus. Die Vorlesung kennt \(\bgdw\) nur als Abkürzung, nicht als aussagenlogisches Symbol.
	\end{alertblock}
\end{frame}

\begin{frame}{Grundlagen}
	\begin{block}{Bindungsstärken}
		\begin{itemize}
			\item $\bnot$ bindet am stärksten
  			\item $\bund$ bindet am zweitstärksten
			\item $\boder$ bindet am drittstärksten
			\item $\bimp$ bindet am viertstärksten
			\item $\bgdw$ bindet am schwächsten
		\end{itemize}
	\end{block}
	\pause
	\begin{exampleblock}{Aufgabe}
		Finde die Klammern für \\
		\(	P	\bimp	Q \bund \bnot R \bgdw Q \boder R	\)\\[1ex]
		\pause
		\(	\bleftBr \bleftBr  P	\bimp	\bleftBr  Q \bund \bleftBr  \bnot R \brightBr \brightBr \brightBr \bgdw \bleftBr Q \boder R	\brightBr \brightBr \)
	\end{exampleblock}
\end{frame}

\subsection{Semantik}

\begin{frame}{Semantik aussagenlogischer Formeln}
	\begin{block}{Def.: Boolsche Funktionen}
		Für ``Wahrheitswerte'' \( \BB = \set{\mathbf{w}, \mathbf{f}} \)
		 ist eine \textbf{boolsche Funktion} eine Funktion
		 $f: \BB^n \to \BB$
	\end{block}
	\pause
	\begin{exampleblock}{Beispiele}
		\begin{center}

		  	\begin{tabular}{cc|cccc}
		    \toprule
		    $x_1$ & $x_2$ & $\bfnot{x_1}$ & $\bfand{x_1}{x_2}$ & $\bfor{x_1}{x_2}$ & $\bfimp{x_1}{x_2}$ \\
		    \midrule
		    $f$ & $f$ & $w$ & $f$ & $f$ & $w$ \\
		    $f$ & $w$ & $w$ & $f$ & $w$ & $w$ \\
		    $w$ & $f$ & $f$ & $f$ & $w$ & $f$ \\
		    $w$ & $w$ & $f$ & $w$ & $w$ & $w$ \\
		    \bottomrule
  			\end{tabular}
		\end{center}
	\end{exampleblock}
\end{frame}

\begin{frame}{Semantik aussagenlogischer Formeln}
	\begin{block}{Def.: Interpretation}
		Für eine Menge \(V\) von Aussagevariablen ist eine \textbf{Interpretation} ist eine Abbildung \(I: V \to \nB\), wobei \( \BB = \set{w, f} \). 
	\end{block}

	\begin{exampleblock}{Beim Aufstellen einer Wahrheitstabelle}
		Anzahl Interpretationen für eine Variablenmenge mit \(k \in \nN_+ \) Aussagevariablen: \(2^k\).
	\end{exampleblock}
\end{frame}

\begin{frame}{Semantik aussagenlogischer Formeln: $\vali{F}$}
	\begin{block}{Auswertung/Validierung aussagenlogischer Formeln}
		Sei \(I: V \to \nB\) eine Interpretation.\\
		Für jede aussagelogische Formel $F$ definiere $\vali{F}$ wie folgt:\\[2ex]

		Für jedes $X \in V$, $G$, $H$ weitere aussagelogische Formeln sei:
		\begin{itemize}
			\item $\vali{X}         := I(X) $
  			\item $\vali{\bnot G}   := \bfnot{\vali{G}} $
  			\item $\vali{G \bund H} := \bfand{\vali{G}}{\vali{H}}$
  			\item $\vali{G \boder H} := \bfor{\vali{G}}{\vali{H}}$
  			\item $\vali{G \bimp H} := \bfimp{\vali{G}}{\vali{H}}$
		\end{itemize}
	\end{block}
\end{frame}

% TODO: Aufgabe zum formellen Vorgehen, mit konkreter Interpretation?

\begin{frame}{Semantik aussagenlogischer Formeln: Wahrheitstabellen}
	\begin{exampleblock}{Aufgabe}
		Stelle für eine aussagenlogische Formel eine Wahrheitstabelle auf.
	\end{exampleblock}

	\begin{exampleblock}{Vorgehen}
	\pause
		\begin{enumerate}[<+->]
			\item Wahrheitswerte für Variablen
			\item Wahrheitswerte für Teilformeln
			\item Wahrheitswerte für ganze Formeln
		\end{enumerate}
	\end{exampleblock}
\end{frame}

\begin{frame}{Semantik aussagenlogischer Formeln: Wahrheitstabellen}
	\begin{exampleblock}{Aufgabe}
	Stelle für folgende aussagenlogische Formel eine Wahrheitstabelle auf: \( \bleftBr{} \bleftBr{} \bnot{} A \bimp{} B \brightBr{} \bund \bleftBr{} \bnot{} \bleftBr{} A \bgdw{} B \brightBr{} \boder{} A \brightBr{} \brightBr{}\).
	\end{exampleblock}
	%\pause
	\begin{block}{Lösung}
\begin{center}

		  	\begin{tabular}{cc|cccccccc}
		    \toprule
		    $A$ & $B$ & $\bleftBr{} \bleftBr{} \bnot{} A$ & $\bimp{}$ & $B\brightBr{}$ & $\bund $ & $\bleftBr{} \bnot{}$ & $\bleftBr{} A \bgdw{} B \brightBr{}$ & $\boder{}$ & $A \brightBr{} \brightBr{}$\\
		    \midrule
		    \pause
		    $ f $ & $ f $ & $ w $ & $ f $ & $ f $ & $ \textcolor{kit-green100}{\mathbf{f}} $ & $ f $ & $ w $ & $ f $ & $ f $\\
		    $ f $ & $ w $ & $ w $ & $ w $ & $ w $ & $ \textcolor{kit-green100}{\mathbf{w}} $ & $ w $ & $ f $ & $ w $ & $ f $\\
		    $ w $ & $ f $ & $ f $ & $ w $ & $ f $ & $ \textcolor{kit-green100}{\mathbf{w}} $ & $ w $ & $ f $ & $ w $ & $ w $ \\
		    $ w $ & $ w $ & $ f $ & $ w $ & $ w $ & $ \textcolor{kit-green100}{\mathbf{w}} $ & $ f $ & $ w $ & $ w $ & $ w $ \\
		    \bottomrule
  			\end{tabular}
		\end{center}

		\pause
		Beobachtung: \( \bleftBr{} \bleftBr{} \bnot{} A \bimp{} B \brightBr{} \bund \bleftBr{} \bnot{} \bleftBr{} A \bgdw{} B \brightBr{} \boder{} A \brightBr{} \brightBr{}\) und \(\bleftBr{} A \boder B \brightBr{}\) sind äquivalent.
	\end{block}
\end{frame}

\begin{frame}{Semantik aussagenlogischer Formeln: Äquivalenz}
	\begin{block}{Def.: äquivalente Formeln}
		Zwei Formeln $A$ und $B$ heißen \textbf{äquivalent}, wenn für jede Interpretation $I$ gilt
		\[\val_I(A)=\val_I(B)\]
		Man schreibt auch $A\equiv B$.
	\end{block}

	\begin{exampleblock}{Beispiele}
		\begin{itemize}
			\item \( \bleftBr{} \bleftBr{} \bnot{} A \bimp{} B \brightBr{} \bund \bleftBr{} \bnot{} \bleftBr{} A \bgdw{} B \brightBr{} \boder{} A \brightBr{} \brightBr{} \equiv \bleftBr{} A \boder B \brightBr{}\)
			\item \( \bleftBr{} A \bimp{} B \brightBr{} \equiv \bleftBr{} \bnot{} A \boder{} B \brightBr{} \)
		\end{itemize}
	\end{exampleblock}
\end{frame}

\begin{frame}{Semantik aussagenlogischer Formeln: Modell}
	\begin{block}{Def.: Modell}
		Eine Interpretation $I$ heißt \textbf{Modell} einer Formel $G$, wenn $\vali{G} =\mathbf{w}$.\\[2ex]

		Eine Interpretation $I$ heißt \textbf{Modell} einer Formelmenge $\Gamma$, wenn $I$ Modell jeder Formel $G\in \Gamma$ ist.
	\end{block}

	\begin{exampleblock}{}
	Wir sagen:\\
	\textcolor{black!50!red}{$\Gamma \models G$}, wenn jedes Modell von $\Gamma$ auch ein Modell von $G$ ist.\\
	\textcolor{black!50!red}{$\models G$} (statt \textcolor{black!50!red}{$\set{}\models G$}), wenn $G$ für \emph{alle} Interpretationen wahr ist.
	\end{exampleblock}
\end{frame}

\begin{frame}{Semantik aussagenlogischer Formeln: Tautologie}
	\begin{block}{Def.: Tautologie}
		Eine Formel $G$ heißt \textbf{Tautologie}, wenn jede Interpreation $I$ Modell ist. Das heißt, für jede Belegung (Interpretation $I$) der Aussagevariablen ist die gesamte aussagenlogische Formel wahr:
			\[	\vali{G}=\mathbf{w} \quad \text{ für alle } I	\]
	\end{block}

	\begin{exampleblock}{Beispiel}
		\begin{itemize}
			\item \( \bnot{} P \boder P \)
		\end{itemize}
	\end{exampleblock}
\end{frame}


\begin{frame}{}
	\begin{exampleblock}{Aufgabe (WS 16/17)}
		% Übungsblatt 2016-2, Aufgabe 2
		Es sei $\VarAL$ eine Menge von Aussagevariablen und $\ForAL$ die Menge aller aussagenlogischen Formeln über $\VarAL$. Beweise, dass für alle \(G, H \in \ForAL\) die aussagenlogische Formel
				\[\mathcal{F} :=	\bleftBr{} \bnot{} H\bimp{} \bnot{} G \brightBr{} \bimp{} \bleftBr{} G \bimp{} H \brightBr{}\]
		eine Tautologie ist. Verwende nicht das aussagenlogische Kalkül, sondern die formellen Definitionen der Auswertung von aussagenlogischen Formeln und boolschen Funktionen.
	\end{exampleblock}
	\begin{block}{Ansatz:}
		\zz Für alle Interpretationen $I$ ist die aussagenlogische Formel wahr.\\[1ex]
		Durch \emph{Abbilden}, \emph{Umformen} und \emph{Substituieren} (\emph{Ersetzen}) auf bekannte Tautologien schließen.\\[1ex]
		\textbf{Lösung:} s. Tafel.
	\end{block}
\end{frame}

\subsection{Beweisbarkeit}