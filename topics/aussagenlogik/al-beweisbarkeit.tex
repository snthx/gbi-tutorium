\subsection{Beweisbarkeit}
\begin{frame}{Aussagenlogik: Beweisbarkeit}
	\begin{block}{Hilbertkalkül für die Aussagenlogik}
		Der \textbf{Hilbertkalkül für die Aussagenlogik}, (vereinfacht) auch \textbf{Aussagenkalkül}, besteht aus
		\begin{itemize}
			\item dem Alphabet $\AAL$,
			\item der Menge der syntakisch korrekten Formeln $\LAL \subseteq \AAL^*$,
			\item einer Menge von \textbf{Axiomen} $\AxAL \subseteq \LAL$,
			\item der (einzigen) \textbf{Schlussregel} \textbf{Modus Ponens} $\MP \subseteq \LAL^3$
		\end{itemize}
	\end{block}
\end{frame}

\begin{frame}{Aussagenlogik: Beweisbarkeit}
	\begin{block}{Axiome}
		\begin{align*}
  			\AxAL &= \bigl\{\alka G\alimpl \alka H\alimpl  G\alkz\alkz
          			\bigm| G,H\in\LAL \bigr\} \\
        		&\mathrel{\hphantom{=}} \cup \bigl\{\alka G\alimpl \alka H\alimpl  K\alkz\alkz
          			\alimpl \alka\alka G\alimpl H\alkz\alimpl \alka G\alimpl  K\alkz\alkz \bigm| G,H,K\in\LAL \bigr\}\\
        		&\mathrel{\hphantom{=}} \cup \bigl\{
          			\alka\alnot H\alimpl \alnot G\alkz\alimpl \alka\alka\alnot H\alimpl G\alkz\alimpl  H\alkz
          			\bigm| G,H \in\LAL 
          			\bigr\}
		\end{align*}

		Kurz: $\AxAL{}_1$, $\AxAL{}_2$ und $\AxAL{}_3$ für die drei Zeilen.
	\end{block}
\end{frame}

\begin{frame}{Aussagenlogik: Beweisbarkeit}
	\begin{block}{Modus Ponens $\MP \subseteq \LAL^3$ (Schlussregel)}
		\begin{itemize}
			\item $\MP = \{ (G\alimpl H, G, H) \mid  G, H \in\LAL \}$
			\item $\MP:$ \quad \begin{tabular}{c}
                $G \alimpl H$ \qquad $G$ \\
                \midrule
                $H$
              \end{tabular}
             \item Aus einer Formel der Form $G \alimp H$ und einer Formel der Form $G$ darf man auf eine Formel der Form $H$ schließen.
		\end{itemize}
	\end{block}

	\begin{exampleblock}{Beispiel}
		\begin{description}
			\item[\emph{Prämissen}:] $G \alimpl H$: \enquote{Wenn es regnet, wird die Straße nass},\\
								 $G$: \enquote{Es regnet}
			% Sind diese beiden Prämissen gültig, so ist auch der Schluss "Conclusio" gültig.
			\item[\emph{Conclusio}:] $H$: \enquote{Die Straße wird nass}.
		\end{description}
	\end{exampleblock}

	% \begin{exampleblock}{Beispiel}
	% 	\begin{itemize}
	% 		\item[Prämissen:] $G \alimpl H$: \enquote{Wenn du Mensch bist, stirbst du},\\
	% 							 $G$: \enquote{Du bist ein Mensch}
	% 		\item[Conclusio:] $H$: \enquote{Du stirbst}.
	% 	\end{itemize}
	% \end{exampleblock}
\end{frame}	

\begin{frame}{Aussagenlogik: Beweisbarkeit}
	\begin{block}{Def.: Ableitung}
		Sei $\Gamma \subseteq \LAL$ eine Menge von \textbf{Hypothesen} oder \textbf{Prämissen} und $G$ eine Formel.\\
		Eine \textbf{Ableitung} von $G$ aus $\Gamma$ ist eine endliche Folge $(G_1, \dots, G_n)$ mit
		\begin{itemize}
			\item $G_n = G$ und 
			\item für jedes $G_i (1 \leq i \leq n)$ gilt einer der folgenden Fälle:
			\begin{itemize}
				\item $G_i\in\AxAL$ oder% $G_i$ ist ein Axiom
				\item $G_i\in\Gamma$ oder% $G_i$ ist eine Prämisse
				\item es gibt $i_1,i_2 < i$ mit $(G_{i_1},G_{i_2},G_i)\in\MP$.
			\end{itemize}
		\end{itemize}

		Geschrieben: $\Gamma\vdash G$
	\end{block}

	\begin{block}{Def.: Beweis und Theorem}
	Ist $\Gamma=\{\}$, so heißt eine entsprechende Ableitung auch ein \textbf{Beweis} von $G$ und $G$ ein \textbf{Theorem} des Kalküls, in Zeichen: $\vdash G$
	\end{block}
\end{frame}

\begin{frame}{Aussagenlogik: Beweisbarkeit}
	\begin{exampleblock}{Beispiel zum Modus Ponens}
		Ableitung/Beweis des Theorems $\alka\alP\alimpl\alP\alkz$:\\[1em]
		\begin{tabular}{rll}
			1. & $\alka \alka \alP \alimpl \alka \alka \alP \alimpl  \alP \alkz\alimpl  \alP \alkz\alkz\alimpl 
       			\alka \alka \alP \alimpl \alka \alP \alimpl  \alP \alkz\alkz\alimpl \alka \alP \alimpl  \alP \alkz\alkz\alkz$ & $\AxAL{}_2$ \\
			2. & $\alka \alP \alimpl \alka \alka \alP \alimpl  \alP \alkz\alimpl  \alP \alkz\alkz$ & $\AxAL{}_1$ \\
			3. & $\alka \alka \alP \alimpl \alka \alP \alimpl  \alP \alkz\alkz\alimpl \alka \alP \alimpl  \alP \alkz\alkz$ & $\MP(1,2)$ \\
			4. & $\alka \alP \alimpl \alka \alP \alimpl  \alP \alkz\alkz$ & $\AxAL{}_1$ \\
			5. & $\alka \alP \alimpl  \alP \alkz$ & $\MP(3,4)$
			\end{tabular}

			\qedwhite{}
	\end{exampleblock}
\end{frame}

\begin{frame}{Aussagenlogik: Beweisbarkeit}    
	\begin{alertblock}{Achtung}
		\begin{itemize}
			\item \emph{Tautologie}, \emph{Modell} und \emph{Theorem} sind unterschiedliche Wörter
			\item $\models$ und $\vdash$ sind unterschiedliche Symbole
		\end{itemize}
	\end{alertblock}
\end{frame}