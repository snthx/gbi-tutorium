\subsection{Tupel und Paare}

	\begin{frame}{Relationen}
		\begin{block}{Def.: Tupel bzw. Paare}
			Ein n-Tupel ist eine Liste von n Elementen. Die Reihenfolge ist dabei wichtig!\\
			Ein Paar ist ein Tupel der Länge 2, also ein 2-Tupel.			
		\end{block}
		\begin{exampleblock}{Beispiele:}
			\begin{itemize}
				\item Paare: $(1,2),(5,3),(Apfel, Birne)$
				\item 3-Tupel: $(1,2,3)$, 4-Tupel: $(4,3,2,1)$, ...
				\item $\set{1,2}=\set{2,1}$, aber $(1,2)\neq(2,1)$
			\end{itemize}			
		\end{exampleblock}	
	\end{frame}

	\begin{frame}{Relationen}
		\begin{block}{Def.: Kartesisches Produkt}
			Das \textbf{kartesische Produkt $A \times B$} der Mengen A und B ist die Menge aller Paare (a,b) mit $a \in A$ und $b \in B$, also:
			$$A \times B = \setC{(a,b)}{$a \in A$ und $b \in B$}$$
		\end{block}
	
		\begin{exampleblock}{Beispiele:}
			\begin{itemize}
				\item $\set{*, +} \times \set{1, 2, 3} = \set{(*,1), (*,2), (*,3), (+,1), (+,2), (+,3)}$
				\item $\set{0,1}^2 = \set{(0,0),(0,1),(1,0),(1,1)}$ 
				\item Für $A \neq B$ gilt im Allgemeinen: $A \times B \neq B \times A$
				\item Ist $A = \emptyset$ oder $B = \emptyset$, so gilt $A \times B = B \times A = \emptyset$
			\end{itemize}			
		\end{exampleblock}
	\end{frame}
%subsection Tupel und Paare (end)	
\subsection{Relationen}
	\begin{frame}{Relationen}
		\begin{block}{Def.: Relationen}
			Eine \textbf{binäre Relation R zwischen A und B} ist eine Teilmenge des kartesischen Produktes $A \times B$, also $R \subseteq A \times B$.\\
			Ist $A=B$, so heißt R eine Relation auf A.\\
			Erfüllt $(a,b) \in A \times B$ die Relation R, so schreibt man $(a,b) \in R$ oder $aRb$.
		\end{block}
	
		\begin{exampleblock}{Beispiele}
			\begin{itemize}
				\item Es sei $M:=\set{1,2,3,4}$. Es bezeichne $R_<$ die \q{kleiner-als}-Relation auf M. Dann ist $R_< = \set{(1,2),(1,3),(1,4),(2,3),(2,4),(3,4)}$.
				\item Statt $(1,4) \in R_<$ schreibt man auch $1 < 4$.
			\end{itemize}
		\end{exampleblock}
	\end{frame}

	\begin{frame}{Relationen}
		\begin{exampleblock}{Aufgabe}
			Seien $A:=\set{2,3,4,75}$ und $B:=\set{1,3,4,10,16}$.\\ 
			Es sei $R \subseteq A \times B $ definiert durch $R:=\setC{(a,b) \in A \times B}{Die Quersumme von $a \cdot b$ ist gleich 3.}$\\
			Gib die Relation explizit an.
		\end{exampleblock}
	\pause
		\begin{block}{Lösung}		
			$R=\set{(3,1),(3,4),(3,10),(4,3),(75,4),(75,16)}$
		\end{block}
	\end{frame}
