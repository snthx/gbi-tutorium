\subsection{Funktionen von Funktionen}

\begin{frame}{Funktionsception: Funktionen von Funktionen}
	\begin{block}{Def.: $B^A$}
		Seien A und B Mengen. Mit $B^A$ bezeichnen wir die Menge aller Abbildungen von $A$ nach $B$, also $$B^A = \set{f \mid f : A \to B }$$
	\end{block}

	\begin{block}{Beobachtung}
		Für endliche Mengen $A$ und $B$ gilt:
		$$|B^A| = |B|^{|A|}$$
	\end{block}
\end{frame}

\begin{frame}{Funktionsception${}^2$: Funktionen als Argument von Funktionen}

	\begin{exampleblock}{Beispiel}
		\small Es sei $M$ eine Menge. Jede Teilmenge $L \subseteq M$ kann man mit einer Abbildung $f_L \colon M \to \set{0,1}$, also einem $f_L \in \set{0,1}^M$ in Beziehung setzen.

		Für die Teilmenge $L \subseteq M$ ist 
		\begin{equation*} 
			f_L \colon M \to \set{0,1}, x \mapsto
			\begin{cases}
				1 & \text{, falls } x \in L \\
				0 & \text{, falls } x \notin L
			\end{cases}
		\end{equation*}
	\end{exampleblock}
\end{frame}

\begin{frame}{Funktionsception${}^2$: Funktionen als Argument von Funktionen}
	\begin{exampleblock}{Beispiel}
		Dann kann man die Vereinigung $\sqcup$ als Abbildung darstellen:
		$$\sqcup \colon \{0,1\}^M \times \{0,1\}^M\to \{0,1\}^M$$
		\pause
		\begin{columns}
			\begin{column}{0.4\textwidth}
				Beispielbild für $A=\{a,c,d\}$ \\ und $B=\{b,c\}$
			\end{column}
			\begin{column}{0.4\textwidth}
				\begin{tabular}{*{4}{>{$}c<{$}}}
					& A & B & A\cup B \\
					x & f_A(x) & f_B(x) & (\sqcup(f_A,f_B))(x) \\
					\hline
					a & 1 & 0 & 1 \\
					b & 0 & 1 & 1 \\
					c & 1 & 1 & 1 \\
					d & 1 & 0 & 1 \\
					e & 0 & 0 & 0 \\
				\end{tabular}
			\end{column}
			\end{columns}	
		\medskip
		\centering Wie könnte nun eine Abbildungsvorschrift aussehen?
	\end{exampleblock}
\end{frame}
\begin{frame}{Funktionsception${}^2$: Funktionen als Argument von Funktionen}
	\begin{exampleblock}{Wie könnte eine Abbildungsvorschrift aussehen?}
	\begin{align*}
		\sqcup\colon \{0,1\}^M \times \{0,1\}^M &\to \{0,1\}^M \\
		(f_A,f_B) &\mapsto (x \mapsto \max(f_A(x),f_B(x))) \\[2ex]
		\text{also }  \sqcup(f_A,f_B) (x) &= \max(f_A(x),f_B(x)) \\
	\end{align*}
	\end{exampleblock}
\end{frame}

\subsection{Speicher}

\begin{frame}{Bits und Bytes}
	\begin{block}{Def.: Bit und Byte}
		Ein \textbf{Bit} ist ein Zeichen des Alphabets $\set{0,1}$. \\
		Ein Wort aus 8 Bits wird \textbf{Byte} genannt. 
	\end{block}
\end{frame}
\begin{frame}{Speicher}
	\begin{block}{Speicher}
		Ein \textbf{Speicher} $m$ bildet Adressen ($Adr$) auf Werte ($Val$) ab.\\
		Also $m \in Val^{Adr}$ bzw. $m \colon Adr \to Val$.
	\end{block}

	\begin{exampleblock}{Dazu sei einiges gesagt:}
		\begin{itemize}
			\item Statt ``Speicher'' trifft es \textbf{aktueller Zustand des Speichers} besser.
			\item Vorstellung als Tabelle. Linke Spalte Adressen, rechte Spalte Werte.
			\item Uns interessiert (noch) nicht die konkrete Realisierung, sondern nur die abstrakte Funktionsweise.
			\item Im Umgang mit Speicher benutzen wir hier nur Zahlen im Binärsystem, d.h.
				$$Adr = 2^k, Val = 2^l \quad k,l \in \nN_+$$
				Bei der \emph{MIMA}: $k=20, l=24$.
		\end{itemize}
	\end{exampleblock}
\end{frame}

\begin{frame}{Speicher - Methoden}
	\begin{block}{Def.: $memread$}
		Gibt im aktuellen Speicherzustand $m$ den Wert an der Adresse $a$ zurück.
		\begin{align*}
			memread : Val^{Adr} \times Adr &\to Val \\
									(m,a) &\mapsto m(a)
	\end{align*}
	\end{block}
\end{frame}

\begin{frame}{Speicher - Methoden}
	\begin{block}{Def.: $memwrite$}
		Speichert einen Wert $v$ in die Adresse $a$.
		\begin{align*}
			memwrite : Val^{Adr} \times Adr \times Val &\to Val^{Adr} \\
			(m,a,v) &\mapsto m' \\
		\end{align*}

		$m'$ ist nun ein neuer Speicherzustand, für den gilt:
		\begin{equation*}
			m'(a') =
				\begin{cases}
					v & \text{, falls } a'=a \\
					m(a') & \text{, sonst}
				\end{cases}
		\end{equation*}
		%\phantom{Moritz ist ein kek}\\
		Das Resultat ist eine neue Abbildung, bei der sich nur der Funktionswert an $a$ geändert hat.			
	\end{block}
\end{frame}

% \begin{frame}{Speicher}
% 	\begin{block}{Beispiel}
		
% 	$memread(m, 01) = \only<2->{00000111}$\\
% 	\visible<2-|handout:2>{$memwrite(m, 01, 11111100)$\\}
% 	\visible<4-|handout:2>{$memread(memwrite(m, 01, 11111100), 01) = 11111100$\\}
% 	\bigskip
	
% 	\begin{table}
% 		\begin{tabular}{|c|c|}
% 			\hline
% 			\multicolumn{2}{|c|}{Speicher $m$} \\
% 			\hline
% 			00 & 00101000 \\
% 			\only<-2|handout:1>{01 & 00000111}\only<3-|handout:2>{\color{blue} 01 & \color{blue} 11111100}  \\
% 			10 & 10010110 \\
% 			11 & 00100101 \\
% 			\hline
% 		\end{tabular} \par
% 		\bigskip
% 		Zustand bei \only<-2|handout:1>{$t=0$}\only<3-|handout:2>{\color{blue}$t=1$}
% 	\end{table}

% 	\end{block}
% \end{frame}

\begin{frame}{Speicher - Methoden}
	\begin{exampleblock}{Aufgabe}		
		Sei $m \in Val^{Adr}$. Seien $ a,a' \in Adr \text{ und } v,v' \in Val$. Gib an: \\
		\begin{itemize}
			\item $memread(memwrite(m,a,v),a)$
			\item $memread(memwrite(memwrite(m,a,v),a',v'),a)$
			\item $memread(memwrite(memwrite(m,a,v),a,v'),a)$
		\end{itemize}
	\end{exampleblock}
	\pause
	\begin{block}{Lösung}
	\small
		\begin{itemize}
			\item $memread(memwrite(m,a,v),a) = v$
			\item $memread(memwrite(memwrite(m,a,v),a',v'),a) = \begin{cases}
																	v' & \text{, falls } a = a' \\
																	v  & \text{, falls } a \neq a' \\
																\end{cases}$
			\item $memread(memwrite(memwrite(m,a,v),a,v'),a) = v' $
		\end{itemize}
	\end{block}
\end{frame}

\begin{frame}{Schnittstellen}
	Man kann die Formalisierung des Speichers verwenden, um unterschiedliche \emph{Schnittstellen} zu implementieren.\\[1em]

	\begin{block}{Def.: Schnittstelle}
		Schnittstellen sind abstrakte Sammlungen von Funktionalitäten, die ein Ganzes ergeben. Das heißt alle \textbf{Implementierungen} einer Schnittstelle, setzen die in der Schnittstelle vorgegebenen Funktionalitäten praktisch um. 
	\end{block}

	\begin{exampleblock}{Beispiel: Fahrzeug}
		Ein Fahrzeug sollte folgende Funktionalitäten haben:
		\begin{enumerate}
			\item Beschleunigen
			\item Bremsen
		\end{enumerate}

		Ein Fahrrad implementiert die Schnittstelle Fahrzeug, indem es z.B. mit den Pedalen und dem Getriebe das Beschleunigen umsetzt.
	\end{exampleblock}
\end{frame}

\begin{frame}{Schnittstelle Stack}
	\begin{block}{Def.: Stack}
		Ein \textbf{Stack} ist eine Schnittstelle für eine Datenstruktur, die folgende Funktionalitäten vorgibt:
		\begin{enumerate}
			\item peek(): Element  -  Gib das oberste Element des Stapels aus.
			\item push(Element e)  -  Schiebe ein Element oben auf den Stapel.
			\item pop()  -  Nimm das oberste Element des Stapels herunter.
		\end{enumerate}

		\medskip

		Man kann also neue Elemente mit \textit{push} nur oben auf den Stapel schreiben, mit \textit{pop} nur Elemente oben von Stapel löschen und nur auf das zuletzt geschriebene Element mit \textit{peek} zugreifen. Insgesamt ist \textbf{Lesen}, \textbf{Schreiben} und \textbf{Entfernen} also nur an einer bestimmten Stelle möglich.
	\end{block}
\end{frame}

\begin{frame}{Stack-Implementierung}
	Wir wollen in unserem formalisierten Speicher einen Stack implementieren, der maximal $2^8 - 1$ Elemente enthält. Wir wählen eine Adresse $a$, in der immer steht, wie viele Elemente aktuell im Stack enthalten sind. Die Elemente des Stacks stehen dann in den auf $a$ folgenden Addressen.
\end{frame}
	
\begin{frame}{Stack-Implementierung}
	\emph{Überlegungen:}
	\begin{enumerate}
		\item Um den Stack zu initialisieren, müssen wir nur den Wert $0$ für keine Elemente in $a$ speichern. Also:
			\[init: Mem \times Adr \to Mem, (m,a) \mapsto memwrite(m,a,bin_8(0))\]
		\item Das oberste Element des Stack befindet sich immer bei der Addresse: $Add_{adr}(a, memread(m,a))$
	\end{enumerate}
\end{frame}

\begin{frame}{Stack-Implementierung}
	\begin{exampleblock}{Aufgabe}
		Implementiere die Methoden \textbf{peek}, \textbf{push} und \textbf{pop} mit den folgenden Funktionsköpfen:
		\begin{enumerate}
			\item $peek: Mem \times Adr \to Val$
			\item $push: Mem \times Adr \times Val \to Mem$
			\item $pop: Mem \times Adr \to Mem$
		\end{enumerate}
		Der Wert in $Adr$ ist dabei immer die Addresse $a$.
	\end{exampleblock}
\end{frame}