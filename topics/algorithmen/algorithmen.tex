\subsection{Algorithmusbegriff}

\begin{frame}{Algorithmen}
\begin{block}{Eigenschaften eines Algorithmus}
\begin{itemize}
  \item endliche Beschreibung
  \item elementare Anweisungen
  \item Determinismus
  \item zu endlichen Eingabe wird endliche Ausgabe berechnet
  \item endliche viele Schritte
  \item funktioniert für beliebig große Eingaben
  \item Nachvollziehbarkeit/Verständlichkeit für jeden (mit der Materie vertrauten)
  \end{itemize}
\end{block}
    


\end{frame}


\subsection{Pseudocode}
\begin{frame}{Pseudocode}
\begin{block}{Zuweisung}
	\begin{columns}
		\begin{column}{0.45\textwidth}
   			\begin{tabular}{ll}
   				Java: & $a = b + c$
   			\end{tabular}
		\end{column}
		\begin{column}{0.45\textwidth}
    		\begin{tabular}{ll}
   				GBI: & $a \leftarrow b + c$
   			\end{tabular}
		\end{column}
	\end{columns}
\end{block}

\begin{block}{Kontrollstrukturen}
	\small
	\begin{columns}
		\begin{column}{0.45\textwidth}
   			\begin{tabular}[t]{ll}
   				Java: & $\text{\textbf{for }} (i=0;i<10;i++) \{ $\\
   				& $\dots$ \\
   				& $\}$\\
   				& \\
   				& \textbf{if} (\dots) \\
   				& \\
   				& $\{\dots\}$ \\
   				& \textbf{else} \\
   				& $\{\dots\}$ \\
   				& \\
   			\end{tabular}
		\end{column}
		\begin{column}{0.45\textwidth}
    		\begin{tabular}[t]{ll}
   				GBI: & $\text{\textbf{for }} (i \leftarrow 0 \text{ to } 9) \text{\textbf{ do}}$ \\
   				& \dots \\
   				& \textbf{od}\\
   				& \\
   				& \textbf{if} $\dots$ \\
   				& \textbf{then} \\
   				& $\dots$ \\
   				& \textbf{else}\\
   				& $\dots$\\
   				& \textbf{fi}



   			\end{tabular}
		\end{column}
	\end{columns}

	Schaut in den Folien, macht es eindeutig, schreibt Kommentare!
\end{block}
    


\end{frame}