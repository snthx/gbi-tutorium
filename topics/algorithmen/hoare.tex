\subsection{Einführung}
\begin{frame}{Hoare-Kalkül}
	\begin{block}{Def.: Hoare-Tripel}
		Ein \textbf{Hoare-Tripel} hat die Form \textcolor{kit-red100}{$\set{P}$} {$\set{S}$} \textcolor{kit-red100}{$\set{Q}$}.
		\begin{itemize}
			\item \textcolor{kit-red100}{$P$}: Vorbedingung
			\item $S$: Programmstück
			\item \textcolor{kit-red100}{$Q$}: Nachbedingung
		\end{itemize}
		$P$ und $Q$ sind sog. Zusicherungen.
	\end{block}

	\begin{block}{Def.: Zusicherung}
		\textbf{Zusicherungen} sind prädikatenlogische Formeln, deren Wahrheit von Interpretationen und Variablenbelegungen abhängt.\\
		Achtung: Variablen in Zusicherungen kommen nur frei vor!
	\end{block}
\end{frame}

\begin{frame}{Hoare-Kalkül}
	\begin{block}{Gültigkeit v. Hoare-Tripeln}
		$\set{P} \, S \, \set{Q}$ ist \textbf{gültig}, wenn für jede relevante (passende) Interpretation $(D,I)$ und jede Variablenbelegung $\beta$ gilt:
		\begin{itemize}
			\item wenn $\valDIb(P) = \mathbf{w}$ und
			\item wenn die Ausführung von $S$ für $(D,I)$ und $\beta$ terminiert und hinterher die Variablenbelegung $\beta'$ vorliegt
			\item dann $\val_{D,I,\beta'}(Q) = \mathbf{w}$
		\end{itemize}
	\end{block}
\end{frame}




\subsection{Ableitungsregeln}

\begin{frame}{Regeln im Hoare-Kalkül}
	Es gibt einige ``Regeln'' im Hoare-Kalkül, die für gültige Hoare-Tripel gelten.

	\begin{block}{HT-E: Verstärkung der Vorbedingung und Abschwächung der Nachbedingung}
		Es gelte $P' \vdash P$ und $Q\vdash Q'$. Dann gilt:
		 \begin{center}\begin{tabular}{c}
						$\set{P} \, S \, \set{Q}$\\
						\midrule
						$\set{P'} \, S \, \set{Q'}$
		\end{tabular}\end{center}
		(Wenn $\set{P} \, S \, \set{Q}$ gültig, dann $\set{P'} \, S \, \set{Q'}$ gültig.)
	\end{block}

\end{frame}

\begin{frame}{Regeln im Hoare-Kalkül}

	\begin{block}{HT-S: Hintereinanderausführung}
		Es gilt: \begin{center}\begin{tabular}{c}
			$\set{P} \, S_1 \, \set{Q}$ \qquad $\set{Q} \, S_2 \, \set{R}$\\
			\midrule
			$\set{P} \, S_1 ; S_2 \, \set{R}$
		\end{tabular}\end{center}
		Das heißt die Gültigkeit von Hoare-Tripeln ist transitiv. 
		
	\end{block}
\end{frame}

\begin{frame}{Regeln im Hoare-Kalkül}
	\begin{block}{HT-A: Zuweisungen}
		\begin{itemize}
			\item Zuweisung $x \leftarrow E$ ($x$ wird durch Ausdruck $E \in L_Ter$ ``ersetzt'')
			\item $Q$ sei Nachbedingung zu $x \leftarrow E$
			\item $\sigma_{\set{x/E}}$ sei auf $Q$ kollisionsfrei.
			\item Dann ist \begin{center}
				$\set{\sigma_{\set{x/E}}(Q)} \, x\leftarrow E \, \set{Q}$
			\end{center} ein gültiges Hoare-Tripel.
		\end{itemize}
	\end{block}
	\pause
	\begin{exampleblock}{Beispiel}
		\only<2>{\HTB{?}\\}
		\only<3>{\HTB{$y^{2}>0$}\\}
		$x \leftarrow y^{2}$\\
		\HTB{$x>0$}
	\end{exampleblock}
	
\end{frame}

\begin{frame}{Schwächste Vorbedingung und stärkste Nachbedingung}
	\begin{block}{Def.: Schwächste Vorbedingung}
		Zu einer Anweisungsfolge $S$ und einer Nachbedingung $Q$ heißt $P$ eine \textbf{schwächste Vorbedingung}, wenn $\set{P} \, S \, \set{Q}$ ein gültiges Hoare-Tripel ist und für jedes gültige Hoare-Tripel $\set{P'} \, S \, \set{Q}$ gilt: $P'$ impliziert $P$.
	\end{block}

	\begin{block}{Def.: Stärkste Nachbedingung}
		Zu einer Anweisungsfolge $S$ und einer Vorbedingung $P$ heißt $Q$ eine \textbf{stärkste Nachbedingung}, wenn $\set{P} \, S \, \set{Q}$ ein gültiges Hoare-Tripel ist und für jedes gültige Hoare-Tripel $\set{P} \, S \, \set{Q'}$ gilt: $Q$ impliziert $Q'$.
	\end{block}
\end{frame}

\begin{frame}{Aufgabe zum Finden einer schwächsten Vorbedingung}
	\begin{exampleblock}{Aufgabe}
		Es seien $x$ und $y$ zwei verschiedene Variablen und es seien $a$ und $b$ zwei ganze
		Zahlen. Bestimme anhand des Hoare-Kalküls eine schwächste Vorbedingung von
		\begin{align*}
			& x \leftarrow x+y\\
			& y \leftarrow x-y\\
			& x \leftarrow x-y\\
			& \HTB{$x=b \, \wedge \, y=a$}
		\end{align*}
		\\[18pt]
		indem du wiederholt die Zuweisungsregel (HT-A) anwendest.
			
	\end{exampleblock}
\end{frame}

\begin{frame}{Lösung zur Aufgabe}
	\begin{block}{Lösung}
		Schwächste Vorbedingung: \HTB{$x=a \, \wedge \, y=b$}\\
		\begin{align*}
			&\HTB{$x=a \, \wedge \, y=b$}\\
			&\HTB{$y=b \, \wedge \, (x+y)-y=x=a$}\\
			& x \leftarrow x+y\\
			&\HTB{$x-(x-y)=y=b \, \wedge \, x-y=a$}\\
			& y \leftarrow x-y\\
			&\HTB{$x-y=b \, \wedge \, y=a$}\\
			& x \leftarrow x-y\\
			&\HTB{$x=b \, \wedge \, y=a$}
		\end{align*}
	\end{block}
\end{frame}

\begin{frame}{Mehr Regeln im Hoare-Kalkül}
	\begin{block}{HT-I: Bedingte Anweisungen}
	\begin{columns}
		\begin{column}{0.3\textwidth}
		\small
		\begin{tabular}{rl}
			&\HTB{P}\\
			&\textbf{if} $B$\\
			&\textbf{then}\\
			&\qquad \HTB{$P \wedge B$}\\
			&\qquad $S_1$\\
			&\qquad \HTB{Q}\\
			&\textbf{else}\\
			&\qquad \HTB{$P \wedge \neg B$}\\
			&\qquad $S_2$\\
			&\qquad \HTB{Q}\\
			&\textbf{fi}\\
			&\HTB{Q}
		\end{tabular}			
		\end{column}

		\begin{column}{0.7\textwidth}
			\begin{center}\begin{tabular}{c}
			$\set{P \wedge B} \, S_1 \, \set{Q}$ \qquad $\set{P \wedge \neg B} \, S_2 \, \set{Q}$\\
			\midrule
			$\set{P} \, $\textbf{if} $B$ \textbf{then} $S_1$ \textbf{else} $S_2$ \textbf{fi}$\, \set{Q}$
		\end{tabular}\end{center}
		\end{column}		
	\end{columns}	

	\end{block}
\end{frame}

\begin{frame}{Aufgabe zu HT-I}
	\begin{exampleblock}{Aufgabe}
		Ergänze die fehlenden Zuweisungen:
		\small 
		\begin{tabular}{rl}
			&\HTB{$x=a \wedge y=b$}\\
			&\textbf{if} $x>y$\\
			&\textbf{then}\\
			&\qquad \HTB{...}\\
			&\qquad $z \leftarrow y$\\
			&\qquad \HTB{...}\\
			&\textbf{else}\\
			&\qquad \HTB{...}\\
			&\qquad $z \leftarrow x$\\
			&\qquad \HTB{...}\\
			&\textbf{fi}\\
			&\HTB{$x=a \wedge y=b \wedge z=min(a,b)$}
		\end{tabular}
	\end{exampleblock}
\end{frame}

\begin{frame}{Lösung zur Aufgabe}
	\begin{block}{Lösung}
		\small 
		\begin{tabular}{rl}
			&\HTB{$x=a \wedge y=b$}\\
			&\textbf{if} $x>y$\\
			&\textbf{then}\\
			&\qquad \HTB{$x=a \wedge y=b \wedge x>y$}\\
			&\qquad $z \leftarrow y$\\
			&\qquad \HTB{$x=a \wedge y=b \wedge z=min(a,b)$}\\
			&\textbf{else}\\
			&\qquad \HTB{$x=a \wedge y=b \wedge x\le y$}\\
			&\qquad $z \leftarrow x$\\
			&\qquad \HTB{$x=a \wedge y=b \wedge z=min(a,b)$}\\
			&\textbf{fi}\\
			&\HTB{$x=a \wedge y=b \wedge z=min(a,b)$}
		\end{tabular}
	\end{block}
\end{frame}


\begin{frame}{Mehr Regeln im Hoare-Kalkül}
	\begin{block}{HT-W: While-Schleifen}
	\begin{columns}
		\begin{column}{0.3\textwidth}
		\small
		\begin{tabular}{rl}
			&\HTB{I}\\
			&\textbf{while} $B$\\
			&\textbf{do}\\
			&\qquad \HTB{$I \wedge B$}\\
			&\qquad $S$\\
			&\qquad \HTB{I}\\
			&\textbf{od}\\
			&\HTB{$I \wedge \neg B$}
		\end{tabular}			
		\end{column}

		\begin{column}{0.7\textwidth}
			\begin{center}\begin{tabular}{c}
			$\set{I \wedge B} \; S \; \set{I}$\\
			\midrule
			$\set{I}$ \textbf{while} $B$ \textbf{do} $S$ \textbf{od} $\set{I \wedge \neg B}$ 
		\end{tabular}\end{center}
		\centering Zusicherung $I$ heißt \textbf{Schleifeninvariante}.
		\end{column}		
	\end{columns}	

	\end{block}
\end{frame}

\begin{frame}{Beispiel zur Schleifeninvariante}
	\begin{exampleblock}{Beispiel}
		Eingabe: $a,b \in \nN_0$
		\begin{align*}
			&X_0 \leftarrow a\\
			&Y_0 \leftarrow b\\
			&\text{\textbf{for }} i \leftarrow 0 \text{\textbf{ to }} b-1 \text{\textbf{ do}}\\
			&\qquad X_{i+1} \leftarrow X_{i} + 1\\
			&\qquad Y_{i+1} \leftarrow Y_{i} - 1\\
			&\text{\textbf{od}}
		\end{align*}

		Sei $a=6$ und $b=4$. Vervollständige zunächst folgende Tabelle:
		\begin{center}\begin{tabular}{l c r}
			 &$X_i$&$Y_i$\\
			\hline
			$i=0$ & & \\
			$i=1$ & & \\
			...& & 
		\end{tabular}\end{center}
	\end{exampleblock}
\end{frame}

\begin{frame}{Beispiel zur Schleifeninvariante}

	\begin{exampleblock}{Beispiel}
		\begin{columns}
			\begin{column}{0.4\textwidth}
				\begin{align*}
				&X_0 \leftarrow a\\
				&Y_0 \leftarrow b\\
				&\text{\textbf{for }} i \leftarrow 0 \text{\textbf{ to }} b-1 \text{\textbf{ do}}\\
				&\qquad X_{i+1} \leftarrow X_{i} + 1\\
				&\qquad Y_{i+1} \leftarrow Y_{i} - 1\\
				&\text{\textbf{od}}
				\end{align*}
			\end{column}

			\begin{column}{0.6\textwidth}
		\begin{center}
		Sei $a=6$ und $b=4$.\\[18pt]
		\begin{tabular}{C C C}
			 &X_i&Y_i\\
			\hline
			i=0&6&4 \\
			i=1&7&3 \\
			i=2&8&2\\
			i=3&9&1\\
			i=4&10&0\\ 
		\end{tabular}\end{center}
			\end{column}
			
		\end{columns}
		Was berechnet der Algorithmus?\\
		Was ist eine passende Schleifeninvariante?
	\end{exampleblock}
\end{frame}

\begin{frame}{Beispiel zur Schleifeninvariante}
	\begin{block}{Lösung}
		\begin{columns}
			\begin{column}{0.4\textwidth}
				\begin{align*}
				&X_0 \leftarrow a\\
				&Y_0 \leftarrow b\\
				&\text{\textbf{for }} i \leftarrow 0 \text{\textbf{ to }} b-1 \text{\textbf{ do}}\\
				&\qquad X_{i+1} \leftarrow X_{i} + 1\\
				&\qquad Y_{i+1} \leftarrow Y_{i} - 1\\
				&\text{\textbf{od}}
				\end{align*}
			\end{column}

			\begin{column}{0.6\textwidth}
		\begin{center}
		Sei $a=6$ und $b=4$.\\[18pt]
		\begin{tabular}{C C C}
			 &X_i&Y_i\\
			\hline
			i=0&6&4 \\
			i=1&7&3 \\
			i=2&8&2\\
			i=3&9&1\\
			i=4&10&0\\ 
		\end{tabular}\end{center}
			\end{column}
			
		\end{columns}
		Der Algorithmus berechnet $a+b$.\\
		Schleifeninvariante: $\forall i \in \nN_0 \text{ mit } i<b: X_i + Y_i = a+b$
	\end{block}
\end{frame}

\begin{frame}{Beispiel zur Schleifeninvariante}

	\begin{exampleblock}{Algorithmus von oben als while-Schleife}
	\small
		\begin{tabular}{rl}
			&\HTB{$x=a \wedge y=b$}\\
			&\HTB{...}\\
			&\textbf{while} $y > 0$\\
			&\textbf{do}\\
			&\qquad \HTB{...}\\
			&\qquad \HTB{...}\\
			&\qquad $y \leftarrow y-1$\\
			&\qquad \HTB{...}\\
			&\qquad $x \leftarrow x+1$\\
			&\qquad \HTB{...}\\
			&\textbf{od}\\
			&\HTB{...}\\
			&\HTB{$x=a+b$}\\
		\end{tabular}
		Erinnerung: Schleifeninvariante $x+y=a+b$.
	\end{exampleblock}
\end{frame}

\begin{frame}{Beispiel zur Schleifeninvariante}
	\begin{block}{Algorithmus von oben als while-Schleife}
		\small
		\begin{tabular}{rl}
			&\HTB{$x=a \wedge y=b$}\\
			&\HTB{$x+y=a+b$}\\
			&\textbf{while} $y > 0$\\
			&\textbf{do}\\
			&\qquad \HTB{$x+y=a+b \wedge y \neq 0$}\\
			&\qquad \HTB{$x+1+y-1=a+b$}\\
			&\qquad $y \leftarrow y-1$\\
			&\qquad \HTB{$x+1+y=a+b$}\\
			&\qquad $x \leftarrow x+1$\\
			&\qquad \HTB{$x+y=a+b$}\\
			&\textbf{od}\\
			&\HTB{$x+y=a+b  \wedge \neg(y \neq 0)$}\\
			&\HTB{$x=a+b$}\\
		\end{tabular}
	\end{block}
\end{frame}

\begin{frame}{Aufgabe zur Schleifeninvariante}

	\begin{exampleblock}{Aufgabe}
		\begin{columns}
			\begin{column}{0.35\textwidth}
				\small
				\begin{align*}
				&X_0 \leftarrow a, Y_0 \leftarrow b, P_0 \leftarrow 1\\
				&Z_0 \leftarrow X_0 \; mod \; 2\\
				&n \leftarrow 1 + \lceil log_2a \rceil\\
				&\text{\textbf{for }} i \leftarrow 0 \text{\textbf{ to }} n-1 \text{\textbf{ do}}\\
				&\qquad P_{i+1} \leftarrow P_{i} \cdot Y_{i}^{Z_i}\\
				&\qquad X_{i+1} \leftarrow X_{i} \; div \; 2\\
				&\qquad Y_{i+1} \leftarrow Y_{i}^{2}\\
				&\qquad Z_{i+1} \leftarrow X_{i+1} \; mod \; 2\\
				&\text{\textbf{od}}
				\end{align*}
			\end{column}

			\begin{column}{0.6\textwidth}
			\begin{center}
			Eingabe: $a,b \in \nN_+$\\[18pt]
			\begin{tabular}{C C C C C C}
			 	&X_i&Y_i&P_i&Z_i&n\\
				\hline
				i=0& & & & &\\
				i=1& & & & &\\
				...& & & & & \\
			\end{tabular}\end{center}
			\end{column}
			
		\end{columns}
		a) Erstelle Tabellen für $(i)a=1,b=2$; $(ii) a=2,b=2$; $(iii) a=2,b=3$.\\
		b) Welchen Wert hat $P_n$ am Ende?\\
		c) Gib eine Schleifeninvariante an.

	\end{exampleblock}
\end{frame}

\begin{frame}{Lösung der Aufgabe zur Schleifeninvariante}
	\begin{block}{Lösung a)}
	\small
	a)
		\begin{center}\begin{tabular}{C C C C C C}
			 	a=1,b=2&X_i&Y_i&P_i&Z_i&n\\
				\hline
				i=0&1&2&1&1&1\\
				i=1&0&4&2&0&1\\
			\end{tabular}\\[10pt]
		\begin{tabular}{C C C C C C}
			 	a=2,b=2&X_i&Y_i&P_i&Z_i&n\\
				\hline
				i=0&2&2&1&0&2\\
				i=1&1&4&1&1&2\\
				i=2&0&16&4&0&2\\
		\end{tabular}\\[10pt]
		\begin{tabular}{C C C C C C}
			 	a=2,b=3&X_i&Y_i&P_i&Z_i&n\\
				\hline
				i=0&2&3&1&0&2\\
				i=1&1&9&1&1&2\\
				i=2&0&81&9&0&2\\
			\end{tabular}\end{center}
	\end{block}
\end{frame}

\begin{frame}{Lösung der Aufgabe zur Schleifeninvariante}
	\begin{block}{Lösung b) + c)}
		b) $P_n = b^{a}$\\
		c) Schleifeninvariante: $P_i \cdot Y_{i}^{X_i} = b^{a}$
	\end{block}
\end{frame}