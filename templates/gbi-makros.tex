\newcommand{\newpar}[1]{\paragraph{#1}\mbox{}\newline}

\newcommand{\nM}{\mathbb{M}}
\newcommand{\nR}{\mathbb{R}}
\newcommand{\nN}{\mathbb{N}}
\newcommand{\nZ}{\mathbb{Z}}
\newcommand{\nQ}{\mathbb{Q}}
\newcommand{\nB}{\mathbb{B}}
\newcommand{\nC}{\mathbb{C}}
\newcommand{\nK}{\mathbb{K}}
\newcommand{\nF}{\mathbb{F}}
\newcommand{\nG}{\mathbb{G}}
\newcommand{\nullel}{\mathcal{O}}
\newcommand{\einsel}{\mathds{1}}
\newcommand{\nP}{\mathbb{P}}
\newcommand{\Pot}{\mathcal{P}}
\renewcommand{\O}{\text{O}}

\newcommand{\set}[1]{\{ #1 \}}
\newcommand{\setc}[2]{\set{#1 \mid #2}}
\newcommand{\setC}[2]{\set{#1 \mid \text{ #2 }}}

\newcommand{\setsize}[1]{\; \mid #1 \mid \; }

\newcommand{\q}[1]{\textquotedblleft #1\textquotedblright}

% Zu zeigen, thx to http://www.matheboard.de/archive/155832/thread.html
\newcommand{\zz}{\ensuremath{\mathrm{z\kern-.29em\raise-0.44ex\hbox{z}}}:}

% Text above symbol
% https://tex.stackexchange.com/a/74132/146825
%
% \newcommand{\eqtext}[1]{\stackrel{\mathclap{\normalfont\mbox{#1}}}{=}}
% \newcommand{\gdwtext}[1]{\stackrel{\mathclap{\normalfont\mbox{#1}}}{\Leftrightarrow}}
% \newcommand{\imptext}[1]{\stackrel{\mathclap{\normalfont\mbox{#1}}}{\Rightarrow}}
% \newcommand{\symbtext}[2]{\stackrel{\mathclap{\normalfont\mbox{#2}}}{#1}}
\newcommand{\eqtext}[1]{\mathrel{\overset{\makebox%[0pt]
{\mbox{\normalfont\tiny #1}}}{=}}}
\newcommand{\gdwtext}[1]{\mathrel{\overset{\makebox%[0pt]
{\mbox{\normalfont\tiny #1}}}{\ensuremath{\Leftrightarrow}}}}
\newcommand{\imptext}[1]{\mathrel{\overset{\makebox%[0pt]
{\mbox{\normalfont\tiny #1}}}{\ensuremath{\Rightarrow}}}}
\newcommand{\symbtext}[2]{\mathrel{\overset{\makebox%[0pt]
{\mbox{\normalfont\tiny #2}}}{#1}}}

% qed symbol
\newcommand{\qedblack}{\hfill \ensuremath{\blacksquare}}
\newcommand{\qedwhite}{\hfill \ensuremath{\Box}}

% Aussagenlogik
% Worsch
\colorlet{alcolor}{blue}
\RequirePackage{tikz}
\usetikzlibrary{arrows.meta}
\newcommand{\alimpl}{\mathrel{\tikz[x={(0.1ex,0ex)},y={(0ex,0.1ex)},>={Classical TikZ Rightarrow[]}]{\draw[alcolor,->,line width=0.7pt,line cap=round] (0,0) -- (15,0);\path (0,-6);}}}
\newcommand{\alimp}{\alimpl}
\newcommand{\aleqv}{\mathrel{\tikz[x={(0.1ex,0ex)},y={(0ex,0.1ex)},>={Classical TikZ Rightarrow[]}]{\draw[alcolor,<->,line width=0.7pt,line cap=round] (0,0) -- (18,0);\path (0,-6);}}}
\newcommand{\aland}{\mathbin{\raisebox{-0.6pt}{\rotatebox{90}{\texttt{\color{alcolor}\char62}}}}}
\newcommand{\alor}{\mathbin{\raisebox{-0.8pt}{\rotatebox{90}{\texttt{\color{alcolor}\char60}}}}}
%\newcommand{\ali}[1]{_{\mathtt{\color{alcolor}#1}}}
\newcommand{\alv}[1]{\mathtt{\color{alcolor}#1}}
\newcommand{\alnot}{\mathop{\tikz[x={(0.1ex,0ex)},y={(0ex,0.1ex)}]{\draw[alcolor,line width=0.7pt,line cap=round,line join=round] (0,0) -- (10,0) -- (10,-4);\path (0,-8) ;}}}
\newcommand{\alP}{\alv{P}} %ali{#1}}
%\newcommand{\alka}{\negthinspace\hbox{\texttt{\color{alcolor}(}}}
\newcommand{\alka}{\negthinspace\text{\texttt{\color{alcolor}(}}}
%\newcommand{\alkz}{\texttt{\color{alcolor})}}\negthinspace}
\newcommand{\alkz}{\text{\texttt{\color{alcolor})}}\negthinspace}

% Thassilo
\newcommand{\BB}{\mathbb{B}}
\newcommand{\boder}{\alor}%{\ensuremath{\text{\;}\textcolor{blue}{\vee}}\text{\;}}
\newcommand{\bund}{\aland}%{\ensuremath{\text{\;}\textcolor{blue}{\wedge}}\text{\;}}
\newcommand{\bimp}{\alimp}%{\ensuremath{\text{\;}\textcolor{blue}{\to}}\text{\;}}
\newcommand{\bnot}{\alnot}%{\ensuremath{\text{\;}\textcolor{blue}{\neg}}\text{}}
\newcommand{\bgdw}{\aleqv}%{\ensuremath{\text{\;}\textcolor{blue}{\leftrightarrow}}\text{\;}}
\newcommand{\bone}{\ensuremath{\textcolor{blue}{1}}\text{}}
\newcommand{\bzero}{\ensuremath{\textcolor{blue}{0}}\text{}}
\newcommand{\bleftBr}{\alka}%{\ensuremath{\textcolor{blue}{(}}\text{}}
\newcommand{\brightBr}{\alkz}%{\ensuremath{\textcolor{blue}{)}}\text{}}

\newcommand{\val}{\hbox{\textit{val}}}

\newcommand{\VarAL}{\hbox{\textit{Var}}_{AL}}
\newcommand{\ForAL}{\hbox{\textit{For}}_{AL}}

% Validierungsfunktion val_i
\newcommand{\vali}[1]{\ensuremath{\val_I(#1)}}

% Boolsche Funktion b_
\newcommand{\bfnot}[1]{\ensuremath{b_{\bnot}(#1)}}
\newcommand{\bfand}[2]{\ensuremath{b_{\bund}(#1,#2)}}
\newcommand{\bfor}[2]{\ensuremath{b_{\boder}(#1,#2)}}
\newcommand{\bfimp}[2]{\ensuremath{b_{\bimp}(#1,#2)}}

% Aussagenkalkül
\newcommand{\AAL}{A_{AL}}
\newcommand{\LAL}{\hbox{\textit{For}}_{AL}}
\newcommand{\AxAL}{\hbox{\textit{Ax}}_{AL}}
\newcommand{\MP}{\hbox{\textit{MP}}}
