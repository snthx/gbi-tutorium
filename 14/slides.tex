% ===== handout mode =====
% Comment/uncomment this line to toggle handout mode
% \newcommand{\handout}{}

% Comment/uncoment this line to toogle Mortitz mode
% \newcommand{\Moritz}{}

% Comment/uncomment this line to toggle handout mode
% \newcommand{\handout}{}

% by Stephan

%% Moritz mode or Stephan mode
\ifdefined \Moritz

% This is a configuration file with private, tutor specific information.
% It is therefore excluded from the Git repository so changes in this file will not conflict in git commits.

% Copy this Template, rename to config.tex and add your information below.

\newcommand{\mymail}{moritz.laupichler@student.kit.edu} % Consider using your named student Mail address to keep your u-Account private.

\newcommand{\myname}{\href{mailto:\mymail}{Moritz Laupichler}}

\newcommand{\mytutnumber}{25}

\newcommand{\mytutinfos}{Dienstags, 5. Block (15:45-17:15 Uhr), SR -120}

\newcommand{\aboutMeFrame}{
	\begin{frame}{Euer Tutor}
		Name: \myname \\
		Alter: 21 Jahre \\
		Studiengang: Master Informatik, 1. Semester \\
		\vspace{1cm}
		\pause 
		\centering{Kontakt: \href{mailto:\mymail}{\mymail}}
	\end{frame}
} % Moritz mode
\else
\ifdefined \Alex

% This is a configuration file with private, tutor specific information.
% It is therefore excluded from the Git repository so changes in this file will not conflict in git commits.

% Copy this Template, rename to config.tex and add your information below.

\newcommand{\mymail}{alexander.klug@student.kit.edu} % Consider using your named student Mail address to keep your u-Account private.

\newcommand{\myname}{\href{mailto:\mymail}{Alexander Klug}}

\newcommand{\mytutnumber}{30}

\newcommand{\mytutinfos}{Mittwochs, 3. Block (11:30-13:00), SR -107}

\newcommand{\aboutMeFrame}{
	\begin{frame}{Euer Tutor}
		Name: \myname \\
		Alter: 19 Jahre \\
		Studiengang: Bachelor Informatik, 3. Semester \\
		\vspace{1cm}
		\pause 
		\centering{Kontakt: \href{mailto:\mymail}{\mymail}}
	\end{frame}
}

% Toggle Handout mode by including the following line before including style_tut
% and removing the % at the start (but do NOT remove it here, otherwise handout mode will always be on!)
% Please keep handout mode on in all commits!

% \newcommand{\handout}{} % Alex Mode
\else

% This is a configuration file with private, tutor specific information.
% It is therefore excluded from the Git repository so changes in this file will not conflict in git commits.

% Copy this Template, rename to config.tex and add your information below.

\newcommand{\mymail}{stephan.bohr@student.kit.edu} % Consider using your named student Mail address to keep your u-Account private.

\newcommand{\myname}{\href{mailto:\mymail}{Stephan Bohr}}

\newcommand{\mytutnumber}{25}

\newcommand{\mytutinfos}{Dienstags, 5. Block (15:45-17:15), SR -119}

\newcommand{\aboutMeFrame}{
	\begin{frame}{Euer Tutor}
		Name: \myname \\
		Alter: 20 Jahre \\
		Studiengang: Bachelor Informatik, 3. Semester \\
		\vspace{1cm}
		\pause 
		\centering{Kontakt: \href{mailto:\mymail}{\mymail}}
	\end{frame}
} % Stephan mode
\fi
\fi

%% Beamer-Klasse im korrekten Modus
\ifdefined \handout
\documentclass[handout]{beamer} % Handout mode
\else
\documentclass{beamer}
\fi
%\documentclass[18pt,parskip]{beamer}

%% SLIDE FORMAT

% use 'beamerthemekit' for standard 4:3 ratio
% for widescreen slides (16:9), use 'beamerthemekitwide'

\usepackage{../templates/KIT-slides/beamerthemekit}
%\usepackage{../templates/KIT-slides/beamerthemekitwide}

%% TITLE PICTURE

% if a custom picture is to be used on the title page, copy it into the 'logos'
% directory, in the line below, replace 'mypicture' with the 
% filename (without extension) and uncomment the following line
% (picture proportions: 63 : 20 for standard, 169 : 40 for wide
% *.eps format if you use latex+dvips+ps2pdf, 
% *.jpg/*.png/*.pdf if you use pdflatex)

\titleimage{../figures/titleimage/brain}

%% TITLE LOGO

% for a custom logo on the front page, copy your file into the 'logos'
% directory, insert the filename in the line below and uncomment it

%\titlelogo{mylogo}

% (*.eps format if you use latex+dvips+ps2pdf,
% *.jpg/*.png/*.pdf if you use pdflatex)

%% TikZ INTEGRATION

% use these packages for PCM symbols and UML classes
% \usepackage{templates/tikzkit}
% \usepackage{templates/tikzuml}

%\usepackage{tikz}
%\usetikzlibrary{matrix}
%\usetikzlibrary{arrows.meta}
%\usetikzlibrary{automata}
%\usetikzlibrary{tikzmark}

%%%%%%%%%%%%%%%%%%%%%%%%%
% Libertine font (Original GBI font)
\usepackage{libertine}
%\renewcommand*\familydefault{\sfdefault}  %% Only if the base font of the document is to be sans serif

%% Schönere Schriften
\usepackage[TS1,T1]{fontenc}

%% Deutsche Silbentrennung und Beschriftungen
\usepackage[ngerman]{babel}

%% UTF-8-Encoding
\usepackage[utf8]{inputenc}

%% Bibliotheken für viele mathematische Symbole
\usepackage{amsmath, amsfonts, amssymb}

%% Anzeigetiefe für Inhaltsverzeichnis: 1 Stufe
\setcounter{tocdepth}{1}

%% Hyperlinks
\usepackage{hyperref}
% I don't know why, but this works and only includes sections and NOT subsections in the pdf-bookmarks.
\hypersetup{bookmarksdepth=subsection}

%% remove navigation symbols
\setbeamertemplate{navigation symbols}{}

%% switch between "ngerman" and "english" for German/English style date and logos
\selectlanguage{ngerman}

%% for invisible pause texts instead of dimming
\setbeamercovered{invisible}

%%%%%%%%%%%% Shortcuts %%%%%%%%%%%%%
\newcommand{\nM}{\mathbb{M}}
\newcommand{\nR}{\mathbb{R}}
\newcommand{\nN}{\mathbb{N}}
\newcommand{\nZ}{\mathbb{Z}}
\newcommand{\nQ}{\mathbb{Q}}
\newcommand{\nB}{\mathbb{B}}
\newcommand{\nC}{\mathbb{C}}
\newcommand{\nK}{\mathbb{K}}
\newcommand{\nF}{\mathbb{F}}
\newcommand{\nG}{\mathbb{G}}
\newcommand{\nullel}{\mathcal{O}}
\newcommand{\einsel}{\mathds{1}}
\newcommand{\nP}{\mathbb{P}}
\newcommand{\Pot}{\mathcal{P}}
\renewcommand{\O}{\text{O}}

\newcommand{\set}[1]{\{ #1 \}}
\newcommand{\setc}[2]{\set{#1 \mid #2}}
\newcommand{\setC}[2]{\set{#1 \mid \text{ #2 }}}

\newcommand{\setsize}[1]{\; \mid #1 \mid \; }

\newcommand{\q}[1]{\textquotedblleft #1\textquotedblright}

%%%%%%%%%%%% INHALT %%%%%%%%%%%%%%%%

%% Wochennummer
%\newcounter{weeknum}

%% Titelinformationen
%\title[GBI Tutorium, Woche \theweeknum]{Grundbegriffe der Informatik \\ Tutorium \mytutnumber}
%\subtitle{Termin \theweeknum \ | \mydate \\ \myname}
\author[\myname]{\myname}
\institute{Fakultät für Informatik}
%\date{\mydate}

%% Titel einfügen
\newcommand{\titleframe}{\frame{\titlepage}\addtocounter{framenumber}{-1}}


%% Alles starten mit \starttut{X}
%\newcommand{\starttut}[1]{\setcounter{weeknum}{#1}\titleframe\frame{\frametitle{Inhalt}\tableofcontents} \AtBeginSection[]{%
%\begin{frame}
%	\tableofcontents[currentsection]
%\end{frame}\addtocounter{framenumber}{-1}}}


%\newcommand{\framePrevEpisode}{
%	\begin{frame}
%		\centering
%		\textbf{In the previous episode of GBI...}
%	\end{frame}
%}

%% Roadmap frame
%table of contents
\newcommand{\roadmap}{
	\frame{\frametitle{Roadmap}\tableofcontents}}

 \AtBeginSection[]{%
\begin{frame}
	\frametitle{Roadmap}
	\tableofcontents[currentsection]
\end{frame}%\addtocounter{framenumber}{-1}
}


%% ShowMessage frame
\newcommand{\showmessage}[1]{\frame{\frametitle{\phantom{1em}}\centering\textbf{#1}}}

%% Fragen
%% Lastframe
\newcommand{\questionframe}{\showmessage{Fragen?}}

%% Lastframe
\newcommand{\lastframe}{\showmessage{Vielen Dank für Eure Aufmerksamkeit! \\Bis nächste Woche :)}}

%% Thanks frame
\newcommand{\slideThanks}{
	\begin{frame}
		\frametitle{Credits}
		\begin{block}{}
			An der Erstellung des Foliensatzes haben mitgewirkt:\\[1em]
			\ifdefined \Moritz
			Stephan Bohr \\
			Alexander Klug \\
			\else
			\ifdefined \Alex
			Stephan Bohr \\
			Moritz Laupichler \\
			\else
			Moritz Laupichler \\
			Alexander Klug \\
			\fi
			\fi
			Katharina Wurz \\
			Thassilo Helmold \\
			Philipp Basler \\
			Nils Braun \\
			Dominik Doerner \\
			Ou Yue \\
		\end{block}
	\end{frame}
}

%% Verbatim
%\usepackage{moreverb}



\title[\Moritz{Reguläre Sprachen und }Wiederholung]{14 . Tutorium\\ \Moritz{Reguläre Sprachen, }Wiederholung und Altklausuren}
\subtitle{Grundbegriffe der Informatik, Tutorium \hashtag\mytutnumber}
\date{\today}

\begin{document}
\titleframe

%\section{Organisatorisches}
\begin{frame}{Organisatorisches}
	\begin{itemize}
		\item \textbf{Für Übungsschein angemeldet?}
		\item \textbf{Für Klausur angemeldet?}
		\item Alle noch nicht abgeholten Übungsblätter könnt ihr beim Übungsleiter\footnote{Augusto Modanese, Raum 231, Geb. 50.34 (Infobau)} abholen
	\end{itemize}
\end{frame}
\Moritz{
\begin{frame}{Übungsklausur}
	\begin{itemize}
		\item Denkt daran: Die Punkte und die Note der Übungsklausur zählen nichts!\begin{itemize}
			\item In der Klausur seid ihr besser vorbereitet (hoffentlich)
			\item Die Notenskala ist ziemlich willkürlich
		\end{itemize}
		\item Ich hoffe ihr könnt euch jetzt besser vorstellen, wie eine Klausur so abläuft und aussieht
	\end{itemize}
\end{frame}
}

\roadmap

%%%%%%%%%% %%%%%%%%%%

\Moritz{
	
\section{Reguläre Sprachen}
\subsection{Reguläre Ausdrücke}

\begin{frame}{Reguläre Ausdrücke}
\begin{block}{Def.: Regulärer Ausdruck}
	Es sei $A$ ein Alphabet, das keines der Symbole aus $Z := \set{\mid, (, ), *, \emptyset}$ enthält.
	Ein \textbf{regulärer Ausdruck} über $A$ ist eine Zeichenfolge über $A \cup Z$, das gewissen Vorschriften genügt.\\
	Die \textbf{Menge der regulären Ausdrücke} ist wie folgt festgelegt:
	\begin{itemize}
		\item $\emptyset$ ist ein regulärer Ausdruck
		\item Für jedes $x \in A$ ist $x$ ein regulärer Ausdruck
		\item Sind $R_1$ und $R_2$ reguläre Ausdrücke, dann auch $(R_1 | R_2 )$ und $(R_1R_2)$
		\item Ist $R$ ein regulärer Ausdruck, dann auch $(R*)$
		\item Nichts anderes sind reguläre Ausdrücke
	\end{itemize}
	Die durch $R$ beschriebene formale Sprache ist $\lang{R}$.
\end{block}
\end{frame}

\begin{frame}{Reguläre Ausdrücke}
\begin{exampleblock}{Aufgabe}
	Gib einen regulären Ausdruck $R_i$ an, für den gilt:
	\begin{enumerate}
		\item $\lang{R_\theenumi}$ enthält genau die Wörter, in denen das Teilwort $\texttt{baa}$ vorkommt,
		\item $\lang{R_\theenumi}$ enthält genau die Wörter, in denen das Teilwort $\texttt{baa}$ nicht vorkommt,
		\item $\lang{R_\theenumi}$ enthält genau die Wörter, in denen das Teilwort $\texttt{baa}$ genau zweimal vorkommt,
		\item $\lang{R_\theenumi}$ enthält genau die Wörter, in denen mindestens drei $\texttt{b}$s vorkommen,
		\item \textbf{Wichtig!} $\lang{R_\theenumi} = \set{\varepsilon}$,
	\end{enumerate}
	mit $i \in \set{1,2,3,4,5}$ und $A := \set{\texttt{a}, \texttt{b}}$.
\end{exampleblock}
\end{frame}

\begin{frame}{Reguläre Ausdrücke}
\newcommand{\any}{(\texttt{a} | \texttt{b}) *}
\begin{block}{Lösung}
	\begin{enumerate}
		\item $R_\theenumi := \any \texttt{baa} \any $
		\item Die Aussage bedeutet umformuliert, dass nach jedem $\texttt{b}$ höchstens ein $\texttt{a}$ kommt, also $R_\theenumi := \texttt{a} * (\texttt{b} | \texttt{ba} ) * $
		\item In jedem Wort muss genau zweimal $\texttt{baa}$ vorkommen. Davor, dazwischen und danach dürfen Teilworte stehen, in denen das Teilwort $\texttt{baa}$ nicht vorkommt (wie in 2.). Also $R_\theenumi := R_2 \texttt{baa} R_2 \texttt{baa} R_2$.
		\item z.B. $R_\theenumi := \any \texttt{b} \any \texttt{b}\any \texttt{b} \any $
		oder $R_\theenumi := \texttt{a} * \texttt{ba} * \texttt{ba} * \texttt{b} \any$
		\item Für $R_\theenumi := \emptyset *$ ist $\lang{R_\theenumi} = \lang{\emptyset *} = \lang{\emptyset}^* = \lang{}^* = \set{}^* = \set{\varepsilon}$
	\end{enumerate}
\end{block}
\end{frame}

\begin{frame}{Reguläre Ausdrücke}
\begin{exampleblock}{Aufgabe}
	Wenn $R$ ein regulärer Ausdruck für die Sprache $L$ ist, wie sieht dann ein regulärer Ausdruck für
	\begin{enumerate}
		\item $L^*$,
		\item $L^+$
	\end{enumerate}
	aus?
\end{exampleblock}
\pause
\begin{block}{Lösung}
	\begin{enumerate}
		\item $(R*)$
		\item $R(R*)$
	\end{enumerate}
\end{block}
\end{frame}

\Moritz{\begin{frame}{Reguläre Ausdrücke}
	Zum Ableiten der Sprache $\lang{R}$ eines regulären Ausdruckes $R$ kann man folgende Regeln anwenden:
	\begin{itemize}
		\item $\lang{\emptyset} = \set{}$
		\item Für $x \in A$ gilt $\lang{x} = \set{x}$
		\item $\lang{R_1,R_2} = \lang{R_1} \cup \lang{R_2}$
		\item $\lang{R_{1}R_{2}} = \lang{R_1} \cdot \lang{R_2}$
		\item $\lang{R *} = \lang{R}^{\ast}$
	\end{itemize}
\end{frame}}
\subsection{Rechtslineare Grammatiken}

\begin{frame}{Rechtslineare Grammatiken}
\begin{block}{Def.: Kontextfreie Grammatik}
	Eine \textbf{kontextfreie Grammatik} $G$ ist ein Tupel $G = (N,T,S,P)$ mit
	\begin{itemize}
		\item $N$ ist das Alphabet der Nonterminalsymbole, auch Nichtterminalsymbole oder Variablen
		\item $T$ ist das Alphabet der Terminalsymbole, auch Zeichen, mit $N \cap T = \emptyset$
		\item $S$ ist das Startsymbol mit $S \in N$, also die Variable, mit der man beginnt
		\item $P$ ist die Menge der Produktionen mit
			\begin{itemize}
				\item $P \subseteq N \times (N \cup T)^*$, d.h. alle Produktionen besitzen folgende Form:\\
				$V \to w \text{ mit } V \in N \text{ und } w \in (N \cup T)^*$
			\end{itemize}
	\end{itemize}
\end{block}
\end{frame}

\begin{frame}{Rechtslineare Grammatiken}
\begin{block}{Def.: Rechtslineare Grammatik}
	Eine \textbf{rechtlineare Grammatik} $G$ ist eine kontextfreie Grammatik $G = (N,T,S,P)$ mit folgenden Einschränkungen für die Produktionen $P$:\\
	\begin{itemize}
			\item $P$ ist entweder von der Form \[
		V \to \omega \text{ mit } \omega \in T^*
	\]
	\item oder \[
		V \to \omega W \text{ mit } \omega \in T^* \text{ und } V,W \in N
	\]
	\item Also: Auf der rechten Seite einer Produktion darf höchstens ein Nonterminalsymbol vorkommen, und wenn dann nur als letztes Symbol.
	\end{itemize}
\end{block}
\pause
\begin{exampleblock}{Beispiel}
	\[
		G = (\set{X,Y}, \set{a,b}, X, \set{X \to aX | bY, Y \to bY | \varepsilon})
	\]
\end{exampleblock}
\end{frame}

\subsection{Reguläre Sprache}
\begin{frame}{Reguläre Sprache}
    \begin{block}{Def.: Reguläre Sprache}
    	Für eine formale Sprache $L$ sind folgende drei Aussagen äquivalent:
    	\begin{itemize}
    		\item $L$ kann von einem endlichen Akzeptor erkannt werden
    		\item $L$ kann durch einen regulären Ausdruck beschrieben werden
    		\item $L$ kann von einer rechtslinearen Grammatik erzeugt werden
    	\end{itemize}
    	Eine solche Sprache $L$ heißt dann auch \textbf{reguläre Sprache} (oder Typ-3-Sprache) 
    \end{block}

    \begin{exampleblock}{Aufgabe}
    	Sei $G=(\set{X,Y,Z},\set{a,b},X,P)$ mit $P = \set{
    		X \to aX | bY | \varepsilon,
    		Y \to aX | bZ | \varepsilon, 
    		Z \to aZ | bZ
    	}$.\\ 
    	Zeichne den Akzeptor, der $L(G)$ akzeptiert.\\
    	Wie kann man die Grammatik vereinfachen?
    \end{exampleblock}
\end{frame}

\begin{frame}{Reguläre Sprache}
    \begin{block}{Lösung}
    	\begin{itemize}
    		\item Akzeptor, der $L(G)$ akzeptiert:\\
    		\begin{figure}[ht]
  			\centering
    			\begin{tikzpicture}[shorten >=1pt,node distance=2cm,auto,initial text=,>=stealth]
      				\node[state,initial,accepting]  (q_0)                       {$X$};
	      			\node[state,accepting]          (q_1) [right of= q_0] {$Y$};
	      			\node[state]                    (q_2) [right of= q_1] {$Z$};
	     			\path[->] (q_0) edge [loop below]      node        {$a$} ()
	     			edge [bend right] node [swap] {$b$} (q_1)
	      			(q_1) edge              node        {$b$} (q_2)
	      			edge [bend right] node [swap] {$a$} (q_0)
	      			(q_2) edge [loop below] node        {$a,b$} ()
	      			;
    			\end{tikzpicture}
    		\end{figure}
    		\item Einfacher: $G=(\set{X,Y},\set{a,b},X,P)$ mit $P=\set{X \to a X \mid b Y \mid \varepsilon,\;\; Y \to a X \mid \varepsilon}$
    		\item Oder noch einfacher:
    		$G=(\set{X},\set{a,b},X,P)$ mit $P=\set{X \to a X\mid baX \mid b \mid \varepsilon}$
    	\end{itemize}
    \end{block}
\end{frame}

\subsection{Regex-Bäume}
\begin{frame}{Regex-Bäume}
    \begin{block}{Def.: Regex-Baum}
    \small
    	Es sei $A$ ein beliebiges Alphabet. Dann heißt ein Baum \textbf{Regex-Baum}, falls:
    	\begin{itemize}
    		\item Entweder ist es ein Baum, dessen Wurzel zugleich Blatt ist und mit einem $x\in A$ oder $\emptyset$ beschriftet ist,
    		\item oder es ist ein Baum, dessen Wurzel mit $*$ beschriftet ist und genau einen Nachfolgeknoten hat, der Wurzel eines Regex-Baumes ist
    		\item oder es ist ein Baum, dessen Wurzel mit $\cdot$ oder mit $|$ beschriftet ist und genau zwei Nachfolgeknoten hat, die Wurzeln zweier Regex-Bäume sind. 
    	\end{itemize}
    \end{block}

    % \only<1|handout:1>{
    % \begin{exampleblock}{Beispiel}
    % 	Zeichne den Kantorowitschbaum zum arithmetischen Ausdruck
    % 	\[
    % 		3+(a+b)*(-c)
    % 	\]
    % \end{exampleblock}
    % }

   %  \begin{center}
   %  \begin{tikzpicture}
   %    [level 1/.style={sibling distance=30mm},
   %    level 2/.style={sibling distance=20mm},
   %    level 3/.style={sibling distance=15mm},
   %    nodes={draw,circle,inner sep=0pt, minimum size=7mm},
   %    ->,>=stealth]
      
   %    \node {$+$}
   %    child { node {$3$}  edge from parent  }
   %    child { node {$*$}  edge from parent
   %      % [level 2/.style={sibling distance=15mm}]
   %      child { node {$+$}  edge from parent
   %        % [level 2/.style={sibling distance=15mm}]
   %        child { node {$a$}  edge from parent
   %          % [level 2/.style={sibling distance=15mm}]
   %        }
   %        child { node {$b$}  edge from parent
   %          % [level 2/.style={sibling distance=15mm}]
   %        }
   %      }
   %      child {node {$-$} edge from parent
   %        child { node {$c$}  edge from parent}
   %      }
   %    }
   %    ;
   %  \end{tikzpicture}
  	% \end{center}


    \only<1|handout:1>{
    	\begin{exampleblock}{Aufgabe}
    		Zeichne den Regex-Baum zu folgendem regulären Ausdruck:
    		\[
    			R = ( ( b | \emptyset * ) a ) (b * )
    		\]
    	\end{exampleblock}
    }

    
\end{frame}

\begin{frame}{Regex-Bäume}
	\begin{block}{Lösung zu $R = ( ( b | \emptyset * ) a ) (b * )$}
		\begin{figure}[ht]
  			\centering
  			\begin{tikzpicture}
    			[level 1/.style={sibling distance=30mm},
    			level 2/.style={sibling distance=20mm},
    			level 3/.style={sibling distance=15mm},
    			nodes={draw,circle,inner sep=0pt, minimum size=7mm},
    			->,>=stealth]
			    
    			\node {$\cdot$}
    			child { node {$\cdot$}  edge from parent
      			% [level 2/.style={sibling distance=15mm}]
      			child { node {$|$}  edge from parent
        			% [level 2/.style={sibling distance=15mm}]
        			child { node {$b$}  edge from parent
          			% [level 2/.style={sibling distance=15mm}]
        			}
              child { node {$*$}  edge from parent
                % [level 2/.style={sibling distance=15mm}]
                child { node {$\emptyset$}  edge from parent
                  % [level 2/.style={sibling distance=15mm}]
                }
              }
      			}
      			child {node {$a$} edge from parent}
    			}
    			child { node {$*$}  edge from parent
      			% [level 2/.style={sibling distance=15mm}]
      			child { node {$b$}  edge from parent
        			% [level 2/.style={sibling distance=15mm}]
      			}
    			}
    			;
			\end{tikzpicture}
  		\end{figure}
	\end{block}
\end{frame}
}

\section{Aufgaben von alten Übungsblättern}

\subsection{Aufgabe 1}
\begin{frame}{Aufgabe 1}
\begin{figure}[h!]
		\centering
		\includegraphics[width=\textwidth]{../topics/weihnachtstut-aufgaben/1.png} 
	\end{figure}     
\end{frame}

\begin{frame}{Aufgabe 1}
\begin{figure}[h!]
		\centering
		\includegraphics[width=\textwidth]{../topics/weihnachtstut-aufgaben/2.png} 
	\end{figure}  
	\begin{figure}[h!]
		\centering
		\includegraphics[width=\textwidth]{../topics/weihnachtstut-aufgaben/3.png} 
	\end{figure}   
\end{frame}

\subsection{Aufgabe 2}
\begin{frame}{Aufgabe 2}
\begin{figure}[h!]
		\centering
		\includegraphics[width=\textwidth]{../topics/weihnachtstut-aufgaben/4.png} 
	\end{figure}     
Vermeide führende Nullen!
\end{frame}

\begin{frame}{Aufgabe 2}
\begin{figure}[h!]
		\centering
		\includegraphics[width=\textwidth]{../topics/weihnachtstut-aufgaben/5.png} 
	\end{figure}    
\end{frame}

\subsection{Aufgabe 3}
\begin{frame}{Aufgabe 3}
\begin{figure}[h!]
		\centering
		\includegraphics[width=\textwidth]{../topics/weihnachtstut-aufgaben/6.png} 
	\end{figure}     
\end{frame}

\begin{frame}{Aufgabe 3}
\begin{figure}[h!]
		%\centering
		%\includegraphics[width=\textwidth]{../topics/weihnachtstut-aufgaben/7.png} 
	\end{figure}    
		$M(x)$: \enquote{x ist ein Mensch} \\[1em]
		\[ \forall x (M(x) \rightarrow \exists y (M(y) \wedge B(x,y) \wedge \forall z (\neg (z \doteq y) \rightarrow \neg B(x,z))) \]
\end{frame}

\subsection{Aufgabe 4}
\begin{frame}{Aufgabe 4}
\begin{figure}[h!]
		\centering
		\includegraphics[width=\textwidth]{../topics/weihnachtstut-aufgaben/8.png} 
	\end{figure}     
\end{frame}

\begin{frame}{Aufgabe 4}
\begin{figure}[h!]
		\centering
		\includegraphics[width=\textwidth]{../topics/weihnachtstut-aufgaben/9.png} 
	\end{figure}    
\end{frame}

\begin{frame}{Aufgabe 4}
\begin{figure}[h!]
		\centering
		\includegraphics[width=\textwidth]{../topics/weihnachtstut-aufgaben/10.png} 
	\end{figure}    
\end{frame}

\subsection{Aufgabe 5}
\begin{frame}{Aufgabe 5}
\begin{figure}[h!]
		\centering
		\includegraphics[width=\textwidth]{../topics/weihnachtstut-aufgaben/11.png} 
	\end{figure}     

	Wir wollen bei dieser Aufgabe die korrekte Notation etwas dehnen und erlauben die (in der intuitiven Bedeutung äquivalenten) Schreibweisen\footnote{vgl. Foliensatz Prädikatenlogik, Folie 44}:
	\begin{itemize}
		\item $\forall x \in F: \dots$ \quad statt \quad $\forall x (p(x) \rightarrow \dots) \quad \text{mit } I(p) = F \subseteq D.$
		\item $\exists x \in F: \dots$ \quad statt \quad $\exists x (p(x) \wedge \dots) \quad \text{mit } I(p) = F \subseteq D.$
	\end{itemize}

\end{frame}

\begin{frame}{Aufgabe 5}
\begin{figure}[h!]
		\centering
		\includegraphics[width=\textwidth]{../topics/weihnachtstut-aufgaben/12.png} 
\end{figure} 

\end{frame}   
   	
\begin{frame}{Aufgabe 5}
	
		Die Notation oben ist natürlich nicht in unserem Sinne, aber mit der korrekten Notation wird es unübersichtlich: \\[1em]

		\begin{enumerate}
			\item \enquote{Jeder Frosch ist glücklich, wenn alle seine Kinder quaken können.} \\[.3em]
				\[ \forall y (F(y) \rightarrow ((\forall x ((F(x) \rightarrow K(x,y)) \rightarrow quak(x) )) \rightarrow happy(y))) \]
		\end{enumerate}

\end{frame}

\subsection{Aufgabe 6}
\begin{frame}{Aufgabe 6}
\begin{figure}[h!]
		\centering
		\includegraphics[width=\textwidth]{../topics/weihnachtstut-aufgaben/13.png} 
	\end{figure}     
\end{frame}

\begin{frame}{Aufgabe 6}
\begin{figure}[h!]
		\centering
		\includegraphics[width=\textwidth]{../topics/weihnachtstut-aufgaben/14.png} 
	\end{figure}     
\end{frame}

\begin{frame}{Aufgabe 6}
\begin{figure}[h!]
		\centering
		\includegraphics[width=\textwidth]{../topics/weihnachtstut-aufgaben/15.png} 
	\end{figure}     
\end{frame}

\begin{frame}{Aufgabe 6}
\begin{figure}[h!]
		\centering
		\includegraphics[width=\textwidth]{../topics/weihnachtstut-aufgaben/16.png} 
	\end{figure}     
\end{frame}

\begin{frame}{Aufgabe 6}
\begin{figure}[h!]
		\centering
		\includegraphics[width=\textwidth]{../topics/weihnachtstut-aufgaben/17.png} 
	\end{figure}     
\end{frame}

\subsection{Aufgabe 7}
\begin{frame}{Aufgabe 7}
\begin{figure}[h!]
		\centering
		\includegraphics[width=\textwidth]{../topics/weihnachtstut-aufgaben/18.png} 
	\end{figure}     
\end{frame}

\begin{frame}{Aufgabe 7}
\begin{figure}[h!]
		\centering
		\includegraphics[width=\textwidth]{../topics/weihnachtstut-aufgaben/19.png} 
	\end{figure}  
	\begin{figure}[h!]
		\centering
		\includegraphics[width=\textwidth]{../topics/weihnachtstut-aufgaben/20.png} 
	\end{figure}   
\end{frame}

\section{Altklausuren}
\begin{frame}{Altklausuren}
    \begin{itemize}
    	\item Im Ilias oder auf 
    	\item \url{http://gbi.ira.uka.de} im Archiv
    \end{itemize}
\end{frame}

%% Zusammenfassung
\section{}

%\subsection{Zusammenfassung}
	\begin{frame}{Was ihr jetzt kennen und können solltet\dots}
			\begin{itemize}
				\item ALLES!!!!
			\end{itemize}
	
	\end{frame}
%% Ausblick
%\subsection{Ausblick}
	\begin{frame}{Ausblick}
		\begin{itemize}
			\item Semester``ferien''
			\item Viele Altklausuren rechnen!
		\end{itemize}
	\end{frame}
%%%%%%%%%% %%%%%%%%%%
\section{}
\questionframe
\showmessage{Vielen Dank für Eure Aufmerksamkeit!\\ Viel Erfolg bei der Klausur und vielleicht bis nächstes Semester! :)}
\mode<handout>{\slideThanks}
\end{document}