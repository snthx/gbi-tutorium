% ===== handout mode =====
% Comment/uncomment this line to toggle handout mode
% \newcommand{\handout}{}

% Comment/uncoment this line to toogle Mortitz mode
% \newcommand{\Moritz}{}

% Comment/uncomment this line to toggle handout mode
% \newcommand{\handout}{}

% by Stephan

%% Moritz mode or Stephan mode
\ifdefined \Moritz

% This is a configuration file with private, tutor specific information.
% It is therefore excluded from the Git repository so changes in this file will not conflict in git commits.

% Copy this Template, rename to config.tex and add your information below.

\newcommand{\mymail}{moritz.laupichler@student.kit.edu} % Consider using your named student Mail address to keep your u-Account private.

\newcommand{\myname}{\href{mailto:\mymail}{Moritz Laupichler}}

\newcommand{\mytutnumber}{25}

\newcommand{\mytutinfos}{Dienstags, 5. Block (15:45-17:15 Uhr), SR -120}

\newcommand{\aboutMeFrame}{
	\begin{frame}{Euer Tutor}
		Name: \myname \\
		Alter: 21 Jahre \\
		Studiengang: Master Informatik, 1. Semester \\
		\vspace{1cm}
		\pause 
		\centering{Kontakt: \href{mailto:\mymail}{\mymail}}
	\end{frame}
} % Moritz mode
\else
\ifdefined \Alex

% This is a configuration file with private, tutor specific information.
% It is therefore excluded from the Git repository so changes in this file will not conflict in git commits.

% Copy this Template, rename to config.tex and add your information below.

\newcommand{\mymail}{alexander.klug@student.kit.edu} % Consider using your named student Mail address to keep your u-Account private.

\newcommand{\myname}{\href{mailto:\mymail}{Alexander Klug}}

\newcommand{\mytutnumber}{30}

\newcommand{\mytutinfos}{Mittwochs, 3. Block (11:30-13:00), SR -107}

\newcommand{\aboutMeFrame}{
	\begin{frame}{Euer Tutor}
		Name: \myname \\
		Alter: 19 Jahre \\
		Studiengang: Bachelor Informatik, 3. Semester \\
		\vspace{1cm}
		\pause 
		\centering{Kontakt: \href{mailto:\mymail}{\mymail}}
	\end{frame}
}

% Toggle Handout mode by including the following line before including style_tut
% and removing the % at the start (but do NOT remove it here, otherwise handout mode will always be on!)
% Please keep handout mode on in all commits!

% \newcommand{\handout}{} % Alex Mode
\else

% This is a configuration file with private, tutor specific information.
% It is therefore excluded from the Git repository so changes in this file will not conflict in git commits.

% Copy this Template, rename to config.tex and add your information below.

\newcommand{\mymail}{stephan.bohr@student.kit.edu} % Consider using your named student Mail address to keep your u-Account private.

\newcommand{\myname}{\href{mailto:\mymail}{Stephan Bohr}}

\newcommand{\mytutnumber}{25}

\newcommand{\mytutinfos}{Dienstags, 5. Block (15:45-17:15), SR -119}

\newcommand{\aboutMeFrame}{
	\begin{frame}{Euer Tutor}
		Name: \myname \\
		Alter: 20 Jahre \\
		Studiengang: Bachelor Informatik, 3. Semester \\
		\vspace{1cm}
		\pause 
		\centering{Kontakt: \href{mailto:\mymail}{\mymail}}
	\end{frame}
} % Stephan mode
\fi
\fi

%% Beamer-Klasse im korrekten Modus
\ifdefined \handout
\documentclass[handout]{beamer} % Handout mode
\else
\documentclass{beamer}
\fi
%\documentclass[18pt,parskip]{beamer}

%% SLIDE FORMAT

% use 'beamerthemekit' for standard 4:3 ratio
% for widescreen slides (16:9), use 'beamerthemekitwide'

\usepackage{../templates/KIT-slides/beamerthemekit}
%\usepackage{../templates/KIT-slides/beamerthemekitwide}

%% TITLE PICTURE

% if a custom picture is to be used on the title page, copy it into the 'logos'
% directory, in the line below, replace 'mypicture' with the 
% filename (without extension) and uncomment the following line
% (picture proportions: 63 : 20 for standard, 169 : 40 for wide
% *.eps format if you use latex+dvips+ps2pdf, 
% *.jpg/*.png/*.pdf if you use pdflatex)

\titleimage{../figures/titleimage/brain}

%% TITLE LOGO

% for a custom logo on the front page, copy your file into the 'logos'
% directory, insert the filename in the line below and uncomment it

%\titlelogo{mylogo}

% (*.eps format if you use latex+dvips+ps2pdf,
% *.jpg/*.png/*.pdf if you use pdflatex)

%% TikZ INTEGRATION

% use these packages for PCM symbols and UML classes
% \usepackage{templates/tikzkit}
% \usepackage{templates/tikzuml}

%\usepackage{tikz}
%\usetikzlibrary{matrix}
%\usetikzlibrary{arrows.meta}
%\usetikzlibrary{automata}
%\usetikzlibrary{tikzmark}

%%%%%%%%%%%%%%%%%%%%%%%%%
% Libertine font (Original GBI font)
\usepackage{libertine}
%\renewcommand*\familydefault{\sfdefault}  %% Only if the base font of the document is to be sans serif

%% Schönere Schriften
\usepackage[TS1,T1]{fontenc}

%% Deutsche Silbentrennung und Beschriftungen
\usepackage[ngerman]{babel}

%% UTF-8-Encoding
\usepackage[utf8]{inputenc}

%% Bibliotheken für viele mathematische Symbole
\usepackage{amsmath, amsfonts, amssymb}

%% Anzeigetiefe für Inhaltsverzeichnis: 1 Stufe
\setcounter{tocdepth}{1}

%% Hyperlinks
\usepackage{hyperref}
% I don't know why, but this works and only includes sections and NOT subsections in the pdf-bookmarks.
\hypersetup{bookmarksdepth=subsection}

%% remove navigation symbols
\setbeamertemplate{navigation symbols}{}

%% switch between "ngerman" and "english" for German/English style date and logos
\selectlanguage{ngerman}

%% for invisible pause texts instead of dimming
\setbeamercovered{invisible}

%%%%%%%%%%%% Shortcuts %%%%%%%%%%%%%
\newcommand{\nM}{\mathbb{M}}
\newcommand{\nR}{\mathbb{R}}
\newcommand{\nN}{\mathbb{N}}
\newcommand{\nZ}{\mathbb{Z}}
\newcommand{\nQ}{\mathbb{Q}}
\newcommand{\nB}{\mathbb{B}}
\newcommand{\nC}{\mathbb{C}}
\newcommand{\nK}{\mathbb{K}}
\newcommand{\nF}{\mathbb{F}}
\newcommand{\nG}{\mathbb{G}}
\newcommand{\nullel}{\mathcal{O}}
\newcommand{\einsel}{\mathds{1}}
\newcommand{\nP}{\mathbb{P}}
\newcommand{\Pot}{\mathcal{P}}
\renewcommand{\O}{\text{O}}

\newcommand{\set}[1]{\{ #1 \}}
\newcommand{\setc}[2]{\set{#1 \mid #2}}
\newcommand{\setC}[2]{\set{#1 \mid \text{ #2 }}}

\newcommand{\setsize}[1]{\; \mid #1 \mid \; }

\newcommand{\q}[1]{\textquotedblleft #1\textquotedblright}

%%%%%%%%%%%% INHALT %%%%%%%%%%%%%%%%

%% Wochennummer
%\newcounter{weeknum}

%% Titelinformationen
%\title[GBI Tutorium, Woche \theweeknum]{Grundbegriffe der Informatik \\ Tutorium \mytutnumber}
%\subtitle{Termin \theweeknum \ | \mydate \\ \myname}
\author[\myname]{\myname}
\institute{Fakultät für Informatik}
%\date{\mydate}

%% Titel einfügen
\newcommand{\titleframe}{\frame{\titlepage}\addtocounter{framenumber}{-1}}


%% Alles starten mit \starttut{X}
%\newcommand{\starttut}[1]{\setcounter{weeknum}{#1}\titleframe\frame{\frametitle{Inhalt}\tableofcontents} \AtBeginSection[]{%
%\begin{frame}
%	\tableofcontents[currentsection]
%\end{frame}\addtocounter{framenumber}{-1}}}


%\newcommand{\framePrevEpisode}{
%	\begin{frame}
%		\centering
%		\textbf{In the previous episode of GBI...}
%	\end{frame}
%}

%% Roadmap frame
%table of contents
\newcommand{\roadmap}{
	\frame{\frametitle{Roadmap}\tableofcontents}}

 \AtBeginSection[]{%
\begin{frame}
	\frametitle{Roadmap}
	\tableofcontents[currentsection]
\end{frame}%\addtocounter{framenumber}{-1}
}


%% ShowMessage frame
\newcommand{\showmessage}[1]{\frame{\frametitle{\phantom{1em}}\centering\textbf{#1}}}

%% Fragen
%% Lastframe
\newcommand{\questionframe}{\showmessage{Fragen?}}

%% Lastframe
\newcommand{\lastframe}{\showmessage{Vielen Dank für Eure Aufmerksamkeit! \\Bis nächste Woche :)}}

%% Thanks frame
\newcommand{\slideThanks}{
	\begin{frame}
		\frametitle{Credits}
		\begin{block}{}
			An der Erstellung des Foliensatzes haben mitgewirkt:\\[1em]
			\ifdefined \Moritz
			Stephan Bohr \\
			Alexander Klug \\
			\else
			\ifdefined \Alex
			Stephan Bohr \\
			Moritz Laupichler \\
			\else
			Moritz Laupichler \\
			Alexander Klug \\
			\fi
			\fi
			Katharina Wurz \\
			Thassilo Helmold \\
			Philipp Basler \\
			Nils Braun \\
			Dominik Doerner \\
			Ou Yue \\
		\end{block}
	\end{frame}
}

%% Verbatim
%\usepackage{moreverb}



\usetikzlibrary{matrix}
\usetikzlibrary{arrows.meta}
\usetikzlibrary{automata}
\usetikzlibrary{tikzmark}

\title[Turingmaschinen]{14 . Tutorium\\ Turingmaschinen, Berechnungskomplexität und Unentscheidbarkeit}
\subtitle{Grundbegriffe der Informatik, Tutorium \mytutnumber}
\date{\today}

\begin{document}
\titleframe

%\section{Organisatorisches}
\begin{frame}{Organisatorisches}
	\begin{itemize}
		\item Für Übungsschein angemeldet?
		\item Für Klausur angemeldet?
		\notStephan{\item Alle noch nicht abgeholten Übungsblätter und das korrigierte Bonusübungsblatt könnt ihr beim Übungsleiter\footnote{
	    	Thomas Zenkel, Gebäude 50.20 (ehem. Kinderklinik), Raum 239} abholen
	    \item Das Bonusübungsblatt müsst ihr aber nicht machen, wenn ihr keine Punkte mehr braucht, ich schreibe auch Klausuren und eine Musterlösung wird es online geben ;)}
	\end{itemize}
\end{frame}

\Stephan{
\begin{frame}[plain]{Übungsbetrieb}
    \centering\includegraphics[height=4cm]{statistiken}
    \begin{itemize}\small
    	\item Alle noch nicht abgeholten Übungsblätter und das korrigierte Bonusübungsblatt könnt ihr beim Übungsleiter\footnote{
	    	Thomas Zenkel, Gebäude 50.20 (ehem. Kinderklinik), Raum 239} abholen
	    \item \textbf{\emph{Bonus-ÜB}} müsst ihr \textbf{\emph{nicht machen}}, wenn ihr keine Punkte mehr braucht, ich schreibe auch Klausuren und eine Musterlösung wird es online geben ;)
	    \item Jeder, der das dritte Übungsblatt abgegeben hat, hat den Schein
	    \item Aufgabe 6.4: Ein gerichteter Baum mit $k$ Knoten hat genau $k-1$ Kanten
    \end{itemize}
\end{frame}
}

\roadmap

%%%%%%%%%% %%%%%%%%%%
\input{fortsetzung_endl-automaten}
\def\parfunc{\dashedrightarrow}
\subsection{Turingmaschinen}

\begin{frame}[fragile]{Turingmaschine}
   \begin{figure}[ht]
  		\centering
  		\begin{tikzpicture}[node distance=0mm,->,>={Latex[open]},thin]
    		\matrix[matrix of math nodes,column sep=0mm,minimum size=8mm,row sep=10mm,nodes={every rectangle node/.style={rectangle,draw,anchor=center}}]   {
      		& & & && \node[circle,draw] (z) {z}; & \\
      		\node[circle] {\cdots}; & \blank & \blank & |(x)| \#a & \#b & \#b & \#a & \#a & \blank & \node[circle] {\cdots};\\
    		};
    		\path[<->] (z) edge [out=270,in=90] (x);
  		\end{tikzpicture}
	\end{figure}
\end{frame}

\begin{frame}{Turingmaschine}
	\begin{block}{Def.: Turingmaschine}
		Eine \textbf{Turingmaschine} $T=(Z,z_0,X,f,$ $g,m)$ ist festgelegt durch
			\begin{itemize}
			\item eine \textbf{Zustandsmenge} $Z$
			\item einen \textbf{Anfangszustand} $z_0\in Z$
			\item ein \textbf{Bandalphabet} $X$
			\item eine partielle \textbf{Zustandsüberführungsfunktion} $f:Z\times X \dashrightarrow Z$
			\item eine partielle \textbf{Ausgabefunktion} $g:Z\times X \dashrightarrow X$ und
			\item eine partielle \textbf{Bewegungsfunktion} $m:Z\times X \dashrightarrow \{ -1, 0, 1\}$
			\begin{itemize}
				\item $f$, $g$ und $m$ seien für die gleichen Paare $(z,x)\in Z\times X$ definiert bzw. nicht definiert
			\end{itemize}
			\item[]
			\item Darstellung einer konkreten Turingmaschine:\\
					Tabelle oder graphisch (ähnlich wie bei Mealy-Automaten)
			\end{itemize} 
	\end{block}
\end{frame}

\begin{frame}{Turingmaschine}
	\dots Und was das ganze soll:\\

	\begin{itemize}
		\item \textbf{Bandalphabet} $X$
		\begin{itemize}
			\item meist mit Blanksymbol \blank
		\end{itemize}
		\item partielle \textbf{Zustandsüberführungsfunktion} $f:Z\times X \dashrightarrow Z$
		\begin{itemize}
			\item in neuen Zustand übergehen
		\end{itemize}
		\item \textbf{Ausgabefunktion} $g:Z\times X \dashrightarrow X$
		\begin{itemize}
			\item Band mit neuem Symbol beschriften
		\end{itemize}
		\item \textbf{Bewegungsfunktion} $m:Z\times X \dashrightarrow \{ -1, 0, 1\} \text{ bzw. } \{ L,0,R\}$
		\begin{itemize}
			\item Kopf bewegen
		\end{itemize}
		\item[]
		\item TM liest von Band, schreibt auf Band und kann ihren Kopf bewegen
	\end{itemize}
\end{frame}

\begin{frame}[fragile]{Turingmaschine}
    \begin{columns}\footnotesize
    	\begin{column}{0.4\textwidth}
    		\begin{tikzpicture}[shorten >=1pt,initial text=,node distance=1.7cm,auto,->,>=stealth,baseline=(B.base)]
          		% \node[state,initial]  (S)                       {$S$};
          		\node[state,initial]  (A)          {$A$};
          		% \node (nix) [right of=A] {};
          		\node[state]          (B) [above right of=A] {$B$};
          		\node[state]          (C) [right of=B] {$C$};
          		\node[state]          (E) [below right of=A] {$E$};
          		\node[state]          (D) [right of=E] {$D$};
          		\path[->]
          		% (S) edge              node  {$\9\io\9R$} (A)
          		(A) edge              node  {$\io{\#1}{\#XR}$} (B)
          		(B) edge [loop above] node  {$\io{\#1}{\#1R}$} ()
          		edge              node  {$\io{\9}{\9R}$} (C)
          		(C) edge [loop above] node  {$\io{\#1}{\#1R}$} ()
          		edge              node  {$\io{\9}{\#1L}$} (D)
          		(D) edge [loop below] node  {$\io{\#1}{\#1L}$} ()
          		edge              node  {$\io{\9}{\9L}$} (E)
          		(E) edge [loop below] node  {$\io{\#1}{\#1L}$} ()
          		edge              node  {$\io{\#X}{\#1R}$} (A)
          		% (B) edge              node        {$\9\io\9R$} (B)
          		% edge [loop right] node        {$\#1\io\#1R$} ()
          		% (B) edge [loop right] node {$\9\io\#1L$} ()
          		% edge  node [pos=0.3]       {$\#1\io\#1L$} (A)
          		;
        \end{tikzpicture}
		\end{column}
		\begin{column}{0.6\textwidth}
			\begin{tabular}[t]{>{$}c<{$}@{\quad}*{6}{>{$}c<{$}}}
          		\toprule
          		& A       & B      & C       & D       & E \\
          		\midrule
          		\9       
          		&         & \9,R,C & \#1,L,D & \9,L,E  &  \\
          		\#1         & \#X,R,B & \#1,R,B& \#1,R,C & \#1,L,D & \#1,L,E \\
          		\#X         &         &        &         &         & \#1,R,A \\
          		\bottomrule
        	\end{tabular}
        	\normalsize
        	\begin{exampleblock}{Aufgabe}
        		\begin{enumerate}
        			\item Was steht bei der Eingabe von \#{111} auf dem Band?
        			\item Was macht die TM allgemein, wenn man auf der ersten \#1 startet?
        		\end{enumerate}
        	\end{exampleblock}
		\end{column}
	\end{columns}
\end{frame}

\begin{frame}{Turingmaschinen}
    \begin{block}{Lösung}
    	\begin{enumerate}
    		\item $\dots \blank \#{111} \blank \#{111} \blank \dots$
    		\item diese TM kopiert ein Wort $\#1^k$ auf einem leeren Band, so dass hinterher $\cdots \blank \#1^k \blank \#1^k \blank \cdots$ da steht
    	\end{enumerate}
    \end{block}

    \textbf{Verallgemeinerung} fürs Kopieren von Wörtern über $\{\#0,\#1\}$:\\
    \begin{itemize}
    	\item auf dem Weg nach rechts merken, was mit \#X überschrieben wurde
    	\item auf dem Weg nach rechts und
    zurück sowohl \#1 als auch \#0 überlaufen
    \end{itemize}
\end{frame}

\begin{frame}{Turingmaschinen}
	\begin{block}{Def.: Konfiguration}
		Die \textbf{Konfiguration} $c = (z,b,p)$ beschreibt den „Gesamtzustand“ in dem sich die TM zu einem bestimmten Zeitpunkt befindet. Sie wird vollständig beschrieben durch:
			\begin{itemize}
			\item den aktuellen Zustand $z\in Z$ der Steuereinheit
			\item die aktuelle Beschriftung des gesamten Bandes, die man als
  			Abbildung $b:\nZ \to X$ formalisieren kann
			\item die aktuelle Position $p\in \nZ$ des Kopfes.
			\end{itemize}
	\end{block}

	\begin{block}{Def.: Endkonfiguration}
		Falls für eine Konfiguration $c$ die Nachfolgekonfiguration nicht definiert ist, heißt $c$ auch eine \textbf{Endkonfiguration} und man sagt, die Turingmaschine habe \textbf{gehalten}.
	\end{block}
\end{frame}

\begin{frame}{Turingmaschine}
    \begin{block}{Def.: $\Delta_t$ und $\Delta_*$}
    	Für $t\in \nN_0$ liefert $\Delta_t(c)$ die ausgehend von Konfiguration $c$ nach $t$ Schritten erreichte Konfiguration.

    		\[
			\Delta_*(c) =
			\begin{cases}
  			\Delta_t(c) & \text{, falls $\Delta_t(c)$ definiert und Endkonfiguration ist} \\
  			\text{undefiniert} & \text{, falls $\Delta_t(c)$ für alle $t\in \nN_0$ definiert ist} \\
			\end{cases}
			\]
    \end{block}
\end{frame}

\begin{frame}{Turingmaschinen}
    \begin{alertblock}{Achtung!}
    	\begin{itemize}
    		\item Turingmaschinen halten \textbf{nicht immer}!
    		\item es gibt auch unendliche Berechnungen, genauso wie z.B. in Java
    	\end{itemize}
    \end{alertblock}

    \begin{block}{Weitere Definitionen}
    	\begin{itemize}
    		\item \textbf{Endliche Berechnung}: endliche Folge von Konfigurationen $(c_0, c_1, c_2,$ $\dots, c_t)$ mit $c_i = \Delta_1(c_{i-1})$ für alle $0<i\leq t$ 
    		\item \textbf{Haltende Berechnung}: endliche Berechung, deren letzte Konfiguration eine Endkonfiguration ist
    		\item \textbf{Unendliche Berechnung/Nicht haltende Berechnung}: für jede Konfiguration ist immer die Nachfolgekonfiguration definiert
    	\end{itemize}
    \end{block}
\end{frame}

\begin{frame}{Turingmaschinen}
	\begin{exampleblock}{Aufgabe}
		 \begin{tabular}[t]{>{$}c<{$}@{\qquad}*{4}{>{$}c<{$}}}
      		\toprule
      		& r & c_0 & c_1 & h \\
      		\midrule
      		\#0 & \#0,R,r   & \#0,L,c_0 & \#1,L,c_0 \\
      		\#1 & \#1,R,r   & \#1,L,c_0 & \#0,L,c_1 \\
      		\9  & \9, L,c_1 & \9 ,R,h   & \#1,L,c_0 & \hphantom{\#1,L,C} \\
      		\bottomrule
    		\end{tabular}\\[2em]

    		$r$ ist der Anfangszustand.
    		\begin{enumerate}
    			\item Stelle diese Turingmaschine graphisch dar
    			\item Gib die Endkonfiguration für die Eingaben \#{100} und \#{111} an
    			\item Was macht die Turingmaschine?
    		\end{enumerate}
	\end{exampleblock}
\end{frame}

\begin{frame}{Turingmaschine}
  
  \begin{block}{Lösung}
    \centering \begin{tikzpicture}[shorten >=1pt,initial text=,node distance=2.25cm,auto,->,>=stealth,baseline=(B.base)]
              \node[state,initial]  (r)          {$r$};
              \node[state]          (c_1) [right of=r] {$c_1$};
              \node[state]          (c_0) [right of=c_1] {$c_0$};
              \node[state]          (h) [right of=c_0] {$h$};
              
              \draw[->] (r) to [loop above] node[above,align=center] {$\io{\#0}{\#0R}$\\$\io{\#1}{\#1R}$} (r);
              \draw[->] (c_1) to [loop above] node[above,align=center] {$\io{\#1}{\#0L}$} (c_1);
              \draw[->] (c_0) to [loop above] node[above,align=center] {$\io{\#0}{\#0L}$\\$\io{\#1}{\#1L}$} (c_0);

              \draw[->] (r) to node[above,align=center] {$\io{\9}{\9L}$} (c_1);
              \draw[->] (c_1) to node[above,align=center] {$\io{\#0}{\#1L}$\\$\io{\9}{\#1L}$} (c_0);
              \draw[->] (c_0) to node[above,align=center] {$\io{\9}{\9R}$} (h);

        \end{tikzpicture}
  \end{block}

\end{frame}

\begin{frame}{Turingmaschinen}
	\begin{block}{Lösung}
		\begin{enumerate}
			\item s. o.
			\item \begin{tabular}{ccccc}
        &h&&&\\
        \hline
        $\9$&$\#{1}$&$\#{0}$&$\#{1}$&$\9$\\
        \hline
      \end{tabular} \quad und \quad 
      \begin{tabular}{cccccc}
        &h&&&&\\
        \hline
        $\9$&$\#{1}$&$\#{0}$&$\#{0}$&$\#{0}$&$\9$\\
        \hline
      \end{tabular}
  %$\9 h \#{101} \9 $, $\9 h \#{1000} \9 $
			\item Die TM erhöht eine binär dargestellte Zahl um 1
		\end{enumerate}
	\end{block}
\end{frame}
\subsection{kekekek}
\begin{frame}{Komplexitätsmaße}
	\begin{block}{Def.: Zeitkomplexität $\ftime_T$ und $\fTime_t$}
		Für die Beurteilung des \textbf{Zeitbedarfs} definiert man zwei Funktionen \\
		$\ftime_T:A^+ \to \nN_+$ und $\fTime_T:\nN_+ \to \nN_+$ wie folgt:
		\begin{align*}
  			\ftime_T(w) &= \text{das $t$, für das $\Delta_t(c_0(w))$ Endkonfiguration ist} \\
  			\fTime_T(n)   &= \max \{\ftime_T(w) \mid w\in A^n\}
		\end{align*}
	\end{block}

	Also: $\fTime_T(n)$ maximale Anzahl Schritte, die bei einer Eingabe der Länge $n$ gemacht werden
\end{frame}

\begin{frame}{Komplexitätsmaße}
    \begin{block}{Def.: Raumkomplexität}
    	Für die Beurteilung des Speicherplatzbedarfs definiert man zwei
		Funktionen $\fspace_T(w):A^+ \to \nN_+$ und $\fSpace_T(n):\nN_+ \to \nN_+$ wie folgt:
			\begin{align*}
  			\fspace_T(w) &= \text{die Anzahl der Felder, die während der }\\
               			&\quad  \text{Berechnung für Eingabe $w$ benötigt werden}\\
  			\fSpace_T(n)   &= \max \{\fspace_T(w) \mid w\in A^n\} 
			\end{align*}
    \end{block}

    Also: $\fSpace_T(n)$ maximale Anzahl Felder, die bei einer Eingabe der Länge $n$ besucht werden\\[1em]
    Bemerkung:\\
    Ein Feld gilt als „benötigt“, wenn es zu Beginn ein Eingabesymbol enthält oder irgendwann vom Kopf der Turingmaschine besucht wird. 
\end{frame}

\begin{frame}{Komplexitätsmaße}
    \textbf{Zusammenhang zwischen Zeit- und Raumkomplexität:}
    \begin{itemize}
    	\item Polynomielle Zeitkomplexität:\\
    	es existiert ein Polynom $p(n)$, sodass $\fTime(n) \in O(p(n))$
    	\item Polynomielle Raumkomplexität:\\
    	es existiert ein Polynom $p(n)$, sodass $\fSpace(n) \in O(p(n))$
    	\pause
    	\item[]
    	\item Welcher Zusammenhang gilt?
    	\begin{itemize}
    		\item Polynomielle Laufzeit $\Rightarrow$ Polynomieller Platzbedarf?
    		\item Polynomieller Platzbedarf $\Rightarrow$ Polynomielle Laufzeit ?
    	\end{itemize}
    	\pause
    	\item[]
    	\item Es gilt:
    	\begin{itemize}
    		\item TM mit polynomieller Laufzeit hat auch nur polynomiellen Platzbedarf
    		\item Umkehrung gilt nicht!
    	\end{itemize}
    \end{itemize}
\end{frame}

\begin{frame}{Turingmaschine}
	\begin{exampleblock}{Aufgabe (WS 2015)}
		Konstruiere eine Turingmaschine $T$ mit Zuständen $A,B,C,D$ und Bandalphabet $X := \set{\#0, \#1, \blank}$, die für jede Eingabe $w \in X^+$ hält und am Ende das Wort $w'\in X^+$ auf dem Band steht, für das gilt:
		\begin{itemize}
			\item $|w|=|w'|$
			\item $\fNum_2(w')=\fNum_2(w)-1$, falls $\fNum_2(w)>0$
			\item $w' = \#0 \cdot ... \cdot \#0$ sonst
		\end{itemize}
		Wo bei der Endkonfiguration der Kopf steht, ist nicht wichtig.\\
		Zeige an Eingabe $w = 010$, dass die TM funktioniert, indem du alle Konfigurationen angibst, die deine TM durchläuft.\\
		Gib außerdem zwei Funktionen $f,g$ an, für die gilt: $\fTime_T(n) \notin O(f(n))$ und $\fTime_T(n)\notin \Omega(g(n))$
	\end{exampleblock}
\end{frame}

\begin{frame}{Komplexitätsklassen}
    \begin{block}{Def.: Komplexitätsklasse}
        Eine \textbf{Komplexitätsklasse} ist eine \textbf{Menge von Problemen}.\\
        Wir beschränken uns auf Entscheidungsprobleme, also auf formale Sprachen.
    \end{block}

    \begin{block}{Def.: $\Pclass$ und $\PSPACE$}
        \Pclass: Menge aller Entscheidungsprobleme, die von TMs entschieden werden können, deren Zeitkomplexität polynomiell ist\\[1ex]
        \PSPACE: Menge aller Entscheidungsprobleme, die von TMs entschieden werden können, deren Raumkomplexität polynomiell ist 
    \end{block}

    Es gilt: $\Pclass \subseteq \PSPACE$
\end{frame}
\subsection{kekekekek}

\begin{frame}{Entscheidbarkeit}
	\begin{block}{Def.: Turingmaschine als Akzeptor}
		Eine TM $T$ kann analog zu endlichen Automaten als Akzeptor für formale Sprachen genutzt werden. Es gilt dann:
		\begin{itemize}
			\item Teilmenge $F \subset Z$ \textbf{akzeptierende} Zustände
			\item $T$ akzeptiert Eingabewort $w$, wenn gilt:
			\begin{itemize}
				\item $T$ \textbf{hält} für die Eingabe $w$ \textbf{und}
				\item der Zustand der Endkonfiguration $\Delta_*(c_0(w))$ ist akzeptierend
			\end{itemize}
			\item $L(T)$ ist die Menge der von $T$ akzeptierten Wörter
		\end{itemize}
	\end{block}
\end{frame}

\begin{frame}{Entscheidbarkeit}
    \begin{block}{Def.: Entscheidbarkeit}
    	Eine Sprache $L \subseteq A^*$ heißt \textbf{entscheidbar}, wenn es eine TM gibt, die auf \textbf{allen} Eingaben stoppt und eine Eingabe $w$ genau dann akzeptiert, wenn $w\in L$ gilt.
    \end{block}

    \begin{exampleblock}{Aufgabe}
		Entwerft eine TM, die die Sprache $L = \set{w \in \set{\#0,\#1} | \text{Num}_2(w) \text{ ist gerade}}$ entscheidet.
	\end{exampleblock}
\pause
    \begin{block}{Def.: Semi-Entscheidbarkeit}
    	Eine Sprache $L \subseteq A^*$ heißt \textbf{semi-etscheidbar} (rekursiv aufzählbar), wenn es eine TM gibt, die die Eingaben $w$ akzeptiert, für die $w\in L$ gilt.\\
    	Für Eingaben $w$ mit $w \notin L$ ist das Verhalten nicht definiert, d.h. die TM stoppt entweder in einem nicht akzeptierenden Zustand oder sie läuft endlos.
    \end{block} 
\end{frame}

\begin{frame}{Unentscheidbarkeit}
    \begin{block}{Def.: Codierung von Turingmaschinen}
    	Ja, das gibt es, wen's interessiert $\rightarrow$ Vorlesungsfolien\\
    	Schreibe $T_w$ für die TM mit Codierung $w$.
    \end{block}
\pause
    \begin{block}{Def.: Halteproblem}
    	Das \textbf{Halteproblem} ist die formale Sprache
    	\[
			H = \{ w\in A^* \mid \text{$w$ ist eine TM-Codierung und $T_w(w)$ hält} \}
		\]

		Das \textbf{Halteproblem ist unentscheidbar}, d.h. es gibt keine TM, die $H$ entscheidet.
    \end{block}
\end{frame}
%%%%%%%%%% %%%%%%%%%%
%% Zusammenfassung
\section{}
%\subsection{Zusammenfassung}
	\begin{frame}{Was ihr jetzt kennen und können solltet\dots}
			\begin{itemize}
				\item Turingmaschinen
				\begin{itemize}
					\item konstruieren
					\item verstehen
				\end{itemize}
				\item Berechnungskomplexität
				\begin{itemize}
					\item Zeit- u. Raumkomplexität sowie versch. Komplexitätsklassen
					\item Diese bei TMs abschätzen
				\end{itemize}
				\item Entscheidbarkeit
				\begin{itemize}
					\item Die TM als Akzeptor
					\item Das Halteproblem ist unentscheidbar
				\end{itemize}
			\end{itemize}
	
	\end{frame}
%% Ausblick
%\subsection{Ausblick}
	\begin{frame}{Ausblick}
		\begin{itemize}
			\item Semesterferien *intense partying*
		\end{itemize}
	\end{frame}
%%%%%%%%%% %%%%%%%%%%
\section{}
\questionframe
\showmessage{Vielen Dank für Eure Aufmerksamkeit!\\ Viel Erfolg bei der Klausur und vlt bis nächstes Semester! :)}
\mode<handout>{\slideThanks}
\end{document}