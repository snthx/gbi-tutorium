% ===== Set up =================
\documentclass[%
  parskip,
  a4paper, %A4-Papier. Optional?
  fontsize=12pt, % Schriftgröße
  %version=last, % Neueste Version von KOMA-Skript verwenden
  %twoside % oneside  (optional?) vs twoside as described here: https://www.sharelatex.com/learn/Single_sided_and_double_sided_documents
]{scrartcl}

% ===== Deutsche Sprache ========
\usepackage[ngerman]{babel}
\usepackage[utf8]{inputenc}
\usepackage[T1]{fontenc}
% ===========================


% ===== Basic information ====
\newcommand\getAuthor{Stephan Bohr}
\newcommand\getTopic{GBI Tutorium}
\newcommand\getTitle{Feedback}
\newcommand\getDate{\today}
% ============================

% ===== PDF functionality =========
\usepackage[pdftex,%
	pdfauthor={\getAuthor},%
	pdftitle={\getTopic : \getTitle}%
		]{hyperref}
% ===========================

% ===== Basic packages ==========
% Check if needed!
\usepackage{booktabs}
\usepackage{tabularx}
\usepackage{framed}
% ===========================

% ===== Look =====
\usepackage[margin=2.5cm]{geometry} %layout
% ================

% ===== Headline / Footline ===
% https://esc-now.de/_/latex-individuelle-kopf--und-fusszeilen/?lang=en
\usepackage[headsepline,%footsepline
  ]{scrpage2}
\pagestyle{scrheadings}
\clearscrheadfoot
\ihead{\getTopic} % inner
\chead{\getTitle} % center
\ohead{\getDate} % outer
%\ifoot{Fußzeile innen} % inner
%\cfoot{Fußzeile mitte} % center
%\ofoot{\pagemark} % outer
% =============================

% ===== Custom packages ========

% ===========================

% === Basic document information ===
    \title{\getTopic : \getTitle}    
    \date{\getDate}
    \author{\getAuthor}
% ===========================

% ===========================
% ========= Document ========
% ===========================

\begin{document}
\newcommand{\punkte}{& $\circ$ & $\circ$ & $\circ$ & $\circ$ & $\circ$ \\}
\begin{tabularx}{\textwidth}{X|ccccc}
\hfill & \makebox[0pt]{\footnotesize$++$} & \makebox[0pt]{\footnotesize$+$} & & \makebox[0pt]{\footnotesize$-$} & \makebox[0pt]{\footnotesize$--$} \\
\midrule
Ich besuche dieses Tutorium gerne. \punkte
Ich fühle mich frei, Fragen und Anmerkungen zu äußern. \punkte
Fragen werden zufriedenstellend beantwortet. \punkte
Ich fühle mich in das Tutorium mit einbezogen. \punkte
Das Tutorium ist zu schnell (+) / zu langsam (-). \punkte
Die Aufgaben im Tutorium sind zu leicht (+) / zu schwer (-). \punkte
Der Tutor spricht klar und deutlich. \punkte
Das Gesprochene des Tutors ist inhaltlich verständlich. \punkte
Der Stoff wird anschaulich vermittelt. \punkte
Das Tutorium weckt mein Interesse an theoretischer Informatik. \punkte
Die Folien sind für das Tutorium geeignet. \punkte
Ich verwende die Folien zum Nachbereiten / Nachsehen. \punkte
Ich fühle mich durch das Tutorium gut auf die ÜB vorbereitet. \punkte
\small Die Übungsblätter werden zu ausführlich (+) / zu wenig (-) besprochen. \punkte
Die Korrektur der Übungsblätter ist hilfreich und angemessen. \punkte
\end{tabularx}

\small Gut gefallen hat mir
\begin{framed}
  \hfill\vspace{2.5cm}
\end{framed}

\small Nicht gefallen hat mir
\begin{framed}
  \hfill\vspace{2.5cm}
\end{framed}

\small Sonstiges: Ideen, Änderungswünsche, Tipps (z.B. zu Inhalt, Erklärweise, Vortragsstil)
\begin{framed}
  \hfill\vspace{2.5cm}
\end{framed}

Ich gebe dem Tutorium folgende \textbf{Gesamtnote}:

Vielen Dank für Deine Teilnahme! :)



\end{document}