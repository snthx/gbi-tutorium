% ===== handout mode =====
% Comment/uncomment this line to toggle handout mode
% \newcommand{\handout}{}

% Comment/uncoment this line to toogle Mortitz mode
% \newcommand{\Moritz}{}

% Comment/uncomment this line to toggle handout mode
% \newcommand{\handout}{}

% by Stephan

%% Moritz mode or Stephan mode
\ifdefined \Moritz

% This is a configuration file with private, tutor specific information.
% It is therefore excluded from the Git repository so changes in this file will not conflict in git commits.

% Copy this Template, rename to config.tex and add your information below.

\newcommand{\mymail}{moritz.laupichler@student.kit.edu} % Consider using your named student Mail address to keep your u-Account private.

\newcommand{\myname}{\href{mailto:\mymail}{Moritz Laupichler}}

\newcommand{\mytutnumber}{25}

\newcommand{\mytutinfos}{Dienstags, 5. Block (15:45-17:15 Uhr), SR -120}

\newcommand{\aboutMeFrame}{
	\begin{frame}{Euer Tutor}
		Name: \myname \\
		Alter: 21 Jahre \\
		Studiengang: Master Informatik, 1. Semester \\
		\vspace{1cm}
		\pause 
		\centering{Kontakt: \href{mailto:\mymail}{\mymail}}
	\end{frame}
} % Moritz mode
\else
\ifdefined \Alex

% This is a configuration file with private, tutor specific information.
% It is therefore excluded from the Git repository so changes in this file will not conflict in git commits.

% Copy this Template, rename to config.tex and add your information below.

\newcommand{\mymail}{alexander.klug@student.kit.edu} % Consider using your named student Mail address to keep your u-Account private.

\newcommand{\myname}{\href{mailto:\mymail}{Alexander Klug}}

\newcommand{\mytutnumber}{30}

\newcommand{\mytutinfos}{Mittwochs, 3. Block (11:30-13:00), SR -107}

\newcommand{\aboutMeFrame}{
	\begin{frame}{Euer Tutor}
		Name: \myname \\
		Alter: 19 Jahre \\
		Studiengang: Bachelor Informatik, 3. Semester \\
		\vspace{1cm}
		\pause 
		\centering{Kontakt: \href{mailto:\mymail}{\mymail}}
	\end{frame}
}

% Toggle Handout mode by including the following line before including style_tut
% and removing the % at the start (but do NOT remove it here, otherwise handout mode will always be on!)
% Please keep handout mode on in all commits!

% \newcommand{\handout}{} % Alex Mode
\else

% This is a configuration file with private, tutor specific information.
% It is therefore excluded from the Git repository so changes in this file will not conflict in git commits.

% Copy this Template, rename to config.tex and add your information below.

\newcommand{\mymail}{stephan.bohr@student.kit.edu} % Consider using your named student Mail address to keep your u-Account private.

\newcommand{\myname}{\href{mailto:\mymail}{Stephan Bohr}}

\newcommand{\mytutnumber}{25}

\newcommand{\mytutinfos}{Dienstags, 5. Block (15:45-17:15), SR -119}

\newcommand{\aboutMeFrame}{
	\begin{frame}{Euer Tutor}
		Name: \myname \\
		Alter: 20 Jahre \\
		Studiengang: Bachelor Informatik, 3. Semester \\
		\vspace{1cm}
		\pause 
		\centering{Kontakt: \href{mailto:\mymail}{\mymail}}
	\end{frame}
} % Stephan mode
\fi
\fi

%% Beamer-Klasse im korrekten Modus
\ifdefined \handout
\documentclass[handout]{beamer} % Handout mode
\else
\documentclass{beamer}
\fi
%\documentclass[18pt,parskip]{beamer}

%% SLIDE FORMAT

% use 'beamerthemekit' for standard 4:3 ratio
% for widescreen slides (16:9), use 'beamerthemekitwide'

\usepackage{../templates/KIT-slides/beamerthemekit}
%\usepackage{../templates/KIT-slides/beamerthemekitwide}

%% TITLE PICTURE

% if a custom picture is to be used on the title page, copy it into the 'logos'
% directory, in the line below, replace 'mypicture' with the 
% filename (without extension) and uncomment the following line
% (picture proportions: 63 : 20 for standard, 169 : 40 for wide
% *.eps format if you use latex+dvips+ps2pdf, 
% *.jpg/*.png/*.pdf if you use pdflatex)

\titleimage{../figures/titleimage/brain}

%% TITLE LOGO

% for a custom logo on the front page, copy your file into the 'logos'
% directory, insert the filename in the line below and uncomment it

%\titlelogo{mylogo}

% (*.eps format if you use latex+dvips+ps2pdf,
% *.jpg/*.png/*.pdf if you use pdflatex)

%% TikZ INTEGRATION

% use these packages for PCM symbols and UML classes
% \usepackage{templates/tikzkit}
% \usepackage{templates/tikzuml}

%\usepackage{tikz}
%\usetikzlibrary{matrix}
%\usetikzlibrary{arrows.meta}
%\usetikzlibrary{automata}
%\usetikzlibrary{tikzmark}

%%%%%%%%%%%%%%%%%%%%%%%%%
% Libertine font (Original GBI font)
\usepackage{libertine}
%\renewcommand*\familydefault{\sfdefault}  %% Only if the base font of the document is to be sans serif

%% Schönere Schriften
\usepackage[TS1,T1]{fontenc}

%% Deutsche Silbentrennung und Beschriftungen
\usepackage[ngerman]{babel}

%% UTF-8-Encoding
\usepackage[utf8]{inputenc}

%% Bibliotheken für viele mathematische Symbole
\usepackage{amsmath, amsfonts, amssymb}

%% Anzeigetiefe für Inhaltsverzeichnis: 1 Stufe
\setcounter{tocdepth}{1}

%% Hyperlinks
\usepackage{hyperref}
% I don't know why, but this works and only includes sections and NOT subsections in the pdf-bookmarks.
\hypersetup{bookmarksdepth=subsection}

%% remove navigation symbols
\setbeamertemplate{navigation symbols}{}

%% switch between "ngerman" and "english" for German/English style date and logos
\selectlanguage{ngerman}

%% for invisible pause texts instead of dimming
\setbeamercovered{invisible}

%%%%%%%%%%%% Shortcuts %%%%%%%%%%%%%
\newcommand{\nM}{\mathbb{M}}
\newcommand{\nR}{\mathbb{R}}
\newcommand{\nN}{\mathbb{N}}
\newcommand{\nZ}{\mathbb{Z}}
\newcommand{\nQ}{\mathbb{Q}}
\newcommand{\nB}{\mathbb{B}}
\newcommand{\nC}{\mathbb{C}}
\newcommand{\nK}{\mathbb{K}}
\newcommand{\nF}{\mathbb{F}}
\newcommand{\nG}{\mathbb{G}}
\newcommand{\nullel}{\mathcal{O}}
\newcommand{\einsel}{\mathds{1}}
\newcommand{\nP}{\mathbb{P}}
\newcommand{\Pot}{\mathcal{P}}
\renewcommand{\O}{\text{O}}

\newcommand{\set}[1]{\{ #1 \}}
\newcommand{\setc}[2]{\set{#1 \mid #2}}
\newcommand{\setC}[2]{\set{#1 \mid \text{ #2 }}}

\newcommand{\setsize}[1]{\; \mid #1 \mid \; }

\newcommand{\q}[1]{\textquotedblleft #1\textquotedblright}

%%%%%%%%%%%% INHALT %%%%%%%%%%%%%%%%

%% Wochennummer
%\newcounter{weeknum}

%% Titelinformationen
%\title[GBI Tutorium, Woche \theweeknum]{Grundbegriffe der Informatik \\ Tutorium \mytutnumber}
%\subtitle{Termin \theweeknum \ | \mydate \\ \myname}
\author[\myname]{\myname}
\institute{Fakultät für Informatik}
%\date{\mydate}

%% Titel einfügen
\newcommand{\titleframe}{\frame{\titlepage}\addtocounter{framenumber}{-1}}


%% Alles starten mit \starttut{X}
%\newcommand{\starttut}[1]{\setcounter{weeknum}{#1}\titleframe\frame{\frametitle{Inhalt}\tableofcontents} \AtBeginSection[]{%
%\begin{frame}
%	\tableofcontents[currentsection]
%\end{frame}\addtocounter{framenumber}{-1}}}


%\newcommand{\framePrevEpisode}{
%	\begin{frame}
%		\centering
%		\textbf{In the previous episode of GBI...}
%	\end{frame}
%}

%% Roadmap frame
%table of contents
\newcommand{\roadmap}{
	\frame{\frametitle{Roadmap}\tableofcontents}}

 \AtBeginSection[]{%
\begin{frame}
	\frametitle{Roadmap}
	\tableofcontents[currentsection]
\end{frame}%\addtocounter{framenumber}{-1}
}


%% ShowMessage frame
\newcommand{\showmessage}[1]{\frame{\frametitle{\phantom{1em}}\centering\textbf{#1}}}

%% Fragen
%% Lastframe
\newcommand{\questionframe}{\showmessage{Fragen?}}

%% Lastframe
\newcommand{\lastframe}{\showmessage{Vielen Dank für Eure Aufmerksamkeit! \\Bis nächste Woche :)}}

%% Thanks frame
\newcommand{\slideThanks}{
	\begin{frame}
		\frametitle{Credits}
		\begin{block}{}
			An der Erstellung des Foliensatzes haben mitgewirkt:\\[1em]
			\ifdefined \Moritz
			Stephan Bohr \\
			Alexander Klug \\
			\else
			\ifdefined \Alex
			Stephan Bohr \\
			Moritz Laupichler \\
			\else
			Moritz Laupichler \\
			Alexander Klug \\
			\fi
			\fi
			Katharina Wurz \\
			Thassilo Helmold \\
			Philipp Basler \\
			Nils Braun \\
			Dominik Doerner \\
			Ou Yue \\
		\end{block}
	\end{frame}
}

%% Verbatim
%\usepackage{moreverb}



\title[Formale Sprachen, Aussagenlogik]{2. Tutorium\\ Formale Sprachen, Aussagenlogik}
\Stephan{\title[Relationen, Formale Sprachen]{2. Tutorium\\ Relationen, Formale Sprachen}}
\subtitle{Grundbegriffe der Informatik, Tutorium \hashtag\mytutnumber}
\date{\today}

\begin{document}
\titleframe
\roadmap

\section{Organisatorisches}
\subsection{Tipps}
\begin{frame}{Allgemeine Tipps}
    \begin{itemize}[<+->]
    	\item Jedes abgegebene Blatt: \emph{Kopfzeile} mit \\{\centering``GBI ÜB 1, \quad Tut. \hashtag\mytutnumber\ (Tutorname), \quad Max Mustermann, 1234567''}
    	\item Aufgabenblatt muss \emph{nicht ausgedruckt} werden
		\item Kein Bleistift, kein roter oder grüner Stift
		\item Achtet auf die \emph{Aufgabenstellung}: Unterschied  ``Gib an'', ``Beweise'' oder ``Begründe''
		\item Bei Beweis \alert{NIEMALS} Beispiel, Widerlegen oft einfaches Gegenbeispiel
		\item Gebt nur \emph{eine Version} an, sonst bewerte ich die schlechtere
    	\end{itemize}
\end{frame}

\begin{frame}{Allgemeine Tipps}
    \begin{itemize}[<+->]
    	\item \emph{Strukturiert vorgehen!} Schreibt ``Beh.:'', ``\zz'', ``Ann.:'',\\
    	setzt ``q.e.d oder $\qedsymbol$'' ans Ende eines Beweises
    	\item \emph{Führt den Leser} bei Beweisen, macht klar, was ihr meint, was ihr in einem Schritt tut
    	\item Formalisiert so viel wie nötig, aber nicht mehr
    	\item Neue Definitionen mit \emph{$:=$}, z.B. $A_n := c^2$
    	\item \emph{Funktionen vollständig} angeben: $f \from \nN_0 \to \nR, x \mapsto x^2$ o.ä.
    \end{itemize}
\end{frame}
\Moritz{
	
\begin{frame}{Zum ersten Übungsblatt}
	
	\textbf{Aufg. 1:}
	\begin{itemize}
		\item Denkweise bei Beweis von Implikation $A \Rightarrow B$ \\(\enquote{Es gilt A, \textbf{wenn} B gilt}): \begin{itemize}
			\item Nehme an, dass $A$ gilt
			\item Dann folgere $B$
		\end{itemize}
		\item Denkweise bei Beweis von Äquivalenz $A \Leftrightarrow B$ \\(\enquote{Es gilt A \textbf{genau dann, wenn} B gilt}): \begin{itemize}
			\item Zeige $A \Rightarrow B$ und
			\item Zeige $B \Rightarrow A$
		\end{itemize}
		\item Immer nur die Informationen verwenden, die bekannt sind aus \begin{itemize}
			\item Aufgabenstellung
			\item Annahmen (z.B. bei Implikation)
		\end{itemize}
		\item Speziell durften hier die Mengen $A,B,C$ nicht festgelegt werden, da die Aufgabenstellung das nicht hergibt
		\item Relation auf Mengen gegeben, z.B. $B \subseteq C$ in Annahme $\Rightarrow$ Informationen über \textit{Elemente der Mengen} bekannt!
	\end{itemize}

\end{frame}

}

\Moritz{
\section{Relationen}
\input{../topics/mengen-relationen-abbildungen/relationen1.tex}
\subsection{Relationen und Abbildungen}
\begin{frame}{Relationen und Abbildungen}
	\begin{block}{Def.: linkstotal}
		Eine Relation $R \subseteq A \times B$ heißt \textbf{linkstotal}, wenn es für jedes Element $a \in A$ ein zugehöriges Element $b \in B$ gibt, mit $$(a,b) \in R$$
		Sprich: \enquote{Jedes Element aus A hat mind. einen Partner in B}
	\end{block}
	\pause
	\begin{block}{Def.: rechtseindeutig}
		Eine Relation $R \subseteq A \times B$ heißt \textbf{rechtseindeutig}, wenn für alle Elemente $a \in A$ und $b_1, b_2 \in B$ gilt: $$\text{Aus }(a,b_1) \in R \text{ und } (a,b_2) \in R \text{ folgt }  b_1 = b_2$$
		Sprich: \enquote{Jedes Element aus A hat höchstens einen Partner in B}
	\end{block}
\end{frame}

\begin{frame}{Relationen und Abbildungen}
	\begin{block}{Def.: Abbildung/Funktion}
		Ist eine Relation $f \subseteq A \times B$ rechtseindeutig und linkstotal, so nennt man sie \textbf{Funktion} oder \textbf{Abbildung}. 
		Man schreibt
		\[
			f : A \to B, a \mapsto b %\, (\text{oder } f(a) = b)
		\]
		\pause
		Dann ist:
		\begin{itemize}
			\item \(A\) der \textbf{Definitionsbereich}
			\item \(B\) der \textbf{Zielbereich}
			\item \(f(A) := \setc{f(a)}{a \in A}\) der \textbf{Bildbereich}
		\end{itemize}
		Für \(x \in A\), \(y \in B\) mit \(f(x)=y\) heißt \(y\) \textbf{Bild} von \(x\) und \(x\) \textbf{Urbild} von \(y\).
	\end{block}
\end{frame}

\begin{frame}{Relationen und Abbildungen}
	\begin{alertblock}{Achtung!}
		Funktionen immer \textbf{vollständig} angeben, also Definitionsbereich, Zielbereich sowie Abbildungsvorschrift. \\
		Auf die unterschiedlichen Pfeile achten (zwischen Definitions- und Zielbereich kein Strich am linken Ende des Pfeils)!
	\end{alertblock}
\end{frame}

\begin{frame}{}%Relationen und Abbildungen}
	\begin{block}{Def.: linkseindeutig}
		Eine Relation $R \subseteq A \times B$ heißt \textbf{linkseindeutig}, wenn für alle Elemente $b \in B$ und $a_1, a_2 \in A$ gilt: $$\text{Aus } (a_1,b) \in R \text{ und } (a_2,b) \in R \text{ folgt } a_1 = a_2$$
		Sprich: \enquote{Jedes Element aus B hat höchstens einen Partner in A}\\
		Eine linkseindeutige Funktion heißt \textbf{injektiv}.
	\end{block}
	\pause
	\begin{block}{Def.: rechtstotal}
		Eine Relation $R \subseteq A \times B$ heißt \textbf{rechtstotal}, wenn es für jedes Element $b \in B$ ein zugehöriges Element $a \in A$ gibt, mit $$(a,b) \in R$$
		Sprich: \enquote{Jedes Element aus B hat mind. einen Partner in A}\\
		Eine rechtstotale Funktion heißt \textbf{surjektiv}.
	\end{block}
\end{frame}
\begin{frame}{Relationen und Abbildungen}
	\begin{block}{Def.: bijektiv}
		Eine Funktion heißt \textbf{bijektiv}, wenn sie linkseindeutig und rechtstotal ist.
	\end{block}
\end{frame}
\begin{frame}{Relationen und Abbildungen}
	\begin{exampleblock}{Beispiel \hashtag1}
		\begin{minipage}{0.5\textwidth}
			\begin{tikzpicture}
			[scale=0.6,auto=left,every node/.style={circle,fill=kit-blue30}]
			\node (a1) at (0,10)  {$a_1$};
			\node (a2) at (0,8)  {$a_2$};
			\node (a3) at (0,6)  {$a_3$};
			\node (b1) at (2,10)  {$b_1$};
			\node (b2) at (2,8)  {$b_2$};
			\node (b3) at (2,6)  {$b_3$};
			
			
			\foreach \from/\to in {b1/a2, a1/b1, a3/b3}
			\draw (\from) -- (\to);
			\end{tikzpicture}
		\end{minipage} \hfill
		\begin{minipage}{0.45\textwidth}
			\raggedright
			% \begin{align*}
			% Linkstotal&: \only<2->{Ja} \\
			% Rechtseindeutig&: \only<3->{Ja} \\
			% Rechtstotal bzw. surjektiv&: \only<4->{Nein} \\
			% Linkseindeutig bzw. injektiv&: \only<5->{Nein} \\
			% \end{align*}
			\begin{tabular}{rl}
			Linkstotal: & \only<2->{\textcolor{kit-green100}{Ja}} \\
			Rechtseindeutig: & \only<3->{\textcolor{kit-green100}{Ja}} \\ 
			Rechtstotal: & \only<4->{\textcolor{kit-red100}{Nein}} \\
			Linkseindeutig: & \only<5->{\textcolor{kit-red100}{Nein}} \\
			\end{tabular}
		\end{minipage}
	\end{exampleblock}
\end{frame}
\begin{frame}{Relationen und Abbildungen}
	\begin{exampleblock}{Beispiel \hashtag2}
		\begin{minipage}{0.5\textwidth}
			\begin{tikzpicture}
			[scale=0.6,auto=left,every node/.style={circle,fill=kit-blue30}]
			\node (a1) at (0,10)  {$a_1$};
			\node (a2) at (0,8)  {$a_2$};
			% \node (a3) at (0,6)  {$a_3$};
			\node (b1) at (2,10)  {$b_1$};
			\node (b2) at (2,8)  {$b_2$};
			\node (b3) at (2,6)  {$b_3$};
			
			
			\foreach \from/\to in {a1/b2, a2/b3, a2/b1}
			\draw (\from) -- (\to);
			\end{tikzpicture}
		\end{minipage} \hfill
		\begin{minipage}{0.45\textwidth}
			\raggedright
			% \begin{align*}
			% Linkstotal&: \only<2->{Ja} \\
			% Rechtseindeutig&: \only<3->{Nein} \\
			% Rechtstotal bzw. surjektiv&: \only<4->{Nein} \\
			% Linkseindeutig bzw. injektiv&: \only<5->{Nein} \\
			% \end{align*}
			\begin{tabular}{rl}
			Linkstotal: & \only<2->{\textcolor{kit-green100}{Ja}} \\
			Rechtseindeutig: & \only<3->{\textcolor{kit-red100}{Nein}}\\ 
			Rechtstotal: & \only<4->{\textcolor{kit-green100}{Ja}} \\
			Linkseindeutig: & \only<5->{\textcolor{kit-green100}{Ja}} \\
			Bijektive Abbildung: & \only<6->{\textcolor{kit-red100}{Nein}} \\
			\only<7->{Abbildung: }& \only<7->{\textcolor{kit-red100}{Nein}}
			\end{tabular}

		\end{minipage}
	\end{exampleblock}
\end{frame}
%\subsection{Aufgabe}
\begin{frame}{Aufgabe}
\begin{figure}[h!]
		\centering
		\includegraphics[width=\textwidth]{../topics/mengen-relationen-abbildungen/1.png} 
	\end{figure}     
\end{frame}

\begin{frame}{Aufgabe}
\begin{figure}[h!]
		\centering
		\includegraphics[width=\textwidth]{../topics/mengen-relationen-abbildungen/2.png} 
	\end{figure}  
	\begin{figure}[h!]
		\centering
		\includegraphics[width=\textwidth]{../topics/mengen-relationen-abbildungen/3.png} 
	\end{figure}   
\end{frame}
\subsection{Binäre Operationen}

\begin{frame}{Binäre Operation}
	\begin{block}{Def.: Binäre Operation}
		Eine \textbf{binäre Operation auf einer Menge $M$} ist eine Abbildung $f: M \times M \to M$. Meist wird ein Operationssymbol eingeführt, um die binäre Operation in \textit{Infix}-Schreibweise zu verwenden. Also z.B.
		\medskip
		\[f(x,y) =: x \diamond y \text{ für } x,y \in M\]
	\end{block}

	\begin{exampleblock}{Beispiel}
		Aus der Mathematik sind viele binäre Operationen bekannt, z.B.: \begin{align*}
			&f_{add}: \nN_0^2 \to \nN_0, \quad (x,y) \mapsto x+y \\
			&f_{sub}: \nZ^2 \to \nZ, \quad (x,y) \mapsto x-y \\
			&f_{mult}: \nN_0^2 \to \nN_0, \quad (x,y) \mapsto x \cdot y \\
			&f_{div}: \nQ^2 \to \nQ, \quad (x,y) \mapsto x / y \\
		\end{align*}
	\end{exampleblock}
	
\end{frame}

\begin{frame}{Binäre Operationen}
	\begin{block}{Def.: Kommutative Binäre Operation}
		Eine binäre Operation $\diamond:M \times M \to M$  heißt \textbf{kommutativ}, wenn für alle Elemente $x,y \in M$ gilt: 
		\medskip
		\[f(x,y) = f(y,x) \; (\text{bzw. } x \diamond y = y \diamond x)\]
		
		\medskip
		Anschaulich: \enquote{Die Operanden dürfen vertauscht werden, ohne das Ergebnis zu beeinflussen}
	\end{block}

	\begin{exampleblock}{Beispiele}
		\begin{itemize}
			\item Addition ist kommutativ $x + y = y + x$
			\item Multiplikation ist kommutativ $x \cdot y = y \cdot x$
			\item Subtraktion ist nicht kommutativ $ x-y \ne y-x$
			\item $f_{Abstand}: \nZ^2 \to \nZ, \; (x,y) \mapsto |x-y|$ ist kommutativ!
		\end{itemize}
	\end{exampleblock}
\end{frame}

\begin{frame}{Binäre Operationen}
	\begin{block}{Def.: Assoziative Binäre Operation}
		Eine binäre Operation $\diamond:M \times M \to M$  heißt \textbf{assoziativ}, wenn für alle Elemente $x,y,z \in M$ gilt: 
		\medskip
		\[f(x,f(y,z)) = f(f(x,y),z) \; (\text{bzw. } x \diamond (y \diamond z) = (x \diamond y) \diamond z )\]

		\medskip
		Anschaulich: \enquote{Klammern um Operanden dürfen gesetzt werden, wie man will (oder weggelassen werden!), ohne das Ergebnis zu beeinflussen}
	\end{block}

	\begin{exampleblock}{Beispiele}
		\begin{itemize}
			\item Addition ist assoziativ $x + (y + z) = (x + y) + z = x + y + z$
			\item Subtraktion ist nicht assoziativ $x - (y - z) \ne (x - y) - z$ 
			\item $f_{Abstand}: \nZ^2 \to \nZ, \; (x,y) \mapsto |x-y|$ ist nicht assoziativ
		\end{itemize}
	\end{exampleblock}
\end{frame}
}

\section{Formale Sprachen}
\subsection{Alphabete, Wörter, Sprachen}
\begin{frame}{Wdh.: Alphabete, Wörter, Sprachen}
	\begin{block}{Def.: Alphabet}
		Ein \textbf{Alphabet} ist eine endliche, nichtleere Menge von Zeichen.
	\end{block}
	\pause
	\begin{block}{Def.: Wort}
		Ein \textbf{Wort} über einem Alphabet A ist eine Folge von Zeichen aus A.
	\end{block}
	\pause
	\begin{block}{Def.: leeres Wort}
		Das \textbf{leere Wort} \(\varepsilon\) ist das Wort, das aus null Zeichen besteht, d.h. es hat die Länge 0.
	\end{block}
\end{frame}

\begin{frame}{Wdh.: Alphabete, Wörter, Sprachen}
	\begin{block}{Def.: \(A^{n}\)}
	\pause
		\textbf{\(A^{n}\)} ist die Menge aller Wörter der Länge n über dem Alphabet A.
	\end{block}
	\pause
	\begin{block}{Def.: \(A^{*}\)}
	\pause
		\textbf{\(A^{*}\)} ist die Menge aller Wörter beliebiger Länge über dem Alphabet A.		
		\[
			A^{*}= \bigcup_{i=0}^{\infty} A^{i}
		\]
	\end{block}
	\pause
	\begin{exampleblock}{Beispiele}
		\(A^{0}=\set{\varepsilon}, \quad \set{a,b}^{*} = \set{\varepsilon, a, b, aa, ab, ba, bb, aaa, ...}\)
	\end{exampleblock}
\end{frame}

\begin{frame}{Wdh.: Alphabete, Wörter, Sprachen}
	\begin{block}{Def.: Konkatenation}
		Die \textbf{Konkatenation} ist die Operation der Verknüpfung von Wörtern. Jedes Wort kann als Konkatenation seiner Zeichen dargestellt werden, z.B. \(GBI = G \cdot B \cdot I\).
	\end{block}
	\pause
	\begin{block}{Induktive Def.: Potenz von Wörtern}
		Die n-te \textbf{Potenz} eines Wortes w ist induktiv definiert durch:
		\begin{align*}
				w^{0} &= \varepsilon \hphantom{000000000000}\\
				\text{Für jedes } n \in \nN_{0}: \; w^{n+1} & = w^{n} \cdot w
		\end{align*}
	\end{block}
	\pause
	\begin{exampleblock}{Aufgabe}
	\begin{itemize}
		\item Ist die Konkatenation assoziativ?\pause
		\item Ist die Konkatenation kommutativ?\pause
		\item \(a\varepsilon\varepsilon\varepsilon\varepsilon\varepsilon\varepsilon b\varepsilon\varepsilon\varepsilon\varepsilon = \, ?\)
	\end{itemize}		
	\end{exampleblock}
\end{frame}

\begin{frame}{Wdh.: Alphabete, Wörter, Sprachen}
	\begin{block}{Def.: Formale Sprache}
		Eine \textbf{formale Sprache} L über einem Alphabet A ist eine Teilmenge der Wörter über A, also \(L \subseteq A^{*}\).
	\end{block}

	\begin{exampleblock}{Aufgabe}
		Gib die Sprache L über dem Alphabet \(\set{a,b}\) an, die alle Wörter enthält, in denen die Zeichenfolge \q{ab} nicht vorkommt.\\
		\pause
		\textbf{Lösung:} \(L = \setc{w_{1}w_{2}}{w_{1} \in \set{b}^{*} \text{ und } w_{2} \in \set{a}^{*}}\)
	\end{exampleblock}
\end{frame}


% \subsection{Postsches Korrespondenzproblem}
\begin{frame}{Postsches Korrespondenzproblem}
	
	\begin{itemize}
		\item Übungsblatt 2 beschäftigt sich mit dem Postschen Korrespondenzproblem (PKP)
		\item Die Definition auf dem ÜB ist vollständig, aber etwas kompliziert
		\item Das PKP kann man sich allerdings schön bildlich vorstellen
		\item Wir gehen kleinschrittig durch die Definition des PKP im ÜB und verstehen sie
	\end{itemize}

\end{frame}

\begin{frame}{Postsches Korrespondenzproblem}
	
	\textbf{Definition von P}
	\begin{itemize}
		\item Sei $A=\set{a,b}$ ein Alphabet.
		\item Definiere die Menge aller Paare von Wörtern über A mit $P = A^\ast \times A^\ast$
	\end{itemize}

	\pause

	\begin{exampleblock}{Beispiele für Elemente in $P$}
		$(aa, bb), (a, b), (ba, \varepsilon), (bab, aaaaaaab), (\varepsilon, \varepsilon) \in P$
	\end{exampleblock}

	Bildlich schreiben wir die Wörter als ``Dominosteine'' in einem Paar $(v,w) \in P$ übereinander ($v$ oben, $w$ unten). Z.B. für $(aab, ba)$:\\[1em]

	\texttt{
	\begin{tabular}{|c|c|c|}
		\hline
		a & a & b \\
		\hline
		b & a & \\
		\hline
	\end{tabular}
	}
\end{frame}
	
\begin{frame}{Postsches Korrespondenzproblem}

	\textbf{Definition von $\diamond$}

	\begin{itemize}
		\item Die Abbildung $\diamond: P \times P \to P$ ist definiert als \[ (t_1, b_1) \diamond (t_2, b_2) = (t_1 t_2, b_1 b_2) \]
		\item $\diamond$ hängt also für zwei Wortpaare jeweils das erste und zweite Wort aneinander, um ein neues Wortpaar zu erzeugen
	\end{itemize}

	\pause

	\begin{exampleblock}{Beispiele für $\diamond$}
		\begin{align*}
			(a,b) \diamond (a,b) &= (aa, bb) \\
			(ab,a) \diamond (a,ba) &= (aba, aba) \\
			(\varepsilon, aba) \diamond (bab, \varepsilon) &= (bab, aba)
		\end{align*}
	\end{exampleblock}

	Als ``Dominosteine'' für $(ab,a) \diamond (a,ba)$:\\[1em]

	\begin{columns}
		\begin{column}{.2\textwidth}
			\texttt{
	\begin{tabular}{|c|c|}
		\hline
		a & b \\
		\hline
		a & \\
		\hline
	\end{tabular}
	}
		\end{column}

		\begin{column}{.05\textwidth}
			$\diamond$
		\end{column}
	
		\begin{column}{.1\textwidth}
			\texttt{
	\begin{tabular}{|c|c|}
		\hline
		a &  \\
		\hline
		b & a \\
		\hline
	\end{tabular}
	}
		\end{column}

		\begin{column}{.05\textwidth}
			$=$
		\end{column}

		\begin{column}{.2\textwidth}
			\texttt{
	\begin{tabular}{|c|c|c|}
		\hline
		a & b & a \\
		\hline
		a & b & a \\
		\hline
	\end{tabular}
	}
		\end{column}
	\end{columns}

\end{frame}

\begin{frame}{Postsches Korrespondenzproblem}
	
	\begin{itemize}
		\item Problem selbst ist jetzt: \begin{itemize}
			\item Kann man aus gegebener Liste von Paaren $D = (d_1, d_2, \dots, d_n) \in P^{n}$ so Paare auswählen, dass sie aneinandergehängt im ersten und zweiten Element das gleiche Wort ergeben?
		\item Man wählt also eine Folge von Paaren aus, die aneinandergehangen werden sollen, indem man eine Folge von Indizes in $D$ definiert
		\end{itemize}
	\end{itemize}

	\pause

	\begin{exampleblock}{Beispiel}
	\begin{itemize}
		\item Sei $D = ((ab,a), (a, ba), (ba, b))$
		\item Die Indexfolge $(2,3,3)$ heißt dann, dass $(a, ba)$ und $(ba,b)$ und $(ba,b)$ aneinander gehangen werden: \[ (a,ba) \diamond (ba,b) \diamond (ba,b) = (ababa, babb) \]
		\item Wegen $ababa \ne babb$ ist also die Indexfolge $(2,3,3)$ keine Lösung des PKP für $D$
	\end{itemize}
	\pause
	\textbf{Frage:} Welche Indexfolge würde das PKP für $D$ lösen?
		
	\end{exampleblock}

\end{frame}

\begin{frame}{Postsches Korrespondenzproblem}
	
	Für längere Folgen von Wortpaaren ist dann die Darstellung übereinander gut, weil man gut sieht, welche Stelle im oberen Wort mit welcher Stelle im unteren Wort zusammenpassen muss. Z.B. sieht man schnell das $(2,3,3)$ keine Lösung von $D$ sein kann:

	\begin{columns}
		\begin{column}{.05\textwidth}
		\centering
			\texttt{
	\begin{tabular}{|c|c|}
		\hline
		a & \\
		\hline
		b & a \\
		\hline
	\end{tabular}
	}
		\end{column}

		\begin{column}{.02\textwidth}
		\centering
			$\diamond$
		\end{column}
	
		\begin{column}{.05\textwidth}
		\centering
			\texttt{
	\begin{tabular}{|c|c|}
		\hline
		b & a \\
		\hline
		b & \\
		\hline
	\end{tabular}
	}
		\end{column}

	\begin{column}{.02\textwidth}
	\centering
			$\diamond$
		\end{column}
	
		\begin{column}{.05\textwidth}
		\centering
			\texttt{
	\begin{tabular}{|c|c|}
		\hline
		b & a \\
		\hline
		b & \\
		\hline
	\end{tabular}
	}
		\end{column}

		\begin{column}{.02\textwidth}
		\centering
			$=$
		\end{column}

		\begin{column}{.25\textwidth}
		\centering
			\texttt{
	\begin{tabular}{|c|c|c|c|c|}
		\hline
		a & b & a & b & a \\
		\hline
		b & a & b & b & \\
		\hline
	\end{tabular}
	}
		\end{column}
	\end{columns}

\end{frame}

\Moritz{
\section{Tipps für 2. ÜB}
	\subsection{ÜB-Tipps}
\begin{frame}{Hinweise zum Übungsblatt}
	\begin{block}{Def.: Notwendige Bedingung}
		Eine \textbf{notwendige Bedingung} zu einer Aussage $K$ ist eine Aussage $A$, für die gilt:

		{\center $K$ kann nur erfüllt sein, wenn die Bedingung $A$ gilt.}
		\\[0.5em]
		Das heißt \textit{nicht}, dass $K$ erfüllt sein muss.
	\end{block}

	\begin{block}{Def.: Hinreichende Bedingung}
		Eine \textbf{hinreichende Bedingung} zu einer Aussage $K$ ist eine Aussage $B$, für die gilt:
		\center Wenn die Bedingung $B$ erfüllt ist, so ist garantiert auch $K$ erfüllt.
	\end{block}
\end{frame}
}

\section{Aussagenlogik}
\subsection{Grundlagen: Aussage, Syntax}

\begin{frame}{Grundlagen}
	\begin{block}{Def.: Aussage}
		\textbf{Aussagen} sind Sätze, die "`objektiv"' wahr oder falsch sind. Man spricht auch von der Zweiwertigkeit der Aussagenlogik.
	\end{block}
	\pause
	\begin{block}{Def.: Aussagenvariable}
		Eine \textbf{Aussagenvariable} steht für eine (elementare) Aussage. Sie kann entweder \emph{wahr} oder \emph{falsch} sein.
	\end{block}
\end{frame}

\begin{frame}{Grundlagen}
	\begin{block}{Aussagenlogische Konnektive}
		Seien \(G\) und \(H\) Aussagevariablen. Dann kann man wie folgt größere Formeln konstruieren:
		\pause
		\begin{itemize}[<+->]
			\item \(	\bleftBr 	\bnot  G \brightBr 	\) heißt \enquote{\textcolor{blue}{nicht} \(G\)}
			\item \(	\bleftBr 	G \bund  H		\brightBr 	\)	heißt \enquote{\(G\) \textcolor{blue}{und} \(H\)}
			\item \(	\bleftBr 	G \boder H		\brightBr 	\)	heißt \enquote{\(G\) \textcolor{blue}{oder} \(H\)}
			\item \(	\bleftBr 	G  \bimp H		\brightBr 	\)	heißt \enquote{\(G\) \textcolor{blue}{impliziert} \(H\)}
			\item \(	\bleftBr 	G  \bgdw H		\brightBr := \bleftBr	G  \bimp H		\brightBr  \bund  \bleftBr 	H  \bimp G		\brightBr	\)	heißt \\ \enquote{\(G\) \textcolor{blue}{impliziert} \(H\) und \(H\) \textcolor{blue}{impliziert} \(G\)}
		\end{itemize}
	\end{block}
	\pause
	\begin{alertblock}{Beachte}
		Hinter dieser Syntax steckt natürlich viel Formalismus. Die Vorlesung kennt \(\bgdw\) nur als Abkürzung, nicht als aussagenlogisches Symbol.
	\end{alertblock}
\end{frame}

\begin{frame}{Grundlagen}
	\begin{block}{Bindungsstärken}
		\begin{itemize}
			\item $\bnot$ bindet am stärksten
  			\item $\bund$ bindet am zweitstärksten
			\item $\boder$ bindet am drittstärksten
			\item $\bimp$ bindet am viertstärksten
			\item $\bgdw$ bindet am schwächsten
		\end{itemize}
	\end{block}
	\pause
	\begin{exampleblock}{Aufgabe}
		Finde die Klammern für \\
		\(	P	\bimp	Q \bund \bnot R \bgdw Q \boder R	\)\\[1ex]
		\pause
		\(	\bleftBr \bleftBr  P	\bimp	\bleftBr  Q \bund \bleftBr  \bnot R \brightBr \brightBr \brightBr \bgdw \bleftBr Q \boder R	\brightBr \brightBr \)
	\end{exampleblock}
\end{frame}

\subsection{Semantik}

\begin{frame}{Semantik aussagenlogischer Formeln}
	\begin{block}{Def.: Boolsche Funktionen}
		Für ``Wahrheitswerte'' \( \BB = \set{\mathbf{w}, \mathbf{f}} \)
		 ist eine \textbf{boolsche Funktion} eine Funktion
		 $f: \BB^n \to \BB$
	\end{block}
	\pause
	\begin{exampleblock}{Beispiele}
		\begin{center}

		  	\begin{tabular}{cc|cccc}
		    \toprule
		    $x_1$ & $x_2$ & $\bfnot{x_1}$ & $\bfand{x_1}{x_2}$ & $\bfor{x_1}{x_2}$ & $\bfimp{x_1}{x_2}$ \\
		    \midrule
		    $f$ & $f$ & $w$ & $f$ & $f$ & $w$ \\
		    $f$ & $w$ & $w$ & $f$ & $w$ & $w$ \\
		    $w$ & $f$ & $f$ & $f$ & $w$ & $f$ \\
		    $w$ & $w$ & $f$ & $w$ & $w$ & $w$ \\
		    \bottomrule
  			\end{tabular}
		\end{center}
	\end{exampleblock}
\end{frame}

\begin{frame}{Semantik aussagenlogischer Formeln}
	\begin{block}{Def.: Interpretation}
		Für eine Menge \(V\) von Aussagevariablen ist eine \textbf{Interpretation} ist eine Abbildung \(I: V \to \nB\), wobei \( \BB = \set{w, f} \). 
	\end{block}

	\begin{exampleblock}{Beim Aufstellen einer Wahrheitstabelle}
		Anzahl Interpretationen für eine Variablenmenge mit \(k \in \nN_+ \) Aussagevariablen: \(2^k\).
	\end{exampleblock}
\end{frame}

\begin{frame}{Semantik aussagenlogischer Formeln: $\vali{F}$}
	\begin{block}{Auswertung/Validierung aussagenlogischer Formeln}
		Sei \(I: V \to \nB\) eine Interpretation.\\
		Für jede aussagelogische Formel $F$ definiere $\vali{F}$ wie folgt:\\[2ex]

		Für jedes $X \in V$, $G$, $H$ weitere aussagelogische Formeln sei:
		\begin{itemize}
			\item $\vali{X}         := I(X) $
  			\item $\vali{\bnot G}   := \bfnot{\vali{G}} $
  			\item $\vali{G \bund H} := \bfand{\vali{G}}{\vali{H}}$
  			\item $\vali{G \boder H} := \bfor{\vali{G}}{\vali{H}}$
  			\item $\vali{G \bimp H} := \bfimp{\vali{G}}{\vali{H}}$
		\end{itemize}
	\end{block}
\end{frame}

% TODO: Aufgabe zum formellen Vorgehen, mit konkreter Interpretation?

\begin{frame}{Semantik aussagenlogischer Formeln: Wahrheitstabellen}
	\begin{exampleblock}{Aufgabe}
		Stelle für eine aussagenlogische Formel eine Wahrheitstabelle auf.
	\end{exampleblock}

	\begin{exampleblock}{Vorgehen}
	\pause
		\begin{enumerate}[<+->]
			\item Wahrheitswerte für Variablen
			\item Wahrheitswerte für Teilformeln
			\item Wahrheitswerte für ganze Formeln
		\end{enumerate}
	\end{exampleblock}
\end{frame}

\begin{frame}{Semantik aussagenlogischer Formeln: Wahrheitstabellen}
	\begin{exampleblock}{Aufgabe}
	Stelle für folgende aussagenlogische Formel eine Wahrheitstabelle auf: \( \bleftBr{} \bleftBr{} \bnot{} A \bimp{} B \brightBr{} \bund \bleftBr{} \bnot{} \bleftBr{} A \bgdw{} B \brightBr{} \boder{} A \brightBr{} \brightBr{}\).
	\end{exampleblock}
	%\pause
	\begin{block}{Lösung}
\begin{center}

		  	\begin{tabular}{cc|cccccccc}
		    \toprule
		    $A$ & $B$ & $\bleftBr{} \bleftBr{} \bnot{} A$ & $\bimp{}$ & $B\brightBr{}$ & $\bund $ & $\bleftBr{} \bnot{}$ & $\bleftBr{} A \bgdw{} B \brightBr{}$ & $\boder{}$ & $A \brightBr{} \brightBr{}$\\
		    \midrule
		    \pause
		    $ f $ & $ f $ & $ w $ & $ f $ & $ f $ & $ \textcolor{kit-green100}{\mathbf{f}} $ & $ f $ & $ w $ & $ f $ & $ f $\\
		    $ f $ & $ w $ & $ w $ & $ w $ & $ w $ & $ \textcolor{kit-green100}{\mathbf{w}} $ & $ w $ & $ f $ & $ w $ & $ f $\\
		    $ w $ & $ f $ & $ f $ & $ w $ & $ f $ & $ \textcolor{kit-green100}{\mathbf{w}} $ & $ w $ & $ f $ & $ w $ & $ w $ \\
		    $ w $ & $ w $ & $ f $ & $ w $ & $ w $ & $ \textcolor{kit-green100}{\mathbf{w}} $ & $ f $ & $ w $ & $ w $ & $ w $ \\
		    \bottomrule
  			\end{tabular}
		\end{center}

		\pause
		Beobachtung: \( \bleftBr{} \bleftBr{} \bnot{} A \bimp{} B \brightBr{} \bund \bleftBr{} \bnot{} \bleftBr{} A \bgdw{} B \brightBr{} \boder{} A \brightBr{} \brightBr{}\) und \(\bleftBr{} A \boder B \brightBr{}\) sind äquivalent.
	\end{block}
\end{frame}

\begin{frame}{Semantik aussagenlogischer Formeln: Äquivalenz}
	\begin{block}{Def.: äquivalente Formeln}
		Zwei Formeln $A$ und $B$ heißen \textbf{äquivalent}, wenn für jede Interpretation $I$ gilt
		\[\val_I(A)=\val_I(B)\]
		Man schreibt auch $A\equiv B$.
	\end{block}

	\begin{exampleblock}{Beispiele}
		\begin{itemize}
			\item \( \bleftBr{} \bleftBr{} \bnot{} A \bimp{} B \brightBr{} \bund \bleftBr{} \bnot{} \bleftBr{} A \bgdw{} B \brightBr{} \boder{} A \brightBr{} \brightBr{} \equiv \bleftBr{} A \boder B \brightBr{}\)
			\item \( \bleftBr{} A \bimp{} B \brightBr{} \equiv \bleftBr{} \bnot{} A \boder{} B \brightBr{} \)
		\end{itemize}
	\end{exampleblock}
\end{frame}

\begin{frame}{Semantik aussagenlogischer Formeln: Modell}
	\begin{block}{Def.: Modell}
		Eine Interpretation $I$ heißt \textbf{Modell} einer Formel $G$, wenn $\vali{G} =\mathbf{w}$.\\[2ex]

		Eine Interpretation $I$ heißt \textbf{Modell} einer Formelmenge $\Gamma$, wenn $I$ Modell jeder Formel $G\in \Gamma$ ist.
	\end{block}

	\begin{exampleblock}{}
	Wir sagen:\\
	\textcolor{black!50!red}{$\Gamma \models G$}, wenn jedes Modell von $\Gamma$ auch ein Modell von $G$ ist.\\
	\textcolor{black!50!red}{$\models G$} (statt \textcolor{black!50!red}{$\set{}\models G$}), wenn $G$ für \emph{alle} Interpretationen wahr ist.
	\end{exampleblock}
\end{frame}

\begin{frame}{Semantik aussagenlogischer Formeln: Tautologie}
	\begin{block}{Def.: Tautologie}
		Eine Formel $G$ heißt \textbf{Tautologie}, wenn jede Interpreation $I$ Modell ist. Das heißt, für jede Belegung (Interpretation $I$) der Aussagevariablen ist die gesamte aussagenlogische Formel wahr:
			\[	\vali{G}=\mathbf{w} \quad \text{ für alle } I	\]
	\end{block}

	\begin{exampleblock}{Beispiel}
		\begin{itemize}
			\item \( \bnot{} P \boder P \)
		\end{itemize}
	\end{exampleblock}
\end{frame}


\begin{frame}{}
	\begin{exampleblock}{Aufgabe (WS 16/17)}
		% Übungsblatt 2016-2, Aufgabe 2
		Es sei $\VarAL$ eine Menge von Aussagevariablen und $\ForAL$ die Menge aller aussagenlogischen Formeln über $\VarAL$. Beweise, dass für alle \(G, H \in \ForAL\) die aussagenlogische Formel
				\[\mathcal{F} :=	\bleftBr{} \bnot{} H\bimp{} \bnot{} G \brightBr{} \bimp{} \bleftBr{} G \bimp{} H \brightBr{}\]
		eine Tautologie ist. Verwende nicht das aussagenlogische Kalkül, sondern die formellen Definitionen der Auswertung von aussagenlogischen Formeln und boolschen Funktionen.
	\end{exampleblock}
	\begin{block}{Ansatz:}
		\zz Für alle Interpretationen $I$ ist die aussagenlogische Formel wahr.\\[1ex]
		Durch \emph{Abbilden}, \emph{Umformen} und \emph{Substituieren} (\emph{Ersetzen}) auf bekannte Tautologien schließen.\\[1ex]
		\textbf{Lösung:} s. Tafel.
	\end{block}
\end{frame}

\subsection{Beweisbarkeit}

\section{}
\questionframe
\lastframe
\mode<handout>{\slideThanks}
\end{document}