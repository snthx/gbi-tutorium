\subsection{ÜB-Tipps}
\begin{frame}{Hinweise zum Übungsblatt}
	\begin{block}{Def.: Notwendige Bedingung}
		Eine \textbf{notwendige Bedingung} zu einer Aussage $K$ ist eine Aussage $A$, für die gilt:

		{\center $K$ kann nur erfüllt sein, wenn die Bedingung $A$ gilt.}
		\\[0.5em]
		Das heißt \textit{nicht}, dass $K$ erfüllt sein muss.
	\end{block}

	\begin{block}{Def.: Hinreichende Bedingung}
		Eine \textbf{hinreichende Bedingung} zu einer Aussage $K$ ist eine Aussage $B$, für die gilt:
		\center Wenn die Bedingung $B$ erfüllt ist, so ist garantiert auch $K$ erfüllt.
	\end{block}
\end{frame}