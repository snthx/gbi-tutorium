\documentclass[a4paper]{article}

\usepackage[TS1,T1]{fontenc}
\usepackage[ngerman]{babel}
\usepackage[utf8]{inputenc}
\usepackage{libertine}
\usepackage{amsmath, amsfonts, amssymb}
\usepackage[pdftex,%
	pdfauthor={Stephan Bohr},%
	pdftitle={Loesung zur Tautologieaufgabe aus Tut 03}%
		]{hyperref}
\usepackage[margin=2.5cm]{geometry}
\usepackage{parskip}
\usepackage{mathtools}
\usepackage{xcolor}
\usepackage{enumitem}
%\setlist[enumerate]{topsep=0pt,itemsep=-1ex,partopsep=1ex,parsep=1ex}
\setlist[itemize]{noitemsep, nolistsep}
\setlist[enumerate]{noitemsep, nolistsep}

\newcommand{\newpar}[1]{\paragraph{#1}\mbox{}\newline}

\newcommand{\nM}{\mathbb{M}}
\newcommand{\nR}{\mathbb{R}}
\newcommand{\nN}{\mathbb{N}}
\newcommand{\nZ}{\mathbb{Z}}
\newcommand{\nQ}{\mathbb{Q}}
\newcommand{\nB}{\mathbb{B}}
\newcommand{\nC}{\mathbb{C}}
\newcommand{\nK}{\mathbb{K}}
\newcommand{\nF}{\mathbb{F}}
\newcommand{\nG}{\mathbb{G}}
\newcommand{\nullel}{\mathcal{O}}
\newcommand{\einsel}{\mathds{1}}
\newcommand{\nP}{\mathbb{P}}
\newcommand{\Pot}{\mathcal{P}}
\renewcommand{\O}{\text{O}}

\newcommand{\set}[1]{\{ #1 \}}
\newcommand{\setc}[2]{\set{#1 \mid #2}}
\newcommand{\setC}[2]{\set{#1 \mid \text{ #2 }}}

\newcommand{\setsize}[1]{\; \mid #1 \mid \; }

\newcommand{\q}[1]{\textquotedblleft #1\textquotedblright}

% Zu zeigen, thx to http://www.matheboard.de/archive/155832/thread.html
\newcommand{\zz}{\ensuremath{\mathrm{z\kern-.29em\raise-0.44ex\hbox{z}}}:}

% Text above symbol
% https://tex.stackexchange.com/a/74132/146825
%
% \newcommand{\eqtext}[1]{\stackrel{\mathclap{\normalfont\mbox{#1}}}{=}}
% \newcommand{\gdwtext}[1]{\stackrel{\mathclap{\normalfont\mbox{#1}}}{\Leftrightarrow}}
% \newcommand{\imptext}[1]{\stackrel{\mathclap{\normalfont\mbox{#1}}}{\Rightarrow}}
% \newcommand{\symbtext}[2]{\stackrel{\mathclap{\normalfont\mbox{#2}}}{#1}}
\newcommand{\eqtext}[1]{\mathrel{\overset{\makebox%[0pt]
{\mbox{\normalfont\tiny #1}}}{=}}}
\newcommand{\gdwtext}[1]{\mathrel{\overset{\makebox%[0pt]
{\mbox{\normalfont\tiny #1}}}{\ensuremath{\Leftrightarrow}}}}
\newcommand{\imptext}[1]{\mathrel{\overset{\makebox%[0pt]
{\mbox{\normalfont\tiny #1}}}{\ensuremath{\Rightarrow}}}}
\newcommand{\symbtext}[2]{\mathrel{\overset{\makebox%[0pt]
{\mbox{\normalfont\tiny #2}}}{#1}}}

% qed symbol
\newcommand{\qedblack}{\hfill \ensuremath{\blacksquare}}
\newcommand{\qedwhite}{\hfill \ensuremath{\Box}}

% Aussagenlogik
\newcommand{\BB}{\mathbb{B}}
\newcommand{\boder}{\ensuremath{\text{\;}\textcolor{blue}{\vee}}\text{\;}}
\newcommand{\bund}{\ensuremath{\text{\;}\textcolor{blue}{\wedge}}\text{\;}}
\newcommand{\bimp}{\ensuremath{\text{\;}\textcolor{blue}{\to}}\text{\;}}
\newcommand{\bnot}{\ensuremath{\text{\;}\textcolor{blue}{\neg}}\text{}}
\newcommand{\bgdw}{\ensuremath{\text{\;}\textcolor{blue}{\leftrightarrow}}\text{\;}}
\newcommand{\bone}{\ensuremath{\textcolor{blue}{1}}\text{}}
\newcommand{\bzero}{\ensuremath{\textcolor{blue}{0}}\text{}}
\newcommand{\bleftBr}{\ensuremath{\textcolor{blue}{(}}\text{}}
\newcommand{\brightBr}{\ensuremath{\textcolor{blue}{)}}\text{}}
\newcommand{\val}{\hbox{\textit{val}}}

\newcommand{\VarAL}{\hbox{\textit{Var}}_{AL}}
\newcommand{\ForAL}{\hbox{\textit{For}}_{AL}}

% Validierungsfunktion val_i
\newcommand{\vali}[1]{\ensuremath{\val_I(#1)}}

% Boolsche Funktion b_
\newcommand{\bfnot}[1]{\ensuremath{b_{\bnot}(#1)}}
\newcommand{\bfand}[2]{\ensuremath{b_{\bund}(#1,#2)}}
\newcommand{\bfor}[2]{\ensuremath{b_{\boder}(#1,#2)}}
\newcommand{\bfimp}[2]{\ensuremath{b_{\bimp}(#1,#2)}}
%

% ===== Document =====
\begin{document}
\newpar{Aufgabe}
Es sei $\VarAL$ eine Menge von Aussagevariablen und $\ForAL$ die Menge aller aussagenlogischen Formeln über $\VarAL$. Beweise, dass für alle \(G, H \in \ForAL\) die aussagenlogische Formel
				\[\mathcal{F} :=	\bleftBr{} \bnot{} H\bimp{} \bnot{} G \brightBr{} \bimp{} \bleftBr{} G \bimp{} H \brightBr{}\]
		eine Tautologie ist. Verwende nicht das aussagenlogische Kalkül, sondern die formellen Definitionen der Auswertung von aussagenlogischen Formeln und boolschen Funktionen.

\newpar{Lösung}
\emph{Behauptung:} \(\mathcal{F} :=	\bleftBr{} \bnot{} H\bimp{} \bnot{} G \brightBr{} \bimp{} \bleftBr{} G \bimp{} H \brightBr{}\) ist eine Tautologie.

\emph{Beweis:} Sei $I$ eine bel. Interpretation. Dann gilt:
\begin{align*}
	&\vali{\mathcal{F}}&\\
	&=\vali{(\bnot{} H \bimp{} \bnot{} G) \bimp{} (G \bimp{} H)}&\\
	&=\bfimp{\vali{\bnot{} H \bimp{} \bnot{} G}}{&\vali{G\bimp{} H}}&\\
	&=\bfor{\bfnot{\vali{\bnot{} H \bimp{} \bnot{} G}}}%
		{&\bfimp{\vali{G}}{\vali{H}}}&\\
	&=\bfor{\bfnot{\bfimp{\vali{\bnot{} H}}{\vali{\bnot{} G}}}}%
		{&\bfor{\bfnot{\vali{G}}}{\vali{H}}}&\\
	&=\bfor{\bfnot{\bfor{\bfnot{\vali{\bnot H}}}{\vali{\bnot G}}}}%
	{&\bfor{\bfnot{\vali{G}}}{\vali{H}}}&\\
	&=\bfor{\bfnot{\bfor{\bfnot{\bfnot{\vali{H}}}}{\bfnot{\vali{G}}}}}%
	{&\bfor{\bfnot{\vali{G}}}{\vali{H}}}&\\
	&\eqtext{$\neg\neg$}%
	\bfor{\bfnot{\bfor{\vali{H}}{\bfnot{\vali{G}}}}}{&\bfor{\bfnot{\vali{G}}}{\vali{H}}}&\\
	&\eqtext{$\vee$ komm.}%
	\bfor{\bfnot{\underbrace{{\bfor{\bfnot{\vali{G}}}{\vali{H}}}}_{:=E}}}{&\underbrace{{\bfor{\bfnot{\vali{G}}}{\vali{H}}}}_{=E}}&\\
	&\eqtext{Def.} \bfor{\bfnot{\vali{E}}}{\vali{E}}\\
	&\imptext{Tut.} \text{Tautologie} & \qedwhite{}
\end{align*}
% \newpar{Allgemein}
% Wir kennen für den Beweis von \emph{Tautologien} drei Möglichkeiten:
% \begin{enumerate}
% 	\item Modus Ponens
% 	\item Fallunterscheidung, was im Grunde einer Wahrheitstabelle gleichkommt
% 	\item Formelles Abbilden, danach dann umformen, substitieren, und auf bekannte Tautologien schließen.
% \end{enumerate}
% Im 2. und 3. Fall sollte zumindest formell korrekt abgebildet werden, so lange es geht, und danach (!) erst gedacht werden, damit die Tutanten mit dem Formalismen-Wahn aus GBI warm werden. Klar machen, dass das wichtig ist, und erst bis zum bitteren Ende durchgeführt werden sollte.
\end{document}