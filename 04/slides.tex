% ===== handout mode =====
% Comment/uncomment this line to toggle handout mode
% \newcommand{\handout}{}

% Comment/uncoment this line to toogle Mortitz mode
% \newcommand{\Moritz}{}

% Comment/uncomment this line to toggle handout mode
% \newcommand{\handout}{}

% by Stephan

%% Moritz mode or Stephan mode
\ifdefined \Moritz

% This is a configuration file with private, tutor specific information.
% It is therefore excluded from the Git repository so changes in this file will not conflict in git commits.

% Copy this Template, rename to config.tex and add your information below.

\newcommand{\mymail}{moritz.laupichler@student.kit.edu} % Consider using your named student Mail address to keep your u-Account private.

\newcommand{\myname}{\href{mailto:\mymail}{Moritz Laupichler}}

\newcommand{\mytutnumber}{25}

\newcommand{\mytutinfos}{Dienstags, 5. Block (15:45-17:15 Uhr), SR -120}

\newcommand{\aboutMeFrame}{
	\begin{frame}{Euer Tutor}
		Name: \myname \\
		Alter: 21 Jahre \\
		Studiengang: Master Informatik, 1. Semester \\
		\vspace{1cm}
		\pause 
		\centering{Kontakt: \href{mailto:\mymail}{\mymail}}
	\end{frame}
} % Moritz mode
\else
\ifdefined \Alex

% This is a configuration file with private, tutor specific information.
% It is therefore excluded from the Git repository so changes in this file will not conflict in git commits.

% Copy this Template, rename to config.tex and add your information below.

\newcommand{\mymail}{alexander.klug@student.kit.edu} % Consider using your named student Mail address to keep your u-Account private.

\newcommand{\myname}{\href{mailto:\mymail}{Alexander Klug}}

\newcommand{\mytutnumber}{30}

\newcommand{\mytutinfos}{Mittwochs, 3. Block (11:30-13:00), SR -107}

\newcommand{\aboutMeFrame}{
	\begin{frame}{Euer Tutor}
		Name: \myname \\
		Alter: 19 Jahre \\
		Studiengang: Bachelor Informatik, 3. Semester \\
		\vspace{1cm}
		\pause 
		\centering{Kontakt: \href{mailto:\mymail}{\mymail}}
	\end{frame}
}

% Toggle Handout mode by including the following line before including style_tut
% and removing the % at the start (but do NOT remove it here, otherwise handout mode will always be on!)
% Please keep handout mode on in all commits!

% \newcommand{\handout}{} % Alex Mode
\else

% This is a configuration file with private, tutor specific information.
% It is therefore excluded from the Git repository so changes in this file will not conflict in git commits.

% Copy this Template, rename to config.tex and add your information below.

\newcommand{\mymail}{stephan.bohr@student.kit.edu} % Consider using your named student Mail address to keep your u-Account private.

\newcommand{\myname}{\href{mailto:\mymail}{Stephan Bohr}}

\newcommand{\mytutnumber}{25}

\newcommand{\mytutinfos}{Dienstags, 5. Block (15:45-17:15), SR -119}

\newcommand{\aboutMeFrame}{
	\begin{frame}{Euer Tutor}
		Name: \myname \\
		Alter: 20 Jahre \\
		Studiengang: Bachelor Informatik, 3. Semester \\
		\vspace{1cm}
		\pause 
		\centering{Kontakt: \href{mailto:\mymail}{\mymail}}
	\end{frame}
} % Stephan mode
\fi
\fi

%% Beamer-Klasse im korrekten Modus
\ifdefined \handout
\documentclass[handout]{beamer} % Handout mode
\else
\documentclass{beamer}
\fi
%\documentclass[18pt,parskip]{beamer}

%% SLIDE FORMAT

% use 'beamerthemekit' for standard 4:3 ratio
% for widescreen slides (16:9), use 'beamerthemekitwide'

\usepackage{../templates/KIT-slides/beamerthemekit}
%\usepackage{../templates/KIT-slides/beamerthemekitwide}

%% TITLE PICTURE

% if a custom picture is to be used on the title page, copy it into the 'logos'
% directory, in the line below, replace 'mypicture' with the 
% filename (without extension) and uncomment the following line
% (picture proportions: 63 : 20 for standard, 169 : 40 for wide
% *.eps format if you use latex+dvips+ps2pdf, 
% *.jpg/*.png/*.pdf if you use pdflatex)

\titleimage{../figures/titleimage/brain}

%% TITLE LOGO

% for a custom logo on the front page, copy your file into the 'logos'
% directory, insert the filename in the line below and uncomment it

%\titlelogo{mylogo}

% (*.eps format if you use latex+dvips+ps2pdf,
% *.jpg/*.png/*.pdf if you use pdflatex)

%% TikZ INTEGRATION

% use these packages for PCM symbols and UML classes
% \usepackage{templates/tikzkit}
% \usepackage{templates/tikzuml}

%\usepackage{tikz}
%\usetikzlibrary{matrix}
%\usetikzlibrary{arrows.meta}
%\usetikzlibrary{automata}
%\usetikzlibrary{tikzmark}

%%%%%%%%%%%%%%%%%%%%%%%%%
% Libertine font (Original GBI font)
\usepackage{libertine}
%\renewcommand*\familydefault{\sfdefault}  %% Only if the base font of the document is to be sans serif

%% Schönere Schriften
\usepackage[TS1,T1]{fontenc}

%% Deutsche Silbentrennung und Beschriftungen
\usepackage[ngerman]{babel}

%% UTF-8-Encoding
\usepackage[utf8]{inputenc}

%% Bibliotheken für viele mathematische Symbole
\usepackage{amsmath, amsfonts, amssymb}

%% Anzeigetiefe für Inhaltsverzeichnis: 1 Stufe
\setcounter{tocdepth}{1}

%% Hyperlinks
\usepackage{hyperref}
% I don't know why, but this works and only includes sections and NOT subsections in the pdf-bookmarks.
\hypersetup{bookmarksdepth=subsection}

%% remove navigation symbols
\setbeamertemplate{navigation symbols}{}

%% switch between "ngerman" and "english" for German/English style date and logos
\selectlanguage{ngerman}

%% for invisible pause texts instead of dimming
\setbeamercovered{invisible}

%%%%%%%%%%%% Shortcuts %%%%%%%%%%%%%
\newcommand{\nM}{\mathbb{M}}
\newcommand{\nR}{\mathbb{R}}
\newcommand{\nN}{\mathbb{N}}
\newcommand{\nZ}{\mathbb{Z}}
\newcommand{\nQ}{\mathbb{Q}}
\newcommand{\nB}{\mathbb{B}}
\newcommand{\nC}{\mathbb{C}}
\newcommand{\nK}{\mathbb{K}}
\newcommand{\nF}{\mathbb{F}}
\newcommand{\nG}{\mathbb{G}}
\newcommand{\nullel}{\mathcal{O}}
\newcommand{\einsel}{\mathds{1}}
\newcommand{\nP}{\mathbb{P}}
\newcommand{\Pot}{\mathcal{P}}
\renewcommand{\O}{\text{O}}

\newcommand{\set}[1]{\{ #1 \}}
\newcommand{\setc}[2]{\set{#1 \mid #2}}
\newcommand{\setC}[2]{\set{#1 \mid \text{ #2 }}}

\newcommand{\setsize}[1]{\; \mid #1 \mid \; }

\newcommand{\q}[1]{\textquotedblleft #1\textquotedblright}

%%%%%%%%%%%% INHALT %%%%%%%%%%%%%%%%

%% Wochennummer
%\newcounter{weeknum}

%% Titelinformationen
%\title[GBI Tutorium, Woche \theweeknum]{Grundbegriffe der Informatik \\ Tutorium \mytutnumber}
%\subtitle{Termin \theweeknum \ | \mydate \\ \myname}
\author[\myname]{\myname}
\institute{Fakultät für Informatik}
%\date{\mydate}

%% Titel einfügen
\newcommand{\titleframe}{\frame{\titlepage}\addtocounter{framenumber}{-1}}


%% Alles starten mit \starttut{X}
%\newcommand{\starttut}[1]{\setcounter{weeknum}{#1}\titleframe\frame{\frametitle{Inhalt}\tableofcontents} \AtBeginSection[]{%
%\begin{frame}
%	\tableofcontents[currentsection]
%\end{frame}\addtocounter{framenumber}{-1}}}


%\newcommand{\framePrevEpisode}{
%	\begin{frame}
%		\centering
%		\textbf{In the previous episode of GBI...}
%	\end{frame}
%}

%% Roadmap frame
%table of contents
\newcommand{\roadmap}{
	\frame{\frametitle{Roadmap}\tableofcontents}}

 \AtBeginSection[]{%
\begin{frame}
	\frametitle{Roadmap}
	\tableofcontents[currentsection]
\end{frame}%\addtocounter{framenumber}{-1}
}


%% ShowMessage frame
\newcommand{\showmessage}[1]{\frame{\frametitle{\phantom{1em}}\centering\textbf{#1}}}

%% Fragen
%% Lastframe
\newcommand{\questionframe}{\showmessage{Fragen?}}

%% Lastframe
\newcommand{\lastframe}{\showmessage{Vielen Dank für Eure Aufmerksamkeit! \\Bis nächste Woche :)}}

%% Thanks frame
\newcommand{\slideThanks}{
	\begin{frame}
		\frametitle{Credits}
		\begin{block}{}
			An der Erstellung des Foliensatzes haben mitgewirkt:\\[1em]
			\ifdefined \Moritz
			Stephan Bohr \\
			Alexander Klug \\
			\else
			\ifdefined \Alex
			Stephan Bohr \\
			Moritz Laupichler \\
			\else
			Moritz Laupichler \\
			Alexander Klug \\
			\fi
			\fi
			Katharina Wurz \\
			Thassilo Helmold \\
			Philipp Basler \\
			Nils Braun \\
			Dominik Doerner \\
			Ou Yue \\
		\end{block}
	\end{frame}
}

%% Verbatim
%\usepackage{moreverb}



\title[Übersetzungen und Codierungen]{4. Tutorium\\ Sprachen, Vollständige Induktion}
\subtitle{Grundbegriffe der Informatik, Tutorium \hashtag\mytutnumber}
\date{\today}

\begin{document}
\titleframe
\roadmap

\Moritz{
\section{Organisatorisches}
\subsection{Zu ÜB 3}
\Kilian{}

\Moritz{}

\Stephan{
	\begin{frame}{Zweites Übungsblatt}
		\emph{Aufgabe 1}
		\begin{itemize}
			\item $M_y = \setC{x \in \nR_0^+}{f(x)=y} \neq \set{0,1} $
			\item Oft solche Angaben: $M_0 = \setc{x \in \nN_0}{f(x)=0}$.\\
	    	      Die Einschränkung ist unnötig, $M_0 = \nN_0$ reicht.
		\end{itemize}


		\emph{Aufgabe 3d}
		\begin{itemize}
	    	\item Häufige Lösung: $((k \diamond n) \diamond (k-n-1)) + ((n \diamond k) \diamond (n-k-1))$
	    	\item Problem: $k-n-1$ und $n-k-1$ können negativ werden, dafür die Operation $\diamond$ nicht definiert
	    \end{itemize}	
	\end{frame}
}
}

\section{Formale Sprachen}
\subsection{Sprachen} % Produkt v. Sprachen, Potenzen, L^*, L^+ und dabei induktives Definieren klarmachen
\begin{frame}{Formale Sprachen}
	\begin{block}{Def.: Formale Sprache}
		Eine \textbf{formale Sprache} L über einem Alphabet A ist eine Teilmenge der Wörter über A, also \(L \subseteq A^{*}\).
	\end{block}

	\begin{block}{Def.: Produkt formaler Sprachen}
		Es sei $A$ ein Alphabet, sowie $L,S \subseteq A^*$ formale Sprachen.\\
		Das \textbf{Produkt} der Sprachen ist definiert durch
		\[
			L \cdot S := \setc{u \cdot v}{u \in L \text{ und } v \in S}
		\]
	\end{block}

	\begin{exampleblock}{Beispiel}
		Es sei $L:=\set{\texttt{aa}, \texttt{bb}}$. Dann ist $L\cdot L = \set{\texttt{aaaa}, \texttt{aabb}, \texttt{bbbb}, \texttt{bbaa}}$
	\end{exampleblock}
\end{frame}

\begin{frame}{Formale Sprachen}
	\begin{block}{Def.: Potenzen von Sprachen}
		Die \textbf{Potenzen} einer formalen Sprache $L$ sind induktiv definiert:
		\begin{align*}
			L^0 &= \set{\varepsilon} \\
			\text{Für alle $n \in \nN_0$ gilt: } L^{n+1} &= L^n \cdot L
		\end{align*}
	\end{block}

	\begin{block}{Def.: Konkatenationsabschluss}
		Sei $L$ eine formale Sprache. Dann ist \\
		$L^* = \bigcup_{i=0}^{\infty} L^{i}$ der \textbf{Konkatenationsabschluss} $L^*$ von $L$ und \\
		$L^+ = \bigcup_{i=1}^{\infty} L^{i}$ der \textbf{$\varepsilon$-freie Konkatenationsabschluss} $L^+$ von $L$.
	\end{block}

	\begin{alertblock}{Achtung}
		Auch beim $\varepsilon$-freien Konkatenationsabschluss kann $\varepsilon$ enthalten sein, nämlich gdw. $\varepsilon \in L$
	\end{alertblock}
\end{frame}

% \begin{frame}{Formale Sprachen}
%     Tipp für's ÜB:
%     \[
%     	L^+ = L \cdot L^*
%     \]
% \end{frame}

\begin{frame}{Formale Sprachen}
	\begin{exampleblock}{Aufgabe}
		Es sei $A = {a,b}$. Drücke folgende Sprachen $L_1, L_2, L_3 \subseteq A^*$ formell aus:
		\begin{enumerate}
			\item $L_1$ enthält nur Wörter mit mindestens drei Vorkommen von $a$
			\item $L_2$ enthält nur Wörter, in denen die Zeichenfolge $bb$ nicht vorkommt
			\item Für alle Wörter $w$ in $L_3$ gilt: Wenn $w$ mindestens drei Vorkommen von $a$ hat, so hat $w$ auch mindestens zwei Vorkommen von $b$ 
		\end{enumerate}
	\end{exampleblock}
	\pause
	\begin{block}{Lösung}
		\begin{enumerate}
			\item $L_1 = \set{a,b}^\ast \cdot \set{a} \cdot \set{a,b}^\ast \cdot \set{a} \cdot \set{a,b}^\ast \cdot \set{a} \cdot \set{a,b}^\ast$
			\item $L_2 = \set{a}^\ast \cdot \set{b, \varepsilon} \cdot (\set{a}^+ \cdot \set{b})^\ast \cdot \{ a \}^*$ 
			\item \begin{align*}L_3 = & \set{\varepsilon,b} \cup \set{a, ab, ba} \cup \set{aa, aab, aba, baa} \\ & \cup (\set{a,b}^\ast \cdot \set{b} \cdot \set{a,b}^\ast \cdot \set{b} \cdot \set{a,b}^\ast) \end{align*}
		\end{enumerate}
	\end{block}
\end{frame}

\Moritz{

\section{Vollständige Induktion}
\subsection{Einstieg u. Definition}
\begin{frame}{Vollständige Induktion}
	\begin{block}{Vollständige Induktion}
		Vollständige Induktion beschreibt ein Beweisverfahren, das man verwendet um die Gültigkeit einer Aussage \(A(n)\) für alle \(n \in \nN_+ \text{ (oder } \nN_{0}\)) zu beweisen. Sie besteht aus: \\[.5em]
		\begin{description}
			\item [\textbf{Induktionsanfang (I.A.):}] Zeige, dass \(A(n)\) für \(n=1\) (bzw. \(n=0\)) gilt.
			\item [\textbf{Induktionsschritt (I.S.):}] Zeige, dass, wenn \(A(n)\) gilt, auch \(A(n+1)\) gilt.
				\begin{enumerate}
					\item \enquote{Sei $n \in \nN_+$ (oder $n \in \nN_0$) \textcolor{red}{beliebig}.}
					\item \textbf{Induktionsvoraussetzung (I.V.):} $A(n)$ sei wahr.
					\item Zeige mit der \textbf{I.V.}, dass dann auch $A(n+1)$ wahr ist.
				\end{enumerate}
		\end{description}
	\end{block}

	\begin{alertblock}{}
		Es gibt einige Varianten vollständiger Induktion. Immer darauf achten, was man braucht und will. %Tipp: \textbf{Nachdenken!}
	\end{alertblock}
\end{frame}
\subsection{Vorgerechnetes Beispiel}
\begin{frame}{Vollständige Induktion - Ein dummes Beispiel}
	\begin{exampleblock}{Aufgabe}
		Sei $x_n$ für $x \in \nN_0$ wie folgt definiert:
		\[
			x_n := (-1)^n.
		\]
		Zeige mit vollständiger Induktion, dass 
		$x_n = 
		\begin{cases}
			-1, &\text{ falls $n$ ungerade}\\
			1, &\text{ sonst}
		\end{cases}$
	\end{exampleblock}
\end{frame}
\begin{frame}{Vollständige Induktion - Ein dummes Beispiel}
	\begin{block}{Lösung}
		\begin{itemize}
			\item[I.A.] $n=0: (-1)^0=1 \quad \checkmark$
			\item[I.S.:] $A(n) \rightarrow A(n+1)$.\\
			\begin{itemize}
				\item Sei $n \in \nN_{0}$ beliebig.
				\item \textbf{I.V.:} Es gelte: 
				 			$x_n = 
					\begin{cases}
						-1, &\text{ falls $n$ ungerade}\\
						1, &\text{ sonst}
					\end{cases}$\\[1.5em]
				\item Fall 1: $n+1 \text{ ist ungerade } \Rightarrow n \text{ ist gerade}$\\
					$x_{n+1} = (-1)^{n+1} = (-1) \cdot (-1)^n \eqtext{\textbf{I.V.}} (-1) \cdot 1 = -1 \quad \checkmark$
				\item Fall 2: $n+1 \text{ ist gerade } \Rightarrow n \text{ ist ungerade}$\\
					$x_{n+1} = (-1)^{n+1} = (-1) \cdot (-1)^n \eqtext{\textbf{I.V.}} (-1) \cdot (-1) = 1 \quad \checkmark$
			\end{itemize}
			Mit Fall 1 und 2 folgt die Behauptung. \qedwhite{}
		\end{itemize}
	\end{block}
\end{frame}
\subsection{Ein leichter induktiver Beweis auf einer Sprache}
\begin{frame}{Induktiv Beweisen I.: Zum Warmwerden}
	\begin{exampleblock}{Behauptung}
		Es sei w ein Wort. Es gilt:\\[.5em]
		\centering{Für alle $k \in \nN_0$ gilt $|w^k| = k \cdot |w|$.}
	\end{exampleblock}
\pause
	\begin{block}{Beweis}
		\begin{itemize}
			\item[I.A.:] Es sei $k=0$. Es gilt $|w^k|=|w^0|=|\varepsilon|=0=0 \cdot |w| = k \cdot |w|$.
			\item[I.S.:] $A(k) \rightarrow A(k+1)$:
				\begin{itemize}
					\item Sei $k \in \nN_{0}$ beliebig.
					\item \textbf{I.V.:} Es gelte $|w^k|=k \cdot |w|$. \\[1em]
					\item $|w^{k+1}|=|w^k \cdot w|=|w^k|+|w| \eqtext{I.V.} k \cdot |w| + |w| = (k+1) \cdot |w|$
				\end{itemize}						
		\end{itemize}

		Nach den Regeln der vollständigen Induktion folgt aus obigem Induktionsschritt die Behauptung.
	\end{block}
\end{frame}
\subsection{Variante vollständiger Induktion} % z.B. gilt für alle <=n oder mit größerem Anfang und n->n+2, etc
\begin{frame}{Induktiv Beweisen II.: Schwieriger}
	\begin{exampleblock}{Aufgabe}
		Eine Funktion $T:\nN_0 \rightarrow \nN_0$ sei wie folgt definiert:
		\begin{equation*}
		T(n) = 
			\begin{cases}
			 2, &\text{ wenn } n=0 \\
			 3, &\text{ wenn } n=1 \\
			 3 \cdot T(n-1) - 2 \cdot T(n-2) &\text{ sonst}
		\end{cases} \pause
		\end{equation*}
		\begin{enumerate}
			\item Gib die Funktionswerte $T(n)$ für $n \in \set{2,3,4,5,6}$ an.
			\item Gib eine geschlossene Formel $F(n)$ (einen arithmetischen Ausdruck ohne Rekursion) für $T(n)$ an.
			\item Zeige durch vollständige Induktion, dass für alle $n \in \nN_0$ gilt $F(n) = T(n)$.
		\end{enumerate}
	\end{exampleblock}
\end{frame}
\subsection{Ein schwieriger mathematischer Beweis}
\begin{frame}{Induktiv Beweisen II.: Schwieriger}
	\begin{block}{Lösung}
		\begin{enumerate}
			\item $ T(2)=5,T(3)=9, T(4)=17, T(5)=33, T(6)=65$
			\item $F(n) = 2^n+1$
			\newcounter{kevin}
			\setcounter{kevin}{\value{enumi}}
		\end{enumerate}
	\end{block}
\end{frame}
\begin{frame}{Induktiv Beweisen II.: Schwieriger}
	\begin{block}{Lösung}
		\begin{enumerate}
			\setcounter{enumi}{\value{kevin}}
			\item \begin{itemize}
		 		\item Beh.: Für alle $n \in \nN_0$ gilt $F(n) = T(n)$.
		 		\item Bew. (induktiv): 
				\begin{itemize}
					\item[I.A.:]
								Sei $n=0$. Dann gilt: \\ 
								\qquad $F(n)=F(0)=2^0+1=1+1=2=T(0)=T(n)$.\\
							 	Sei $n=1$. Dann gilt: \\
							 	\qquad $F(n)=F(1)=2^1+1=2+1=3=T(1)=T(n)$.
				\end{itemize}
		 	\end{itemize}
		\end{enumerate}		 					  
	\end{block}
\end{frame}
\begin{frame}{Induktiv Beweisen II.: Schwieriger}
	\begin{block}{Lösung}
	\begin{enumerate}
		\setcounter{enumi}{\value{kevin}}
		\item
		\begin{itemize}
				\item[I.S.:] $A(n) \wedge A(n+1) \rightarrow A(n+2)$ %\zz $T(n+2) = F(n+2)$, also $n \leadsto n+2$.
					\begin{itemize}
						\item Sei $n \in \nN_{0}$ beliebig.
						\item \textbf{I.V.:} Es gelte
				 					\[T(n)=F(n)=2^n+1 \text{ und } T(n+1)=F(n+1)=2^{n+1}+1\]
				 		\item \zz $T(n+2) = F(n+2)$
				 		\begin{align*}
							T(n+2) &= 3 \cdot T(n+1) - 2 \cdot T(n) \\
								   &\eqtext{I.V.} 3 \cdot F(n+1) - 2 \cdot F(n) \\
								   &= 3 \cdot (2^{n+1}+1) - 2 \cdot (2^n+1) \\
								   &= 3 \cdot 2^{n+1} + 3 - 2 \cdot 2^n - 2 \\
								   &= 3 \cdot 2^{n+1} - 2 \cdot 2^n + 1 \\
								   &= 3 \cdot 2^{n+1} - 2^{n+1} +1 \\
								   &= 2 \cdot 2^{n+1} + 1 = 2^{n+2} + 1 = F(n+2)
						\end{align*}
					\end{itemize}				
			\end{itemize}
		\end{enumerate}
	\end{block}
\end{frame}
\subsection{Eine Knobelaufgabe}
\begin{frame}{Induktiv Beweisen III.: Zum Knobeln}
	\begin{exampleblock}{Aufgabe}
		Alice und Bob feiern ihren Hochzeitstag. Auf ihrer Party befinden sich $n \in \nN_+$ Paare. Dabei begrüßen\footnote{Eine Begrüßung ist das gegenseitige Händeschütteln. Mit einer Begrüßung haben sich beide Personen begrüßt.} sich alle Paare mit Ausnahme des eigenen Partners.\\
		\begin{enumerate}
			\item Gib die Anzahl der Begrüßungen für $i \in \set{1,2,3,4,5}$ Paare an.
			\item Stelle für die Anzahl der Begrüßungen einen geschlossenen Ausdruck $p_n$ auf.
			\item Gib für die Anzahl der Begrüßungen eine induktive Definiton $q_n$ an.
			\item Zeige per vollständiger Induktion, dass für alle $n \in \nN_+$ gilt: $p_n$ = $q_n$.
		\end{enumerate}
	\end{exampleblock}
\end{frame}

\begin{frame}{Induktiv Beweisen III.: Zum Knobeln}
	\begin{block}{Lösung}
		\begin{enumerate}
			\item \begin{itemize}
				\item 1 Paar $\rightarrow$ 0 Begrüßungen
				\item 2 Paare $\rightarrow$ 4 Begrüßungen
				\item 3 Paare $\rightarrow$ 12 Begrüßungen
				\item 4 Paare $\rightarrow$ 24 Begrüßungen
				\item 5 Paare $\rightarrow$ 40 Begrüßungen
			\end{itemize} \pause
			\item $p_n = 4 \cdot \frac{1}{2} \cdot n \cdot (n-1) = 2 \cdot (n^2 -n)$ \pause
			\item Jedes Paar, das neu hinzukommt, muss die $n$ sich im Raum befindenden Paare begrüßen. Das bedeutet $2n$ Begrüßungen (ein Paar besteht aus $2$ Partnern) pro neu hinzugekommenen Partner, also insgesamt $4n$ Begrüßungen. Formal heißt das: 
			\begin{align*}
				q_1 &= 0 \\
				\text{Für alle } n \in \nN_+ \text{ gilt: } q_{n+1} &= q_n + 4n
			\end{align*}
			\setcounter{kevin}{\value{enumi}}
		\end{enumerate}
	\end{block}
\end{frame}

\begin{frame}{Induktiv Beweisen III.: Zum Knobeln}
	\begin{block}{Lösung}
	\begin{enumerate}
		\setcounter{enumi}{\value{kevin}}
		\item \begin{itemize}
			\item[I.A.] $n=1: p_n = p_1 = 2 \cdot (1^2 -1 ) = 0 = q_1 = q_n \quad \checkmark$ \\[1em]
			%\item[I.V.] Für ein beliebiges, aber festes $n \in \nN_+$ gelte $p_n=q_n$.
			\item[I.S.] $A(n) \rightarrow A(n+1)$ %\zz $p_{n+1}=q_{n+1}$, also $n \leadsto n+1$
				\begin{itemize}
					\item Sei $n \in \nN_{+}$ beliebig.
					\item \textbf{I.V.:} Es gelte $p_n = q_n$.
					\item \zz $p_{n+1}=q_{n+1}$
						\begin{align*}
							p_{n+1} &= 2 \cdot (n+1) \cdot (n+1-1) \\
									&= 2n^2 + 2n \\
									&= 2n^2 -2n + 4n \\
									&= 2 \cdot n \cdot (n-1) + 4n \\
									&= p_n + 4n \\
									&\eqtext{I.V.} q_n + 4n \\
									&\eqtext{Def.} q_{n+1} & \qedwhite{}
						\end{align*}
				\end{itemize}
		\end{itemize}
	\end{enumerate}		
	\end{block}
\end{frame}
\subsection{VI Aufgaben zu Sprachen}

\begin{frame}{Induktiv Beweisen: Sprachen I}
	\small \begin{exampleblock}{Aufgabe}
		Sei $L$ eine formale Sprache über einem Alphabet $A$. $L$ sei induktiv definiert durch:
		\begin{align*}
		 	L_1 &= A \\
		 	L_{n+1} &= L_n \cdot L_n \text{ für } n\in \nN^+ \\
		 	L &= \bigcup_{i=1}^\infty L_i
		 \end{align*}

		 \smallskip
		 Zeige per vollständiger Induktion, dass folgender geschlossener Ausdruck für $L$ gilt: \[L = \setc{w\in A^\ast}{|w|=2^n, n\in \nN_0}\]
		 
	\end{exampleblock}

	\smallskip
	Tipp: Zeige $\forall n\in \nN^+: L_n = \setc{w\in A^\ast}{|w|=2^{n-1}}$, denn dann folgt:
	\[
		L = \bigcup_{i=1}^\infty L_i 
		= \bigcup_{n=1}^\infty \setc{w\in A^\ast}{|w|=2^{n-1}} 
		= \bigcup_{n=0}^\infty \setc{w\in A^\ast}{|w|=2^{n}}
	\]

\end{frame}
	
\begin{frame}{Induktiv Beweisen: Sprachen I}

	\small \begin{block}{Lösung}
		\textbf{Induktionsanfang (I.A.):} Betrachte $n=1$. Es gilt $L_1 = A$, d.h. für jedes $w \in L_1$ gilt $|w|=1=2^0=2^{n-1}$ \checkmark\\
		\textbf{Induktionsschritt (I.S.):}\\
		\textit{Induktionsvorauss. (I.V.):} Sei $n \in \nN^+$ bel. und gelte $L_n = \setc{w\in A^\ast}{|w|=2^{n-1}}$.\\[.25em]
		$\zz L_{n+1} = \setc{w\in A^\ast}{|w|=2^{n}}$ \quad bzw. \quad $w\in L_{n+1} \textit{ gdw. } |w|=2^n$		
		\begin{align*}
			&w\in L_{n+1} \\
			\textit{gdw. \quad}& w \in L_n \cdot L_n\\
			\textit{gdw. \quad}& \exists w',w'' \in L_n: w = w' \cdot w'' \\
			\textit{mit I.V. gdw. \quad}& \exists w',w'' \in \setc{x\in A^\ast}{|x|=2^{n-1}}: w = w' \cdot w'' \\
			\textit{gdw. \quad}& \exists w',w'' \in A^\ast: |w'|=|w''|=2^{n-1} \wedge w = w' \cdot w''\\
			%\textit{gdw. }& \exists w',w'' \in A^\ast: |w'|=|w''|=2^{n-1} \wedge w = w' \cdot w'' \wedge |w|=|w'|+|w''|\\
			\textit{gdw. \quad}& |w|=2^{n-1}+2^{n-1}=2*2^{n-1}\\
			\textit{gdw. \quad}& |w|=2^n
		\end{align*}
	\end{block}
\end{frame}
}


% \section{Darstellung von Zahlen}
% \subsection{Num}

\begin{frame}{Von Wörtern zu Zahlen}
	
	\begin{block}{Def.: $Z_k$}
		Das Alphabet $Z_k$ enthält Zeichen, die Ziffern mit den Werten von $0$ bis $k-1$ (inklusive) entsprechen. $Z_k$ enthält somit keine Zahlen, sondern Zeichen, die der Repräsentation von Zahlen als Wort dienen.
	\end{block}

	\begin{exampleblock}{Beispiele}
		\begin{align*}
			Z_{2} &= \set{\texttt{\textcolor{blue}{0,1}}}\\
			Z_{10} &= \set{\texttt{\textcolor{blue}{0,1,2,3,4,5,6,7,8,9}}}\\
			Z_{16} &= \set{\texttt{\textcolor{blue}{0,1,2,3,4,5,6,7,8,9,A,B,C,D,E,F}}}
		\end{align*}
	\end{exampleblock}

\end{frame}

\begin{frame}{Von Wörtern zu Zahlen}
	\begin{block}{Def.: $\text{Num}_k$}
		Zu einer Zahlenbasis $k$ definiere die Abbildung $\text{Num}_k : Z_k^* \to \nZ$ 

		\begin{align*}
			\text{Num}_k(\varepsilon) &= 0 \\
			\text{Num}_k(wx) &= k\cdot \text{Num}_k(w) + \text{num}_k(x) \text{ für alle } w\in Z_k^*, x\in Z_k\\[2ex]
			\text{wobei }\text{num}_k(x) &= x \text{ für alle } x \in \set{1,\dots,k-1}
		\end{align*}
	\end{block}
	\pause
	\begin{exampleblock}{Beispiel}
		\begin{align*}
			\text{Num}_2(11) &= 2\cdot \text{Num}_2(1) + \text{num}_2(1) \\
				&= 2\cdot 1 + 1 \\
				&= 3
		\end{align*}
	\end{exampleblock}
\end{frame}

\begin{frame}{Von Wörtern zu Zahlen}
	\begin{exampleblock}{Aufgabe: Berechne die Zahlenwerte von $321_4, B2_{16}$}
	\begin{align*}
	\text{Num}_4(321) &= \visible<2->{ 4\cdot \text{Num}_4(32) + \text{num}_4(1) \\
	&= 4\cdot \left( 4\cdot \text{Num}_4(3) + \text{num}_4(2) \right) + \text{num}_4(1) \\
	&= 4^2\cdot \text{num}_4(3) + 4 \cdot \text{num}_4(2) + \text{num}_4(1) \\
	&= 57 \\}
	\visible<3->{\text{Num}_{16}(B2) &=} \visible<4->{ 16 \cdot \text{Num}_{16}(B) + \text{num}_{16}(2) \\
	&= 16\cdot 11 + 2 \\
	&= 178}
	\end{align*}
	\end{exampleblock}
	\visible<5->{
	\begin{exampleblock}{Aufgabe}
		$\text{Num}_2(1), \text{ Num}_2(11), \text{ Num}_2(111), \text{ Num}_2(1111)$\\
		Was gilt allgemein für $\text{Num}_2(1^m)$ für $m \in \nN_0$?
	\end{exampleblock}}
\end{frame}

\subsection{mod und div}
\begin{frame}{Division und Modulo}
	\begin{block}{\div und \mod}
		$ x \div y$ ist die ganzzahlige Division von x durch y.\\
		$ x \mod y$ liefert den Rest dieser Division.
	\end{block} 
	\begin{block}{Beobachtung}
		$$ 0 \leq (x \mod y ) < y$$
	\end{block}
	\begin{block}{Lemma}
		$$ x = y \cdot (x \div y ) + \left( x \mod y \right)$$ 
	\end{block}
\end{frame}

\begin{frame}{Division und Modulo}
	\begin{exampleblock}{Beispiel}
		\begin{table}[h!]
			\begin{tabular}{c|cc}
				& $x\div y$ & $x\mod y$ \\ \hline
				$x=2,y=3$ \only<2-|handout:0>{&  0 & 2 } \only<1>{&&}\\
				$x=5 ,y=2$ \only<3-|handout:0>{ & 2 & 1 } \only<1-2>{&&} \\
				$x=8,y=2$ \only<4-|handout:0>{ & 4 & 0 } \only<1-3>{&&} \\	
			\end{tabular}
		\end{table}
	\end{exampleblock}
\end{frame}

\begin{frame}{Division und Modulo}
	\begin{exampleblock}{Aufgabe}
		Fülle die folgende Tabelle aus
		\begin{table}[h!]
		\begin{tabular}{c|cccccccccccc}
			$x$ & 0 & 1 & 2 & 3 & 4 & 5 & 6 & 7 & 8 & 9 & 10 & 11 \\ \hline
			$x\div 4 $ & \only<1>{ &&&&&&&&&&} \only<2-|handout:0>{0 & 0 & 0 & 0 & 1 & 1 & 1 & 1 & 2 & 2 & 2 & 2}\\
			$4\left( x\div 4\right) $ & \only<1-2>{&&&&&&&&&&} \only<3-|handout:0>{0&0&0&0&4&4&4&4&8&8&8&8}\\
			$x\mod 4$ & \only<1-3>{&&&&&&&&&&} \only<4-|handout:0>{0&1&2&3&0&1&2&3&0&1&2&3}
		\end{tabular}
		\end{table}
	\end{exampleblock}
\end{frame}

\subsection{Repr}
\begin{frame}{Repräsentation}
	\begin{block}{Def.: $Repr_k$}
		\begin{align*}
			Repr_k : \; &\nN_0 \to Z_k  \\
			n&\mapsto \begin{cases} repr_k(n) & \text{, falls } n<k \\
				Repr_k\left( n\div k \right) \cdot repr_k\left( n \mod 	k \right) & \text{, falls } n\geq k 
			\end{cases}\\
			\text{Wobei für alle } &x \in \nZ_k : \text{repr}_k(x) = x.
		\end{align*}
	\end{block}
	\pause
	\begin{exampleblock}{Aufgabe}
		Berechne folgende Darstellungen:\\
		$Repr_2(42) = \pause 101010$ \\
		$Repr_4(42) = \pause 222$ \\
		$Repr_8(42) = \pause 52$ \\
		$Repr_{16}(42) = \pause 2A$
	\end{exampleblock}
\end{frame}

% \begin{frame}{Beispiel: Lösung}
% 	\begin{align*}
% 		Repr_8(42) &= Repr_8(42 \div 8) \cdot repr_8(42 \mod 8) \\
% 		&= Repr_8(5) \cdot repr_8(2)\\
% 		&= repr_8(5) \cdot 2\\
% 		&= 5 \cdot 2\\
% 		&= 52_8
% 	\end{align*}
% \end{frame}

\begin{frame}{Repräsentation}
	\begin{block}{Korrolar}
		$Repr_k(n)$ ist das kürzeste Wort $w\in Z_k^\ast$ mit $Num_k(w)=n$, also 
		$$ Num_k\left( Repr_k(n)\right) = n $$
	\end{block}

	\begin{alertblock}{Anmerkung}
		Im Allgemeinen  $Repr_k\left(Num_k(w)\right) \neq w $, da überflüssige Nullen wegfallen.
	\end{alertblock}
	
	\Stephan{
	\begin{exampleblock}{Schaubild}
		s. Tafel
	\end{exampleblock}
	}
	%TODO
	% \begin{exampleblock}{Zusammenhang}
	% 	\begin{tikzpicture}[
 %    		node distance=2mm,
 %    		>=stealth, auto,
	% 		every state/.style={draw=none, inner sep=0pt}
 %                		]

	% 			\node[state] (q1) {Dezimaldarstellung};
	% 			\node[state] (q2) [above right=of q1] {$\text{Repr}_k$};
	% 			\node[state] (q3) [below right=of q2] {k-äre Darstellung};
	% 			\node[state] (q4) [below right=of q1] {$\text{Num}_k$};

 %    		\begin{scope}[bend left]%
	% 	\path[->]   (q1) edge node {c} (q2)
 %            		(q2) edge node {c} (q3)
 %            		(q3) edge node {c} (q4)
 %            		(q4) edge node {c} (q1);
 %    		\end{scope}
	% 	\end{tikzpicture}
	% \end{exampleblock}	 
\end{frame}

\subsection{Zweierkomplement}
\begin{frame}{Ein asymetrischer Zahlenbereich}
	\[
	\nK_{\ell} := \{ x\in \nZ\mid -2^{\ell-1} \leq x \leq 2^{\ell-1} -1 \} \;.
	\]
	\\[0.2cm]
	
	\begin{figure}
		\centering
		\includegraphics[scale=0.35]{../topics/codierung/ZK_K4}
	\end{figure}

	\begin{exampleblock}{Frage}
		Wie sieht $\nK_5$ aus?
	\end{exampleblock}
\end{frame}

\begin{frame}{Zweierkomplement} % TODO nötig?
    Das Zweierkomplement ist eine Möglichkeit, negative Zahlen binär darzustellen. Im Vergleich zu anderen Darstellungsarten ist es besonders vorteilhaft bei arithmetischen Rechnungen mit Hardware \textit{(mehr dazu in Technischer Informatik)}.
\end{frame}

\begin{frame}{Zweierkomplement}
	\begin{block}{Def.: $\text{Zkpl}_\ell$}
		Die Zweierkomplementdarstellung $ \text{Zkpl}_\ell: \nK \to \set{\texttt{0}, \texttt{1}}^\ell$ der Länge $\ell$:
		$$\text{Zkpl}_\ell(x) = \begin{cases} 0 \text{bin}_{l-1}(x) & \text{falls } x \geq 0 \\ 1 \text{bin}_{l-1}(2^{l-1}+x) & \text{falls } x < 0\end{cases}$$
		Äquivalent:
		$$\text{Zkpl}_\ell(x) = \begin{cases} \text{bin}_{l}(x) & \text{, falls } x \geq 0 \\ \text{bin}_{l}(2^{l}+x) & \text{, falls } x < 0\end{cases}$$
		wobei
		\[
			\text{bin} \colon \nZ_{2^{\ell}} \to \{\texttt{0},\texttt{1}\}^{\ell}, n \mapsto \texttt{0}^{\ell- |\text{Repr}_2(n)|} \text{Repr}_2(n) 
		\]
	\end{block}
\end{frame}
\subsection{Trick}
\begin{frame}{Zweierkomplement}
	\begin{exampleblock}{Ein Trick}
		Zum ``intuitiven'' Berechnen des Zweierkomplement können wir so vorgehen (für $x < 0$):
		\begin{enumerate}
			\item Binärdarstellung von $|x|$ berechnen: $\text{Repr}_2(|x|)$
			\item Mit führenden Nullen auffüllen bis zur Länge $\ell$
			\item Alle binären Ziffern negieren
			\item 1 addieren
		\end{enumerate}
	\end{exampleblock}

	\begin{exampleblock}{Beispiel}
		$$\text{Zkpl}_4(-2): 2 \rightarrow 10 \rightarrow 0010 \rightarrow 1101 \rightarrow 1110 = \text{Zkpl}_4(-2)$$
	\end{exampleblock}
\end{frame}
\subsection{Aufgabe}
\begin{frame}{Zweierkomplement}
	\begin{exampleblock}{Aufgaben}
	Berechne
		\begin{itemize}
			\item $\text{Zkpl}_5(0) \only<2->{= 00000}$ \\
			\item $\text{Zkpl}_5(2) \only<3->{= 00010}$ \\
			\item $\text{Zkpl}_5(15) \only<4->{= 01111}$ \\
			\item $\text{Zkpl}_5(-1) \only<5->{= 11111} $\\
			\item $\text{Zkpl}_5(-6) \only<6->{= 11010} $\\
			\item $\text{Zkpl}_5(-16) \only<7->{= 10000}$
		\end{itemize}
	\end{exampleblock}
\end{frame}

\begin{frame}{Darstellung von Zahlen}
    \begin{exampleblock}{Aufgabe}
    	Es seien die Wörter $u := \texttt{10010}, v := \texttt{01011}$ aus $Z_2^*$.
    	\begin{enumerate}
     		\item Gib die Dezimaldarstellung an, die $u$ und $v$ als Binärdarstellung haben. Gib die Binärdarstellung von $u+v=:w \in Z_2^*$ an.
    		\item Gib den Dezimalwert der Zkpl-Interpretation von $u, v, w$ an, also $\text{Num}_{Zkpl}(x)$ für $x\in \set{u,v,w}$.
    		\item Ist $w$ die Zweierkomplementdarstellung der Summe der Zahlen mit den Zweierkomplementdarstellungen $u$ und $v$?
  		\end{enumerate}
    \end{exampleblock}
\end{frame}
	
\begin{frame}
    \begin{block}{Lösung}
    	\begin{enumerate}
     		\item $\text{Num}_2(u)=18, \text{Num}_2(v)=11, \text{Num}_2(w) = \text{Num}_2(u)+\text{Num}_2(v)=29, w=\texttt{11101}$
    		\pause \item $\text{Num}_{Zkpl}(u)=-14, \text{Num}_{Zkpl}(v)=\text{Num}_2(v)=11, \text{Num}_{Zkpl}(w)=-3$
    		\pause \item Ja, denn $-14+11=-3$.
  		\end{enumerate}
    \end{block}
\end{frame}


% \begin{frame}{ZK: Einfache Berechnung}
% 	Die einzelnen Schritte können wir auch formal angeben:\\
% 	(Wir operieren jeweils auf Wörtern aus $\{0, 1\}^* = Z_2^*$) \\[0.5em]
% 	1. Binärdarstellung von $\setsize{x}$ berechnen: $Repr_2(\setsize{\cdot})$ \\
% 	2. Mit führenden Nullen auffüllen bis zur Länge $\ell$\\ \pause
% 	\begin{align*}
% 		Fill_\ell : Z_2^m &\to Z_2^\ell \qquad (m \le \ell) \\ \visible<3-> {
% 		w &\mapsto \begin{cases}
% 		0^\ell & w = \varepsilon \\
% 		Fill_{\ell-1}(w') \cdot \mu & w = w' \cdot \mu, w' \in Z_2^*, \mu \in Z_2
% 		\end{cases} \\
% 		&\text{oder deutlich einfacher} \\
% 		w &\mapsto \begin{cases}
% 		w & \setsize{w} = \ell \\
% 		Fill_\ell(0w) & \text{sonst}
% 		\end{cases} \\
% 	}
% 	\end{align*}
% 	\pause[4] 3. / 4. Analog %TODO
% \end{frame}

% \section{Homomorphismen}
% 
% \begin{frame}
% 	\textbf{Übersetzung:} Bedeutungserhaltende Abbildung \\[0.5em] \pause
% 	\textbf{Codierung:} Injektive Übersetzung \\ \pause
% 	Es reicht eine injektive Abbildung: Dann können wir für jedes $f(w)$ eindeutig das erzeugende $w$ angeben und somit die Bedeutung von $f(w)$ als die Bedeutung von $w$ festlegen. \\[1em]
	
% 	\pause
% 	\emph{Beliebige }Codierungen zu speichern ist sehr aufwendig, bei unendlichem Definitionsbereich sogar unmöglich.\\
% 	Also bringen wir etwas Struktur ins Spiel!
% \end{frame}

% \begin{frame}{Homomorphismen}
% 	Ein Homomorphismus ist eine strukturerhaltende Abbildung. \\
% 	\begin{align*}
% 		\Phi : A &\to B \\
% 		\text{ mit } \forall a \in A, b \in B: \Phi(a \; \square \; b) &= \Phi(a) \; \triangle \; \Phi(b)
% 	\end{align*} 
% \end{frame}
\subsection{Definitionen und alles}
\begin{frame}{Homomorphismen}
	\begin{block}{Def.: Homomorphismus}
		Ein \textbf{Homomorphismus} ist eine strukturerhaltende Abbildung. Wir betrachten Homomorphismen auf Wörtern, dabei muss die Konkatenation erhalten werden.\\
		Seien $A, B$ Alphabete, dann ist Abbildung $h: A^* \to B^*$ ein \textbf{Homomorphismus}, wenn
		$$ \forall\ x, y\in A^* : h(x \cdot y) = h(x) \cdot h(y) $$
	\end{block}
	
	\begin{exampleblock}{Beispiel}
		Sei $h$ ein Homomorphismus mit $h(a) = 2, h(b) = 3$. \\
		Dann gilt $h(aba) = h(a) \cdot h(b) \cdot h(a) = 232 $
	\end{exampleblock}
\end{frame}

\begin{frame}{Homomorphismen}
	\begin{exampleblock}{Aufgabe}
		Gegeben sei die folgende Abbildung über dem Alphabet $A:=\set{a, b, c, \dots, z}$:
		\begin{align*}
			R(\varepsilon) &= \varepsilon,\\
			\text{für alle } w \in A^* gilt: R(wx) &= x \cdot R(w).
		\end{align*}
		\begin{enumerate}
			\item Ist $R$ ein Homomorphismus?
			\item \only<2->{Gib ein Alphabet $A'$ an, sodass $R$ ein Homomorphismus ist.}
		\end{enumerate}
	\end{exampleblock}

	\begin{block}{Lösung}
		\begin{enumerate}
			\item \only<2->{$R$ ist kein Homomorphismus, denn $R(a \cdot b) = ba \neq ab = R(a) \cdot R(b)$.}
			\item \only<3->{$R$ ist ein Homomorphismus, wenn $\setsize{A'}=1$.}
		\end{enumerate}
	\end{block}
\end{frame}

% \begin{frame}{Homomorphismen konstruieren}
% 	Wir können uns aber aus einer Abbildung der einzelnen Zeichen einen Homomorphismus auf Wörtern konstruieren.
% 	\begin{Definition}
% 		Sei $f: A \to B^*,$ \pause definiere $f^{**}:A^* \to B^*$ als
% 		\begin{align*}
% 		f^{**}(\varepsilon) &= \varepsilon  \\
% 		\forall w\in A^*, x\in A: f^{**}(wx) &= f^{**}(w) f(x)       
% 		\end{align*}
% 	\end{Definition}

% 	$f^{**}$ ist der durch $f$ \textbf{induzierte} Homomorphismus.
% \end{frame}

\begin{frame}{Homomorphismen}
	\begin{block}{$\varepsilon$-freier Homomorphismus}
		Ein Homomorphismus heißt \textbf{$\mathbf{\varepsilon}$-frei}, wenn für jedes $\ x\in A$ gilt:
		$$ h(x) \neq \varepsilon. $$
	\end{block}

	\begin{exampleblock}{Vorteil}
		\pause Es gehen keine Informationen verloren:\\
		\begin{itemize}
			\item Betrachte $h: \set{a,b}^* \to \set{0,1}^*$ mit $h(a)=001, h(b)=\varepsilon$
			\item Für welches $w$ gilt $h(w)=001$?
			\item Klar: in $w$ muss ein $a$ sein, aber wie viele $b$s?
		\end{itemize}
	\end{exampleblock}
\end{frame}

\begin{frame}{Homomorphismen}
	\begin{exampleblock}{Wann gehen noch Informationen verloren?}
		\begin{itemize}
			\item Betrachte $h: \set{a, b, c}^* \to \set{0,1}^*$ mit $h(a)=0, h(b)=1, h(c)=10$
			\item Für welches $w$ gilt $h(w)=10$?
			\item \pause Für $w=ba$ und $w=c$
		\end{itemize}
	\end{exampleblock}
	\pause
	\begin{block}{Def.: Präfixfreier Homomorphismus}
		Sei $h: A^* \to B^* $ ein Homomorphismus. h heißt \textbf{präfixfrei}, wenn für
		keine zwei verschiedenen Symbole $x_1,x_2\in A$ gilt: $h(x_1)$
		ist ein Präfix von $h(x_2)$.
	\end{block}
	\begin{exampleblock}{Beispiele}
		\begin{itemize}
			\item $h(a)=001, h(b)=1101$ ist präfixfrei
			\item $h(a)=01, h(b)=011$ ist nicht präfixfrei 
		\end{itemize}
	\end{exampleblock}
\end{frame}

\begin{frame}{Homomorphismen}
	\begin{exampleblock}{Beobachtung}
		Präfixfreie Homomorphismen sind $\varepsilon-$frei
	\end{exampleblock}

	\begin{block}{Def.: Codierung}
		Präfixfreie Homomorphismen sind injektiv. Das wollen wir \textbf{Codierungen} nennen.
	\end{block}

	\begin{exampleblock}{Beobachtung}
		Präfixfreie Codes kann man einfach decodieren\footnote{Das liegt daran, dass zu injektiven Abbildungen  Umkehrabbildung existieren}:
		\[
		u(w) = 
		\begin{cases}
		\varepsilon, & \text{ falls } w=\varepsilon\\
		x\,u(w'), & \text{ falls } w=h(x)w' \text{ für ein } x\in A \\
		\bot,  & \text{ sonst }\\
		\end{cases}
		\]
	\end{exampleblock}
\end{frame}

% \section{Huffman-Codierung}
% \subsection{Definition}
% \begin{frame}
% 	Kann man mit einer Codierung die benötigte Anzahl der Zeichen für ein Wort reduzieren und trotzdem den Sinn erhalten?\\[0.5em]
% 	\pause
% 	Natürlich geht das (manchmal), dieses Verfahren ist überall im Einsatz:\\
% 	Komprimierung!
% \end{frame}

\begin{frame}{Huffman-Codierung}
	\textbf{Ziel}: möglichst kurze, präfixfreie Codierungen für beliebige Wörter.
	\pause
	\begin{exampleblock}{}
		Eine Huffman-Codierung ist ein präfixfreier (und demnach einfach zu decodierenden) Homomorphismus, bei der die Codierung eines Zeichens umso länger wird, je seltener das Zeichen vorkommt.
	
		Die Huffman-Codierung für ein Wort ist dabei nicht eindeutig, wie wir gleich im Konstruktionsverfahren sehen werden.
	\end{exampleblock}
	% \begin{block}{Lemma}
	% 	Unter allen präfixfreien Codes führen Huffman-Codes zu kürzesten Codierungen
	% 	\textbf{des Wortes, für das die Huffman-Codierung konstruiert wurde.}
	% \end{block}
\end{frame}
\begin{frame}
	\begin{block}{Nützliche (informelle) Definitionen zu Bäumen}
		\begin{itemize}
			\item \textbf{Baum}: besteht aus Knoten, die über Kanten miteinander verbunden sind
			\item \textbf{Wurzel}: Knoten, von dem aus alle anderen Knoten erreichbar sind
			\item \textbf{Blatt}: Ein Knoten ohne ausgehende Kanten
			\item \textbf{Innerer Knoten}: Alle Knoten, die keine Blätter sind
		\end{itemize}
	\end{block}
\end{frame}

\begin{frame}{Huffman-Codierung}
	\begin{block}{Baum konstruieren (informell)}
		\begin{enumerate}
			\item \textbf{Alle Zeichen} mit ihrer \textbf{Häufigkeit} als Blätter in die unterste Ebene zeichnen.
			\item Jeweils zwei Knoten/Bäume (nicht unbeding Blätter!) mit den \textbf{geringsten Häufigkeiten} verbinden, indem man ihnen eine gemeinsame Wurzel gibst, die die Summe der Häufigkeiten erhält.
			\item Fortfahren, bis nur noch \textbf{ein Baum} übrig ist.
			\item Die linken Äste jeweils mit \bzero \ beschriften, die rechten Äste mit \bone.
			\item Die Codierung einzelner Zeichen kann man entlang der Pfade von der Wurzel zum jeweiligen Zeichen ablesen.
		\end{enumerate}
	\end{block}
	\begin{exampleblock}{Am Beispiel}
		Codiere das Wort $w= cabadcdaac $
	\end{exampleblock}
\end{frame}

\begin{frame}{Huffman-Codierung}
    \begin{block}{Lösung}
    \begin{columns}[T] % align columns
    	\begin{column}{.58\textwidth}
			\begin{figure}[b]
				%\centering
				\begin{tikzpicture}
				[level 1/.style={sibling distance=40mm},
				level 2/.style={sibling distance=20mm},
				level 3/.style={sibling distance=15mm}]
				\node {$10$}
				child {
					node{$6$}
					child{
						node{$3$}
						child{
							node{$1,b$}
							edge from parent node[left] {\bzero}
						}
						child {
							node{$2,d$}
							edge from parent node[right] {\bone}
						}
						edge from parent node[left] {\bzero};
					}
					child{
						node{$3,c$}
						edge from parent node[right] {\bone}
					}
					edge from parent node[left] {\bzero};
				}
				child{
					node{$4,a$}
					edge from parent node[right] {\bone}
				};
				\end{tikzpicture}
			\end{figure}  
		\end{column}
			\hfill
		\begin{column}{.44\textwidth}
			\hfill
			\hfill 
			\vspace*{0.1\linewidth}
			\begin{table}[H]
				\begin{tabular}{c|cccc}
					\hline
					x & a & b & c & d  \\ \hline
					$N_x(w)$  & 4 & 1 & 3 & 2 \\ \hline
					h(x) & 1 & 000 & 01 & 001 \\ \hline 
				\end{tabular}
			\end{table}
		\end{column}
	\end{columns}
    \end{block}
\end{frame}

\begin{frame}{Huffman-Codierung}
	\begin{block}{``Operator'' Huffman-Codierung}
		\begin{enumerate}
			\item Häufigkeiten zählen
			\item Baum konstruieren
			\item Abbildung definieren
			\item Codierung des Wortes
		\end{enumerate}
	\end{block}

	\begin{exampleblock}{Aufgabe}
		Betrachte das Alphabet $A:=\set{a,b,c,d,e}.$
		\begin{enumerate}
			\item Bestimme die Huffman-Codierung $h_1(w)$ des Wortes $w:=bbeacdbbea$. Wie ist $|h_1(w)|?$
			\pause \item Bestimme die Huffman-Codierung $h_2(u)$ mit $u:=a^2 b^4 c^8 d^{16} e^{32}$.
			\pause \item Zerlege $w$ in Zweierblöcke und fasse diese als Zeichen auf. Wie ist nun die Blockcodierung $h_3(w)$? Wie verhält sich $|h_1(w)|$ zu $|h_3(w)|$?
		\end{enumerate}
	\end{exampleblock}
\end{frame}

%%%%%%%%%% %%%%%%%%%%
%% Zusammenfassung
% \section{}
%\subsection{Zusammenfassung}
% 	\begin{frame}{Was ihr jetzt kennen und können solltet\dots}
% 			\begin{itemize}
% 				\item Zwischen Zahlensystemen und Darstellungen jonglieren
% 				\item Codieren
% 			\end{itemize}
	
% 	\end{frame}
% %% Ausblick
% %\subsection{Ausblick}
% 	\begin{frame}{Ausblick}
% 		\begin{itemize}
% 			\item Never forgetti - \sout{Moms Spaghetti} \textbf{Speicher}
% 			\item Die Bits beim Arbeiten beobachten - \textbf{MIMA}
% 		\end{itemize}
% 	\end{frame}
%%%%%%%%%% %%%%%%%%%%
\section{}
\questionframe
\lastframe
\mode<handout>{\slideThanks}
\end{document}