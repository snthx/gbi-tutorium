% Comment/uncoment this line to toogle Mortitz mode
% \newcommand{\Moritz}{}

% Comment/uncomment this line to toggle handout mode
% \newcommand{\handout}{}

% by Stephan

%% Moritz mode or Stephan mode
\ifdefined \Moritz

% This is a configuration file with private, tutor specific information.
% It is therefore excluded from the Git repository so changes in this file will not conflict in git commits.

% Copy this Template, rename to config.tex and add your information below.

\newcommand{\mymail}{moritz.laupichler@student.kit.edu} % Consider using your named student Mail address to keep your u-Account private.

\newcommand{\myname}{\href{mailto:\mymail}{Moritz Laupichler}}

\newcommand{\mytutnumber}{25}

\newcommand{\mytutinfos}{Dienstags, 5. Block (15:45-17:15 Uhr), SR -120}

\newcommand{\aboutMeFrame}{
	\begin{frame}{Euer Tutor}
		Name: \myname \\
		Alter: 21 Jahre \\
		Studiengang: Master Informatik, 1. Semester \\
		\vspace{1cm}
		\pause 
		\centering{Kontakt: \href{mailto:\mymail}{\mymail}}
	\end{frame}
} % Moritz mode
\else
\ifdefined \Alex

% This is a configuration file with private, tutor specific information.
% It is therefore excluded from the Git repository so changes in this file will not conflict in git commits.

% Copy this Template, rename to config.tex and add your information below.

\newcommand{\mymail}{alexander.klug@student.kit.edu} % Consider using your named student Mail address to keep your u-Account private.

\newcommand{\myname}{\href{mailto:\mymail}{Alexander Klug}}

\newcommand{\mytutnumber}{30}

\newcommand{\mytutinfos}{Mittwochs, 3. Block (11:30-13:00), SR -107}

\newcommand{\aboutMeFrame}{
	\begin{frame}{Euer Tutor}
		Name: \myname \\
		Alter: 19 Jahre \\
		Studiengang: Bachelor Informatik, 3. Semester \\
		\vspace{1cm}
		\pause 
		\centering{Kontakt: \href{mailto:\mymail}{\mymail}}
	\end{frame}
}

% Toggle Handout mode by including the following line before including style_tut
% and removing the % at the start (but do NOT remove it here, otherwise handout mode will always be on!)
% Please keep handout mode on in all commits!

% \newcommand{\handout}{} % Alex Mode
\else

% This is a configuration file with private, tutor specific information.
% It is therefore excluded from the Git repository so changes in this file will not conflict in git commits.

% Copy this Template, rename to config.tex and add your information below.

\newcommand{\mymail}{stephan.bohr@student.kit.edu} % Consider using your named student Mail address to keep your u-Account private.

\newcommand{\myname}{\href{mailto:\mymail}{Stephan Bohr}}

\newcommand{\mytutnumber}{25}

\newcommand{\mytutinfos}{Dienstags, 5. Block (15:45-17:15), SR -119}

\newcommand{\aboutMeFrame}{
	\begin{frame}{Euer Tutor}
		Name: \myname \\
		Alter: 20 Jahre \\
		Studiengang: Bachelor Informatik, 3. Semester \\
		\vspace{1cm}
		\pause 
		\centering{Kontakt: \href{mailto:\mymail}{\mymail}}
	\end{frame}
} % Stephan mode
\fi
\fi

%% Beamer-Klasse im korrekten Modus
\ifdefined \handout
\documentclass[handout]{beamer} % Handout mode
\else
\documentclass{beamer}
\fi
%\documentclass[18pt,parskip]{beamer}

%% SLIDE FORMAT

% use 'beamerthemekit' for standard 4:3 ratio
% for widescreen slides (16:9), use 'beamerthemekitwide'

\usepackage{../templates/KIT-slides/beamerthemekit}
%\usepackage{../templates/KIT-slides/beamerthemekitwide}

%% TITLE PICTURE

% if a custom picture is to be used on the title page, copy it into the 'logos'
% directory, in the line below, replace 'mypicture' with the 
% filename (without extension) and uncomment the following line
% (picture proportions: 63 : 20 for standard, 169 : 40 for wide
% *.eps format if you use latex+dvips+ps2pdf, 
% *.jpg/*.png/*.pdf if you use pdflatex)

\titleimage{../figures/titleimage/brain}

%% TITLE LOGO

% for a custom logo on the front page, copy your file into the 'logos'
% directory, insert the filename in the line below and uncomment it

%\titlelogo{mylogo}

% (*.eps format if you use latex+dvips+ps2pdf,
% *.jpg/*.png/*.pdf if you use pdflatex)

%% TikZ INTEGRATION

% use these packages for PCM symbols and UML classes
% \usepackage{templates/tikzkit}
% \usepackage{templates/tikzuml}

%\usepackage{tikz}
%\usetikzlibrary{matrix}
%\usetikzlibrary{arrows.meta}
%\usetikzlibrary{automata}
%\usetikzlibrary{tikzmark}

%%%%%%%%%%%%%%%%%%%%%%%%%
% Libertine font (Original GBI font)
\usepackage{libertine}
%\renewcommand*\familydefault{\sfdefault}  %% Only if the base font of the document is to be sans serif

%% Schönere Schriften
\usepackage[TS1,T1]{fontenc}

%% Deutsche Silbentrennung und Beschriftungen
\usepackage[ngerman]{babel}

%% UTF-8-Encoding
\usepackage[utf8]{inputenc}

%% Bibliotheken für viele mathematische Symbole
\usepackage{amsmath, amsfonts, amssymb}

%% Anzeigetiefe für Inhaltsverzeichnis: 1 Stufe
\setcounter{tocdepth}{1}

%% Hyperlinks
\usepackage{hyperref}
% I don't know why, but this works and only includes sections and NOT subsections in the pdf-bookmarks.
\hypersetup{bookmarksdepth=subsection}

%% remove navigation symbols
\setbeamertemplate{navigation symbols}{}

%% switch between "ngerman" and "english" for German/English style date and logos
\selectlanguage{ngerman}

%% for invisible pause texts instead of dimming
\setbeamercovered{invisible}

%%%%%%%%%%%% Shortcuts %%%%%%%%%%%%%
\newcommand{\nM}{\mathbb{M}}
\newcommand{\nR}{\mathbb{R}}
\newcommand{\nN}{\mathbb{N}}
\newcommand{\nZ}{\mathbb{Z}}
\newcommand{\nQ}{\mathbb{Q}}
\newcommand{\nB}{\mathbb{B}}
\newcommand{\nC}{\mathbb{C}}
\newcommand{\nK}{\mathbb{K}}
\newcommand{\nF}{\mathbb{F}}
\newcommand{\nG}{\mathbb{G}}
\newcommand{\nullel}{\mathcal{O}}
\newcommand{\einsel}{\mathds{1}}
\newcommand{\nP}{\mathbb{P}}
\newcommand{\Pot}{\mathcal{P}}
\renewcommand{\O}{\text{O}}

\newcommand{\set}[1]{\{ #1 \}}
\newcommand{\setc}[2]{\set{#1 \mid #2}}
\newcommand{\setC}[2]{\set{#1 \mid \text{ #2 }}}

\newcommand{\setsize}[1]{\; \mid #1 \mid \; }

\newcommand{\q}[1]{\textquotedblleft #1\textquotedblright}

%%%%%%%%%%%% INHALT %%%%%%%%%%%%%%%%

%% Wochennummer
%\newcounter{weeknum}

%% Titelinformationen
%\title[GBI Tutorium, Woche \theweeknum]{Grundbegriffe der Informatik \\ Tutorium \mytutnumber}
%\subtitle{Termin \theweeknum \ | \mydate \\ \myname}
\author[\myname]{\myname}
\institute{Fakultät für Informatik}
%\date{\mydate}

%% Titel einfügen
\newcommand{\titleframe}{\frame{\titlepage}\addtocounter{framenumber}{-1}}


%% Alles starten mit \starttut{X}
%\newcommand{\starttut}[1]{\setcounter{weeknum}{#1}\titleframe\frame{\frametitle{Inhalt}\tableofcontents} \AtBeginSection[]{%
%\begin{frame}
%	\tableofcontents[currentsection]
%\end{frame}\addtocounter{framenumber}{-1}}}


%\newcommand{\framePrevEpisode}{
%	\begin{frame}
%		\centering
%		\textbf{In the previous episode of GBI...}
%	\end{frame}
%}

%% Roadmap frame
%table of contents
\newcommand{\roadmap}{
	\frame{\frametitle{Roadmap}\tableofcontents}}

 \AtBeginSection[]{%
\begin{frame}
	\frametitle{Roadmap}
	\tableofcontents[currentsection]
\end{frame}%\addtocounter{framenumber}{-1}
}


%% ShowMessage frame
\newcommand{\showmessage}[1]{\frame{\frametitle{\phantom{1em}}\centering\textbf{#1}}}

%% Fragen
%% Lastframe
\newcommand{\questionframe}{\showmessage{Fragen?}}

%% Lastframe
\newcommand{\lastframe}{\showmessage{Vielen Dank für Eure Aufmerksamkeit! \\Bis nächste Woche :)}}

%% Thanks frame
\newcommand{\slideThanks}{
	\begin{frame}
		\frametitle{Credits}
		\begin{block}{}
			An der Erstellung des Foliensatzes haben mitgewirkt:\\[1em]
			\ifdefined \Moritz
			Stephan Bohr \\
			Alexander Klug \\
			\else
			\ifdefined \Alex
			Stephan Bohr \\
			Moritz Laupichler \\
			\else
			Moritz Laupichler \\
			Alexander Klug \\
			\fi
			\fi
			Katharina Wurz \\
			Thassilo Helmold \\
			Philipp Basler \\
			Nils Braun \\
			Dominik Doerner \\
			Ou Yue \\
		\end{block}
	\end{frame}
}

%% Verbatim
%\usepackage{moreverb}



\title[Aussagenkalkül, vollständige Induktion, formale Sprachen]{4. Tutorium\\ Aussagenkalkül, vollständige Induktion, formale Sprachen}
\subtitle{Grundbegriffe der Informatik, Tutorium \#\mytutnumber}
\date{\today}

\begin{document}
\titleframe
\roadmap

%%%%%%%%%% %%%%%%%%%%
\section*{Übungsblatt}
\subsection{Rückmeldungen, Statikstik, whatever}
	\begin{frame}{Zum 1. Übungsblatt}
	    \Alex{
	    	\begin{alertblock}{Aufgabe 1.4 b)}
	    		A und B waren beliebig, d.h. kein konkretes Beispiel!
	    	\end{alertblock}
    	\pause
			\begin{alertblock}{Aufgabe 1.5}				
				Beim Widerlegen: Möglichst einfaches Gegenbeispiel finden. \\
				Beim Zeigen: Formaler Beweis (Bitte mit $\subseteq$ und  $\supseteq$).
			\end{alertblock}    
    }
	\end{frame}

\section[Aussagenlogik]{Aussagenlogik: Wiederholung, Beweisbarkeit}
\subsection{Wiederholung}
\begin{frame}{Aussagenlogik: Wiederholung}
	\begin{block}{Def.: Interpretation}
		Für eine Menge \(V\) von Aussagevariablen ist eine \textbf{Interpretation} ist eine Abbildung \(I: V \to \nB\), wobei \( \BB = \set{w, f} \). % Man sagt auch "Belegung"
	\end{block}

	\begin{block}{Auswertung/Validierung aussagenlogischer Formeln}
		Sei \(I: V \to \nB\) eine Interpretation.\\
		Für jede aussagelogische Formel $F$ definiere $\vali{F}$ wie folgt:\\[2ex]

		Für jedes $X \in V$, $G$, $H$ weitere aussagelogische Formeln sei:
		\begin{itemize}
			\item $\vali{X}         := I(X) $
  			\item $\vali{\bnot G}   := \bfnot{\vali{G}} $
  			\item $\vali{G \bund H} := \bfand{\vali{G}}{\vali{H}}$
  			\item $\vali{G \boder H} := \bfor{\vali{G}}{\vali{H}}$
  			\item $\vali{G \bimp H} := \bfimp{\vali{G}}{\vali{H}}$
		\end{itemize}
	\end{block}
\end{frame}

\begin{frame}{Aussagenlogik: Wiederholung}
	\begin{block}{Def.: Modell}
		Eine Interpretation $I$ heißt \textbf{Modell} einer Formel $G$, wenn $\vali{G} =\mathbf{w}$.\\

		Eine Interpretation $I$ heißt \textbf{Modell} einer Formelmenge $\Gamma$, wenn $I$ Modell jeder Formel $G\in \Gamma$ ist.
	\end{block}

	\begin{exampleblock}{}
	Wir sagen:\\
	\textcolor{black!50!red}{$\Gamma \models G$}, wenn jedes Modell von $\Gamma$ auch ein Modell von $G$ ist.\\
	\textcolor{black!50!red}{$\models G$} (statt \textcolor{black!50!red}{$\set{}\models G$}), wenn $G$ für \emph{alle} Interpretationen wahr ist.
	\end{exampleblock}

	\begin{block}{Def.: Tautologie}
		Eine Formel $G$ heißt \textbf{Tautologie}, wenn jede Interpreation $I$ Modell ist. Das heißt, für jede Belegung (Interpretation $I$) der Aussagevariablen ist die gesamte aussagenlogische Formel wahr:
			\[	\vali{G}=\mathbf{w} \quad \text{ für alle } I	\]
	\end{block}
\end{frame}




\subsection{Beweisbarkeit}
\begin{frame}{Aussagenlogik: Beweisbarkeit}
	\begin{block}{Hilbertkalkül für die Aussagenlogik}
		Das \textbf{Hilbertkalkül für die Aussagenlogik}, (vereinfacht) auch \textbf{Aussagenkalkül}, besteht aus
		\begin{itemize}
			\item dem Alphabet $\AAL$,
			\item der Menge der syntakisch korrekten Formeln $\LAL \subseteq \AAL^*$,
			\item einer Menge von \textbf{Axiomen} $\AxAL \subseteq \LAL$,
			\item der (einzigen) \textbf{Schlussregel} \textbf{Modus Ponens} $\MP \subseteq \LAL^3$
		\end{itemize}
	\end{block}
\end{frame}

\begin{frame}{Aussagenlogik: Beweisbarkeit}
	\begin{block}{Axiome}
		\begin{align*}
  			\AxAL &= \bigl\{\alka G\alimpl \alka H\alimpl  G\alkz\alkz
          			\bigm| G,H\in\LAL \bigr\} \\
        		&\mathrel{\hphantom{=}} \cup \bigl\{\alka G\alimpl \alka H\alimpl  K\alkz\alkz
          			\alimpl \alka\alka G\alimpl H\alkz\alimpl \alka G\alimpl  K\alkz\alkz \bigm| G,H,K\in\LAL \bigr\}\\
        		&\mathrel{\hphantom{=}} \cup \bigl\{
          			\alka\alnot H\alimpl \alnot G\alkz\alimpl \alka\alka\alnot H\alimpl G\alkz\alimpl  H\alkz
          			\bigm| G,H \in\LAL 
          			\bigr\}
		\end{align*}

		Kurz: $\AxAL{}_1$, $\AxAL{}_2$ und $\AxAL{}_3$ für die drei Zeilen.
	\end{block}
\end{frame}

\begin{frame}{Aussagenlogik: Beweisbarkeit}
	\begin{block}{Modus Ponens $\MP \subseteq \LAL^3$ (Schlussregel)}
		\begin{itemize}
			\item $\MP = \{ (G\alimpl H, G, H) \mid  G, H \in\LAL \}$
			\item $\MP:$ \quad \begin{tabular}{c}
                $G \alimpl H$ \qquad $G$ \\
                \midrule
                $H$
              \end{tabular}
             \item Aus einer Formel der Form $G \alimp H$ und einer Formel der Form $G$ darf man auf eine Formel der Form $H$ schließen.
		\end{itemize}
	\end{block}

	\begin{exampleblock}{Beispiel}
		\begin{description}
			\item[\emph{Prämissen}:] $G \alimpl H$: \enquote{Wenn es regnet, wird die Straße nass},\\
								 $G$: \enquote{Es regnet}
			% Sind diese beiden Prämissen gültig, so ist auch der Schluss "Conclusio" gültig.
			\item[\emph{Conclusio}:] $H$: \enquote{Die Straße wird nass}.
		\end{description}
	\end{exampleblock}

	% \begin{exampleblock}{Beispiel}
	% 	\begin{itemize}
	% 		\item[Prämissen:] $G \alimpl H$: \enquote{Wenn du Mensch bist, stirbst du},\\
	% 							 $G$: \enquote{Du bist ein Mensch}
	% 		\item[Conclusio:] $H$: \enquote{Du stirbst}.
	% 	\end{itemize}
	% \end{exampleblock}
\end{frame}	

\begin{frame}{Aussagenlogik: Beweisbarkeit}
	\begin{block}{Def.: Ableitung}
		Sei $\Gamma \subseteq \LAL$ eine Menge von \textbf{Hypothesen} oder \textbf{Prämissen} und $G$ eine Formel.\\
		Eine \textbf{Ableitung} von $G$ aus $\Gamma$ ist eine endliche Folge $(G_1, \dots, G_n)$ mit
		\begin{itemize}
			\item $G_n = G$ und 
			\item für jedes $G_i (1 \leq i \leq n)$ gilt einer der folgenden Fälle:
			\begin{itemize}
				\item $G_i\in\AxAL$ oder% $G_i$ ist ein Axiom
				\item $G_i\in\Gamma$ oder% $G_i$ ist eine Prämisse
				\item es gibt $i_1,i_2 < i$ mit $(G{i_1},G_{i_2},G_i)\in\MP$.
			\end{itemize}
		\end{itemize}

		Geschrieben: $\Gamma\vdash G$
	\end{block}

	\begin{block}{Def.: Beweis und Theorem}
	Ist $\Gamma=\{\}$, so heißt eine entsprechende Ableitung auch ein \textbf{Beweis} von $G$ und $G$ ein \textbf{Theorem} des Kalküls, in Zeichen: $\vdash G$
	\end{block}
\end{frame}

\begin{frame}{Aussagenlogik: Beweisbarkeit}
	\begin{exampleblock}{Beispiel zum Modus Ponens}
		Ableitung/Beweis des Theorems $\alka\alP\alimpl\alP\alkz$:\\[1em]
		\begin{tabular}{rll}
			1. & $\alka \alka \alP \alimpl \alka \alka \alP \alimpl  \alP \alkz\alimpl  \alP \alkz\alkz\alimpl 
       			\alka \alka \alP \alimpl \alka \alP \alimpl  \alP \alkz\alkz\alimpl \alka \alP \alimpl  \alP \alkz\alkz\alkz$ & $\AxAL{}_2$ \\
			2. & $\alka \alP \alimpl \alka \alka \alP \alimpl  \alP \alkz\alimpl  \alP \alkz\alkz$ & $\AxAL{}_1$ \\
			3. & $\alka \alka \alP \alimpl \alka \alP \alimpl  \alP \alkz\alkz\alimpl \alka \alP \alimpl  \alP \alkz\alkz$ & $\MP(1,2)$ \\
			4. & $\alka \alP \alimpl \alka \alP \alimpl  \alP \alkz\alkz$ & $\AxAL{}_1$ \\
			5. & $\alka \alP \alimpl  \alP \alkz$ & $\MP(3,4)$
			\end{tabular}

			\qedwhite{}
	\end{exampleblock}
\end{frame}

\begin{frame}{Aussagenlogik: Beweisbarkeit}    
	\begin{alertblock}{Achtung}
		\begin{itemize}
			\item \emph{Tautologie}, \emph{Modell} und \emph{Theorem} sind unterschiedliche Wörter
			\item $\models$ und $\vdash$ sind unterschiedliche Symbole
		\end{itemize}
	\end{alertblock}
\end{frame}

\section[Vollständige Induktion]{Beweisverfahren d. vollständigen Induktion}
\subsection{Einstieg u. Definition}
\begin{frame}{Vollständige Induktion}
	\begin{block}{Vollständige Induktion}
		Vollständige Induktion beschreibt ein Beweisverfahren, das man verwendet um die Gültigkeit einer Aussage \(A(n)\) für alle \(n \in \nN_{0} \text{ (oder } \nN\)) zu beweisen.\\
		
		Sie besteht aus:
		\begin{description}
			\item [\textbf{Induktionsanfang (I.A.):}] Zeige, dass \(A(n)\) für \(n=0\) (bzw. \(n=1\)) gilt.
			\item [\textbf{Induktionsvoraussetzung (I.V.):}] Stelle die Annahme auf, dass \(A(n)\) für \textcolor{red}{ein beliebiges, aber festes} \(n \in \nN_{0}\) (oder \(n \in \nN\)) gilt.
			\item [\textbf{Induktionsschritt (I.S.):}] Zeige, dass, wenn \(A(n)\) gilt, auch \(A(n+1)\) gilt. Verwende hierbei die Induktionsvorraussetzung!
		\end{description}
	\end{block}

	\begin{alertblock}{}
		Es gibt einige Varianten vollständiger Induktion. Immer darauf achten, was man braucht und will. Tipp: \textbf{Nachdenken!}
	\end{alertblock}
\end{frame}
\subsection{Vorgerechnetes Beispiel}
\begin{frame}{Vollständige Induktion - Ein dummes Beispiel}
	\begin{exampleblock}{Aufgabe}
		Sei $x_n$ für $x \in \nN_0$ wie folgt definiert:
		\[
			x_n := (-1)^n.
		\]
		Zeige mit vollständiger Induktion, dass 
		$x_n = 
		\begin{cases}
			-1, &\text{ falls $n$ ungerade}\\
			1, &\text{ sonst}
		\end{cases}$
	\end{exampleblock}
\end{frame}
\begin{frame}{Vollständige Induktion - Ein dummes Beispiel}
	\begin{block}{Lösung}
		\begin{itemize}
			\item[I.A.] $n=0: (-1)^0=1 \quad \checkmark$
			\item[I.V.:] Für ein bel., aber festes $n \in \nN_0$ gelte: \\
				 			$x_n = 
					\begin{cases}
						-1, &\text{ falls $n$ ungerade}\\
						1, &\text{ sonst}
					\end{cases}.$
			\item[I.S.:] $n \notStephan{\leadsto} \Stephan{\curvearrowright} n+1$.\\
			\begin{itemize}
				\item Fall 1: $x_{n+1} \text{ ist ungerade } \Rightarrow x_n \text{ ist gerade}$\\
					$x_{n+1} = (-1)^{n+1} = (-1) \cdot (-1)^n \eqtext{I.V.} (-1) \cdot 1 = -1 \quad \checkmark$
				\item Fall 2: $x_{n+1} \text{ ist gerade } \Rightarrow x_n \text{ ist ungerade}$\\
					$x_{n+1} = (-1)^{n+1} = (-1) \cdot (-1)^n \eqtext{I.V.} (-1) \cdot (-1) = 1 \quad \checkmark$
			\end{itemize}
			Mit Fall 1 und 2 folgt die Behauptung. \qedwhite{}
		\end{itemize}
	\end{block}
\end{frame}
\subsection{Einfaches Mathematisches Beispiel zum Selberrechnen}

\section{Formale Sprachen}
\subsection{Wiederholung Sprachen} % Produkt v. Sprachen, Potenzen, L^*, L^+ und dabei induktives Definieren klarmachen
\begin{frame}{Wdh.: Formale Sprachen}
	\begin{block}{Def.: Formale Sprache}
		Eine \textbf{formale Sprache} L über einem Alphabet A ist eine Teilmenge der Wörter über A, also \(L \subseteq A^{*}\).
	\end{block}

	\begin{block}{Def.: Produkt formaler Sprachen}
		Es sei $A$ ein Alphabet, sowie $L,S \subseteq A^*$ formale Sprachen.\\
		Das \textbf{Produkt} der Sprachen ist definiert durch
		\[
			L \cdot S := \setc{u \cdot v}{u \in L \text{ und } v \in S}
		\]
	\end{block}

	\begin{exampleblock}{Beispiel}
		Es sei $L:=\set{\texttt{aa}, \texttt{bb}}$. Dann ist $L\cdot L = \set{\texttt{aaaa}, \texttt{aabb}, \texttt{bbbb}, \texttt{bbaa}}$
	\end{exampleblock}
\end{frame}

\begin{frame}{Wdh.: Formale Sprachen}
	\begin{block}{Def.: Potenzen von Sprachen}
		Die \textbf{Potenzen} einer formalen Sprache $L$ sind induktiv definiert:
		\begin{align*}
			L^0 &= \set{\varepsilon} \\
			\text{Für alle $n \in \nN_0$ gilt: } L^{n+1} &= L^n \cdot L
		\end{align*}
	\end{block}

	\begin{block}{Def.: Konkatenationsabschluss}
		Sei $L$ eine formale Sprache. Dann ist \\
		$L^* = \bigcup_{i=0}^{\infty} L^{i}$ der \textbf{Konkatenationsabschluss} $L^*$ von $L$ und \\
		$L^+ = \bigcup_{i=1}^{\infty} L^{i}$ der \textbf{$\varepsilon$-freie Konkatenationsabschluss} $L^+$ von $L$.
	\end{block}

	\begin{alertblock}{Achtung}
		Auch beim $\varepsilon$-freien Konkatenationsabschluss kann $\varepsilon$ enthalten sein, nämlich gdw. $\varepsilon \in L$
	\end{alertblock}
\end{frame}



\section{Übungsaufgaben}
\subsection{Ein leichter induktiver Beweis auf einer Sprache}
\begin{frame}{Induktiv Beweisen I.: Zum Warmwerden}
	\begin{exampleblock}{Behauptung}
		Es sei w ein Wort. Es gilt:\\
		\centering{Für alle $k \in \nN_0$ gilt $|w^k| = k \cdot |w|$.}
	\end{exampleblock}
\pause
	\begin{block}{Beweis}
		\begin{itemize}
			\item[I.A.:] Es sei $k=0$. Es gilt $|w^k|=|w^0|=|\varepsilon|=0=0 \cdot |w| = k \cdot |w|$.
			\item[I.V.:] Es sei $|w^k|=k \cdot |w|$ für ein bel., aber festes $k\in \nN_0$.
			\item[I.S.:] $k \Stephan{\curvearrowright} \notStephan{\leadsto} k+1$:\\
						 $|w^{k+1}|=|w^k \cdot w|=|w^k|+|w| \eqtext{I.V.} k \cdot |w| + |w| = (k+1) \cdot |w|$
		\end{itemize}
	\end{block}
\end{frame}
\subsection{Variante vollständiger Induktion} % z.B. gilt für alle <=n oder mit größerem Anfang und n->n+2, etc
\begin{frame}{Induktiv Beweisen II.: Schwieriger}
	\begin{exampleblock}{Aufgabe}
		Eine Funktion $T:\nN_0 \rightarrow \nN_0$ sei wie folgt definiert:
		\begin{equation*}
		T(n) = 
			\begin{cases}
			 2, &\text{ wenn } n=0 \\
			 3, &\text{ wenn } n=1 \\
			 3 \cdot T(n-1) - 2 \cdot T(n-2), &\text{ wenn }n \in \nN_0 \setminus \set{0,1}
		\end{cases} \pause
		\end{equation*}
		\begin{enumerate}
			\item Gib die Funktionswerte $T(n)$ für $n \in \set{2,3,4,5,6}$ an.
			\item Gib eine geschlossene Formel $F(n)$ (einen arithmetischen Ausdruck ohne Rekursion) für $T(n)$ an.
			\item Zeige durch vollständige Induktion, dass für alle $n \in \nN_0$ gilt $F(n) = T(n)$.
		\end{enumerate}
	\end{exampleblock}
\end{frame}
\subsection{Ein schwieriger mathematischer Beweis}
\begin{frame}{Induktiv Beweisen II.: Schwieriger}
	\begin{block}{Lösung}
		\begin{enumerate}
			\item $ T(2)=5,T(3)=9, T(4)=17, T(5)=33, T(6)=65$
			\item $F(n) = 2^n+1$
			\newcounter{kevin}
			\setcounter{kevin}{\value{enumi}}
		\end{enumerate}
	\end{block}
\end{frame}
\begin{frame}{Induktiv Beweisen II.: Schwieriger}
	\begin{block}{Lösung}
		\begin{enumerate}
			\setcounter{enumi}{\value{kevin}}
			\item \begin{itemize}
		 		\item Beh.: Für alle $n \in \nN_0$ gilt $F(n) = T(n)$.
		 		\item Bew. (induktiv): 
				\begin{itemize}
					\item[I.A.:]
								Sei $n=0$. Dann gilt: \\ 
								\qquad $F(n)=F(0)=2^0+1=1+1=2=T(0)=T(n)$.\\
							 	Sei $n=1$. Dann gilt: \\
							 	\qquad $F(n)=F(1)=2^1+1=2+1=3=T(1)=T(n)$.
				\end{itemize}
		 	\end{itemize}
		\end{enumerate}		 					  
	\end{block}
\end{frame}
\begin{frame}{Induktiv Beweisen II.: Schwieriger}
	\begin{block}{Lösung}
	\begin{enumerate}
		\setcounter{enumi}{\value{kevin}}
		\item
		\begin{itemize}
				\item[I.V.:] Für ein bel., aber festes $n \in \nN_0$ gelte: \\
				 $T(n)=F(n)=2^n+1$ und $T(n+1)=F(n+1)=2^{n+1}+1$.
				\item[I.S.:] \zz $T(n+2) = F(n+2)$, also $n \notStephan{\leadsto} \Stephan{\curvearrowright} n+2$.
				\begin{align*}
					T(n+2) &= 3 \cdot T(n+1) - 2 \cdot T(n) \\
						   &\eqtext{I.V.} 3 \cdot F(n+1) - 2 \cdot F(n) \\
						   &= 3 \cdot (2^{n+1}+1) - 2 \cdot (2^n+1) \\
						   &= 3 \cdot 2^{n+1} + 3 - 2 \cdot 2^n - 2 \\
						   &= 3 \cdot 2^{n+1} - 2 \cdot 2^n + 1 \\
						   &= 3 \cdot 2^{n+1} - 2^{n+1} +1 \\
						   &= 2 \cdot 2^{n+1} + 1 = 2^{n+2} + 1 \\
						   &=F(n+2)
				\end{align*}
			\end{itemize}
		\end{enumerate}
	\end{block}
\end{frame}
\subsection{Eine Knobelaufgabe}
\begin{frame}{Induktiv Beweisen III.: Zum Knoblen}
	\begin{exampleblock}{Aufgabe}
		Alice und Bob feiern ihren Hochzeitstag. Auf ihrer Party befinden sich $n \in \nN_+$ Paare. Dabei begrüßen sich alle Paare mit Ausnahme des eigenen Partners.\\
		\begin{enumerate}
			\item Gib die Anzahl der Begrüßungen für $i \in \set{1,2,3,4,5}$ Paare an.
			\item Stelle für die Anzahl der Begrüßungen einen geschlossenen Ausdruck $p_n$ auf.
			\item Gib für die Anzahl der Begrüßungen eine induktive Definiton $q_n$ an.
			\item Zeige per vollständiger Induktion, dass für alle $n \in \nN_+$ gilt: $p_n$ = $q_n$.
		\end{enumerate}
	\end{exampleblock}
\end{frame}

\begin{frame}{Induktiv Beweisen III.: Zum Knoblen}
	\begin{block}{Lösung}
		\begin{enumerate}
			\item \begin{itemize}
				\item 1 Paar $\rightarrow$ 0 Begrüßungen
				\item 2 Paare $\rightarrow$ 4 Begrüßungen
				\item 3 Paare $\rightarrow$ 12 Begrüßungen
				\item 4 Paare $\rightarrow$ 24 Begrüßungen
				\item 5 Paare $\rightarrow$ 40 Begrüßungen
			\end{itemize} \pause
			\item $p_n = 2 \cdot n \cdot (n-1) = 2 \cdot (n^2 -n)$ \pause
			\item Jedes Paar, das neu hinzukommt, muss die $n$ sich im Raum befindenden Paare begrüßen. Das bedeutet $2n$ Begrüßungen (ein Paar besteht aus $2$ Partnern) pro neu hinzugekommenen Partner, also insgesamt $4n$ Begrüßungen. Formal heißt das: 
			\begin{align*}
				q_1 &= 0 \\
				\text{Für alle } n \in \nN_+ \text{ gilt: } q_{n+1} &= q_n + 4n
			\end{align*}
			\setcounter{kevin}{\value{enumi}}
		\end{enumerate}
	\end{block}
\end{frame}

\begin{frame}{Induktiv Beweisen III.: Zum Knoblen}
	\begin{block}{Lösung}
	\begin{enumerate}
		\setcounter{enumi}{\value{kevin}}
		\item \begin{itemize}
			\item[I.A.] $n=1: p_n = p_1 = 2 \cdot (1^2 -1 ) = 0 = q_1 = q_n \quad \checkmark$
			\item[I.V.] Für ein beliebiges, aber festes $n \in \nN_+$ gelte $p_n=q_n$.
			\item[I.S.] \zz $p_{n+1}=q_{n+1}$, also $n \notStephan{\leadsto} \Stephan{\curvearrowright} n+1$
			\begin{align*}
				p_{n+1} &= 2 \cdot (n+1) \cdot (n+1-1) \\
						&= 2n^2 + 2n \\
						&= 2n^2 -2n + 4n \\
						&= 2 \cdot n \cdot (n-1) + 4n \\
						&= p_n + 4n \\
						&\eqtext{I.V.} q_n + 4n \\
						&\eqtext{Def.} q_{n+1} & \qedwhite{}
			\end{align*}
		\end{itemize}
	\end{enumerate}		
	\end{block}
\end{frame}




%%%%%%%%%% %%%%%%%%%%
%% Zusammenfassung
\section{}
%\subsection{Zusammenfassung}
	\begin{frame}{Was ihr jetzt kennen und können solltet\dots}
			\begin{itemize}
				\item mit dem Aussagenkalkül arbeiten
				\item Induktionsbeweise, auch in Varianten
				\item Mit formalen Sprachen umgehen
			\end{itemize}
	
	\end{frame}
%% Ausblick
%\subsection{Ausblick}
	\begin{frame}{Ausblick}
		\begin{itemize}
			\item aus \texttt{2} mach \texttt{10}: \emph{Zahlensysteme übersetzen}\Stephan{, GBI-flavoured ;-)} 
			\item Weniger ist mehr: \emph{Huffmann-Codierung}
		\end{itemize}
	\end{frame}
%%%%%%%%%% %%%%%%%%%%
\section{}
\questionframe
\lastframe
\mode<handout>{\slideThanks}
\end{document}